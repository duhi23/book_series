\chapter{Modelo de Correcci\'{o}n del error (MCE)}
%\label{sec:mylabel2}
El modelo de correcci\'{o}n del error\index{Modelos MCE} (MCE) recibe su nombre debido a que mediante este, se corrigen los errores o desviaciones que las variables presentan en el corto plazo con respecto a su relaci\'{o}n de equilibrio a largo plazo.\newline

Del p\'{a}rrafo anterior se puede determinar que el MCE est\'{a} relacionado con la cointegraci\'{o}n de las variables; si las variables no est\'{a}n cointegradas, no se puede plantear el MCE.\newline

Al utilizar este m\'{e}todo, se considera el comportamiento din\'{a}mico de las series; adem\'{a}s, presenta ventajas con respecto a los m\'{e}todos en los que se utiliza la diferenciaci\'{o}n de las series para convertirlas en estacionarias, pues pueden dar lugar a casos de sobrediferenciaci\'{o}n y p\'{e}rdida de informaci\'{o}n.

\section{Planteamiento del MCE}
%\label{subsec:mylabel3}
Se considera el modelo VAR(p) de $k$ variables, donde\index{Modelos MCE!Planteamiento} por simplicidad se omiten las posibles componentes deterministas: 
\[
X_{t}=A_{1}X_{t-1}+A_{2}X_{t-2}+\ldots +A_{p}X_{t-p}+u_{t}
\]
con $u_{t}$ ruido blanco.\newline

Consid\'{e}rese, 
\[
A(B)=I-A_{1}B-\ldots -A_{p}B^{p}
\]

Recu\'{e}rdese que si las ra\'{i}ces del polinomio caracter\'{i}stico est\'{a}n fuera del c\'{i}rculo unidad, $X_{t}$ es estacionario; es decir, es $I(0)$. Si $|A(1)|=0$, se dice que $X_{t}$ tiene una ra\'{i}z unitaria. Por simplicidad de la exposici\'{o}n, se asume que $X_{t}$ es integrada de orden 1 $(I(1))$; esto significa que: 
\[
(1-B)X_{t}
\]
es estacionaria, si $X_{t}$ no lo es.\newline

Un modelo de correcci\'{o}n del error (MCE) para un proceso $VAR(p)$ es:
\[
\Delta X_{t}=\Omega X_{t-1}+A_{1}^{\ast}\Delta X_{t-1}+\ldots +A_{p-1}^{\ast}\Delta X_{t-p+1}+u_{t}
\]
donde, 
\[
A_{j}^{\ast }=-\sum\limits_{i=j+1}^p A_{i} ,\qquad j=1,\ldots ,p-1
\]
\[
\Omega =\alpha \beta^{'} = A_{p}+A_{p-1}+\ldots +A_{1}-I=-A(1)
\]
$\alpha$ y $\beta$ son matrices de dimensi\'{o}n $k*r$ (r es el n\'{u}mero de factores de cointegraci\'{o}n y $r<k)$, de rango completo.\newline

Al t\'{e}rmino $\mathrm{\Omega }X_{t-1}$ se lo conoce como el\index{Modelos MCE!Termino de correcci\'{o}n del error} \textbf{t\'{e}rmino de correcci\'{o}n del error}, que desempe\~{n}a un papel clave en el estudio de cointegraci\'{o}n. La existencia del t\'{e}rmino $\beta^{'}X_{t-1}$ es natural en la representaci\'{o}n de correcci\'{o}n del error; adem\'{a}s, es estacionario. Las columnas de $\beta$ son los vectores de cointegraci\'{o}n de $X_{t}$.\newline

La estacionariedad de $\beta^{'} X_{t-1}$ puede ser justificada de la siguiente manera: La teor\'{i}a de series con ra\'{i}ces unitarias muestra que el coeficiente de correlaci\'{o}n simple entre una serie no estacionaria con ra\'{i}z unitaria y una estacionaria tiende a 0 cuando el tama\~{n}o de la muestra tiende a infinito $(T\to \infty )$. En un MCE $X_{t-1}$ es no estacionario, pero $\Delta X_{t}$ es estacionario; por lo tanto, solo existe una forma en que $\Delta X_{t}$ puede relacionarse significativamente con $X_{t-1}$: a trav\'{e}s de la serie estacionaria $\beta^{'} X_{t-1}$.\newline

Para poder recuperar $A_{i}$ a partir de MCE se tiene:
\[
A_{1}=I+\Omega +A_{j}^{\ast }
\]
\[
A_{i}=A_{i}^{\ast}-A_{i-1}^{\ast },\qquad i=2,\ldots ,p
\]
donde $A_{p}^{\ast}=0$, la matriz cero. En lo que sigue se supone que $X_{t}$ es al menos $I(1)$. Por lo tanto, se consideran tres casos en el MCE:

\begin{enumerate}
      \item Si $rg(\Omega)=0$, implica que $\Omega=0$ y las $X_{t}$ son no cointegradas. Con esto, se estima un modelo VAR cl\'{a}sico en primeras diferencias a fin de eliminar la tendencia.
      \item Si $rg(\Omega)=k$, implica que $X_{t}$ no contiene ra\'{i}ces unitarias; esto es, $X_{t}$ es I(0). El modelo de correcci\'{o}n del error no es informativo y se debe estudiar directamente $X_{t}$. 
      \item Si $0<rg(\Omega)=r < k$, existen $r$ vectores (fila) linealmente independientes que recogen $r$ relaciones de cointegraci\'{o}n diferentes; entonces, se puede descomponer la matriz $\Omega$ en el producto de dos submatrices $\alpha$ y $\beta$, de orden (k x r) de forma tal que:
\end{enumerate}

\[
\Omega =\alpha\beta^{'}
\]
donde $\beta^{'}$ contiene los coeficientes de los vectores de cointegraci\'{o}n y $\alpha$ los par\'{a}metros de la velocidad de ajuste. Sustituyendo la matriz $\Omega$ en la expresi\'{o}n general del modelo, se tiene:
\[
\Delta X_{t}=\Omega X_{t-1}+A_{1}^{\ast}\Delta X_{t-1}+\ldots +A_{p-1}^{\ast}\Delta X_{t-p+1}+u_{t}
\]
donde, el producto $\beta^{'} X_{t-1}$ es estacionario.

\section{Prueba de Johansen}
%\label{subsec:mylabel4}
Para determinar el n\'{u}mero de relaciones de cointegraci\'{o}n Johansen (1988) propuso dos pruebas fundamentadas en los valores propios de la matriz $\Omega$.\newline

En la pr\'{a}ctica, el procedimiento de aplicaci\'{o}n de la prueba de Johansen\index{Prueba!Johansen} se realiza mediante las siguientes etapas:

\begin{enumerate}
      \item[1.] Determinaci\'{o}n del orden \'{o}ptimo del modelo VAR con las series que son integradas del mismo orden. Esta etapa es muy importante, ya que los resultados del contraste son muy sensibles frente a una mala especificaci\'{o}n de retardos. Posteriormente, se realizar\'{a} un ajuste con respecto a la tendencia com\'{u}n.
      
      En algunos paquetes estad\'{\i}sticos como el EViews, se escoge el retardo del VAR pero con las variables diferenciadas una vez. Es decir, el paquete realiza una diferenciaci\'{o}n en el VAR de manera autom\'{a}tica.

      \item[2.] Estimaci\'{o}n del modelo. Para realizar la descomposici\'{o}n de la matriz $\Omega$ en las submatrices $\alpha$ y $\beta$, se utiliza el m\'{e}todo de m\'{a}xima verosimilitud propuesto por Johansen (1988).
      \item[3.] Determinaci\'{o}n del rango de cointegraci\'{o}n, que vendr\'{a} dado por el propio rango de la matriz $\Omega$ . Teniendo en cuenta que se trata de una matriz de coeficientes estimados, y por lo tanto aleatorios, el rango de dicha matriz no puede determinarse de forma absoluta, sino que habr\'{a} que definirlo dentro de un entorno probabil\'{i}stico.
\end{enumerate}

Se definen dos estad\'{i}sticos alternativos, basados en los valores propios $\lambda_{i}$ de la matriz $\hat{\Omega }$ y formulados como:


\begin{enumerate}
\item[a.] {\bf Traza}
\[
\lambda_{\text{traza}}( r )=-T\sum_{i=r+1}^k \ln(1-\lambda_{i})
\]
donde:\newline

$\lambda_{i}=$ es el i-\'{e}simo valor propio de la matriz $\hat{\Omega}$ (en orden creciente),\newline
$r=$ rango de la matriz $\hat{\Omega }$,\newline
$k=$ n\'{u}mero de variables,\newline
$T=$ n\'{u}mero de observaciones.
\end{enumerate}

Con este estad\'{i}stico se contrasta la hip\'{o}tesis nula de existencia de un n\'{u}mero de vectores de cointegraci\'{o}n menor o igual a $r$ frente a la hip\'{o}tesis alternativa de existencia de m\'{a}s de $r$ relaciones de cointegraci\'{o}n.

\begin{enumerate}
\item[b.] {\bf M\'{a}ximo valor propio}
\[
\lambda_{max}(r, r+1)=-T \ln(1-\lambda_{r+1})
\]
donde:\newline
$\lambda_{i}=$ es el i-\'{e}simo valor propio de la matriz $\hat{\Omega}$ (en orden creciente),\newline
$r=$ rango de la matriz $\hat{\Omega }$,\newline
$k=$ n\'{u}mero de variables,\newline
$T=$ n\'{u}mero de observaciones.
\end{enumerate}

Este estad\'{i}stico contrasta la hip\'{o}tesis nula de existencia de $r$ vectores de cointegraci\'{o}n frente a la hip\'{o}tesis alternativa de existencia de $r + 1$ vectores de cointegraci\'{o}n.\newline

Estos estad\'{i}sticos tienen una ley de probabilidad (similar a una $\chi^{2})$ tabulada mediante simulaciones por Johansen y Juselius (1990).

\begin{ejemplo}
A partir del ejemplo 6.1, se pide analizar una eventual cointegraci\'{o}n entre las variables y estimar un modelo VAR o un modelo VEC si fuese el caso.
\end{ejemplo}

\textbf{Resoluci\'{o}n}\newline

\begin{enumerate}
\item[a.] \textbf{Primera etapa:} Determinar el n\'{u}mero de retardos de la representaci\'{o}n VAR.
\end{enumerate}

Para iniciar, se presentan los gr\'{a}ficos de las series temporales a analizarse:

\begin{figure}[H]
\centering
\includegraphics[width=3.99in,height=2.81in]{Graficos/Cap6-7/STcap64.eps}
\caption{Series consideradas en el ejemplo}
\label{fig4}
\end{figure}

Como se vio en el ejemplo 6.1, las series son cointegradas de orden 1. Ahora, se determina el mejor retardo para el modelo VAR (considerando diferentes criterios):

\begin{table}[H]
\centering
\begin{tabular}{ccccccc}\hline\hline
~Lag & LogL & LR & FPE & AIC & SC & HQ \\ \hline\hline
0 & -170.1455 & NA & 1.441464 & 8.879257 & 9.007223* & 8.925170* \\
1 & -158.4805 & 20.93714 & 1.260025* & 8.742591 & 9.254456 & 8.926244 \\
2 & -150.7736 & 12.64724 & 1.359430 & 8.808904 & 9.704668 & 9.130296 \\
3 & -146.6524 & 6.129033 & 1.787069 & 9.059096 & 10.33876 & 9.518228 \\
4 & -143.2875 & 4.486496 & 2.493999 & 9.348077 & 11.01164 & 9.944949 \\
5 & -124.5157 & 22.14106* & 1.627928 & 8.846961 & 10.89442 & 9.581572 \\
6 & -119.0699 & 5.585502 & 2.196024 & 9.029224 & 11.46058 & 9.901575 \\
7 & -116.5909 & 2.161115 & 3.663332 & 9.363638 & 12.17890 & 10.37373 \\
8 & -98.07554 & 13.29311 & 2.931050 & 8.875669 & 12.07483 & 10.02350 \\
9 & -88.11999 & 5.615947 & 4.160914 & 8.826666 & 12.40972 & 10.11224 \\
10& -76.43760 & 4.792776 & 6.776135 & 8.689108* & 12.65606 & 10.11242 \\ \hline\hline
\end{tabular}
\caption{Elecci\'{o}n del retardo del VAR}
\label{tab7}
\end{table}

Como se puede observar, los criterios difieren de cu\'{a}l es el mejor retardo del VAR a estimar (cabe considerar que en este caso se consideran las series en primeras diferencias). Los valores dados en la tabla se pueden considerar como cotas superiores del retardo del VAR. Debido a que se tienen pocos datos se desechar\'{a} el retardo 10, por lo que, se puede estimar un VAR(5).\newline

Esto da lugar a 2 posibles modelos: 
\begin{enumerate}
\item[1.] VAR(5), considerando solamente el retardo 5.
\item[2.] VAR(5), considerando los retardos del 1 al 5.
\item[b.] \textbf{Segunda Etapa}: Prueba de Johansen
\end{enumerate}

Como ya se mencion\'{o}, las series son $I(1)$. Se realizan las pruebas sobre los supuestos de la tendencia de los datos (\'{e}stas est\'{a}n implementadas en EViews):

\begin{figure}[H]
\includegraphics[width=5.16in,height=4.02in]{Graficos/Cap6-7/STcap65.eps}
\caption{Prueba de cointegraci\'{o}n de Johansen}
\label{fig5}
\end{figure}


N\'{o}tese, que esta prueba es v\'{a}lida solo para series que no son estacionarias. Adem\'{a}s, los retardos se fijan para la series en primeras diferencias (ver figura 9.3, \textit{lag intervals}) y \underline {no en niveles}. Por ejemplo, si se escribe ``1 2'' la prueba hace la regresi\'{o}n de $\Delta X_{t}$ sobre $\Delta X_{t-1}$, $\Delta X_{t-2}$ y sobre las variables ex\'{o}genas especificadas, de ser el caso. Para $X_{t}$ tendr\'{i}a 3 retardos. Si se desea correr la prueba con un retardo en niveles, se debe escribir ``0 0''. 
% 
% \begin{table}[H]
% \begin{center}
% \begin{tabular}{|p{13pt}|p{129pt}|p{129pt}|}
% \hline
% \multicolumn{3}{|p{271pt}|}{Resumen de los 5 conjuntos de supuestos} \\
% \hline
% & 
% CE& 
% VAR \\
% \hline
% \multicolumn{3}{|p{271pt}|}{No tendencia determin\'{\i}stica en los datos} \\
% \hline
% 1& 
% Ni intercepto ni Tendencia& 
% Ni intercepto ni Tendencia \\
% \hline
% 2& 
% Intercepto-no Tendencia & 
% No intercepto \\
% \hline
% \multicolumn{3}{|p{271pt}|}{Tendencia determin\'{\i}stica lineal en los datos} \\
% \hline
% 3& 
% Intercepto-no Tendencia & 
% Intercepto-no Tendencia \\
% \hline
% 4& 
% Intercepto y Tendencia & 
% No Tendencia \\
% \hline
% \multicolumn{3}{|p{271pt}|}{Tendencia determin\'{\i}stica cuadr\'{a}tica en los datos} \\
% \hline
% 5& 
% Intercepto y Tendencia & 
% Tendencia lineal \\
% \hline
% \end{tabular}
% \label{tab8}
% \end{center}
% \end{table}
% 
% La prueba de Johansen propone 5 modelos del VEC que se describen de la 
% siguiente manera:
% 
% \begin{enumerate}
% \item Es el modelo m\'{a}s restrictivo: No admite ni constante ni tendencia. Es decir, los datos de nivel $X_{t}$ no tienen tendencia determin\'{\i}stica y las ecuaciones de cointegraci\'{o}n no tienen interceptos.
% \item Incluye una constante en el vector de cointegraci\'{o}n; pero, los datos de nivel no tienen tendencia determin\'{\i}stica.
% \item Incluye una constante en el vector de cointegraci\'{o}n y adem\'{a}s una tendencia lineal en las componentes de $X_{t}$.
% \item Incluye una constante en el vector de cointegraci\'{o}n, una tendencia lineal en las componentes de $X_{t}$, y adem\'{a}s, una tendencia lineal en el vector de cointegraci\'{o}n.
% \item Finalmente el \'{u}ltimo modelo, incluye una constante en el vector de cointegraci\'{o}n, una tendencia lineal en las componentes de $X_{t}$, una tendencia lineal en el vector de cointegraci\'{o}n y una tendencia cuadr\'{a}tica en las variables en nivel. Este es el modelo menos restrictivo.
% \end{enumerate}
% En la figura siguiente se muestran los supuestos y las relaciones de 
% cointegraci\'{o}n, para las pruebas de la traza y del m\'{a}ximo valor 
% absoluto para los dos modelos descritos anteriormente:
% 
% \begin{table}[H]
% \begin{center}
% \begin{tabular}{|p{58pt}|l|l|l|l|l|}
% \hline
% & 
% & 
% & 
% & 
% & 
%  \\
% \hline
% & 
% & 
% & 
% & 
% & 
%  \\
% \hline
% Data Trend:& 
% None& 
% None& 
% Linear& 
% Linear& 
% Quadratic \\
% \hline
% Test Type& 
% No Intercept& 
% Intercept& 
% Intercept& 
% Intercept& 
% Intercept \\
% \hline
% & 
% No Trend& 
% No Trend& 
% No Trend& 
% Trend& 
% Trend \\
% \hline
% Trace& 
% 2& 
% 3& 
% 3& 
% 2& 
% 0 \\
% \hline
% Max-Eig& 
% 2& 
% 1& 
% 0& 
% 0& 
% 0 \\
% \hline
% & 
% & 
% & 
% & 
% & 
%  \\
% \hline
% & 
% & 
% & 
% & 
% & 
%  \\
% \hline
% \end{tabular}
% \label{tab9}
% \end{center}
% \end{table}
% 
% \begin{center}
% Tabla 7.2: Resumen de la prueba de cointeraci\'{o}n de Johansen para VAR(5), 
% con solo el retardo 5
% \end{center}
% 
% \begin{table}[H]
% \begin{center}
% \begin{tabular}{|p{58pt}|l|l|l|l|l|}
% \hline
% & 
% & 
% & 
% & 
% & 
%  \\
% \hline
% & 
% & 
% & 
% & 
% & 
%  \\
% \hline
% Data Trend:& 
% None& 
% None& 
% Linear& 
% Linear& 
% Quadratic \\
% \hline
% Test Type& 
% No Intercept& 
% Intercept& 
% Intercept& 
% Intercept& 
% Intercept \\
% \hline
% & 
% No Trend& 
% No Trend& 
% No Trend& 
% Trend& 
% Trend \\
% \hline
% Trace& 
% 0& 
% 0& 
% 0& 
% 0& 
% 0 \\
% \hline
% Max-Eig& 
% 0& 
% 0& 
% 0& 
% 0& 
% 0 \\
% \hline
% & 
% & 
% & 
% & 
% & 
%  \\
% \hline
% & 
% & 
% & 
% & 
% & 
%  \\
% \hline
% \end{tabular}
% \label{tab10}
% \end{center}
% \end{table}
% 
% \begin{center}
% Tabla 7.3: Resumen de la prueba de cointeraci\'{o}n de Johansen para VAR(5), 
% con los retardos desde el 1 al 5 \newline
% 
% \end{center}
% 
% Como se puede observar, dependiendo del n\'{u}mero de retardos y la forma de 
% ingresarlos se tienen diferentes relaciones de cointegraci\'{o}n. En el 
% modelo VAR(5) con los retardos del 1 al 5 no se tienen relaciones de 
% cointegraci\'{o}n. Ahora, se tiene un solo modelo posible:
% 
% \begin{enumerate}
% \item Modelo VAR(5) solo con el retardo 5, con dos relaci\'{o}n de cointegraci\'{o}n sin intercepto ni tendencia.
% \end{enumerate}
% 
% \textbf{\textit{Primera prueba}}: El rango de la matriz $\mathrm{\Omega }$ 
% igual a 0. Sea $H_{0}:\thinspace r=0\thinspace contra\thinspace 
% H_{1}:\thinspace r>0$.
% 
% Los valores propios de la matriz $\mathrm{\Omega }$, estimados por 
% m\'{a}xima verosimilitud, son iguales a: $\lambda_{3}=0,4451,\thinspace 
% \thinspace \lambda_{2}=0,2734\thinspace y\thinspace \lambda_{1}=0,0257$
% 
% Se calcula el estad\'{\i}stico de Johansen: 
% \[
% \lambda_{traza}=-T\sum\limits_{i=r+1}^k {Ln\left( 1-\lambda_{i} \right)} 
% ;\thinspace para\thinspace r=0
% \]
% \[
% \lambda_{traza}=-T\ast \left\{ Ln\left( 1-\lambda_{1} \right)+Ln\left( 
% 1-\lambda_{2} \right) \right\}\thinspace \thinspace \thinspace 
% \]
% \[
% \thinspace =-50\ast \left\{ \ln \left( 1-0,4451 \right)+\ln \left( 1-0,2734 
% \right)+\ln \left( 1-0,0257 \right) \right\}=41,11
% \]
% Tambi\'{e}n, se calcula el del m\'{a}ximo valor propio:
% \[
% \lambda_{max}\left( r,\thinspace r+1 \right)=-T\ln \left( 1-\lambda_{r+1} 
% \right)para\thinspace r=0
% \]
% \[
% \lambda_{max}\left( 0,1 \right)=-T\ln \left( 1-\lambda_{2} \right)=-50\ast 
% \ln \left( 1-0,2734 \right)=25,91
% \]
% El valor cr\'{\i}tico para la traza es igual a 24,27 para un nivel de 
% significaci\'{o}n del 5{\%}. Por lo tanto para el caso se la traza, se 
% rechaza la hip\'{o}tesis nula; el rango de la matriz no es 0 (las series no 
% son estacionarias). Mientras que, la prueba del m\'{a}ximo valor propio 
% tiene un valor cr\'{\i}tico de 25,91, se rechaza la hip\'{o}tesis nula; el 
% rango de la matriz no es 0.
% 
% Se acepta que la hip\'{o}tesis de que existe al menos una relaci\'{o}n de 
% cointegraci\'{o}n.
% 
% \textbf{\textit{Segunda prueba}}: El rango de la matriz $\mathrm{\Omega }$ 
% igual a 1. Sea $H_{0}:\thinspace r=1\thinspace contra\thinspace 
% H_{1}:\thinspace r>1$.
% \[
% \begin{array}{l}
%  \lambda_{traza}=-15,19 \\ 
%  \lambda_{max}\left( 1,2 \right)=14,05 \\ 
%  \end{array}
% \]
% El valor cr\'{\i}tico es igual a 12,32 y de 11,22 para la prueba de la traza 
% y del m\'{a}ximo valor propio, respectivamente, con un nivel de 
% significaci\'{o}n del 5{\%}. Por lo tanto, se rechazar $H_{0}$; se considera 
% que el rango de la matriz no es 1. Se tiene al menos 2 relaciones de 
% cointegraci\'{o}n.
% 
% \textbf{\textit{Tercera prueba}}: El rango de la matriz $\mathrm{\Omega }$ 
% igual a 2. Sea $H_{0}:\thinspace r=2\thinspace contra\thinspace 
% H_{1}:\thinspace r>2$.
% \[
% \begin{array}{l}
%  \lambda_{traza}=1,14 \\ 
%  \lambda_{max}\left( 1,2 \right)=1,14 \\ 
%  \end{array}
% \]
% El valor cr\'{\i}tico es igual a 4,12 tanto para la prueba de la traza como 
% para la del m\'{a}ximo valor propio con un nivel de significaci\'{o}n del 
% 5{\%}. Por lo tanto, no se rechazar $H_{0}$; se considera que el rango de la 
% matriz es 2. 
% 
% Se acepta la hip\'{o}tesis que existen dos relaciones de cointegraci\'{o}n.
% 
% \textbf{Observaci\'{o}n 6.2. }A pesar de que la prueba de Johansen est\'{a} 
% implementada en algunos paquetes estad\'{\i}sticos, tiene algunos problemas 
% al momento de realizar las predicciones. Hay observaciones sobre la 
% especificaci\'{o}n de los modelos derivados de la prueba de Johansen (Ver, 
% Tsay 2008).
% 
% \begin{enumerate}
% \item \textbf{Tercera Etapa}: Estimaci\'{o}n de un modelo vectorial de correcci\'{o}n del error (MCE)
% \end{enumerate}
% 
% Se presentan los resultados del modelo encontrado. La estimaci\'{o}n final 
% de un VEC con 50 observaciones es el siguiente:
% 
% \begin{table}[H]
% \begin{center}
% \begin{tabular}{|p{111pt}|l|l|l|}
% \hline
% & 
% & 
% & 
%  \\
% \hline
% & 
% & 
% & 
%  \\
% \hline
% Cointegrating Eq:~& 
% CointEq1& 
% CointEq2& 
%  \\
% \hline
% & 
% & 
% & 
%  \\
% \hline
% & 
% & 
% & 
%  \\
% \hline
% X1(-1)& 
% ~1.000000& 
% ~0.000000& 
%  \\
% \hline
% & 
% & 
% & 
%  \\
% \hline
% X2(-1)& 
% ~0.000000& 
% ~1.000000& 
%  \\
% \hline
% & 
% & 
% & 
%  \\
% \hline
% X3(-1)& 
% -1.087335& 
% -0.349631& 
%  \\
% \hline
% & 
% ~(0.01946)& 
% ~(0.07942)& 
%  \\
% \hline
% & 
% [-55.8619]& 
% [-4.40212]& 
%  \\
% \hline
% & 
% & 
% & 
%  \\
% \hline
% & 
% & 
% & 
%  \\
% \hline
% Error Correction:& 
% D(X1)& 
% D(X2)& 
% D(X3) \\
% \hline
% & 
% & 
% & 
%  \\
% \hline
% & 
% & 
% & 
%  \\
% \hline
% CointEq1& 
% -0.141775& 
% -0.068674& 
% ~0.323563 \\
% \hline
% & 
% ~(0.24049)& 
% ~(0.09656)& 
% ~(0.22752) \\
% \hline
% & 
% [-0.58952]& 
% [-0.71123]& 
% [ 1.42214] \\
% \hline
% & 
% & 
% & 
%  \\
% \hline
% CointEq2& 
% ~0.026182& 
% ~0.019842& 
% ~0.139864 \\
% \hline
% & 
% ~(0.06048)& 
% ~(0.02428)& 
% ~(0.05721) \\
% \hline
% & 
% [ 0.43293]& 
% [ 0.81719]& 
% [ 2.44458] \\
% \hline
% & 
% & 
% & 
%  \\
% \hline
% D(X1(-5))& 
% -0.640939& 
% -0.117720& 
% -0.159909 \\
% \hline
% & 
% ~(0.24748)& 
% ~(0.09936)& 
% ~(0.23413) \\
% \hline
% & 
% [-2.58981]& 
% [-1.18474]& 
% [-0.68298] \\
% \hline
% & 
% & 
% & 
%  \\
% \hline
% D(X2(-5))& 
% ~0.375685& 
% ~0.330682& 
% ~0.512122 \\
% \hline
% & 
% ~(0.55345)& 
% ~(0.22221)& 
% ~(0.52359) \\
% \hline
% & 
% [ 0.67880]& 
% [ 1.48817]& 
% [ 0.97809] \\
% \hline
% & 
% & 
% & 
%  \\
% \hline
% D(X3(-5))& 
% ~0.219208& 
% -0.074681& 
% -0.280249 \\
% \hline
% & 
% ~(0.26134)& 
% ~(0.10493)& 
% ~(0.24724) \\
% \hline
% & 
% [ 0.83877]& 
% [-0.71173]& 
% [-1.13349] \\
% \hline
% & 
% & 
% & 
%  \\
% \hline
% & 
% & 
% & 
%  \\
% \hline
% \end{tabular}
% \label{tab11}
% \end{center}
% \end{table}
% 
% Las expresiones de cointegraci\'{o}n tienen la siguiente forma:
% \[
% \begin{array}{l}
%  X_{1,t-1}-1,087\ast X_{3,t-1} \\ 
%  X_{1,t-2}-0,350\ast X_{3,t-1} \\ 
%  \end{array}
% \]
% Las ecuaciones de correcci\'{o}n del error son:
% \[
% \mathrm{\Delta }X_{1t}=-0,641\mathrm{\Delta }X_{1,t-5}+0,376\mathrm{\Delta 
% }X_{2,t-5}+0,219\mathrm{\Delta }X_{3,t-5}-0,142\left( 3X_{1,t-1}-1,087\ast 
% X_{3,t-1} \right)+0,026(X_{1,t-2}-0,350\ast X_{3,t-1})
% \]
% \[
% \mathrm{\Delta }X_{2t}=-0,118\mathrm{\Delta }X_{1,t-5}+0,331\mathrm{\Delta 
% }X_{2,t-5}-0,075\mathrm{\Delta }X_{3,t-5}-0,069\left( 3X_{1,t-1}-1,087\ast 
% X_{3,t-1} \right)+0,020(X_{1,t-2}-0,350\ast X_{3,t-1})
% \]
% \[
% \mathrm{\Delta }X_{3t}=-0,160\mathrm{\Delta }X_{1,t-5}+0,512\mathrm{\Delta 
% }X_{2,t-5}-0,280\mathrm{\Delta }X_{3,t-5}+0,324\left( 3X_{1,t-1}-1,087\ast 
% X_{3,t-1} \right)+0,140(X_{1,t-2}-0,350\ast X_{3,t-1})
% \]
% \begin{enumerate}
% \item \textbf{Diagn\'{o}stico y verificaci\'{o}n del modelo}
% \end{enumerate}
% Para verificar el modelo, se consideran \index{Modelos MCE!Diagn\'{o}stico y 
% verificaci\'{o}n}los mismos estad\'{\i}sticos que se utilizaron en el caso 
% del VAR.
% 
% \begin{table}[H]
% \begin{center}
% \begin{tabular}{|p{43pt}|l|l|l|l|l|}
% \hline
% & 
% & 
% & 
% & 
% & 
%  \\
% \hline
% & 
% & 
% & 
% & 
% & 
%  \\
% \hline
% Lags& 
% Q-Stat& 
% Prob.& 
% Adj Q-Stat& 
% Prob.& 
% df \\
% \hline
% & 
% & 
% & 
% & 
% & 
%  \\
% \hline
% & 
% & 
% & 
% & 
% & 
%  \\
% \hline
% 1& 
% ~7.902984& 
% NA*& 
% ~8.086774& 
% NA*& 
% NA* \\
% \hline
% 2& 
% ~24.71026& 
% NA*& 
% ~25.69440& 
% NA*& 
% NA* \\
% \hline
% 3& 
% ~29.93520& 
% NA*& 
% ~31.30165& 
% NA*& 
% NA* \\
% \hline
% 4& 
% ~32.89659& 
% NA*& 
% ~34.55918& 
% NA*& 
% NA* \\
% \hline
% 5& 
% ~35.08583& 
% NA*& 
% ~37.02909& 
% NA*& 
% NA* \\
% \hline
% 6& 
% ~41.49298& 
% ~0.7349& 
% ~44.44789& 
% ~0.6192& 
% 48 \\
% \hline
% 7& 
% ~52.58789& 
% ~0.6411& 
% ~57.64185& 
% ~0.4513& 
% 57 \\
% \hline
% 8& 
% ~63.54760& 
% ~0.5627& 
% ~71.03705& 
% ~0.3137& 
% 66 \\
% \hline
% 9& 
% ~71.85129& 
% ~0.5817& 
% ~81.47597& 
% ~0.2849& 
% 75 \\
% \hline
% 10& 
% ~79.78680& 
% ~0.6099& 
% ~91.74545& 
% ~0.2639& 
% 84 \\
% \hline
% 11& 
% ~86.84222& 
% ~0.6601& 
% ~101.1527& 
% ~0.2644& 
% 93 \\
% \hline
% 12& 
% ~90.95700& 
% ~0.7751& 
% ~106.8105& 
% ~0.3527& 
% 102 \\
% \hline
% 13& 
% ~96.73747& 
% ~0.8306& 
% ~115.0150& 
% ~0.3779& 
% 111 \\
% \hline
% 14& 
% ~101.9794& 
% ~0.8817& 
% ~122.7032& 
% ~0.4144& 
% 120 \\
% \hline
% 15& 
% ~107.4967& 
% ~0.9161& 
% ~131.0742& 
% ~0.4325& 
% 129 \\
% \hline
% & 
% & 
% & 
% & 
% & 
%  \\
% \hline
% & 
% & 
% & 
% & 
% & 
%  \\
% \hline
% \end{tabular}
% \label{tab12}
% \end{center}
% \end{table}
% 
% \begin{center}
% Tabla 7.4: Prueba \textit{Portmanteau} \newline
% 
% \end{center}
% 
% La prueba expresa que no existe autocorrelaci\'{o}n de los residuos.
% 
% \textbf{Prueba de Jarque-Bera}
% 
% \begin{table}[H]
% \begin{center}
% \begin{tabular}{|p{64pt}|l|l|l|}
% \hline
% Component& 
% Jarque-Bera& 
% df& 
% Prob. \\
% \hline
% & 
% & 
% & 
%  \\
% \hline
% & 
% & 
% & 
%  \\
% \hline
% 1& 
% ~0.994354& 
% 2& 
% ~0.6082 \\
% \hline
% 2& 
% ~1.961766& 
% 2& 
% ~0.3750 \\
% \hline
% 3& 
% ~5.545940& 
% 2& 
% ~0.0625 \\
% \hline
% & 
% & 
% & 
%  \\
% \hline
% & 
% & 
% & 
%  \\
% \hline
% Joint& 
% ~8.502061& 
% 6& 
% ~0.2036 \\
% \hline
% & 
% & 
% & 
%  \\
% \hline
% & 
% & 
% & 
%  \\
% \hline
% \end{tabular}
% \label{tab13}
% \end{center}
% \end{table}
% 
% \begin{center}
% Tabla 7.5: Prueba de Jarque-Bera
% \end{center}
% 
% La prueba de Jarque-Bera concluye que los residuos tienen una 
% distribuci\'{o}n normal, tanto de manera marginal como global. Finalmente, 
% se muestra la prueba LM de correlaci\'{o}n serial:
% 
% \begin{table}[H]
% \begin{center}
% \begin{tabular}{|p{43pt}|l|l|}
% \hline
% & 
% & 
%  \\
% \hline
% & 
% & 
%  \\
% \hline
% Lags& 
% LM-Stat& 
% Prob \\
% \hline
% & 
% & 
%  \\
% \hline
% & 
% & 
%  \\
% \hline
% 1& 
% ~7.935467& 
% ~0.5407 \\
% \hline
% 2& 
% ~16.71692& 
% ~0.0533 \\
% \hline
% 3& 
% ~5.199048& 
% ~0.8166 \\
% \hline
% 4& 
% ~2.892506& 
% ~0.9684 \\
% \hline
% 5& 
% ~13.83981& 
% ~0.1281 \\
% \hline
% 6& 
% ~8.614968& 
% ~0.4735 \\
% \hline
% 7& 
% ~14.98006& 
% ~0.0915 \\
% \hline
% 8& 
% ~12.22423& 
% ~0.2010 \\
% \hline
% 9& 
% ~13.94869& 
% ~0.1242 \\
% \hline
% 10& 
% ~11.32514& 
% ~0.2541 \\
% \hline
% 11& 
% ~10.29224& 
% ~0.3273 \\
% \hline
% 12& 
% ~7.225709& 
% ~0.6136 \\
% \hline
% 13& 
% ~10.47207& 
% ~0.3136 \\
% \hline
% 14& 
% ~7.876060& 
% ~0.5467 \\
% \hline
% 15& 
% ~9.297943& 
% ~0.4102 \\
% \hline
% & 
% & 
%  \\
% \hline
% & 
% & 
%  \\
% \hline
% \end{tabular}
% \label{tab14}
% \end{center}
% \end{table}
% 
% \begin{center}
% Tabla 7.6: Prueba LM \newline
% 
% \end{center}
% 
% Como se puede observar, no existe correlaci\'{o}n serial en los residuos.
% 
% \begin{enumerate}
% \item \textbf{Predicciones a partir del MCE}
% \end{enumerate}
% Las predicciones obtenidas a partir del modelo son las siguientes:
% 
% \begin{table}[H]
% \begin{center}
% \begin{tabular}{|p{60pt}|p{60pt}|p{60pt}|p{60pt}|p{60pt}|p{60pt}|p{60pt}|}
% \hline
% & 
% \multicolumn{2}{|p{120pt}|}{$X_{1}$} & 
% \multicolumn{2}{|p{120pt}|}{$X_{2}$} & 
% \multicolumn{2}{|p{120pt}|}{$X_{3}$} \\
% \hline
% obs& 
% Real& 
% Predicci\'{o}n& 
% Real& 
% Predicci\'{o}n& 
% Real& 
% Predicci\'{o}n \\
% \hline
% 51& 
% 165,071& 
% 167,8575& 
% 130,579& 
% 130,9032& 
% 169,377& 
% 171,4889 \\
% \hline
% 52& 
% 169,493& 
% 171,6238& 
% 132,612& 
% 133,0139& 
% 171,449& 
% 174,5759 \\
% \hline
% 53& 
% 172& 
% 173,1461& 
% 133,298& 
% 134,601& 
% 175,665& 
% 176,6129 \\
% \hline
% 54& 
% 175,099& 
% 176,1222& 
% 136,343& 
% 136,5069& 
% 177,327& 
% 178,8983 \\
% \hline
% 55& 
% 178,819& 
% 180,4314& 
% 138,931& 
% 138,8625& 
% 184,029& 
% 182,4962 \\
% \hline
% 56& 
% 180,759& 
% 183,9694& 
% 140,683& 
% 141,5225& 
% 183,299& 
% 186,2786 \\
% \hline
% \end{tabular}
% \label{tab15}
% \end{center}
% \end{table}
% 
% \begin{center}
% Tabla 7.7: Predicciones para las tres series utilizando el MCE
% \end{center}
% 
% A continuaci\'{o}n, se presenta gr\'{a}ficamente la comparaci\'{o}n con los 
% datos reales guardados:
% 
% \begin{figure}[H]
% \includegraphics[width=4.58in,height=3.42in]{STcap66.eps}
% \label{fig6}
% \end{figure}
% 
% \begin{center}
% Figura 7.3: Predicci\'{o}n para $X_{1t}$
% \end{center}
% 
% \begin{figure}[H]
% \includegraphics[width=4.47in,height=3.34in]{STcap67.eps}
% \label{fig7}
% \end{figure}
% 
% \begin{center}
% Figura 7.4: Predicci\'{o}n para $X_{2t}$
% \end{center}
% 
% \begin{figure}[H]
% \includegraphics[width=4.58in,height=3.42in]{STcap68.eps}
% \label{fig8}
% \end{figure}
% 
% \begin{center}
% Figura 7.5: Predicci\'{o}n para $X_{3t}$
% \end{center}
% 
% En los gr\'{a}ficos las leyendas significan:
% 
% LCI: L\'{\i}mite de confianza inferior.
% 
% Xi: Serie real de datos, i$=$1,2,3.
% 
% Xi (MCE): Predicci\'{o}n de la serie Xi, i$=$1,2,3.
% 
% LCS: L\'{\i}mite de confianza superior.
% 
% Adicionalmente, se presentan las predicciones de los modelos univariantes y 
% se realiza la comparaci\'{o}n con las predicciones del modelo de 
% correcci\'{o}n del error. As\'{\i}, para $X_{1t}$ se estim\'{o} el modelo: 
% ${\Delta \hat{X}}_{1t}=0,92X_{1t-1}+\hat{u}_{1t}-0,96u_{1t-1}+3,05$. Para 
% $X_{2t}$ se obtuvo: ${\Delta \hat{X}}_{2t}=0,50X_{1t-1}+\hat{u}_{1t}+1,75$ 
% y; finalmente, para $X_{3t}$ se encontr\'{o}: ${\Delta 
% \mathrm{log?}(\hat{X}}_{3t})=0,78X_{1t-6}+\hat{u}_{1t}$.
% 
% A continuaci\'{o}n, se presentan los gr\'{a}ficos con la comparaci\'{o}n de 
% las predicciones:
% 
% \begin{figure}[H]
% \includegraphics[width=4.10in,height=2.95in]{STcap69.eps}
% \label{fig9}
% \end{figure}
% 
% \begin{center}
% Figura 7.6: Comparaci\'{o}n de las predicciones para $X_{1t}$ entre el MCE y 
% el modelo univariante.
% \end{center}
% 
% \begin{figure}[H]
% \includegraphics[width=4.26in,height=3.31in]{STcap610.eps}
% \label{fig10}
% \end{figure}
% 
% \begin{center}
% Figura 7.7: Comparaci\'{o}n de las predicciones para $X_{2t}$ entre el MCE y 
% el modelo univariante.
% \end{center}
% 
% \begin{figure}[H]
% \includegraphics[width=4.26in,height=3.31in]{STcap611.eps}
% \label{fig11}
% \end{figure}
% 
% \begin{center}
% Figura 7.8: Comparaci\'{o}n de las predicciones para $X_{1t}$ entre el MCE y 
% el modelo univariante.
% \end{center}
% 
% Se puede observar, en las tres series, que el ajuste de las predicciones 
% realizadas por el MCE son mejores que las que se obtienen con el modelo 
% univariante.
% 
% \section{7.3 PRUEBA DE EXOGENEIDAD D\'{E}BIL}
% \label{subsec:mylabel5}
% Esta prueba\index{Prueba!Exogeneidad d\'{e}bil} busca ver si las variables 
% son end\'{o}genas o no. Esta prueba se refiere al coeficiente $\alpha $ de 
% la fuerza de empuje (si existe una sola relaci\'{o}n de cointegraci\'{o}n) o 
% los coeficientes $\alpha $ cuando existe m\'{a}s de una relaci\'{o}n de 
% cointegraci\'{o}n. Realizar una prueba sobre $\alpha $ comprueba si la 
% relaci\'{o}n de cointegraci\'{o}n est\'{a} presente en todas las ecuaciones 
% del modelo. Esta prueba se realiza sobre la matriz $\pi =\alpha^{'}\beta $, 
% donde $\beta $ son las relaciones de cointegraci\'{o}n y $\alpha $ los pesos 
% del estas relaciones en cada una de las ecuaciones del sistema.
% 
% Por ejemplo, considere un MCE con dos relaciones de cointegraci\'{o}n:
% \[
% \left( {\begin{array}{*{20}c}
% \Delta X_{1t}\\
% \Delta X_{2t}\\
% \Delta X_{3t}\\
% \end{array} } \right)=A\left( {\begin{array}{*{20}c}
% \Delta X_{1,t-1}\\
% \Delta X_{2,t-1}\\
% \Delta X_{3,t-1}\\
% \end{array} } \right)+\left( {\begin{array}{*{20}c}
% \alpha_{11} & \alpha_{12}\\
% \alpha_{21} & \alpha_{22}\\
% \alpha_{31} & \alpha_{32}\\
% \end{array} } \right)\left( {\begin{array}{*{20}c}
% \beta_{11} & \beta_{21} & \beta_{31}\\
% \beta_{12} & \beta_{22} & \beta_{32}\\
% \end{array} } \right)\left( {\begin{array}{*{20}c}
% X_{1,t-1}\\
% X_{2,t-1}\\
% X_{3,t-1}\\
% \end{array} } \right)
% \]
% Para probar la exogeneidad de la variable $X_{2t}$, se debe realizar la 
% prueba de hip\'{o}tesis:
% \[
% \left\{ {\begin{array}{l}
%  H_{0}:\thinspace \alpha_{21}=\alpha_{22}=0 \\ 
%  H_{a}:\thinspace \alpha_{21},\alpha_{22}\ne 0\thinspace \thinspace 
% \thinspace \thinspace \\ 
%  \end{array}} \right.
% \]
% Si se acepta la hip\'{o}tesis nula, esto significa que la fuerza de empuje 
% $\alpha $ no interviene en ninguna relaci\'{o}n de cointegraci\'{o}n y que 
% la variable $X_{2t}$ es d\'{e}bilmente ex\'{o}gena. En este caso, se estima 
% un MCE imponiendo la restricci\'{o}n $\alpha_{21}=\alpha_{22}=0$.
% 
% \section{7.4 COINTEGRACI\'{O}N ESTACIONAL}
% \label{subsec:mylabel6}
% Los modelos descritos\index{Cointegraci\'{o}n!Estacional} anteriormente no 
% cubren la variaci\'{o}n estacional de las series de tiempo. Sin embargo, los 
% resultados b\'{a}sicos de invertir los polinomios matriciales pueden ser 
% extendidas para cubrir este caso tambi\'{e}n, con el resultado que se 
% obtiene de una formulaci\'{o}n de correcci\'{o}n del error y la posibilidad 
% de calcular el rango de cointegraci\'{o}n sobre diversas frecuencias 
% complejas. Las as\'{\i}ntotas son, aproximadamente, las mismas que para el 
% modelo usual, pero implican movimientos Brownianos complejos, como 
% consecuencia de permitir ra\'{\i}ces en las frecuencias complejas. El lector 
% puede indagar m\'{a}s sobre el tema, por ejemplo, en ``Likelihood analysis 
% of seasonal cointegration'' (Johansen and Schaumburg,1998) o en 
% ``Cointegration: an overview'' (Johansen, 2004).