\chapter*{ANEXO A}
%\label{sec:anexo}
\section*{A.1: NOCIONES SOBRE PROCESOS ESTOC\'{A}STICOS}
%\label{subsec:anexo}

Sea el proceso estoc\'{a}stico $\left( {X_{t} } \right)_{t\in T}$, definido sobre el espacio de probabilidad $(\Omega ,A ,P)$ y a valores en $(E,\beta )$. $(\Omega ,A ,P)$ se dice espacio de base y $(E,\beta )$ espacio de estados. Se definen adem\'{a}s:

\begin{itemize}
\item $t\to X_{t} \left( w \right):$ trayectoria de $w$ $\left( {w\in \Omega } \right)$
\item $w\to X_{t} \left( w \right):$ estado del proceso en el tiempo (o instante) t ($t\in T)$
\item $T$: Conjunto de tiempos
\end{itemize}

Un proceso estoc\'{a}stico es una aplicaci\'{o}n $X$:
\[
X: (\Omega ,A ,P) \quad \to\limits (E^{T}, S ,P_{X} )
\]
\[
w\to \left( {X_{t} \left( w \right),t\in T} \right)
\]

$P_{X}$ es la ley del proceso, $\pi_{t_{0}}$ es la proyecci\'{o}n tal que $\pi_{t_{0} } \left( {X_{t} , t\in T} \right)=X_{t_{0} }$ y $S=\sigma \left( {\pi_{t}, t\in T} \right)$ la $\sigma$- \'{a}lgebra generada por las $\left( {\pi_{t} } \right)_{t\in T} $.

\begin{itemize}
      \item X es medible pues: 
\[
X_{t} =\pi_{t} \circ X\Rightarrow \forall \quad B\in \beta,\quad X_{t}^{-1} \left( B \right)=X^{-1}\left( {\pi_{t}^{-1} \left( B \right)} \right)
\]
Pero los conjuntos $\pi_{t}^{-1} (B)$ generan S y $X_{t}^{-1} (B)\in A$, por lo cual $X^{-1} (\pi_{t}^{-1} (B))\in A$.
      
      \item $\left( {\pi_{t}, t\in T} \right)$ se dice el proceso can\'{o}nico asociado a $\left( {X_{t}, t\in T} \right)$.
      \item Leyes de dimensi\'{o}n finita son aquellas de $\left( {X_{t_{1} } ,\ldots, X_{t_{k} } } \right), k=1,2,\ldots;t_{1},\ldots,t_{k} \in T$
\end{itemize}

A continuaci\'{o}n se presentan dos resultados importantes, cuyas demostraciones se admiten.\newline

\textbf{Teorema de Kolmogorov:} Sean $E=R^{d}$, $\beta =\beta_{R^{d}} $. Entonces, la ley del proceso est\'{a} determinada por las leyes de dimensi\'{o}n finita (bajo ciertas condiciones llamadas de proyectividad).\newline

\textbf{Observaci\'{o}n A1:} Cuando estas leyes de dimensi\'{o}n finita son gaussianas, la ley del proceso X est\'{a} determinada por:
\begin{itemize}
      \item media: $t\to EX(t)$.
      \item covarianza: $\left( {s,t} \right)\to cov\left( {X_{s}, X_{t} } \right)$.
\end{itemize}

\section*{A.2: DEMOSTRACIONES DE ALGUNOS TEOREMAS DEL CAP\'{I}TULO 1}
%\label{subsec:mylabel2}
\textbf{Teorema 1.1:} Existe una medida \'{u}nica $\mu$ acotada y sim\'{e}trica sobre $\left[ {-\pi ,\pi } \right]$ tal que:
\[
\gamma_{t} =\int\limits_{\left[ {-\pi ,\pi } \right]} {\cos \lambda t d\mu \left( \lambda \right)}, \quad t\in Z
\]
$\mu$ se dice la medida espectral de $\left( {X_{t} } \right)$.\newline

\textbf{Demostraci\'{o}n:}\newline

\begin{itemize}
      \item \textbf{Caso particular:} $\sum\limits_t {\left| {\gamma_{t} } \right|<\infty }$\newline

Se puede definir $f\left( \lambda \right)=\frac{1}{2\pi }\sum\limits_{t\in Z} {\gamma_{t} \cos \lambda t}$, $\lambda \in \left[ {-\pi ,\pi } \right]$ (f est\'{a} bien definida).\newline

Entonces: 
\[
\gamma_{t} =\int_{-\Pi }^\Pi {\cos \lambda t} 
f\left( \lambda \right)d\lambda 
\]
Si se llega a demostrar que $f\ge 0$, entonces $f\left( \lambda \right){\kern 1pt}{\kern 1pt}d\lambda $ puede considerarse como $d\mu \left( \lambda \right)$.

      \item En caso contrario, consid\'{e}rese: 
\[
I_{T} (\lambda)=\frac{1}{2\pi T}\sum\limits_{1\le s,t\le T} {X_{s} } X_{t} e^{i\lambda (t-s)},\quad \lambda \in \left[-\pi, \pi \right], \quad T=1,2,\ldots
\]
$I_{T}(\lambda)$ se llama el periodograma

Se puede observar que:
\begin{align*}
I_{T} \left( \lambda \right) & = \frac{1}{2\pi T}\left| {\sum\limits_{t=1}^T {X_{t} e^{i\lambda t}} } \right|^{2}\ge 0\\
                             & = \frac{1}{2\pi T}\sum\limits_{1\le s,t\le T} {X_{s} X_{t} \cos ( t-s)} \quad \text{pues}\quad \limits \sum\limits_{1\le s,t\le T}^ {X_{s} X_{t} } sen\lambda (t-s)=0
\end{align*}

Se considera ahora: 

\begin{align*}
f_{T} (\lambda) &= EI_{T} (\lambda)=\frac{1}{2\pi T}\sum\limits_{1\leq s,t\leq T} {E\left[ {\left(X_{t} X_{s} \right)\cos \lambda (t-s)} \right]\ge 0}\\
                &= \frac{1}{2\pi T}\sum\limits_{s,t} {\gamma_{t-s} } \cos \lambda \left( {t-s} \right)\\
                &= \frac{1}{2\pi T}\sum\limits_{h=-(T-1)}^{T-1} {\left(T-|h| \right)}\gamma_{h} \cos \lambda h\quad \text{(poniendo $h=t-s$)}\\
                &= \frac{1}{2\pi }\sum\limits_{t=-(T-1)}^{T-1} {\left( {1-\frac{\left| t \right|}{T}} \right)} \gamma_{t} \cos \lambda t
\end{align*}

Sea $F_{T}(\lambda)=\int_{-\pi }^\lambda {f_{T} (v)dv=\frac{\gamma_{o}}{2\pi}(\lambda +\pi)+\frac{1}{\pi}\displaystyle\sum\limits_{t=1}^{T-1} {\left( {1-\frac{t}{T}} \right)}\gamma_{t} \frac{sen\lambda t}{t}}$ entonces: $F_{T}$ es creciente (pues $f_{T} \ge 0$),  $F_{T}(-\pi)=0$ y $F_{T}(\pi)=\gamma_{0}$.

\end{itemize}

Se recuerda el lema de Helly -- Bray:\newline
% 
% Dada una sucesi\'{o}n $\left( {F_{T} ,T\in N^{\ast }} \right)$ con la misma 
% masa total, se puede extraer una subsucesi\'{o}n $\left( F_{T^{1}} \right)$, 
% que converge estrictamente (en ley); es decir, existe una funci\'{o}n de 
% distribuci\'{o}n F, de una medida $\mu $, de masa total $\gamma_{o} $, t.q. 
% $F_{T^{'}} \to F$ en todo punto de continuidad de F
% 
% As\'{\i}, si $\mu_{T} $ es la medida asociada a $F_{T} ,$ entonces $\forall 
% \phi $ continua (acotada ) sobre $\left[ {-\pi ,\pi } \right]$, 
% 
% $\int \phi $ d $\mu_{T^{'}} \to \int \phi $ d$\mu $
% 
% As\'{\i}, tomando $\varphi \left( \lambda \right)=\cos \lambda t$, se 
% tiene que:
% \[
% \int\limits_{\left[ {-\pi ,\pi } \right]} {\cos \lambda t\limits d\mu 
% _{T^{'}} \left( \lambda \right)=} \int\limits_{\left[ {-\pi ,\pi } \right]} 
% {\cos \lambda t} \text{\thinspace f}_{\text{T}^{\text{'}}} (\lambda 
% )d\lambda 
% \]
% pero $\int\limits_{\left[ {-\pi ,\pi } \right]} {\left( {\cos \lambda t} 
% \right)d\mu_{T^{'}} \left( \lambda \right)\to } \int\limits_{\left[ {-\pi 
% ,\pi } \right]} {\cos \lambda t} d\mu \left( \lambda \right)$ cuando 
% $T^{'}\to \infty $
% 
% y $\int\limits_{\left[ {-\pi ,\pi } \right]} {\cos {\kern 1pt}\;\lambda 
% {\kern 1pt}t} f_{T^{'}} \left( \lambda \right)=\left( {1-\frac{\left| t 
% \right|}{T'}} \right)\gamma_{\text{t}} \text{\thinspace \thinspace }\to 
% \gamma_{t} $ cuando $T^{'}\to \infty $
% \[
% \int_{\left[ {-\pi ,\pi } \right]} {\cos \lambda td\mu \left( \lambda 
% \right)=\gamma_{t} } 
% \]
% Sean: $G_{T} \left( \lambda \right)=F_{T} (\lambda )-\frac{\gamma_{O} 
% }{2\pi }\left( {\lambda +\pi } \right)=\frac{1}{\pi }\sum\limits_{t=1}^{T-1} 
% {\left( {1-\frac{t}{T}} \right)} \gamma_{t} \quad \frac{sen\lambda t}{t}$ y 
% $G\left( \lambda \right)=F\left( \lambda \right)-\frac{\gamma_{O} }{2\pi 
% }\left( {\lambda +\pi } \right)$
% 
% Ahora
% \[
% \int_{-\pi }^{\rm T} {sen\lambda tG_{T^{1}} } \left( \lambda \right)d\lambda 
% \to \int_{-\pi }^\pi {sen\;\lambda tG\left( \lambda \right)d\lambda } 
% \]
% (por el teorema de convergencia dominada (TCD), pues $G_{T^{'}} $ est\'{a} 
% mayorada por una constante y $G_{T^{'}} \to G$, en todo punto de continuidad 
% de G). 
% 
% $\left( {1-\frac{t}{T^{'}}} \right)\frac{\gamma_{t} }{t}\to \frac{\gamma 
% _{t} }{t}$ cuando $T^{'}\to \infty $
% \[
% \limits F\left( \lambda \right)=\frac{\gamma_{O} }{2\pi }\left( {\lambda 
% +\pi } \right)+\frac{1}{\pi }\sum\limits_{t=1}^\infty {\gamma_{t} } 
% \frac{sen\lambda t}{t}
% \]
% Puesto que F es creciente y de variaci\'{o}n acotada, entonces la serie 
% converge en todo punto de continuidad de F.
% 
% Una medida $\mu $ asociada a F se dice, medida central del proceso.
% 
% \begin{itemize}
% \item \textbf{?`Unicidad de }$\mu $\textbf{?}
% \end{itemize}
% 
% En general $\mu $ no es \'{u}nica. Por ejemplo:
% 
% Sea $X_{t} =\left( {-1} \right)^{t}X$ con EX$=$0 y EX$^{\mathrm{2}}=$1. Se 
% ha visto que $\left( {X_{t} } \right)$ es d.e. y $\gamma_{t} =\left( {-1} 
% \right)^{t},\limits t\in {\rm Z}$
% 
% Sea $\mu_{\alpha } =\alpha \delta_{\left( {-\pi } \right)} 
% +$(1-$\alpha ) \quad \delta_{\left( \pi \right)} ,$ con $\alpha \in \left[ 
% {0,1} \right]$, donde $\delta_{\left( a \right)} $ denota a la medida de 
% Dirac en a.
% 
% Dado que $\int_{-\pi }^\pi {sen\lambda td\mu_{\alpha } \left( \lambda 
% \right)=0} $ se tiene que:
% \[
% \int_{-\pi }^\pi {\cos \lambda t} 
% d\mu_{\alpha } \left( \lambda \right)=\cos \pi t=\left( {-1} \right)^{t}
% \quad
% \forall \alpha \in \left[ {0,1} \right],\limits \forall\limits t\in 
% {\rm Z}
% \]
% Por tanto, $\gamma_{t} $ se expresa de la forma requerida, pero hay un 
% n\'{u}mero infinito de medidas centrales; sin embargo, si se exige que la 
% medida sea sim\'{e}trica, se tiene la unicidad. En el ejemplo precedente: 
% \[
% \mu =\frac{1}{2}\left[ {\delta_{\left( {-\pi } \right)} +\delta_{\left( 
% \pi \right)} } \right]
% \]
% La funci\'{o}n de repartici\'{o}n $F$ que se ha obtenido es la funci\'{o}n de 
% repartici\'{o}n de una medida sim\'{e}trica sobre $\left] {-\pi ,\pi } 
% \right].$ Sea $\mu $ la medida central y se define:
% \[
% \nu \left( {\text{A}} \right)=\mu \left( A \right)
% \forall 
% \quad
% A\in \beta (\left] {-\pi ,\pi } \right])
% \]
% \[
% \nu \left( {-\pi } \right)=\nu \left( \pi \right)=\frac{1}{2}\left[ {\mu 
% \left( {-\pi } \right)+\mu \left( \pi \right)} \right]
% \quad
% \left( {\int_{-\pi }^\pi {\cos \lambda t} d\nu \left( \lambda 
% \right)=\int_{-\pi }^\pi {\cos \lambda td\mu \left( \lambda \right)} } 
% \right)
% \]
% Entonces, se puede considerar $\mu $ sim\'{e}trica
% 
% \begin{itemize}
% \item \textbf{Unicidad de la medida espectral sim\'{e}trica sobre }$\left] {-\pi ,\pi } \right]$
% \end{itemize}
% $\int {\cos \lambda td\nu \left( t \right)=\gamma_{t} } $ y $\int 
% {sen\lambda td\nu \left( t \right)=0} $ (conocidos los coeficientes de 
% Fourier)
% 
% entonces $\nu $ es conocida sobre los polinomios trigonom\'{e}tricos y por 
% el teorema de Stone-Weierstrass, $\nu $ est\'{a} determinada para las 
% funciones continuas. 
% 
% \textbf{Teorema 1.6: }Condici\'{o}n suficiente de convergencia c.s.
% 
% Sea $\left( {X_{t} ,t\in {\rm Z}} \right)$ de segundo orden, centrado t.q.
% 
% $V\left( {X_{t} +....+X_{t+p-1} } \right)\le kp^{\gamma }
% \quad
% k=cte;
% \quad
% 0\le \gamma <2;$ p$=$1,2,.....; $\forall t\in {\rm Z}$
% 
% Entonces: $\bar{{X}}\to 0$ en media cuadr\'{a}tica y casi 
% seguramente 
% 
% \textbf{Demostraci\'{o}n:}
% 
% Sea $Sn=\sum\limits_{i=1}^n {X_{i} } \Rightarrow E\left( {\frac{Sn}{n}} 
% \right)^{2}\le \frac{kn^{\gamma }}{n^{2}}=\frac{k}{n^{2-\gamma }}\to 0$ 
% 
%  existe convergencia en m.c.
% 
% \begin{itemize}
% \item Convergencia casi segura:
% \end{itemize}
% 
% Sea $k>\frac{1}{2-\gamma },\mathop\nolimits k\in {\rm Z}$
% 
% entonces: $P\left[ {\left| {\frac{S_{m^{k}} }{m^{k}}} \right|\ge \varepsilon 
% } \right]\le\limits \frac{1}{\varepsilon^{2}}E\left( {\frac{S_{m^{k}} 
% }{m^{k}}} \right)^{2}\le \frac{1}{\varepsilon^{2}m^{2k}}\bar{{k}}m^{k\gamma 
% }=\frac{\bar{{k}}}{\varepsilon^{2}}\frac{1}{m^{k\left( {2-\gamma } 
% \right)}}$
% \[
% \sum\limits_ {P\left( {\left| {\frac{S_{m}^{k}}{m^{k}}} \right|\ge 
% \varepsilon } \right)} <\infty 
% \]
% La primera desigualdad se justifica por la desigualdad de Chebyshev
% 
% y por el Teorema de Borel-Cantelli. $\frac{S_{m^{k}} }{m^{k}}\buildrel 
% {c.s.} \over \longrightarrow 0$ 
% 
% Ahora, Sea $Y_{m} ={\max }\limits_{m^{k}<n\le \left( {m+1} \right)^{k}} 
% \left| {\frac{S_{n} -S_{m^{k}} }{n}} \right|^{2}$
% \[
% P\left( {Y_{m} \ge \varepsilon } \right)\le \frac{EY_{m} }{\varepsilon }
% \quad
% \left( {Y_{m} \ge 0} \right)
% \]
% pero: $E\left( {Y_{m} } \right)\le \frac{E\left( {\left| {X_{m^{k}+1} } 
% \right|+....+\left| {X_{\left( {m-1} \right)^{k}} } \right|} 
% \right)^{2}}{m^{2k}}$
% 
% $=\frac{1}{m^{2k}}\sum\limits_{j,j^{'}} {E\left| {X_{j} X_{j^{'}} } \right|} 
% $ \newline
%  ${\le\limits^{Schw.} \frac{1}{m^{2k}}\sum\limits_{j,j^{'}} {\sqrt 
% {EX_{j}^{2} } \sqrt {EX_{j^{\textasciiacute '}}^{2} } 
% =\frac{1}{m^{2k}}\left( {\sum\limits_j {\sqrt {EX_{j}^{2} } } } \right)} 
% }\limits^{2}$
% \[
% \le \frac{K}{m^{2k}}\left( {\left( {m+1} \right)^{k}-m^{k}} 
% \right)^{2}=K\left[ {\left( {1+\frac{1}{m}} \right)^{k}-1} \right]^{2}
% \]
% $\le K\left( {e^{\frac{k}{m}}-1} \right)^{2}$ (pues $\left( {1+\frac{1}{m}} 
% \right)^{k}\le e^{\frac{k}{m}})$
% 
% $\le K\left( {\frac{2k}{m}} \right)^{2}$ (m suficientemente grande)
% 
% As\'{\i} $E\left( {Y_{m} } \right)\le \frac{cte}{m^{2}}$ y por tanto 
% $\sum\limits_m {P(Y_{m} \ge )\varepsilon <\infty \Rightarrow\limits^{B.C} 
% Y_{m} \mathop \to\limits_{m\to \infty }^{c.s.} 0} $
% 
% Sea $m=m(n)$ t.q. $\left( {m\left( n \right)} \right)^{k}<n<\left( {m\left( n 
% \right)+1} \right)^{k}$; \textit{m(n) es \'{u}nico y }$m\left( n \right)\to \infty $ cuando $n\to \infty 
% $ entonces: 
% \[
% Y_{m\left( n \right)} \buildrel {c.s.} \over \longrightarrow 0
% \]
% \[
% \frac{S_{n} }{n}-\frac{S_{m\left( n \right)^{k}} }{n}\buildrel {c.s.} \over 
% \longrightarrow 0
% \]
% pues: $\frac{S_{m\left( n \right)^{k}} }{n}=\frac{S_{m\left( n \right)^{k}} 
% }{m\left( n \right)^{k}}\ast \frac{m\left( n \right)^{k}}{n}\Rightarrow 
% \frac{S_{n} }{n}\buildrel {c.s.} \over \longrightarrow 0$ (dado que 
% $\frac{S_{(m(n))^{k}} }{(m(n))^{k}}\buildrel {c.s.} \over \longrightarrow 
% 0)$
% 
% \textbf{Teorema 1.8: }Sea$(X_{t} ,t\in {\rm Z})$ d.e, centrado y 
% regular $\left( {\sigma^{2}=E\left( {X_{t} -\hat{{X}}_{t} } \right)^{2}>0} 
% \right)$
% 
% (Ejercicio: Verificar que $E\left( {X_{t} -\hat{{X}}_{t} } 
% \right)^{2}$ no depende de t)
% 
% Entonces: $X_{t} =\sum\limits_{j=0}^\infty {\lambda_{j} u_{t-j} 
% +v_{t} } t\in {\rm Z}$
% 
% donde $(u_{t} )$ r.b. (d\'{e}bil) de varianza $\sigma 
% ^{2}$, $\lambda_{o} =1,\;\sum {\lambda_{j}^{2} <\infty } , \quad u_{t} \in \mu 
% _{t} ,\;u_{t} \bot \mu_{t-1} $
% 
% $\left( {v_{t} } \right)$ centrado, los $u_{t} $ son ortogonales a los $v_{s} 
% $, $v_{t} \in \mathop \cap\limits_{s=0}^{\infty } \mu_{t-s} $
% 
% y la descomposici\'{o}n es \'{u}nica.
% 
% \textbf{Observaci\'{o}n B.1: }$u_{t} =X_{t} -\hat{{X}}_{t} \Rightarrow X_{t} 
% =\hat{{X}}_{t} +u_{t} . \quad u_{t} $ se dice la innovaci\'{o}n del proceso y 
% v$_{\mathrm{t}}$ la parte determinista del proceso.
% 
% \textbf{Demostraci\'{o}n: }
% 
% Sea $u_{t} =X_{t} -\hat{{X}}_{t} $
% 
% \begin{itemize}
% \item $\{u_{t} \}$ es centrado, pues $\hat{{X}}_{t} $ es centrada, por ser proyecci\'{o}n ortogonal sobre un s.e.v. generado por v.a. centradas.
% \item $Eu_{t}^{2} =\sigma^{2}$
% \item $\hat{{X}}_{t} \in \mu_{t-1} \subset \mu_{t} \Rightarrow u_{t} \in \mu_{t} $
% \item Sea $s<t \quad \left. {\begin{array}{l}
%  u_{t} =X_{t} -\hat{{X}}_{t} \bot \mu_{t-1} \supset \mu_{s} \\ 
%  u_{s} =X_{s} -\hat{{X}}_{s} \in \mu_{s} \\ 
%  \end{array}} \right\} \Rightarrow u_{t} \bot u_{s} $
% \item Consideramos $\left( {\frac{\mu_{t} }{\sigma },t\in {\rm Z}} \right),$ sistema ortonormal de $L^{2}$
% \end{itemize}
% Los coeficientes de Fourier de $X_{t} =\sum\limits_{j=0}^\infty {\lambda 
% _{j} u_{t-j} +\upsilon_{t} } , \quad t\in {\rm Z},$ satisfacen:
% \[
% \left( {\int {X_{t} \frac{\mu_{t-j} }{\sigma }} } \right)\frac{\mu_{t-j} 
% }{\sigma }dP=\lambda_{j} u_{t-j} 
% \]
% \[
% \lambda_{j} =\frac{1}{\sigma^{2}}\int {X_{t} u_{t-j} } dP
% \quad
% j=0,1,......
% \]
% \[
% \lambda_{o} =\frac{1}{\sigma^{2}}\int {X_{t} u_{t} \;dP=\frac{1}{\sigma 
% ^{2}}\int {X_{t} \left( {X_{t} -\hat{{X}}_{t} } \right)\;dP=\frac{1}{\sigma 
% ^{2}}\int {\left( {X_{t} -\hat{{X}}_{t} +\hat{{X}}_{t} } \right)\;\left( 
% {X_{t} -\hat{{X}}_{t} } \right)\;dP} } } 
% \]
% \[
% =\frac{1}{\sigma^{2}}\int {\left( {X_{t} -\hat{{X}}_{t} } 
% \right)^{2}dP+\int {\hat{{X}}_{t} \left( {X_{t} -\hat{{X}}_{t} } \right)} 
% \;dP} 
% \]
% \[
% =\frac{1}{\sigma^{2}}\sigma^{2}=1
% \]
% \[
% \lambda_{o} =1
% \]
% $\sum {\lambda_{j}^{2} <\infty ,} $ pues $\sum\limits_{j=0}^\infty {\lambda 
% _{j}^{2} } \le \left| {\left| {X_{t} } \right|} \right|^{2}$
% 
% \begin{itemize}
% \item Pongamos $\upsilon_{t} =X_{t} -\sum\limits_{j=0}^\infty {\lambda_{j} u_{t-j} } $ y verifiquemos que $\upsilon_{t} $satisface las propiedades requeridas.
% \item $\upsilon_{t} $ centrada
% \item $\upsilon_{t} \in \mu_{t} $ pues $\left\{ {\begin{array}{l}
%  X_{t} \in \mu_{t} \\ 
%  u_{t-j} \in \mu_{t-j} \subset \mu_{t} \Rightarrow \sum\limits_{j=0}^\infty {\lambda_{j} u_{t-j} \in \mu_{t} } \\ 
%  \end{array}} \right.$
% \item Sea $\upsilon_{t} =\hat{{V}}_{t} +\varepsilon_{t} ,$ donde $\hat{{V}}_{t} =P_{r}^{\mu_{t-1} } \upsilon_{t} $
% \end{itemize}
% 
% Queremos demostrar que $\varepsilon_{t} =0.$ Vamos a suponer que $\upsilon 
% _{t} \bot u_{s} ,{ }s\in {\rm Z}$
% 
% $\varepsilon_{t} \bot \mu_{t-1} $ y $\varepsilon_{t} \bot u_{t} $ (pues: 
% $\upsilon_{t} \bot \mu_{t} , \hat{{\upsilon }}_{t} \in \mu_{t-1} \bot 
% u_{t} \Rightarrow \hat{{\upsilon }}_{t} \bot u_{t} )$
% \[
% \hat{{X}}_{t} \in \mu_{t-1} y\varepsilon_{t} \bot \mu_{t-1} 
% \Rightarrow entonces\varepsilon_{t} \bot \hat{{X}}_{t} \Rightarrow 
% \varepsilon_{t} \bot X_{t} =u_{t} +\hat{{X}}_{t} y\varepsilon_{t} 
% \bot \mu_{s} ,s<t.
% \]
% Por tanto $\varepsilon_{t} \bot \mu_{t} $; pero $\left( {\varepsilon_{t} 
% \bot \mu_{t} ,\varepsilon_{t} \in \mu_{t} } \right)\Rightarrow 
% \varepsilon_{t} =0\Rightarrow \upsilon_{t} \in \mu_{t-1} $
% 
% Con el mismo razonamiento se constata que $\upsilon_{t} \in \mu_{t-2} $
% \[
% \upsilon_{t} \in \mathop \cap\limits_{j=0}^{\infty } \mu_{t-j} 
% \]
% \begin{itemize}
% \item Para concluir, demostremos que $\upsilon_{t} \bot u_{s} ,s\in {\rm Z}$
% \end{itemize}
% $u_{t} ,u_{t-1} ,....$ sistema ortogonal, que puede completarse en una base 
% $\left( {u_{s}^{'} ,s\in {\rm Z}} \right)$ de $L^{2}\left( {\Omega ,{\rm 
% A},P} \right)$. Entonces $\upsilon_{t} =\sum {\alpha_{s} u_{s}^{'} } $
% 
% Si $s\le t,\upsilon_{t} \bot u_{s} $ pues: $u_{s} \bot \mu_{s-1} $ y 
% $\upsilon_{t} \in \mu_{s-1} \quad \left( {V_{t} \in \mathop \cap 
% \limits_{j=0}^{\infty } \mu_{t-j} } \right)$
% 
% Si $s>t \quad \left. {\begin{array}{l}
%  u_{s} \in \mu_{s} yu_{s} \bot \mu_{s-1} \\ 
%  \mathop\nolimits \mathop\nolimits\limits \upsilon_{t} \in \mu 
% _{t} \subset \mu_{s-1} \\ 
%  \end{array}} \right\} \Rightarrow u_{s} \bot \upsilon_{t} $
% 
% \textbf{Unicidad:}
% \[
% X_{t} ={u_{t} }\limits +\underbrace {\left[ {\sum\limits_{j=1}^\infty 
% {\lambda_{j} u_{t-j} +v_{t} } } \right]}_{\in \mu_{t-1} }
% \]
% Puesto que $u_{t} \,\,\bot \mu_{t-1} $, el t\'{e}rmino entre 
% corchetes es igual a $\hat{{X}}_{t} $ y $u_{t} 
% =X_{t} -\hat{{X}}_{t} $
% 
% \chapter{ANEXO B ANEXO ASOCIADO AL CAP\'{I}TULO 2}
% \label{sec:mylabel2}
% \section{ANEXO B.1: OPERADORES DE RETARDO Y AVANCE}
% \label{subsec:mylabel3}
% El operador de retardo B asocia a un proceso $X=\left( {X_{t} ,t\in {\rm Z}} 
% \right)$ el proceso $\left( {Y_{t} ,t\in Z} \right)$ tal que $Y_{t} =BX_{t} 
% =X_{t-1} $.
% 
% Este operador es lineal; es invertible y su inverso $B^{-1}=F$se define 
% por$FX_{t} =X_{t+1} $; F se llama operador de avance. Estos operadores 
% satisfacen:
% \[
% \begin{array}{l}
%  B^{n}X_{t} =X_{t-n} ;\;F^{n}X_{t} =X_{t+n} \quad \quad 
% n=1,\;2,\mathellipsis \\ 
%  \left( {\sum\limits_{i=0}^n {a_{i} B^{i}} } \right)X_{t} 
% =\sum\limits_{i=0}^n {a_{i} X_{t-i} } \\ 
%  \end{array}
% \]
% Esta \'{u}ltima igualdad describe la acci\'{o}n sobre el proceso X de un 
% polinomio en B. Adem\'{a}s, se define $B^{0}=F^{0}=1$ (operador identidad)
% 
% \textbf{SERIES EN B}
% 
% De manera m\'{a}s general se pueden definir series con el operador B (o con 
% F). Para esto, se consideran solamente procesos estacionarios.
% 
% Dado un proceso estacionario $X=\left( {X_{t} ,t\in Z} \right)$ y una 
% sucesi\'{o}n $\left( {a_{i} ,i\in Z} \right)$ absolutamente 
% convergente$\sum\limits_{i=-\infty }^\infty {\left| {a_{i} } \right|} 
% <+\infty $, se sabe que el proceso definido por:
% \[
% Y_{t} =\sum\limits_{i=-\infty }^\infty {a_{i} X_{t-i} } ,
% \quad
% t\in {\rm Z}
% \]
% es estacionario. Se denotar\'{a} por: $\sum\limits_{i=-\infty }^\infty 
% {a_{i} B^{i}} $a la aplicaci\'{o}n que al proceso estacionario X le hace 
% corresponden al proceso estacionario Y.
% 
% Estas series en B tienen propiedades que permiten manejarlas como a las 
% series enteras habituales. En particular se puede sumarlas y realizar la 
% composici\'{o}n entre ellas.
% 
% \textbf{SUMAS DE SERIES EN B}
% \[
% \left( {\sum\limits_{i=-\infty }^\infty {a_{i} B^{i}+\sum\limits_{i=-\infty 
% }^\infty {\alpha_{i} B^{i}} } } \right)X_{t} =\sum\limits_{i=-\infty 
% }^\infty {a_{i} B^{i}} X_{t} +\sum\limits_{i=-\infty }^\infty {\alpha_{i} 
% B^{i}X_{t} } =\sum\limits_{i-\infty }^\infty {a_{i} X_{t-i} 
% +\sum\limits_{i=-\infty }^\infty {\alpha_{i} } X_{t-i} } 
% \]
% \[
% \begin{array}{l}
%  =\lim\limits_{m,n\to \infty } \left( {\sum\limits_{-m}^n {a_{i} X_{t-i} 
% } +\sum\limits_{-m}^n {\alpha_{i} X_{t-i} } } \right) \\ 
%  =\lim\limits_{m,n\to \infty } \left[ {\sum\limits_{-m}^n {\left( {a_{i} 
% +\alpha_{i} } \right)X_{t-i} } } \right] \\ 
%  =\sum\limits_{i=-\infty }^\infty {\left( {a_{i} +\alpha_{i} } 
% \right)X_{t-i} } \\ 
%  =\left[ {\sum\limits_{i=-\infty }^\infty {\left( {a_{i} +\alpha_{i} } 
% \right)B^{i}} } \right]X_{t} \\ 
%  \end{array}
% \]
% pues si cada sucesi\'{o}n es absolutamente convergente la suma de las dos 
% sucesiones tambi\'{e}n lo ser\'{a}. Fundamentalmente se observa que:
% \[
% \sum\limits_{i=-\infty }^\infty {a_{i} B^{i}+\sum\limits_{i=-\infty }^\infty 
% {\alpha_{i} B_{i} =\sum\limits_{i=-\infty }^\infty {\left( {a_{i} +\alpha 
% _{i} } \right)B^{i}} } } 
% \]
% De la misma manera se puede mostrar que:
% \[
% \lambda \sum\limits_{i=-\infty }^\infty {a_{i} B^{i}=} 
% \sum\limits_{i=-\infty }^\infty {\lambda a_{i} B^{i}} 
% \]
% \textbf{COMPOSICI\'{O}N DE SERIES EN B}
% 
% \textbf{Propiedad:}
% \[
% \left[ {\sum\limits_{j=-\infty }^\infty {\alpha_{j} B^{j}} } \right]\left[ 
% {\sum\limits_{i=-\infty }^\infty {a_{i} B^{i}} } \right]X_{t} =\lim\limits_{n,m,n\textasciiacute ,m\textasciiacute \to \infty } \left[ 
% {\sum\limits_{j=-m}^n {\alpha_{j} B^{j}} } \right]\left[ 
% {\sum\limits_{i=-m\textasciiacute }^{n\textasciiacute } {a_{i} B^{i}} } 
% \right]X_{t} 
% \]
% \textbf{Demostraci\'{o}n:}
% 
% Den\'{o}tense por: $S_{n,m} ,S,\tilde{{S}}_{n\textasciiacute 
% ,m\textasciiacute } ,\tilde{{S}}$ las aplicaciones:
% \[
% \sum\limits_{j=-m}^n {\alpha_{j} B^{j},\sum\limits_{j=-\infty }^\infty 
% {\alpha_{j} B^{j\textasciiacute },\sum\limits_{i=-m\textasciiacute 
% }^{n\textasciiacute } {a_{i} B^{i},\sum\limits_{i=-\infty }^\infty {a_{i} 
% B^{i}} } } } 
% \]
% Se tiene que:
% \[
% \left\| {S\;\tilde{{S}}\;X_{t} -S_{n,m} \tilde{{S}}_{n\textasciiacute 
% m\textasciiacute } X_{t} } \right\|
% \le \left\| {S\;\tilde{{S}}\;X_{t} -S_{n,m} \tilde{{S}}X_{t} } 
% \right\|+\left\| {S_{n,m} \tilde{{S}}X_{t} -S_{n,m} 
% \tilde{{S}}_{n\textasciiacute ,m\textasciiacute } X_{t} } \right\|
% \]
% \[
% =\left\| {S\left( {\tilde{{S}}\;X_{t} } \right)-S_{n,m} \left( 
% {\tilde{{S}}X_{t} } \right)} \right\|+\left\| {S_{n,m} \left( 
% {\tilde{{S}}X_{t} -\tilde{{S}}_{n\textasciiacute m\textasciiacute } X_{t} } 
% \right)} \right\|
% \]
% \[
% \le \left\| {S\left( {\tilde{{S}}X_{t} } \right)-S_{n,m} \left( 
% {\tilde{{S}}X_{t} } \right)} \right\|+\left( {\sum\limits_{j=-m}^n {\left| 
% {\alpha_{j} } \right|} } \right)\left\| {\tilde{{S}}X_{t} 
% -\tilde{{S}}_{n\textasciiacute m\textasciiacute } X_{t} } \right\|
% \]
% \[
% \le \left\| {S\left( {\tilde{{S}}X_{t} } \right)-S_{n,m} \left( 
% {\tilde{{S}}X_{t} } \right)} \right\|+\left( {\sum\limits_{j=-\infty 
% }^\infty {\left| {\alpha_{j} } \right|} } \right)\left\| {\tilde{{S}}X_{t} 
% -\tilde{{S}}_{n\textasciiacute m\textasciiacute } X_{t} } \right\|
% \]
% La propiedad es una consecuencia de:
% 
% $S\;Y_{t} =\;\lim\limits_{n,m\to \infty } S_{n,m} \;Y_{t} $ y 
% $\tilde{{S}}\;X_{t} =\lim\limits_{n\textasciiacute m\textasciiacute \to 
% \infty } \tilde{{S}}_{n\textasciiacute ,m\textasciiacute } X_{t} $
% 
% (l\'{\i}mites en el sentido de $L_{2} )$
% 
% \textbf{Corolario:}
% 
% La composici\'{o}n de dos series en B es una serie en B.
% 
% \textbf{Demostraci\'{o}n:}
% 
% $\left[ {\sum\limits_{j=-\infty }^\infty {\alpha_{j} B^{j}} } \right]\left[ 
% {\sum\limits_{i=-\infty }^\infty {a_{i} B^{i}} } \right]X_{t} =\lim\limits_{n,m,n\textasciiacute m,\to \infty } \left[ {\sum\limits_{j=-m}^n 
% {\alpha_{j} B^{j}} } \right]\left[ {\sum\limits_{i=-m\textasciiacute 
% }^{n\textasciiacute } {a_{i} B^{i}} } \right]X_{t} $(l\'{\i}m. en el sentido 
% L$_{\mathrm{2}})$
% \[
% \begin{array}{l}
%  \quad \quad \;=\lim\limits_{n,m,n\textasciiacute ,m\textasciiacute \to 
% \infty } \sum\limits_{k=-m-m\textasciiacute }^{n+n\textasciiacute } {\quad 
% \left[ {\sum\limits_{i=\max \left( {-m\textasciiacute ,k-n} \right)^{a_{i} 
% }}^{\min \left( {n\textasciiacute ,k+m} \right)} {a_{i} } \alpha_{k-i} } 
% \right]B^{k}X_{t} } \\ 
%  \quad \quad \;=\sum\limits_{k=-\infty }^\infty {\left[ 
% {\sum\limits_{i=-\infty }^\infty {a_{i} \alpha_{k-i} } } \right]B^{k}X_{t} 
% } \\ 
%  \end{array}
% \]
% puesto que $b_{k} =\sum\limits_{i=-\infty }^\infty {a_{i} \alpha_{k-i} } $ 
% existe y la serie de t\'{e}rmino general b$_{\mathrm{k}}$ es absolutamente 
% convergente. La sucesi\'{o}n de los $b_{k} $ es la ``convoluci\'{o}n'' de 
% las series en $\alpha_{i} \;\text{y\thinspace a}_{\text{j}} .\;$
% 
% \textbf{Corolario:}
% 
% El producto de series en B es conmutativo:
% 
% \textbf{Demostraci\'{o}n:}
% 
% En efecto: $\sum\limits_{i=-\infty }^\infty {a_{i} \alpha_{k-i} } 
% =\sum\limits_{i=-\infty }^\infty {a_{k-i} \alpha_{i} } $.
% 
% \section{ANEXO B.2: ECUACIONES EN DIFERENCIAS}
% \label{subsec:mylabel4}
% Consid\'{e}rese la ecuaci\'{o}n lineal homog\'{e}nea en diferencias
% 
% $x_{t} -\varphi_{1} x_{t-1} -\varphi_{2} x_{t-2} -\mathellipsis -\varphi 
% _{k} x_{t-k} =0$ (B.2.1)
% 
% Se va a demostrar que su soluci\'{o}n general es:
% 
% $x_{t} =A_{1} G_{1}^{t} +\mathellipsis +A_{k} G_{k}^{t} $ (B.2.2)
% 
% donde las $A_{1} ,\mathellipsis ,A_{k} $ son constantes que dependen de las 
% condiciones iniciales y $G_{1} ,\mathellipsis G_{k} $ son las ra\'{\i}ces de 
% la ecuaci\'{o}n
% 
% $y^{k}-\varphi_{1} y^{k-1}-\mathellipsis -\varphi_{k} =0$ (B.2.3)
% 
% que se suponen son distintas. Para demostrar este resultado, sustit\'{u}yase 
% (B.2.2) en (B.2.1):
% \[
% \sum {A_{i} G_{i}^{t} -\varphi_{1} \sum {A_{i} G_{i}^{t-1} -\mathellipsis 
% -\varphi_{k} \sum {A_{i} G_{i}^{t-k} } =\sum\limits_{i=1}^k {A_{i} 
% G_{i}^{t-k} \left( {G_{i}^{k} -\varphi_{1} G_{i}^{k-1} -\mathellipsis 
% -\varphi_{k} } \right)=0} } } 
% \]
% Con esto se ha demostrado que si $G_{i} $ satisface (B.2.3), entonces la 
% soluci\'{o}n (B.2.2) satisface (B.2.1). Obs\'{e}rvese adem\'{a}s que para 
% que $x_{t} $ tienda a cero cuando $t\to \infty ,G_{i}^{t} $ debe tender a 
% cero, lo que requiere que el valor absoluto de todas las soluciones de 
% (B.2.3) sea menor a la unidad. 
% 
% Las expresiones anteriores pueden escribirse introduciendo el operador $B$de 
% retardo, definido por $(B^{0}X_{t} =1;B^{k}X_{t} =X_{t-k} 
% ,k=1,\;2,\mathellipsis )$. Entonces la ecuaci\'{o}n en diferencias (B.2.1) 
% se expresa
% \[
% \left( {1-\varphi_{1} B-\mathellipsis -\varphi_{k} B^{k}} \right)x_{t} =0
% \]
% y se denomina \textit{ecuaci\'{o}n caracter\'{\i}stica} de la ecuaci\'{o}n en diferencias a:
% 
% $1-\varphi_{1} B-\mathellipsis -\varphi_{k} B^{k}=0$ (B.2.4)
% 
% N\'{o}tese que: $y^{k}-\varphi_{1} y^{k-1}-\mathellipsis -\varphi_{k} 
% =0\Leftrightarrow y^{k}\left( {1-\varphi_{1} y^{-1}-\mathellipsis -\varphi 
% _{k} y^{-k}} \right)=0$ 
% 
% As\'{\i}, si $G_{i} $ satisface (B.2.3)entonces $G_{i}^{-1} $ es la 
% soluci\'{o}n de la ecuaci\'{o}n caracter\'{\i}stica (B.2.4). La soluci\'{o}n 
% (B.2.2) se expresa habitualmente indicando que $G_{1}^{-1} ,\mathellipsis 
% ,G_{k}^{-1} $ son las ra\'{\i}ces de la ecuaci\'{o}n caracter\'{\i}stica.
% 
% Cuando dos ra\'{\i}ces de (B.2.4) \'{o} (B.2.3) son iguales los dos 
% t\'{e}rminos $A_{i} G_{i}^{t} $ correspondientes pueden escribirse como 
% $\left( {A_{1} +A_{2} t} \right)G^{t}$. Para demostrarlo, sup\'{o}ngase el 
% caso m\'{a}s simple, con una ra\'{\i}z doble en la ecuaci\'{o}n:
% 
% $x_{t} -\varphi_{1} x_{t-1} -\varphi_{2} x_{t-2} =0$ (B.2.5)
% 
% cuya ecuaci\'{o}n caracter\'{\i}stica es:
% 
% $\left( {1-\varphi_{1} B-\varphi_{2} B^{2}} \right)=0$ (B.2.6)
% 
% Si existe una ra\'{\i}z doble en esta ecuaci\'{o}n, $G_{0}^{-1} ,$ se 
% verifica:
% \[
% \left( {1-G_{0} B} \right)^{2}=\left( {1-\varphi_{1} B-\varphi_{2} B^{2}} 
% \right)
% \]
% que implica $\varphi_{2} =-G_{0}^{2} ;\;\varphi_{1} =2G_{0} $ y por tanto:
% 
% $\varphi_{1} G_{0} +2\varphi_{2} =0$ (B.2.7)
% 
% Adem\'{a}s si $G_{0}^{t} $ es una soluci\'{o}n doble de (B.2.5) tambi\'{e}n 
% debe serlo $tG_{0}^{t} $. Pues sustituyendo en (B.2.5):
% \[
% tG_{0}^{t} -\varphi_{1} \left( {t-1} \right)G_{0}^{t-1} -\varphi_{2} 
% \left( {t-2} \right)G_{0}^{t-2} =tG_{0}^{t-2} \left( {G_{0}^{2} -\varphi 
% _{1} G_{0} -\varphi_{2} } \right)+G_{0}^{t-2} \left( {\varphi_{1} G_{0} 
% +2\varphi_{2} } \right)=0
% \]
% El primer t\'{e}rmino es cero por ser $G_{0} $ soluci\'{o}n y el segundo por 
% ser ra\'{\i}z de orden 2, con lo que se concluye que la soluci\'{o}n general 
% es:
% \[
% x_{t} =A_{1} G_{o}^{t} +A_{2} tG_{o}^{t} =\left( {A_{1} +A_{2} t} 
% \right)G_{0}^{t} 
% \]
% An\'{a}logamente se puede demostrar que si la ra\'{\i}z $G_{0} $ es 
% m\'{u}ltiple, de orden $h$, la soluci\'{o}n general es:
% 
% $x_{t} =\left( {A_{1} +A_{2} t+\mathellipsis +A_{h} t^{h-1}} \right)G_{0}^{t} 
% $ (B.2.8)
% 
% Para estudiar el comportamiento de las ra\'{\i}ces complejas de la 
% ecuaci\'{o}n caracter\'{\i}stica, consid\'{e}rese el caso simple determinado 
% por las ecuaciones (B.2.5) y (B.2.6).
% 
% Sup\'{o}ngase que $G_{1}^{-1} $ y $G_{2}^{-1} $ son soluciones distintas de 
% la ecuaci\'{o}n caracter\'{\i}stica
% \[
% \left( {1-\varphi_{1} B-\varphi_{2} B^{2}} \right)=\left( {1-G_{1} B} 
% \right)\left( {1-G_{2} B} \right)=0
% \]
% Por tanto, $\varphi_{1} =G_{1} +G_{2} ;\varphi_{2} =-G_{1} G_{2} .$ Se 
% calculan $A_{1} $ y $A_{2} ,$ imponiendo las condiciones iniciales $\rho 
% _{0} =1;\;\rho_{1} =\varphi_{1} \left( {1-\varphi_{2} } \right)=\left( 
% {G_{1} +G_{2} } \right)/\left( {1+G_{1} G_{2} } \right)$. Se obtiene:
% \[
% A_{1} =\frac{G_{1} \left( {1-G_{2}^{2} } \right)}{\left( {1+G_{1} G_{2} } 
% \right)\left( {G_{1} -G_{2} } \right)};\quad A_{2} =\frac{-G_{2} \left( 
% {1-G_{1}^{2} } \right)}{\left( {1+G_{1} G_{2} } \right)\left( {G_{1} -G_{2} 
% } \right)}
% \]
% Sup\'{o}ngase adem\'{a}s que $G_{1} $ y $G_{2} $ son complejas; entonces 
% deben ser conjugadas y pueden escribirse $G_{1} =r\;\exp \left( {i\varpi } 
% \right);\;G_{2} =r\;\exp \left( {-i\varpi } \right),$ donde el m\'{o}dulo se 
% calcula por:
% 
% $\begin{array}{l}
%  r^{2}=G_{1} G_{2} =-\varphi_{2} \\ 
%  r=\sqrt {-\varphi_{2} .} \\ 
%  \end{array}$ (B.2.9)
% 
% y la frecuencia angular con:
% \[
% \varphi_{1} =G_{1} +G_{2} =r\left( {\exp \left( {i\varpi } \right)+\exp 
% \left( {-i\varpi } \right)} \right)=2r\;Cos\;\varpi 
% \]
% Utilizando que $\varpi =2\pi /T_{0} ,$ donde $T_{0} $ es el per\'{\i}odo, y 
% (B.2.9), se obtiene:
% 
% $Cos\;\varpi =Cos\frac{2\pi }{T_{0} }=\frac{\varphi_{1} }{2\sqrt {-\varphi 
% _{2} } }$ (B.2.10)
% 
% Sustituyendo $A_{1} ,A_{2} ,G_{1} \;\text{y}\;G_{2} $ en la ecuaci\'{o}n 
% (B.2.2) y operando, se obtiene:
% \[
% x_{t} =\frac{\left[ {sg\left( {\varphi_{1} } \right)} 
% \right]^{t}r^{t}sen\left( {2\pi t/T_{0} +\alpha } \right)}{sen\alpha }
% \]
% Donde $tg\;\alpha =\left( {1+r^{2}} \right)\;tg\left( {2\pi /T_{0} } 
% \right)/\left( {1-r^{2}} \right)\;\text{y}\;sg\left( {\varphi_{1} } 
% \right)$ es el signo del coeficiente $\varphi_{1} .$ Por tanto $X^{t}$ 
% ser\'{a} una senoide amortiguada con factor de amortiguaci\'{o}n $r$ y 
% per\'{\i}odo $T_{0} $. 
% 
% Por tanto, la soluci\'{o}n general de (B.2.1) tendr\'{a}: 
% 
% \begin{itemize}
% \item Por cada ra\'{\i}z real no repetida un t\'{e}rmino del tipo $A_{i} G_{i}^{t} ;$ 
% \item Por cada ra\'{\i}z real $G_{0} $ repetida $h$ veces un t\'{e}rmino polin\'{o}mico de orden $\left( {h-1} \right)$ en la variable $t$ que multiplica a $G_{0}^{t} $ (ecuaci\'{o}n 2.B.8);
% \item Por cada par de ra\'{\i}ces complejas conjugadas un t\'{e}rmino sinusoidal.
% \end{itemize}
% 
% \section{ANEXO B.3: DEMOSTRACIONES DE ALGUNOS TEOREMAS}
% \label{subsec:mylabel5}
% \textbf{Teorema 2.2: }Una condici\'{o}n necesaria y suficiente para 
% que exista un AR(p) que satisfaga $\pi^{'}\left( B \right)X_{t} =u_{t} $ es 
% que las ra\'{\i}ces de la ecuaci\'{o}n $\pi^{'}\left( z \right)=0$ se 
% encuentren fuera del c\'{\i}rculo unidad. Bajo esta condici\'{o}n.
% \[
% X_{t} =\sum\limits_{j=0}^\infty {\psi_{j} u_{t-j} } ,
% \quad
% t\in {\rm Z}
% \]
% donde los $\psi_{j} $, son los coeficientes de Taylor de $\frac{1}{\pi 
% ^{'}\left( z \right)}$
% 
% \textbf{Demostraci\'{o}n:}
% 
% \begin{itemize}
% \item \textbf{Condici\'{o}n Necesaria} 
% \end{itemize}
% 
% Sean
% \[
% y_{t} =\left( {\begin{array}{l}
%  X_{t} \\ 
%  X_{t-1} \\ 
%  \vdots \\ 
%  X_{t-p+1} \\ 
%  \end{array}} \right)
% \quad
% \delta_{t} =\left( {\begin{array}{l}
%  u_{t} \\ 
%  0 \\ 
%  \vdots \\ 
%  0 \\ 
%  \end{array}} \right)
% \quad
% P=\left[ {\begin{array}{l}
%  \pi_{1} ..............\pi_{p} \\ 
%  \mathop\nolimits I_{p-1} \mathop\nolimits \mathop\nolimits 
% 0\limits_{{\begin{array}{l}
%  \vdots \\ 
%  0 \\ 
%  \end{array}}} \\ 
%  \end{array}} \right]
% \]
% Entonces$y_{t} =Py_{t-1} +\delta_{t} $
% 
% Sean: $\Gamma \quad =$ matriz de covarianza de $y_{t} $ y $\Delta =$matriz de 
% covarianza de $\delta_{t} $
% \[
% \Delta =
% \left[ {{\begin{array}{*{20}c}
%  {\sigma^{2}} \hfill & 0 \hfill & 0 \hfill \\
%  {0\limits_{\vdots } } \hfill & {\limits_{0} } \hfill & \hfill \\
%  0 \hfill & \hfill & \hfill \\
% \end{array} }} \right]
% \]
% Puesto que $\delta_{t} \bot y_{t-1} , \quad \Gamma \quad =P\Gamma P^{'}+\Delta $ 
% (pues $\left( {X_{t} } \right)$ d.e)
% 
% Sea $\phi \left( \lambda \right)$ el polinomio caracter\'{\i}stico de P:
% \[
% \quad
% \phi \left( \lambda \right)=\left[ {{\begin{array}{*{20}c}
%  {\pi_{1} -\lambda } \hfill & {\pi_{2} } \hfill & {\pi_{3} } \hfill & 
% \mathellipsis \hfill & \hfill & {\pi_{p} } \hfill \\
%  1 \hfill & {-\lambda } \hfill & 0 \hfill & \hfill & \hfill & \hfill \\
%  \hfill & 1 \hfill & {-\lambda } \hfill & \hfill & 0 \hfill & \hfill \\
%  {\limits^{0} } \hfill & \hfill & \hfill & \hfill & 1 \hfill & {-\lambda } 
% \hfill \\
% \end{array} }} \right]
% \]
% $=\limits^{ind.} \left( {-1} \right)^{p-1}
% [-1+\sum\limits_{j=1}^P {\pi_{j} \left. {\lambda^{-j}} \right]} \lambda 
% ^{P}$ (desarrollando por la 1\textordfeminine l\'{\i}nea).
% 
% Las ra\'{\i}ces $\left( {\ne 0} \right)$ de $\phi \left( \lambda \right)$son 
% l as ra\'{\i}ces de $\left[ {-1+\sum\limits_{j=1}^P {\pi_{j} \lambda^{-j}} 
% } \right]\lambda^{P}$; es decir, aquellas de $\pi^{'}\left( 
% {\frac{1}{\lambda }} \right)=0$. Entonces los valores propios (v.p.) de 
% \textbf{\textit{P}} son los inversos de las ra\'{\i}ces de $\pi^{'}$; vamos 
% a demostrar que estos son de $\left\| \right\|<1$. 
% 
% Sea $X$ un vector propio de $P^{'}$ y $\lambda $ el v.p. asociado (Recordamos 
% que: $X^{\ast }=\overline X^{'}yX^{\ast }P=\overline \lambda X^{\ast 
% })$; entonces
% \[
% 0\le X^{\ast }\Delta X=X^{\ast }\Gamma X-X^{\ast }P\Gamma P^{'}X
% \]
% \[
% \begin{array}{l}
%  \le \left( {1-\left| \lambda \right|^{2}} \right)\underbrace {X\ast \Gamma 
% X}_{\ge 0} \\ 
%  \left| \lambda \right|^{2}\le 1 \\ 
%  \end{array}
% \]
% Sup\'{o}ngase que existe $\lambda $ valor propio de $P^{'}$ t.q. $\left| 
% \lambda \right|=1\Rightarrow X^{\ast }\Delta X=0$
% 
% Sea $X=\left( {\begin{array}{l}
%  x_{1} \\ 
%  x_{2} \\ 
%  \vdots \\ 
%  x_{p} \\ 
%  \end{array}} \right)\Rightarrow X^{\ast }\Delta X=\sigma^{2}\left| 
% {x_{1} } \right|^{2}\Rightarrow x_{1} =0$
% 
% Por tanto:
% \[
% P^{'}X=\left[ {{\begin{array}{*{20}c}
%  {\pi_{1} } \hfill & 1 \hfill & 0 \hfill & \cdots \hfill & 0 \hfill \\
%  {\pi_{2} } \hfill & 0 \hfill & 1 \hfill & \cdots \hfill & 0 \hfill \\
%  \mathellipsis \hfill & \cdots \hfill & \cdots \hfill & \cdots \hfill & 
% \cdots \hfill \\
%  \cdots \hfill & \cdots \hfill & \cdots \hfill & \cdots \hfill & \cdots 
% \hfill \\
% \end{array} }} \right]\left[ {{\begin{array}{*{20}c}
%  {\begin{array}{l}
%  0 \\ 
%  x_{2} \\ 
%  x_{3} \\ 
%  \vdots \\ 
%  \end{array}} \hfill \\
% \end{array} }} \right]=\lambda \left[ {\begin{array}{l}
%  0 \\ 
%  x_{2} \\ 
%  \vdots \\ 
%  \end{array}} \right]\Rightarrow x_{2} =0
% \]
% continuando con la 2\textordfeminine l\'{\i}nea se obtiene $x_{3} =0$ etc. 
% $\Rightarrow X=0$ (una contradicci\'{o}n, pues un vector propio no puede ser 
% nulo).
% \[
% \left| \lambda \right|<1
% \]
% \begin{itemize}
% \item \textbf{Condici\'{o}n Suficiente}
% \end{itemize}
% 
% Sea $\left( {u_{t} } \right)$ el r.b. y sea $\pi^{'}$ dado. Se busca 
% construir el proceso AR (p).
% 
% $\frac{1}{\pi^{'}\left( z \right)}$ es holomorfa en $\left| z 
% \right|<1+\alpha $ ($\alpha $\textgreater 0, bien elegido)
% 
% Consid\'{e}rese el desarrollo de Taylor de $\frac{1}{\pi^{'}\left( z 
% \right)}$:
% \[
% \frac{1}{\pi^{'}\left( z \right)}=\sum\limits_{j=0}^\infty {\psi_{j} 
% z^{j}} ,
% \quad
% \left| z \right|<1+\alpha 
% \]
% (la serie converge en todo c\'{\i}rculo cerrado contenido en $B\left( 
% {0,1+\alpha } \right)\left. \right)$
% 
% Para $\left| z \right|=1,\limits \sum\limits_{j=0}^\infty {\left| {\psi 
% _{j} } \right|<\infty } $. Se pone: $X_{t} =\sum\limits_{j=0}^\infty {\psi 
% _{j} u_{t-j} } $ y luego se considera $X_{t} -\sum\limits_{j=1}^P {\pi_{j} 
% X_{t-j} =u_{t} } $.
% 
% Se hacen los reemplazos correspondientes y se verifica que este proceso 
% satisface la definici\'{o}n (c\'{a}lculos un poco largos).
% 
% \textbf{Teorema 2.5: }Se tiene que 
% $X_{t}=m_{t}+Y_{t\thinspace ,\thinspace \thinspace \thinspace 
% \thinspace }\forall t\ge -s+1_{\thinspace ,\thinspace }$donde: 
% 
% \begin{enumerate}
% \item $m_{t} $es la soluci\'{o}n de $\phi (B)X_{t} =0$ con $X_{-s+1} ,.....,X_{-s+p^{'}} $ como valores iniciales (v.i.)
% \item $Y_{t} =\left\{ {\begin{array}{l}
%  \sum\limits_{i=0}^{t+s-p^{'}-1} {\psi_{i} u_{t-i} \mathop\nolimits \mathop\nolimits t\ge -s+p^{'}+1} \\ 
%  0\mathop\nolimits \mathop\nolimits \mathop\nolimits \mathop\nolimits \mathop\nolimits \mathop\nolimits\limits -s+1<t<-s+p^{'}+1 \\ 
%  \end{array}} \right.$
% \end{enumerate}
% 
% Los $\psi_{i} $ se obtienen por divisiones creciente de $\Theta $ por $\phi 
% $ (hasta el orden $t+s-p^{\textasciiacute }$\textit{-1}).
% 
% \textbf{Demostraci\'{o}n: }Una soluci\'{o}n particular de:
% \[
% \phi \left( B \right)X_{t} =\Theta \left( B \right)\,u_{t} 
% \quad
% t\ge -s+p^{'}+1
% \]
% es:
% \[
% \left\{ {\begin{array}{l}
%  \sum\limits_{i=0}^{t+s-p^{'}-1} {\psi_{i} u_{t-i} } \mathop\nolimits 
% t\ge -s+p^{'}+1 \\ 
%  0\mathop\nolimits \mathop\nolimits \mathop\nolimits \mathop 
% \nolimits \mathop\nolimits t<-s+p^{'}+1 \\ 
%  \end{array}} \right.
% \]
% pues, escribiendo la identidad de la divisi\'{o}n creciente:
% \[
% \Theta (B)=\phi (B)\left[ {1+\psi_{1} B+.....+\psi_{t+s-p^{'}-1} 
% B^{t+s-p^{'}-1}} \right]+B^{t+s-p^{'}}\lambda \left( B \right)
% \]
% y aplicando a los dos miembros u$_{\mathrm{t\thinspace \thinspace }}$se 
% tiene:
% \[
% \Theta (B)u_{t} =\phi (B)\sum\limits_{i=0}^{t+s-p^{'}-1} {\psi_{i} 
% u_{t-i} +0} 
% \]
% Puesto que m$_{\mathrm{t}}$ es soluci\'{o}n de $\phi \left( B \right)X_{t} 
% =0$, compatible con los v.i., se obtiene el resultado anunciado. En 
% particular:
% \[
% X_{t} =m_{t} +\sum\limits_{i=0}^{t+s-p^{'}-1} {\psi_{i} u_{t-i} } ,
% \quad
% \forall t\ge -s+p^{'}+1
% \]
% \textbf{Observaci\'{o}n B.2:}
% 
% \begin{enumerate}
% \item Puesto, que m$_{\mathrm{t}}$ satisface una ecuaci\'{o}n en diferencias y si $\varphi $ no tiene ra\'{\i}ces m\'{u}ltiples, m$_{\mathrm{t}}$ puede expresarse:
% \end{enumerate}
% \[
% m_{t+s} =\sum\limits_{i=1}^d {a_{i} (t+s)^{i-1}+\sum\limits_{i=1}^p {b_{i} 
% z_{i}^{-(t+s)} } } 
% \]
% $(t+s\ge 1);$ z$_{\mathrm{i}}$ ra\'{\i}z de $\varphi $
% 
% Si algunas ra\'{\i}ces de $\varphi (z)$ son m\'{u}ltiples, hay que 
% introducir ciertos factores polinomiales en la segunda sumatoria; pero, en 
% todos los casos, este t\'{e}rmino se vuelve despreciable para $\forall 
% t\ge 0$ y s suficientemente grande.
% 
% Si s es bastante grande, se puede suponer que m$_{\mathrm{t}}$ es, para todo 
% $t\ge 0$, un polinomio de grado d-1
% 
% \begin{enumerate}
% \item $\forall 
% \quad
% r\in $N, $\Theta (z)=\phi (z)\left[ {1+\sum\limits_{i=1}^r {\psi_{i} z^{i}} } \right]+z^{r+1}\lambda (z)$
% \end{enumerate}
% 
% entonces: $\forall n\ge \max (q+1,p^{'}) \quad \psi_{n} -\phi_{1} \psi 
% _{n-1} -......-\phi_{p} \psi_{n-p^{'}} =0$.
% 
% Razonando de la misma manera que en la parte (i), se puede aproximar $\psi 
% _{n} $, para $n$ suficientemente grande, por un polinomio de grado \textit{d-1} en $n$. Por lo 
% tanto, la serie diverge si \textit{d\textgreater 1.}
% 
% \begin{enumerate}
% \item El teorema anterior no depende del hecho que $u_{t}_{\mathrm{\thinspace }}$sea centrada.
% \end{enumerate}
% 
% Si $u_{t} =\varepsilon_{t} +\lambda $ con $E\varepsilon_{t} =0$, el 
% proceso ARIMA se presenta: $\phi (B)X_{t} =\theta_{0} +\Theta 
% (B)\varepsilon_{t} $ con $\theta_{0} =\lambda (1-\theta_{1} 
% -......-\theta_{q} )$ y entonces:
% \[
% X_{t} =m_{t} +\lambda \sum\limits_{i=0}^{t+s-p^{'}-1} {\psi_{i} +} 
% \sum\limits_{i=0}^{t+s-p^{'}-1} {\psi_{i} \varepsilon_{t-i} } 
% \]
% luego de la parte (2), el segundo t\'{e}rmino es un polinomio de grado $d$ en 
% $t$. Entonces la introducci\'{o}n de una constante, aumenta el grado de $E (X_{t})$ 
% en una unidad.
% 
% \begin{itemize}
% \item \textbf{Representaci\'{o}n AR de un ARIMA:}
% \end{itemize}
% 
% \textbf{Teorema 2.6: }Se tiene :
% \[
% X_{t} =-\sum\limits_{i=1}^{t+s-p^{'}-1} {\pi_{i} X_{t-i} +} 
% \sum\limits_{j=s-p^{'}}^{s-1} {\pi_{j,t}^{\ast } X_{-j} +u_{t} } 
% \quad
% \forall t>-s+p^{'}
% \]
% donde:
% 
% \begin{enumerate}
% \item Los $\pi_{i} $ se obtienen por divisi\'{o}n creciente de $\phi $ por $\Theta (\pi_{0} =1)$
% \item Los $\pi_{s-p^{\textasciiacute '},t}^{\ast } ,.....,\pi_{s-1,t}^{\ast } \to 0$ cuando $s\to \infty $
% \end{enumerate}
% 
% \textbf{Demostraci\'{o}n:}
% 
% Se tiene que:
% 
% \begin{center}
% 1$=\Theta \left( B \right)\left[ {1+\sum\limits_{i=1}^{t+s-p^{'}-1} 
% {\theta_{i}^{\ast } } B^{C}} \right]+B^{t+s-p^{'}} \quad v$(B)
% \end{center}
% 
% y aplicando $u_{t}_{\mathrm{\thinspace }}$a esta expresi\'{o}n se obtiene:
% \[
% \left[ {1+\sum\limits_{i=1}^{t+s-p^{'}-1} {\theta_{i}^{\ast } B^{i}} } 
% \right]\,\Theta (B)u_{t} =u_{t} \mathop\nolimits \mathop\nolimits 
% t>-s+p^{'}
% \]
% $\left[ {1+\sum\limits_{i=1}^{t+s-p^{'}-1} {\theta_{i}^{\ast } B^{i}} } 
% \right]\phi (B)X_{t} =u_{t} $ (B.3.1)
% 
% que tiene la forma de relaci\'{o}n pedida (de la definici\'{o}n del ARIMA).
% 
% De otro lado:
% 
% $\phi (B)=\Theta (B)\left[ {1+\sum\limits_{i=1}^{t+s-p^{'}-1} {\pi_{i} 
% B^{i}} } \right]+B^{t+s-p^{'}}\mu (B)$ (B.3.2)
% 
% Utilizando (2.4)
% 
% $\phi (B)=\Theta (B)\left[ {1+\sum\limits_{i=1}^{t+s-p^{'}-1} {\theta_{i} 
% B^{i}} } \right]\phi (B)+B^{t+s-p^{'}}v(B)\phi (B)$ (B.3.3)
% 
% y de la unicidad de la divisi\'{o}n de $\phi $ por $\Theta ,$ se igualan los 
% coeficientes de B$^{\mathrm{i\thinspace \thinspace \thinspace }}$( 
% i$=$1,....,t$+$s-p$^{\mathrm{\text{\textasciiacute }}}$-1) en (B.3.2) y 
% (B.3.3), entonces los coeficientes en (B.3.1) son los $\pi_{i} $.
% 
% Adem\'{a}s:
% \[
% \quad
% \begin{array}{l}
%  \;\pi_{s-p^{'},t}^{\ast } \;\;=-\phi_{1} \theta_{t+s-p^{'}-1}^{\ast } 
% -......-\phi_{p'} \theta_{t+s-2p'}^{\ast } \\ 
%  \pi_{s-p'+1,t}^{\ast } =-\phi_{2} \theta_{t+s-p'-1}^{\ast } 
% -.....-\phi_{p'} \theta_{{\begin{array}{l}
%  t+s-2p'+ \\ 
%  1 \\ 
%  \end{array}}}^{\ast } \\ 
%  \pi_{s-1,t}^{\ast } \,\quad \,=-\phi_{p'} \theta_{t+s-p'-1}^{\ast } 
% \\ 
%  \end{array}
% \]
% Los $\pi_{s-p',t}^{\ast } ,.....,\pi_{s-1,t}^{\ast } \to 0$ cuando $s\to 
% \infty $, pues los $\theta_{i} $ son coeficientes de la divisi\'{o}n 
% creciente de 1 por un polinomio con ra\'{\i}ces de $\left| \right|>1$ y 
% entonces tienden hacia cero cuando $i\to \infty $
% 
% \textbf{Observaci\'{o}n B.3:}
% 
% \begin{enumerate}
% \item Puesto que las ra\'{\i}ces de $\Theta $ son de $\left| \right|>1$, se tiene que:
% \end{enumerate}
% $\phi (z)=\Theta (z)\left[ {1+\sum\limits_{i=1}^\infty {\pi_{i} z^{i}} } 
% \right]$ para $\left| z \right|\le 1$
% 
% (en realidad para $\left| z \right|<$ min $\left| {z_{j} } 
% \right|conz_{j} $ ra\'{\i}z de $\Theta )$.
% 
% Si d\textgreater 0, $\phi \left( 1 \right)=0$ y entonces:
% \[
% 0=\Theta (1)\left[ {1+\sum\limits_{i=1}^\infty {\pi_{i} } } \right]
% \]
% y, dado que $\Theta $(1)$\ne 0$, se tiene:
% \[
% 1+\sum\limits_{i=1}^\infty {\pi_{i} =0} 
% \]
% \begin{enumerate}
% \item Si se pone: 
% \end{enumerate}
% $\phi (B)X_{t} =\theta_{0} +\Theta (B)\varepsilon_{t} $ con $E\varepsilon 
% _{t} =0$
% 
% $\phi (B)X_{t} =\Theta \left( B \right)u_{t} $ con $u_{t} =\varepsilon_{t} 
% +\lambda $
% 
% y 
% \[
% Eu_{t} =\lambda =\frac{\theta_{0} }{1-\theta_{1} -.....-\theta_{q} }
% \]
% se tiene:
% \[
% X_{t} =\lambda -\sum\limits_{i=1}^{t+s-p'-1} {\pi_{i} X_{t-i} } 
% +\sum\limits_^ {\pi_{j,t}^{\ast } X_{-j} +\varepsilon_{t} } 
% \]
% \begin{enumerate}
% \item La introducci\'{o}n de un mecanismo de inicializaci\'{o}n, hace que el prop\'{o}sito inicial de que el proceso (X$_{\mathrm{t}})$ diferenciado $d$ veces sea ARMA (p,q), se cumpla solo asint\'{o}ticamente . Es decir, el proceso ARIMA
% \end{enumerate}
% \[
% X_{t} =m_{t} +\sum\limits_{i=0}^{t+s-p'-1} {\psi_{i} u_{t-i} } 
% \quad
% \psi_{0} =1,
% \quad
% t>-s+p'+1
% \]
% es t.q. $E(\Delta^{d}X_{t} -Z_{t} )^{2}\to 0$, donde $Z_{t} $ es un ARMA 
% (p, q) definido por: $\Phi (B)Z_{i} =\Theta (B)u_{t} $
% 
% En efecto: 
% \[
% \Delta^{d}X_{t} =\Delta^{d}m_{t} +\nabla^{d}\left( 
% {\sum\limits_{i=0}^{t+s-p'-1} {\psi_{i} B^{i}} } \right)u_{t} 
% \]
% Puesto que:
% \[
% \Theta (B)=\varphi (B)\Delta^{d}\left( {1+\sum\limits_{i=1}^{t+s-p'-1} 
% {\psi_{i} B^{i}} } \right)+B^{t+s-p'}\lambda (B)
% \]
% entonces:
% \[
% \Delta^{d}\left( {1+\sum\limits_{i=1}^{t+s-p'-1} {\psi_{i} B^{i}} } 
% \right)=\frac{\Theta (B)}{\varphi (B)}-\frac{B^{t+s-p'}\lambda (B)}{\varphi 
% (B)}
% \]
% 
% $\quad
% \Delta^{d}X_{t} =\nabla^{d}m_{t} +\frac{\Theta (B)}{\varphi (B)}u_{t} $ 
% (u$_{\mathrm{t}}=$0 si t$\le -s+p')$
% 
% Pero para t suficientemente grande m$_{\mathrm{t}}$ se puede considerar un 
% polinomiado de grado d-1, luego $\Delta^{d}m_{t} \to 0$ si $t\to \infty .$ 
% Por tanto $Z_{t} -\frac{\Theta (B)}{\varphi (B)}u_{t} \,\,(u_{t} 
% =0sit<-s+p')$
% 
% converge a cero en m.c. cuando $t\to \infty .$
% 
% \chapter{ANEXO C ANEXO ASOCIADO AL CAP\'{I}TULO 3}
% \label{sec:mylabel3}
% \section{ANEXO C.1: ESTIMACI\'{O}N DE M\'{A}XIMA VEROSIMILITUD NO CONDICIONAL EN UN 
% PROCESO ARMA }
% \label{subsec:mylabel6}
% \textbf{C.1.1 Caso de un MA (q)}
% 
% Consid\'{e}rese el siguiente modelo.
% \[
% z_{t} =\Theta (B)u_{t} =u_{t} -\sum\limits_{j=1}^q {\theta_{j} u_{t-j} } 
% \quad \quad \quad t=1,\;2,.......,N
% \]
% Sup\'{o}ngase que los u$_{\mathrm{t}}$ son normales (en realidad solo se 
% requiere independencia). Este sistema se puede escribir matricialmente 
% as\'{\i}:
% \[
% z=M(\theta )\,u\;
% \]
% $\;con\;\;z=\left( {{\begin{array}{*{20}c}
%  {z_{1} } \hfill \\
%  {z_{2} } \hfill \\
%  \vdots \hfill \\
%  \hfill \\
%  {z_{N} } \hfill \\
% \end{array} }} \right)\quad ,\quad u=\left( {{\begin{array}{*{20}c}
%  {u_{1-q} } \hfill \\
%  {u_{2-q} } \hfill \\
%  \vdots \hfill \\
%  \hfill \\
%  {u_{N} } \hfill \\
% \end{array} }} \right)$ y M($\theta )$ una matriz que depende de los $\theta 
% _{\mathrm{j}}$
% 
% Adem\'{a}s: z \textasciitilde $N(0\;,\;\sigma^{2}M(\theta )M(\theta )')$
% 
% Una primera idea puede ser encontrar la densidad de z para maximizar la 
% verosimilitud; sin embargo, esto no es muy aconsejable pues se debe invertir 
% la matriz (M($\theta )$M($\theta )$')$_{\mathrm{NxN}}$. Consid\'{e}rese mas 
% bien el sistema:
% 
% u$_{\mathrm{1-q\thinspace }}=$ u$_{\mathrm{1-q}}$ 
% \[
% \vdots 
% \quad
% \vdots 
% \]
% u$_{\mathrm{0\thinspace \thinspace }} \quad =$ u$_{\mathrm{0}}$ 
% 
% u$_{\mathrm{1\thinspace \thinspace \thinspace \thinspace }}=$ 
% z$_{\mathrm{1\thinspace +\thinspace }}\theta 
% _{\mathrm{1}}$u$_{\mathrm{0\thinspace }} \quad +$ 
% ...................$+\theta_{\mathrm{q}}$ u$_{\mathrm{1-q}}$ 
% 
% u$_{\mathrm{2\thinspace \thinspace \thinspace \thinspace }}=$ 
% z$_{\mathrm{2\thinspace +\thinspace }}\theta 
% _{\mathrm{2}}$u$_{\mathrm{1\thinspace }} \quad +$ 
% ...................$+\theta_{\mathrm{q}}$ u$_{\mathrm{2-q}}$
% \[
% \vdots 
% \quad
% \vdots 
% \quad
% \vdots 
% \]
% u$_{\mathrm{N\thinspace }} \quad =$ z$_{\mathrm{N\thinspace +\thinspace 
% }}\theta_{\mathrm{1}}$u$_{\mathrm{N-1\thinspace }} \quad +$ 
% ...................$+\theta_{\mathrm{q}}$ u$_{\mathrm{N-q}}$
% 
% Se obtiene :
% \[
% u=Lz+Xu_{\ast } 
% \]
% donde las matrices L, X y el vector $u_{\ast } $ se definen por:
% 
% \[
% \begin{array}{l}
%  L=\left( {\begin{array}{l}
%  \;\;0 \\ 
%  A_{1} (\theta ) \\ 
%  \end{array}} \right)_{(N+q)xN} \quad \quad \quad \quad \quad \quad A_{1} 
% (\theta )=\left( {{\begin{array}{*{20}c}
%  1 \hfill & \hfill & \hfill & \hfill \\
%  \cdots \hfill & 1 \hfill & 0 \hfill & \hfill \\
%  \vdots \hfill & \vdots \hfill & \hfill & \hfill \\
%  \cdots \hfill & \cdots \hfill & \cdots \hfill & 1 \hfill \\
% \end{array} }} \right)_{NxN} \\ 
%  X=\left( {\begin{array}{l}
%  I_{q} \\ 
%  A_{2} (\theta ) \\ 
%  \end{array}} \right)_{(N+q)xq} \quad \quad \quad \quad \quad \quad u_{\ast 
% } =\left( {\begin{array}{l}
%  u_{1-q} \\ 
%  \;\vdots \\ 
%  \;\vdots \\ 
%  u_{o} \\ 
%  \end{array}} \right) \\ 
%  \end{array}
% \]
% El sistema se escribe entonces:
% \[
% u=\left( {X\quad L} \right)\;\left( {\begin{array}{l}
%  u_{\ast } \\ 
%  z \\ 
%  \end{array}} \right)\;,\quad donde\quad \left( {X\quad L} \right)=\left( 
% {\begin{array}{l}
%  I_{q\quad \quad \quad \quad } \quad 0 \\ 
%  A_{2} (\theta )\quad \quad A_{1} (\theta ) \\ 
%  \end{array}} \right)
% \]
% Se tiene que det ( X L) $=$ 1, pues (X L) es una matriz triangular inferior 
% cuya diagonal tiene como t\'{e}rminos 1. Se obtiene as\'{\i} la densidad del 
% vector aleatorio $\left( {\begin{array}{l}
%  u_{\ast } \\ 
%  z \\ 
%  \end{array}} \right)$, remplazando en la densidad de u (normal 
% multivariante de orden N$+$q), el vector u por Lz$+$Xu$_{\mathrm{\ast }}$:
% 
% $(2\pi \sigma^{2})^{(N+q)/2}\exp \left[ {-\frac{1}{2\sigma^{2}}(Lz+Xu_{\ast 
% } )'\;(Lz+Xu_{\ast } )} \right]$ (C.1.1)
% 
% Puesto que lo que interesa es calcular la distribuci\'{o}n marginal de z, se 
% lo debe hacer a partir de la distribuci\'{o}n conjunta de $\left( 
% {\begin{array}{l}
%  u_{\ast } \\ 
%  z \\ 
%  \end{array}} \right)$.
% 
% La proyecci\'{o}n ortogonal de --Lz sobre el subespacio vectorial de 
% R$^{\mathrm{N+q}}$, generado por las columnas de X, est\'{a} dada por 
% X$u\limits^{\wedge } {\kern 1pt}_{\ast } $, donde $u\limits^{\wedge } {\kern 
% 1pt}_{\ast } =-(X'\;X)^{-1}{\kern 1pt}{\kern 1pt}X'{\kern 1pt}{\kern 
% 1pt}{\kern 1pt}{\kern 1pt}Lz$; adem\'{a}s, el teorema de Pit\'{a}goras 
% permite escribir:
% \[
% (Lz+Xu_{\ast } )'(Lz+Xu_{\ast } )=(-Xu\limits^{\wedge } {\kern 1pt}_{\ast } 
% +Xu_{\ast } ){\kern 1pt}'{\kern 1pt}(-Xu\limits^{\wedge } {\kern 1pt}_{\ast 
% } +Xu_{\ast } )+(Lz+Xu\limits^{\wedge } {\kern 1pt}_{\ast } 
% )'(Lz+Xu\limits^{\wedge } {\kern 1pt}_{\ast } )
% \]
% \[
% =(u_{\ast } -u\limits^{\wedge } {\kern 1pt}_{\ast } )'X'X(u_{\ast } 
% -u\limits^{\wedge } {\kern 1pt}_{\ast } )+(Lz+Xu\limits^{\wedge } {\kern 
% 1pt}_{\ast } )'(Lz+Xu\limits^{\wedge } {\kern 1pt}_{\ast } )
% \]
% de donde (C.1.1) se puede expresar por:
% 
% $(2\pi \sigma^{2})^{-(N+q)/2}\exp \left\{ {-\frac{1}{2\sigma^{2}}\left[ 
% {(Lz+Xu\limits^{\wedge } {\kern 1pt}_{\ast } )'(Lz+Xu\limits^{\wedge } 
% {\kern 1pt}_{\ast } )+(u_{\ast } -u\limits^{\wedge } {\kern 1pt}_{\ast } 
% )'X'X(u_{\ast } -u\limits^{\wedge } {\kern 1pt}_{\ast } )} \right]} 
% \right\}$ (C.1.2)
% 
% Si se integra con respecto a $u_{\ast } $, se obtiene que la densidad de z 
% es:
% \[
% L=(2\pi \sigma^{2})^{-N/2}\left| {X'X} \right|^{-1/2}\exp \left[ 
% {-\frac{1}{2\sigma^{2}}(Lz+u\limits^{\wedge } {\kern 1pt}_{\ast } 
% )'(Lz+Xu\limits^{\wedge } {\kern 1pt}_{\ast } )} \right]
% \]
% N\'{o}tese que (C.1.2) permite interpretar a $u\limits^{\wedge } {\kern 
% 1pt}_{\ast } $ como $E(u\limits {\kern 1pt}_{\ast } \left| {z)} \right.$.
% 
% Sea S ($\theta )=$ (Lz$+$X$u\limits^{\wedge } {\kern 1pt}_{\ast } 
% )'(Lz+Xu\limits^{\wedge } {\kern 1pt}_{\ast } )$. Consid\'{e}rese entonces:
% 
% $\ell =Ln\;L=-\frac{N}{2}Ln\;(2\pi )-\frac{N}{2}Ln(\sigma 
% ^{2})-\frac{1}{2}Ln\left| {X'X} \right|-\frac{S(\theta )}{2\sigma^{2}}$ 
% (C.1.3)
% 
% Para encontrar los estimadores de m\'{a}xima verosimilitud, se debe 
% maximizar $\ell $ con respecto a $\theta $ y $\sigma^{2} $. Derivando 
% con respecto a $\sigma^{2} $ e igualando a cero se obtiene:
% \[
% \frac{\partial {\kern 1pt}\ell }{\partial \sigma^{2}}=-\frac{N}{2\sigma 
% ^{2}}+\frac{S(\theta )}{2\sigma^{4}}=0\Leftrightarrow \sigma 
% ^{2}=\frac{S(\theta )}{N}
% \]
% Reemplazando esta expresi\'{o}n de $\sigma^{2}$ en (C.1.3) se obtiene:
% \[
% \ell\limits^{\mathunderscore \mathunderscore } =-\frac{N}{2}Ln(2\pi 
% )-\frac{N}{2}Ln\frac{S(\theta )}{N}-\frac{1}{2}Ln\left| {X'X} 
% \right|-\frac{N}{2}
% \]
% Por lo cual se debe minimizar:
% \[
% \ell^{\ast }=N\;Ln\;S(\theta )+Ln\left| {X'X} \right|
% \]
% Existen formulaciones que permiten minimizar $\ell^{\ast }$, con la ayuda 
% de m\'{e}todos num\'{e}ricos (m\'{e}todos llamados exactos). Sin embargo, se 
% puede demostrar que cuando N es suficientemente grande, el t\'{e}rmino Ln 
% $\left| {X'X} \right|$ se vuelve despreciable, por lo cual se puede aplicar 
% la minimizaci\'{o}n \'{u}nicamente a S($\theta )$; estos m\'{e}todos se 
% dicen de m\'{\i}nimos cuadrados (son aproximados).
% 
% Sea $u\limits^{\wedge } =E(u\left| {z)} \right.$. Del hecho que u $=$ 
% Lz$+$X$u_{\ast } $ , se obtiene
% \[
% u\limits^{\wedge } =Lz+Xu\limits^{\wedge } {\kern 1pt}_{\ast } 
% \]
% y entonces:
% \[
% S(\theta )=u\limits^{\wedge } 'u\limits^{\wedge } =\sum\limits_{i=1-q}^N 
% {(u\limits^{\wedge } {\kern 1pt}_{i} )^{2}} 
% \]
% Esta interpretaci\'{o}n de S($\theta )$ permite proponer un c\'{a}lculo 
% simple para $\theta $ dado. Se calcula inicialmente por predicci\'{o}n hacia 
% atr\'{a}s, de la siguiente manera:
% \[
% u\limits^{\wedge } {\kern 1pt}_{1-q} =z\limits^{\wedge } {\kern 1pt}_{1-q} 
% +\theta_{1} u\limits^{\wedge } {\kern 1pt}_{1-q-1} +\cdots +\theta_{q} 
% u\limits^{\wedge } {\kern 1pt}_{1-2q} =z\limits^{\wedge } {\kern 1pt}_{1-q} 
% \quad (u\limits^{\wedge } {\kern 1pt}_{1-q-i} =0\;si\;i\ge 1)
% \]
% $\begin{array}{l}
%  u\limits^{\wedge } {\kern 1pt}_{2-q} =z\limits^{\wedge } {\kern 1pt}_{2-q} 
% +\theta_{1} u\limits^{\wedge } {\kern 1pt}_{{\begin{array}{l}
%  \\ 
%  1-q \\ 
%  \end{array}}} \\ 
%  \vdots \\ 
%  u\limits^{\wedge } {\kern 1pt}_{0} =z\limits^{\wedge } {\kern 1pt}_{0} 
% +\theta_{1} u\limits^{\wedge } {\kern 1pt}_{-1} +\cdots +\theta_{q-1} 
% u\limits^{\wedge } {\kern 1pt}_{1-q} \\ 
%  \end{array}$ (C.1.4)
% 
% As\'{\i}, se tiene que se puede conocer $u_{\ast } $ si se conocen 
% tambi\'{e}n los $z\limits^{\wedge } {\kern 1pt}_{i} =E(z_{i} \left| z 
% \right.)$ para i$=$1 -- q,...,0. El c\'{a}lculo de los $z\limits^{\wedge } 
% {\kern 1pt}_{i} $ se realiza utilizando la representaci\'{o}n en avance del 
% proceso (z$_{\mathrm{t}})$:
% \[
% z_{t} =\Theta \,(F)\varepsilon_{t} 
% \]
% Se requiere \'{u}nicamente utilizar las t\'{e}cnicas de predicci\'{o}n 
% descritas anteriormente, pero tomando de manera inversa el sentido del 
% tiempo. En particular se tiene que:
% \[
% z\limits^{\wedge } {\kern 1pt}_{0} =\sum\limits_{i=1}^\infty {\pi_{i} 
% z\limits^{\wedge } {\kern 1pt}_{i} } \approx \sum\limits_{i=1}^N {\pi_{i} 
% z\limits^{\wedge } {\kern 1pt}_{i} } 
% \]
% Los $\pi_{\mathrm{i}}$ son los mismos que se utilizan para la 
% predicci\'{o}n hacia delante ($\pi_{\mathrm{0}}=$1). Luego se calcula:
% \[
% \begin{array}{l}
%  z\limits^{\wedge } {\kern 1pt}_{-1} =-\sum\limits_{i=1}^\infty {\pi_{i} 
% z\limits^{\wedge } {\kern 1pt}_{-1+i} \approx -} \sum\limits_{i=1}^{N+1} 
% {\pi_{i} z\limits^{\wedge } {\kern 1pt}_{-1+i} } \\ 
%  \vdots \\ 
%  z\limits^{\wedge } {\kern 1pt}_{1-q} =-\sum\limits_{i=1}^\infty {\pi_{i} 
% z\limits^{\wedge } {\kern 1pt}_{1-q+i} } \approx \sum\limits_{i=1}^{N+q-1} 
% {\pi_{i} z\limits^{\wedge } {\kern 1pt}_{1-q+i} } \\ 
%  \end{array}
% \]
% Posteriormente se calculan los $u\limits^{\wedge } {\kern 1pt}_{i} $ de 
% manera ascendente, con la f\'{o}rmula:
% \[
% u\limits^{\wedge } {\kern 1pt}_{i} =z_{i} +\theta_{1} u\limits^{\wedge } 
% {\kern 1pt}_{i-1} +\cdots +\theta\limits^{\wedge } {\kern 1pt}_{q} 
% u\limits^{\wedge } {\kern 1pt}_{i-q} \quad \quad \quad i=1,...,N
% \]
% Finalmente, estimados los $u\limits^{\wedge } {\kern 1pt}_{i} $ y los 
% ${z{\kern 1pt}}\limits^{\wedge }_{i} $, la funci\'{o}n a minimizar 
% solamente tiene por inc\'{o}gnitas a $\theta_{\mathrm{1}}$,..., $\theta 
% _{\mathrm{q}}$, los cuales a su vez se pueden obtener con algoritmos 
% num\'{e}ricos; sin embargo, vale la pena mencionar que para obtener el valor 
% \'{o}ptimo de S($\theta )$ con t\'{e}cnicas num\'{e}ricas, se requiere tener 
% valores iniciales de buena calidad.
% 
% \textbf{C.1.2 Caso de un ARMA (p , q)}
% 
% Se utiliza la representaci\'{o}n MA del proceso:
% \[
% X_{t} =\sum\limits_{i=0}^\infty {\psi_{i} u_{t-i} } \approx 
% \sum\limits_{i=0}^Q {\psi_{i} u_{t-i} } 
% \]
% donde Q debe ser suficientemente grande.
% 
% En el caso precedente $z\limits^{\wedge } {\kern 1pt}_{i} =0,\;para\;i\le 
% -q,$ por lo cual se puede tomar Q como el \'{\i}ndice a partir del cual los 
% $z\limits^{\wedge } {\kern 1pt}_{i} $ se vuelvan despreciables ( por 
% ejemplo, menores en valor absoluto a 10$^{\mathrm{-3}})$. As\'{\i} entonces, 
% para $\varphi '=$ ($\phi_{\mathrm{1}}$,............,$\phi 
% _{\mathrm{p}})$ y $\theta '=$ ($\theta 
% _{\mathrm{1}}$,..........,$\theta_{\mathrm{q}})$ se debe minimizar la 
% siguiente expresi\'{o}n 
% \[
% S(\varphi ,\theta )=\sum\limits_{i=1-Q}^N {(u\limits^{\wedge } {\kern 
% 1pt}_{t} )^{2} } 
% \]
% $puesto\;que\;\Phi (B)z_{t} =\Theta (B)u_{t} \;\;se\;puede\;\exp 
% licitar\;esta\;relaci\'{o}n$as\'{\i}:
% \[
% z_{t} -\varphi_{1} z_{t-1} -\cdots \cdots -\varphi_{p} z_{t-p} =u_{t} 
% -\theta_{1} \mu_{t-1} -\cdots \cdots -\theta_{q} u_{t-q} 
% \]
% lo que implica:
% \[
% \hat{{u}}_{t} =z\limits^{\wedge } {\kern 1pt}_{t} -\varphi_{1} 
% z\limits^{\wedge } {\kern 1pt}_{t-1} -\cdots \cdots -\varphi_{p} 
% z\limits^{\wedge } {\kern 1pt}_{t-p} +\theta_{1} u\limits^{\wedge } {\kern 
% 1pt}_{t-1} +\cdots \cdots +\theta_{q} u\limits^{\wedge } {\kern 1pt}_{t-q} 
% \]
% S($\varphi $, $\theta )$ se puede minimizar con diversos algoritmos 
% num\'{e}ricos.
% 
% \textbf{C.1.3 Valores iniciales }
% 
% Para los valores iniciales de $\varphi_{i} $ se puede utilizar el siguiente 
% sistema:
% \[
% \left( {\begin{array}{l}
%  \rho_{q+1} \\ 
%  \;\vdots \\ 
%  \;\vdots \\ 
%  \rho_{q+p} \\ 
%  \end{array}} \right)=\left( {\begin{array}{l}
%  \rho_{q\quad } \quad \quad \cdots \quad \rho_{q-p+1} \\ 
%  \;\vdots \quad \quad \quad \;\quad \;\;\vdots \\ 
%  \;\vdots \quad \quad \quad \;\quad \;\;\vdots \\ 
%  \rho_{q+p-1} \quad \cdots \quad \;\rho_{q} \\ 
%  \end{array}} \right)\left( {\begin{array}{l}
%  \varphi_{1} \\ 
%  \;\vdots \\ 
%  \;\vdots \\ 
%  \varphi_{p} \\ 
%  \end{array}} \right)
% \]
% Los $\rho_{i} $ se estiman por los $\rho\limits^{\wedge }_{i} $ , por lo 
% cual se obtienen estimadores convergentes de los $\varphi_{i} $.
% 
% Para obtener valores iniciales para los $\theta_{\mathrm{j\thinspace 
% \thinspace }}$se utiliza la noci\'{o}n de autocorrelaciones inversas de 
% Cleveland (1972)$_{\mathrm{,\thinspace }}$denotadas por $\rho_{i} (h)$. Se 
% interpretan como las autocorrelaciones del proceso ARMA ( q, p), en donde se 
% han intercambiado los papeles de $\Phi $ y $\Theta $.
% 
% Se establece un sistema lineal para h $=$ p$+$1,. . . . . . ,p$+$q, a partir 
% de la relaci\'{o}n;
% \[
% \rho_{i} (h)=\sum\limits_{j=1}^q {\theta_{j} \rho_{i} (h-j)} \quad \quad 
% \quad (h>p)
% \]
% $\rho_{i} $(j) se reemplaza por las estimaciones (integradas en varios 
% paquetes estad\'{\i}sticos), lo que permite obtener estimaciones 
% convergentes de los $\theta_{j} $
% 
% Si existe una constante $\theta_{\mathrm{o}}$ en el modelo, se la puede 
% estimar por:
% \[
% \theta\limits^{\wedge } {\kern 1pt}_{o} =\bar{{z}}(1-\varphi 
% \limits^{\wedge }_{1} -\cdots \cdots -\varphi\limits^{\wedge }_{p} )
% \]
% \textbf{Observaci\'{o}n C.1:}
% 
% Se puede demostrar que los estimadores obtenidos por el m\'{e}todo de 
% m\'{a}xima verosimilitud (exacto o aproximado) son asint\'{o}ticamente 
% normales (incluso si los u$_{\mathrm{i}}$ son solamente independientes). La 
% matriz de varianza -- covarianza de los estimadores se estima en el proceso 
% iterativo.
% 
% \section{ANEXO C.2: EVOLUCI\'{O}N DE UN \'{I}NDICE BURS\'{A}TIL EN 141 D\'{I}AS DE 
% OPERACI\'{O}N EN LA BOLSA DE VALORES}
% \label{subsec:mylabel7}
% \begin{table}[H]
% \begin{center}
% \begin{tabular}{|p{50pt}|l|l|l|l|l|l|l|l|}
% \hline
% & 
% \textbf{1}& 
% \textbf{2}& 
% \textbf{3}& 
% \textbf{4}& 
% \textbf{5}& 
% \textbf{6}& 
% \textbf{7}& 
% \textbf{8} \\
% \hline
% \textbf{1}& 
% -25,580& 
% 15,331& 
% 27,197& 
% 18,715& 
% 43,048& 
% 69,853& 
% 75,314& 
% 74,428 \\
% \hline
% \textbf{2}& 
% -21,484& 
% 15,536& 
% 26,828& 
% 21,909& 
% 46,353& 
% 71,655& 
% 80,143& 
% 77,244 \\
% \hline
% \textbf{3}& 
% -15,824& 
% 13,154& 
% 25,475& 
% 25,328& 
% 50,250& 
% 74,599& 
% 82,453& 
% 80,708 \\
% \hline
% \textbf{4}& 
% -10,214& 
% 9,987& 
% 24,797& 
% 26,858& 
% 52,347& 
% 77,571& 
% 83,280& 
% 83,294 \\
% \hline
% \textbf{5}& 
% -5,442& 
% 7,723& 
% 25,378& 
% 26,514& 
% 53,548& 
% 78,648& 
% 83,953& 
% 81,862 \\
% \hline
% \textbf{6}& 
% -2,952& 
% 7,140& 
% 23,219& 
% 25,950& 
% 55,201& 
% 79,183& 
% 82,854& 
% 78,619 \\
% \hline
% \textbf{7}& 
% -0,491& 
% 8,029& 
% 19,259& 
% 26,390& 
% 57,268& 
% 80,945& 
% 80,705& 
% 76,782 \\
% \hline
% \textbf{8}& 
% 1,557& 
% 10,268& 
% 14,621& 
% 26,083& 
% 59,306& 
% 82,095& 
% 78,575& 
% 75,892 \\
% \hline
% \textbf{9}& 
% 0,465& 
% 13,353& 
% 10,096& 
% 25,505& 
% 60,986& 
% 80,984& 
% 75,920& 
% 76,443 \\
% \hline
% \textbf{10}& 
% -1,578& 
% 16,705& 
% 8,073& 
% 26,076& 
% 61,983& 
% 78,238& 
% 73,599& 
% 78,542 \\
% \hline
% \textbf{11}& 
% -3,349& 
% 20,622& 
% 8,723& 
% 26,023& 
% 63,650& 
% 75,162& 
% 71,586& 
% 80,511 \\
% \hline
% \textbf{12}& 
% -4,531& 
% 23,238& 
% 8,590& 
% 26,481& 
% 66,618& 
% 72,432& 
% 69,134& 
% 81,544 \\
% \hline
% \textbf{13}& 
% -3,946& 
% 23,815& 
% 8,989& 
% 29,288& 
% 69,616& 
% 70,352& 
% 66,873& 
% 81,539 \\
% \hline
% \textbf{14}& 
% -2,320& 
% 24,098& 
% 12,317& 
% 33,619& 
% 71,716& 
% 68,430& 
% 66,112& 
% 81,584 \\
% \hline
% \textbf{15}& 
% -0,284& 
% 24,619& 
% 15,792& 
% 36,878& 
% 71,538& 
% 66,120& 
% 65,597& 
% 81,510 \\
% \hline
% \textbf{16}& 
% 2,776& 
% 25,250& 
% 16,662& 
% 37,783& 
% 69,114& 
% 65,233& 
% 65,569& 
% ~ \\
% \hline
% \textbf{17}& 
% 8,138& 
% 27,108& 
% 16,551& 
% 38,366& 
% 67,166& 
% 66,273& 
% 67,576& 
% ~ \\
% \hline
% \textbf{18}& 
% 12,843& 
% 28,269& 
% 16,840& 
% 40,670& 
% 67,551& 
% 69,678& 
% 70,969& 
% ~ \\
% \hline
% \end{tabular}
% \label{tab1}
% \end{center}
% \end{table}
% 
% \section{ANEXO C.3: DATOS DE VENTAS MENSUALES DE UN PRODUCTO}
% \label{subsec:mylabel8}
% \begin{center}
% (En miles de d\'{o}lares)
% \end{center}
% 
% \begin{table}[H]
% \begin{center}
% \begin{tabular}{|p{42pt}|l|l|l|l|l|l|l|l|l|l|}
% \hline
% \textbf{MES}& 
% \textbf{2006}& 
% \textbf{2007}& 
% \textbf{2008}& 
% \textbf{2009}& 
% \textbf{2010}& 
% \textbf{2011}& 
% \textbf{2012}& 
% \textbf{2013}& 
% \textbf{2014}& 
% \textbf{2015} \\
% \hline
% \textbf{1}& 
% 1.490& 
% 1.543& 
% 2.184& 
% 2.279& 
% 2.583& 
% 2.911& 
% 2.972& 
% 2.579& 
% 3.681& 
% 3.899 \\
% \hline
% \textbf{2}& 
% 1.597& 
% 1.529& 
% 2.221& 
% 2.256& 
% 2.303& 
% 2.786& 
% 2.841& 
% 3.213& 
% 3.365& 
% 3.217 \\
% \hline
% \textbf{3}& 
% 908& 
% 1.153& 
% 1.167& 
% 1.289& 
% 1.388& 
% 1.571& 
% 1.674& 
% 1.978& 
% 2.265& 
% 2.148 \\
% \hline
% \textbf{4}& 
% 437& 
% 503& 
% 522& 
% 649& 
% 772& 
% 770& 
% 947& 
% 951& 
% 1.043& 
% 1.134 \\
% \hline
% \textbf{5}& 
% 150& 
% 205& 
% 184& 
% 241& 
% 228& 
% 275& 
% 296& 
% 332& 
% 336& 
% 348 \\
% \hline
% \textbf{6}& 
% 112& 
% 132& 
% 141& 
% 170& 
% 169& 
% 188& 
% 187& 
% 212& 
% 236& 
% 256 \\
% \hline
% \textbf{7}& 
% 87& 
% 89& 
% 106& 
% 115& 
% 131& 
% 152& 
% 160& 
% 162& 
% 177& 
% 180 \\
% \hline
% \textbf{8}& 
% 161& 
% 189& 
% 201& 
% 207& 
% 234& 
% 257& 
% 288& 
% 314& 
% 322& 
% 321 \\
% \hline
% \textbf{9}& 
% 498& 
% 636& 
% 652& 
% 815& 
% 853& 
% 938& 
% 1.015& 
% 993& 
% 1.044& 
% 1.152 \\
% \hline
% \textbf{10}& 
% 1.433& 
% 1.185& 
% 1.543& 
% 1.520& 
% 1.653& 
% 1.694& 
% 2.348& 
% 2.255& 
% 2.670& 
% 2.681 \\
% \hline
% \textbf{11}& 
% 1.793& 
% 2.358& 
% 2.434& 
% 2.825& 
% 2.841& 
% 3.269& 
% 3.790& 
% 3.659& 
% 4.405& 
% 4.511 \\
% \hline
% \textbf{12}& 
% 3.195& 
% 3.258& 
% 3.652& 
% 4.323& 
% 4.422& 
% 4.597& 
% 5.816& 
% 5.909& 
% 6.437& 
% 5.327 \\
% \hline
% \end{tabular}
% \label{tab2}
% \end{center}
% \end{table}
% 
% \section{ANEXO C.4: DATOS DE TEMPERATURA EN RIOBAMBA}
% \label{subsec:mylabel9}
% \begin{table}[H]
% \begin{center}
% \begin{tabular}{|l|p{22pt}|p{21pt}|p{25pt}|p{23pt}|p{25pt}|p{22pt}|p{21pt}|p{24pt}|p{21pt}|p{23pt}|p{24pt}|p{21pt}|p{22pt}|p{21pt}|p{25pt}|p{23pt}|p{25pt}|p{22pt}|p{21pt}|p{24pt}|p{20pt}|p{23pt}|p{24pt}|p{21pt}|p{22pt}|p{21pt}|p{25pt}|p{23pt}|p{25pt}|p{22pt}|p{21pt}|p{24pt}|p{20pt}|p{23pt}|p{24pt}|p{21pt}|}
% \hline
%  & 
% \multicolumn{12}{|p{279pt}|}{\textbf{1997}} & 
% \multicolumn{12}{|p{277pt}|}{\textbf{1998}} & 
% \multicolumn{12}{|p{277pt}|}{\textbf{1999}} \\
% \hline
% \textbf{DIA}& 
% \textbf{ENE}& 
% \textbf{FEB}& 
% \textbf{MAR}& 
% \textbf{ABR}& 
% \textbf{MAY}& 
% \textbf{JUN}& 
% \textbf{JUL}& 
% \textbf{AGO}& 
% \textbf{SEP}& 
% \textbf{OCT}& 
% \textbf{NOV}& 
% \textbf{DIC}& 
% \textbf{ENE}& 
% \textbf{FEB}& 
% \textbf{MAR}& 
% \textbf{ABR}& 
% \textbf{MAY}& 
% \textbf{JUN}& 
% \textbf{JUL}& 
% \textbf{AGO}& 
% \textbf{SEP}& 
% \textbf{OCT}& 
% \textbf{NOV}& 
% \textbf{DIC}& 
% \textbf{ENE}& 
% \textbf{FEB}& 
% \textbf{MAR}& 
% \textbf{ABR}& 
% \textbf{MAY}& 
% \textbf{JUN}& 
% \textbf{JUL}& 
% \textbf{AGO}& 
% \textbf{SEP}& 
% \textbf{OCT}& 
% \textbf{NOV}& 
% \textbf{DIC} \\
% \hline
% \textbf{1}& 
% 10,2& 
% 7& 
% 7,2& 
% 10& 
% 8,2& 
% 7,6& 
% 9& 
% 7,8& 
% 10,4& 
% 4& 
% 7& 
% 10,2& 
% 9,6& 
% 5& 
% 10& 
% 11& 
% 12& 
% 9,9& 
% 6,7& 
% 6,4& 
% 2,4& 
% 2,9& 
% 11& 
% 8,2& 
% 8,2& 
% 8,9& 
% 6,9& 
% 8& 
% 3& 
% 8,4& 
% 8,4& 
% 6,2& 
% 6,6& 
% 9,2& 
% 7,6& 
% 9,1 \\
% \hline
% \textbf{2}& 
% 7,1& 
% 7,8& 
% 9,2& 
% 10,4& 
% 8& 
% 8,7& 
% 6& 
% 6,5& 
% 7,5& 
% 10,4& 
% 11& 
% 11,4& 
% 8,2& 
% 8,5& 
% 12& 
% 10& 
% 12& 
% 11& 
% 8,2& 
% 6,5& 
% 0& 
% 7,5& 
% 11& 
% 1,3& 
% 7,3& 
% 10& 
% 9,1& 
% 7,5& 
% 9,8& 
% 10& 
% 8& 
% 4,5& 
% 8,5& 
% 6,7& 
% 6& 
% 11 \\
% \hline
% \textbf{3}& 
% 9& 
% 9,6& 
% 9& 
% 10,5& 
% 9,5& 
% 10,5& 
% 4,5& 
% 5,6& 
% 9,9& 
% 10,2& 
% 10,2& 
% 10& 
% 9,6& 
% 8,3& 
% 11& 
% 11& 
% 9,2& 
% 9,5& 
% 9,4& 
% 6,5& 
% 9,5& 
% 7,7& 
% 9,2& 
% 1,1& 
% 11& 
% 9,5& 
% 9,9& 
% 11& 
% 9,3& 
% 8& 
% 8,7& 
% 6,4& 
% 9,2& 
% 8,4& 
% 10& 
% 10 \\
% \hline
% \textbf{4}& 
% 10,2& 
% 7,5& 
% 11& 
% 4,3& 
% 10& 
% 9,5& 
% 2,9& 
% 5,9& 
% 3,4& 
% 7,4& 
% 10,8& 
% 6,4& 
% 11& 
% 7,5& 
% 10& 
% 11& 
% 11& 
% 9,4& 
% 9,7& 
% 6,6& 
% 9,4& 
% 3,8& 
% 11& 
% 4,8& 
% 7,9& 
% 10& 
% 9& 
% 11& 
% 5& 
% 8,6& 
% 5,5& 
% 3,8& 
% 6,6& 
% 9& 
% 7,4& 
% 8,2 \\
% \hline
% \textbf{5}& 
% 10,2& 
% 8,2& 
% 10,3& 
% 6,4& 
% 8,3& 
% 4,3& 
% 8,4& 
% 5,5& 
% 5,8& 
% 8,2& 
% 11,3& 
% 9& 
% 9,4& 
% 8,2& 
% 11& 
% 11& 
% 11& 
% 8,5& 
% 8,7& 
% 8,2& 
% 6,3& 
% 11& 
% 12& 
% 5,7& 
% 11& 
% 8,5& 
% 8& 
% 9,5& 
% 11& 
% 8& 
% 4,9& 
% 6,5& 
% 6,3& 
% 7,5& 
% 11& 
% 10 \\
% \hline
% \textbf{6}& 
% 8,1& 
% 8,8& 
% 8,9& 
% 7,5& 
% 9,5& 
% 8,6& 
% 7,8& 
% 9& 
% 6,8& 
% 4,3& 
% 11& 
% 10& 
% 9,4& 
% 8,8& 
% 11& 
% 13& 
% 9,5& 
% 10& 
% 7,5& 
% 7,3& 
% 6,8& 
% 9,5& 
% 11& 
% 6,6& 
% 11& 
% 10& 
% 9,7& 
% 9& 
% 11& 
% 7,6& 
% 6& 
% 6,3& 
% 6,8& 
% 8,5& 
% 10& 
% 10 \\
% \hline
% \textbf{7}& 
% 7& 
% 7,2& 
% 6,5& 
% 10,4& 
% 9,4& 
% 10& 
% 8& 
% 8,5& 
% 6,6& 
% 3& 
% 11,3& 
% 10,3& 
% 12& 
% 7,2& 
% 10& 
% 12& 
% 9,6& 
% 5,6& 
% 7,3& 
% 6,5& 
% 5,2& 
% 7,8& 
% 8,7& 
% 3,4& 
% 11& 
% 8,4& 
% 8,3& 
% 9,4& 
% 9,1& 
% 9,3& 
% 5,9& 
% 6& 
% 4,2& 
% 7& 
% 9,5& 
% 12 \\
% \hline
% \textbf{8}& 
% 8& 
% 9,5& 
% 6,4& 
% 10& 
% 8,9& 
% 9& 
% 8,9& 
% 7,1& 
% 6,6& 
% 9,8& 
% 10,8& 
% 10& 
% 11& 
% 8,6& 
% 11& 
% 11& 
% 11& 
% 9,7& 
% 6,5& 
% 6,2& 
% 4,6& 
% 9,6& 
% 9,7& 
% 4,4& 
% 11& 
% 8& 
% 11& 
% 9,1& 
% 7,8& 
% 8,7& 
% 5& 
% 7,4& 
% 5,2& 
% 5,9& 
% 11& 
% 11 \\
% \hline
% \textbf{9}& 
% 7& 
% 7,6& 
% 10& 
% 9,7& 
% 9& 
% 7,4& 
% 7,3& 
% 6& 
% 5,3& 
% 6& 
% 10,4& 
% 10,2& 
% 9,8& 
% 8,4& 
% 11& 
% 9,1& 
% 12& 
% 7,8& 
% 9,7& 
% 6,8& 
% 6& 
% 8,4& 
% 11& 
% 5,5& 
% 11& 
% 8,1& 
% 8,6& 
% 9,2& 
% 8& 
% 8,4& 
% 4,4& 
% 6,5& 
% 5,2& 
% 8,7& 
% 7,1& 
% 11 \\
% \hline
% \textbf{10}& 
% 8,8& 
% 8,7& 
% 10& 
% 9,8& 
% 10,7& 
% 8,4& 
% 3& 
% 6,2& 
% 8,4& 
% 6,9& 
% 7,5& 
% 10,6& 
% 11& 
% 9,3& 
% 9,4& 
% 8& 
% 11& 
% 7,4& 
% 8,8& 
% 5,8& 
% 8,5& 
% 6,6& 
% 9,9& 
% 5,5& 
% 8,5& 
% 10& 
% 9,2& 
% 8,8& 
% 8,8& 
% 7,5& 
% 1,8& 
% 6& 
% 6,6& 
% 6,5& 
% 10& 
% 9,7 \\
% \hline
% \textbf{11}& 
% 9& 
% 9,8& 
% 10,9& 
% 10,7& 
% 10,2& 
% 10,5& 
% 8,5& 
% 3& 
% 8& 
% 11,8& 
% 9,5& 
% 6,5& 
% 8,7& 
% 8,6& 
% 10& 
% 12& 
% 11& 
% 10& 
% 9,7& 
% 7,3& 
% 6& 
% 8,6& 
% 9,6& 
% 8,8& 
% 9,4& 
% 9,2& 
% 8,4& 
% 9,5& 
% 7,4& 
% 8,3& 
% 2,6& 
% 7,4& 
% 6,7& 
% 7,5& 
% 10& 
% 9,2 \\
% \hline
% \textbf{12}& 
% 10,2& 
% 9,3& 
% 10,5& 
% 11& 
% 9,6& 
% 7,9& 
% 5,7& 
% 9,5& 
% 10,2& 
% 9,6& 
% 10,5& 
% 10,5& 
% 9,4& 
% 7,8& 
% 12& 
% 11& 
% 11& 
% 9,6& 
% 9& 
% 5,8& 
% 7,6& 
% 2,5& 
% 11& 
% 7,5& 
% 6,1& 
% 9,8& 
% 9,7& 
% 10& 
% 7,6& 
% 9,8& 
% 6,7& 
% 6,1& 
% 7,6& 
% 9& 
% 8,6& 
% 8 \\
% \hline
% \textbf{13}& 
% 10& 
% 9,5& 
% 11& 
% 7,7& 
% 9,9& 
% 10& 
% 4,8& 
% 5,4& 
% 6,8& 
% 10,5& 
% 8,9& 
% 11,7& 
% 11& 
% 8,3& 
% 11& 
% 11& 
% 11& 
% 7,7& 
% 7,6& 
% 6,7& 
% 6,8& 
% 2,5& 
% 11& 
% 8,5& 
% 4,6& 
% 8& 
% 8,6& 
% 9,5& 
% 9,5& 
% 9,5& 
% 6,4& 
% 8,3& 
% 6,8& 
% 8,4& 
% 8,4& 
% 11 \\
% \hline
% \textbf{14}& 
% 10,4& 
% 7,5& 
% 10& 
% 9,4& 
% 8,9& 
% 9,5& 
% 3,6& 
% 6& 
% 7& 
% 5,6& 
% 8& 
% 11,8& 
% 8,6& 
% 8,7& 
% 11& 
% 12& 
% 9,2& 
% 7& 
% 8,6& 
% 6,3& 
% 7,8& 
% 0,5& 
% 11& 
% 9,8& 
% 10& 
% 9,3& 
% 9,4& 
% 6,5& 
% 9,1& 
% 7,1& 
% 7& 
% 6,1& 
% 4,4& 
% 3& 
% 6,2& 
% 11 \\
% \hline
% \textbf{15}& 
% 10,1& 
% 10,2& 
% 8,5& 
% 10,4& 
% 9,9& 
% 6,2& 
% 5& 
% 5,4& 
% 7,7& 
% 10,4& 
% 9,7& 
% 11,5& 
% 8,6& 
% 11& 
% 12& 
% 12& 
% 11& 
% 9,2& 
% 6,5& 
% 6& 
% 6,5& 
% 8& 
% 11& 
% 8,7& 
% 3,6& 
% 9,4& 
% 9,4& 
% 10& 
% 10& 
% 6,5& 
% 6,7& 
% 5,5& 
% 7,7& 
% 7,7& 
% 7& 
% 11 \\
% \hline
% \textbf{16}& 
% 9,7& 
% 10,2& 
% 10,2& 
% 10& 
% 8,2& 
% 7,8& 
% 3,8& 
% 6,2& 
% 8& 
% 8,5& 
% 9& 
% 10,9& 
% 10& 
% 11& 
% 12& 
% 12& 
% 7,2& 
% 8,8& 
% 8,4& 
% 7,4& 
% 4,5& 
% 4& 
% 11& 
% 8,9& 
% 6,8& 
% 9,8& 
% 9,8& 
% 7& 
% 8,2& 
% 5,5& 
% 8& 
% 6,2& 
% 8,3& 
% 4,7& 
% 9,4& 
% 10 \\
% \hline
% \textbf{17}& 
% 8& 
% 10,4& 
% 10,6& 
% 7,5& 
% 10& 
% 8,6& 
% 6& 
% 5,8& 
% 8,2& 
% 10,2& 
% 10,8& 
% 11& 
% 11& 
% 10& 
% 11& 
% 12& 
% 11& 
% 5& 
% 8,4& 
% 6,5& 
% 7,4& 
% 6,4& 
% 8& 
% 5,1& 
% 2,2& 
% 7,5& 
% 8,6& 
% 7,4& 
% 9,1& 
% 9,5& 
% 5,2& 
% 2& 
% 7& 
% 4,2& 
% 8,3& 
% 11 \\
% \hline
% \textbf{18}& 
% 8,2& 
% 10,1& 
% 10,5& 
% 10,2& 
% 8& 
% 9,6& 
% 1,2& 
% 5,9& 
% 3,7& 
% 7,4& 
% 10,5& 
% 9,7& 
% 10& 
% 11& 
% 11& 
% 12& 
% 9,4& 
% 8,4& 
% 8,5& 
% 6& 
% 9,7& 
% 5,6& 
% 10& 
% 8,9& 
% 9,6& 
% 9& 
% 9& 
% 8,5& 
% 8,8& 
% 8,2& 
% 8& 
% 2& 
% 8,2& 
% 6,9& 
% 7& 
% 10 \\
% \hline
% \textbf{19}& 
% 10& 
% 10,1& 
% 10& 
% 8& 
% 9& 
% 9,5& 
% 6,4& 
% 4,5& 
% 5& 
% 9,2& 
% 10,2& 
% 10,9& 
% 11& 
% 12& 
% 8& 
% 9,5& 
% 7,9& 
% 9,9& 
% 9& 
% 5,7& 
% 6,8& 
% 10& 
% 8,4& 
% 7& 
% 9,2& 
% 10& 
% 8,7& 
% 9,4& 
% 9,7& 
% 9,6& 
% 7,8& 
% 3,1& 
% 6,8& 
% 8& 
% 6,6& 
% 10 \\
% \hline
% \textbf{20}& 
% 9,7& 
% 5,6& 
% 9,4& 
% 8,6& 
% 8& 
% 8,2& 
% 5& 
% 9,1& 
% 6,8& 
% 11& 
% 10,9& 
% 10,4& 
% 9& 
% 12& 
% 12& 
% 9,5& 
% 8,9& 
% 9,4& 
% 9& 
% 5,8& 
% 6,8& 
% 7,2& 
% 9& 
% 7,6& 
% 9,4& 
% 7,2& 
% 8,9& 
% 8& 
% 8,5& 
% 8,8& 
% 5,1& 
% 2,5& 
% 6,8& 
% 9,6& 
% 7,7& 
% 10 \\
% \hline
% \textbf{21}& 
% 10,5& 
% 10,2& 
% 9,3& 
% 9,7& 
% 8& 
% 2,8& 
% 4,4& 
% 7,5& 
% 7,5& 
% 9,5& 
% 11,2& 
% 10,6& 
% 12& 
% 12& 
% 8,7& 
% 12& 
% 3,5& 
% 9,3& 
% 6& 
% 6,5& 
% 7,7& 
% 9,8& 
% 11& 
% 3& 
% 9,5& 
% 10& 
% 8,6& 
% 7,7& 
% 7,6& 
% 9& 
% 8,5& 
% 6,9& 
% 9& 
% 8& 
% 7& 
% 10 \\
% \hline
% \textbf{22}& 
% 8,4& 
% 10,6& 
% 5,5& 
% 9,3& 
% 8,6& 
% -0,6& 
% 6,8& 
% 1,3& 
% 6,9& 
% 10,4& 
% 10,6& 
% 8,9& 
% 11& 
% 11& 
% 9,6& 
% 11& 
% 5,6& 
% 8,1& 
% 6,9& 
% 7,5& 
% 9,2& 
% 11& 
% 9,5& 
% 1,7& 
% 9,5& 
% 10& 
% 8,4& 
% 9,4& 
% 9,2& 
% 8& 
% 4& 
% 6,2& 
% 8,5& 
% 5,5& 
% 3,2& 
% 10 \\
% \hline
% \textbf{23}& 
% 6,8& 
% 9,5& 
% 4,8& 
% 10,6& 
% 10,3& 
% 6,7& 
% 9,6& 
% 6,5& 
% 11,2& 
% 10& 
% 10,5& 
% 10& 
% 9,4& 
% 11& 
% 9,4& 
% 10& 
% 7,8& 
% 8,4& 
% 7,9& 
% 6,5& 
% 7,2& 
% 9,6& 
% 11& 
% -1& 
% 9,8& 
% 10& 
% 7,7& 
% 9& 
% 9,6& 
% 8,1& 
% 9& 
% 6,5& 
% 6,6& 
% 9,8& 
% 9,5& 
% 8 \\
% \hline
% \textbf{24}& 
% 10,3& 
% 9,9& 
% 9,5& 
% 9,4& 
% 11,2& 
% 5,9& 
% 9,6& 
% 5,4& 
% 9,6& 
% 11& 
% 10,5& 
% 7& 
% 9,7& 
% 11& 
% 12& 
% 12& 
% 5,6& 
% 9,5& 
% 7,3& 
% 5,7& 
% 7,3& 
% 11& 
% 7,8& 
% 8,6& 
% 9,7& 
% 10& 
% 9,5& 
% 8,4& 
% 10& 
% 5,7& 
% 5,6& 
% 5,6& 
% 4,4& 
% 9,8& 
% 6,1& 
% 9,4 \\
% \hline
% \textbf{25}& 
% 9,5& 
% 9& 
% 9,3& 
% 9,3& 
% 5,8& 
% 7,9& 
% 8,6& 
% 5,9& 
% 8& 
% 9,5& 
% 11& 
% 8,4& 
% 10& 
% 12& 
% 9,3& 
% 12& 
% 11& 
% 5,1& 
% 8& 
% 7& 
% 4& 
% 11& 
% 3,9& 
% 8,8& 
% 9,7& 
% 10& 
% 11& 
% 9,6& 
% 8,5& 
% 6,8& 
% 6,2& 
% 8,8& 
% 8,5& 
% 9,4& 
% 5,9& 
% 4,7 \\
% \hline
% \textbf{26}& 
% 10& 
% 6& 
% 7,5& 
% 6,8& 
% 9,8& 
% 9& 
% 7,4& 
% 6,8& 
% 10,2& 
% 11,8& 
% 10,6& 
% 10,6& 
% 8,3& 
% 9& 
% 10& 
% 9,6& 
% 11& 
% 9,7& 
% 5,7& 
% 6& 
% 8,1& 
% 10& 
% 2& 
% 4,2& 
% 10& 
% 11& 
% 10& 
% 8,9& 
% 7,4& 
% 6& 
% 7,8& 
% 9,4& 
% 9,5& 
% 10& 
% 5,4& 
% 5,2 \\
% \hline
% \textbf{27}& 
% 10,6& 
% 10& 
% 9,5& 
% 6,8& 
% 10,9& 
% 2,7& 
% 5,8& 
% 8,6& 
% 7,2& 
% 10& 
% 10,2& 
% 10,2& 
% 11& 
% 10& 
% 11& 
% 9,4& 
% 11& 
% 9,6& 
% 8,3& 
% 6,3& 
% 7,2& 
% 6,4& 
% 4,9& 
% 9,7& 
% 10& 
% 8,2& 
% 8,8& 
% 9,7& 
% 9,1& 
% 10& 
% 6,4& 
% 8,2& 
% 6,5& 
% 11& 
% 7,5& 
% 4,7 \\
% \hline
% \textbf{28}& 
% 6,9& 
% 10,5& 
% 5& 
% 4,9& 
% 9,7& 
% 8& 
% 3& 
% 7,5& 
% 7,4& 
% 8,5& 
% 10,6& 
% 10,5& 
% 11& 
% 9& 
% 11& 
% 11& 
% 11& 
% 8,4& 
% 7& 
% 6& 
% 7,6& 
% 7,2& 
% 2,8& 
% 6& 
% 9,2& 
% 9,8& 
% 10& 
% 8,5& 
% 9,5& 
% 8,2& 
% 8& 
% 6& 
% 8,2& 
% 5,9& 
% 8,2& 
% 8,8 \\
% \hline
% \textbf{29}& 
% 4,5& 
% ~& 
% 6,8& 
% 10& 
% 8,7& 
% 4,3& 
% 7& 
% 9& 
% 8& 
% 10,5& 
% 10,3& 
% 10,8& 
% 8,8& 
% ~& 
% 11& 
% 11& 
% 10& 
% 8,2& 
% 5,3& 
% 6,2& 
% 6,5& 
% 9,5& 
% 6,7& 
% 11& 
% 8,5& 
% ~& 
% 11& 
% 9,4& 
% 9,4& 
% 8& 
% 7,9& 
% 6,5& 
% 9,6& 
% 6,3& 
% 9& 
% 10 \\
% \hline
% \textbf{30}& 
% 6,3& 
% ~& 
% 10,7& 
% 9,1& 
% 10,4& 
% 8,7& 
% 7,5& 
% 3,6& 
% 7,2& 
% 10,4& 
% 9,8& 
% 10,5& 
% 11& 
% ~& 
% 10& 
% 9& 
% 9,2& 
% 7& 
% 5& 
% 6,9& 
% 9,6& 
% 10& 
% 7,5& 
% 10& 
% 6,7& 
% ~& 
% 9,5& 
% 9,7& 
% 5,7& 
% 8& 
% 2,6& 
% 6& 
% 7,9& 
% 9,7& 
% 10& 
% 8 \\
% \hline
% \textbf{31}& 
% 8& 
% ~& 
% 10,6& 
% ~& 
% 10,3& 
% ~& 
% 8& 
% 5,8& 
% ~& 
% 9,8& 
% ~& 
% 9,4& 
% 7,3& 
% ~& 
% 12& 
% ~& 
% 10& 
% ~& 
% 6& 
% 6& 
% ~& 
% 11& 
% ~& 
% 11& 
% 9,2& 
% ~& 
% 9& 
% ~& 
% 8,4& 
% ~& 
% 6,7& 
% 5,3& 
% ~& 
% 5,3& 
% ~& 
% 10 \\
% \hline
% \end{tabular}
% \label{tab3}
% \end{center}
% \end{table}
% 
% \chapter{ANEXO D ANEXO ASOCIADO AL CAP\'{I}TULO 5}
% \label{sec:mylabel4}
% \section{ANEXO D.1 DATOS MENSUALES DE SERIES ECON\'{O}MICAS DE UN PA\'{I}S 
% SUDAMERICANO Y SUS RETORNOS}
% \label{subsec:mylabel10}
% producto interno bruto (PIB), denotada por $\left( X_{1t} \right)$; el 
% consumo interno (CI), denotada por $\left( X_{2t} \right)$ y la demanda 
% final interna (DFI), denotada por $\left( X_{3t} \right)$
% 
% \begin{table}[H]
% \begin{center}
% \begin{tabular}{|p{60pt}|l|l|l|l|l|l|}
% \hline
% PIB($X_{1t})$& 
% CI($X_{2t})$& 
% DFI($X_{3t})$& 
% & 
% Ret1 ($Y_{1t})$& 
% Ret2 ($Y_{2t})$& 
% Ret3 ($Y_{3t})$ \\
% \hline
% 49,789& 
% 43,052& 
% 50,514& 
% & 
% -& 
% -& 
% - \\
% \hline
% 51,356& 
% 44,6& 
% 52,101& 
% & 
% 0,0315& 
% 0,0360& 
% 0,0314 \\
% \hline
% 53,659& 
% 45,416& 
% 53,537& 
% & 
% 0,0448& 
% 0,0183& 
% 0,0276 \\
% \hline
% 53,727& 
% 46,146& 
% 54,124& 
% & 
% 0,0013& 
% 0,0161& 
% 0,0110 \\
% \hline
% 54,668& 
% 47,591& 
% 56,663& 
% & 
% 0,0175& 
% 0,0313& 
% 0,0469 \\
% \hline
% 56,068& 
% 48,996& 
% 58,194& 
% & 
% 0,0256& 
% 0,0295& 
% 0,0270 \\
% \hline
% 57,465& 
% 49,581& 
% 58,591& 
% & 
% 0,0249& 
% 0,0119& 
% 0,0068 \\
% \hline
% 57,65& 
% 50,536& 
% 59,465& 
% & 
% 0,0032& 
% 0,0193& 
% 0,0149 \\
% \hline
% 57,648& 
% 51,142& 
% 59,412& 
% & 
% -0,3469& 
% 0,0120& 
% -0,0009 \\
% \hline
% 60,739& 
% 52,261& 
% 62,754& 
% & 
% 0,0536& 
% 0,0219& 
% 0,0563 \\
% \hline
% 63,082& 
% 53,555& 
% 65,145& 
% & 
% 0,0386& 
% 0,0248& 
% 0,0381 \\
% \hline
% 63,854& 
% 54,265& 
% 66,234& 
% & 
% 0,0122& 
% 0,0133& 
% 0,0167 \\
% \hline
% 65,584& 
% 55,165& 
% 68,694& 
% & 
% 0,0271& 
% 0,0166& 
% 0,0371 \\
% \hline
% 66,939& 
% 57,153& 
% 69,244& 
% & 
% 0,0207& 
% 0,0360& 
% 0,0080 \\
% \hline
% 69,157& 
% 58,389& 
% 70,57& 
% & 
% 0,0331& 
% 0,0216& 
% 0,0191 \\
% \hline
% 70,665& 
% 59,925& 
% 73,003& 
% & 
% 0,0218& 
% 0,0263& 
% 0,0345 \\
% \hline
% 73,615& 
% 61,873& 
% 76,152& 
% & 
% 0,0417& 
% 0,0325& 
% 0,0431 \\
% \hline
% 75,169& 
% 62,852& 
% 76,902& 
% & 
% 0,0211& 
% 0,0158& 
% 0,0098 \\
% \hline
% 78,013& 
% 64,196& 
% 79,232& 
% & 
% 0,0378& 
% 0,0214& 
% 0,0303 \\
% \hline
% 80,965& 
% 66,155& 
% 82,622& 
% & 
% 0,0378& 
% 0,0305& 
% 0,0428 \\
% \hline
% 81,368& 
% 67,095& 
% 83,284& 
% & 
% 0,0050& 
% 0,0142& 
% 0,0080 \\
% \hline
% 84,117& 
% 69,144& 
% 85,645& 
% & 
% 0,0338& 
% 0,0305& 
% 0,0283 \\
% \hline
% 87,078& 
% 70,225& 
% 88,845& 
% & 
% 0,0352& 
% 0,0156& 
% 0,0374 \\
% \hline
% 87,593& 
% 71,489& 
% 88,962& 
% & 
% 0,0059& 
% 0,0180& 
% 0,0013 \\
% \hline
% 89,792& 
% 73,724& 
% 92,341& 
% & 
% 0,0251& 
% 0,0313& 
% 0,0380 \\
% \hline
% 94,831& 
% 76,28& 
% 97,514& 
% & 
% 0,0561& 
% 0,0347& 
% 0,0560 \\
% \hline
% 98,877& 
% 78,738& 
% 101,994& 
% & 
% 0,0427& 
% 0,0322& 
% 0,0459 \\
% \hline
% 100,398& 
% 80,28& 
% 103,178& 
% & 
% 0,0154& 
% 0,0196& 
% 0,0116 \\
% \hline
% 103,369& 
% 83,004& 
% 108,791& 
% & 
% 0,0296& 
% 0,0339& 
% 0,0544 \\
% \hline
% 105,182& 
% 85,463& 
% 108,448& 
% & 
% 0,0175& 
% 0,0296& 
% -0,0032 \\
% \hline
% 108,961& 
% 87,826& 
% 112,744& 
% & 
% 0,0359& 
% 0,0276& 
% 0,0396 \\
% \hline
% 113,56& 
% 89,532& 
% 115,105& 
% & 
% 0,0422& 
% 0,0194& 
% 0,0209 \\
% \hline
% 115,231& 
% 91,95& 
% 118,332& 
% & 
% 0,0147& 
% 0,0270& 
% 0,0280 \\
% \hline
% 119,382& 
% 94,137& 
% 121,345& 
% & 
% 0,0360& 
% 0,0238& 
% 0,0255 \\
% \hline
% 123,361& 
% 95,874& 
% 126,404& 
% & 
% 0,0333& 
% 0,0185& 
% 0,0417 \\
% \hline
% 122,113& 
% 97,24& 
% 125,879& 
% & 
% -0,0101& 
% 0,0142& 
% -0,0042 \\
% \hline
% 122,476& 
% 98,605& 
% 126,889& 
% & 
% 0,0030& 
% 0,0140& 
% 0,0080 \\
% \hline
% 125,945& 
% 100,377& 
% 128,092& 
% & 
% 0,0283& 
% 0,0180& 
% 0,0095 \\
% \hline
% 126,949& 
% 101,206& 
% 129,949& 
% & 
% 0,0080& 
% 0,0083& 
% 0,0145 \\
% \hline
% 129,277& 
% 102,411& 
% 130,912& 
% & 
% 0,0183& 
% 0,0119& 
% 0,0074 \\
% \hline
% 131,651& 
% 105,073& 
% 133,456& 
% & 
% 0,0184& 
% 0,0260& 
% 0,0194 \\
% \hline
% 135,177& 
% 107,581& 
% 136,635& 
% & 
% 0,0268& 
% 0,0239& 
% 0,0238 \\
% \hline
% 135,643& 
% 109,688& 
% 139,729& 
% & 
% 0,0034& 
% 0,0196& 
% 0,0226 \\
% \hline
% 142,453& 
% 111,977& 
% 145,07& 
% & 
% 0,0502& 
% 0,0209& 
% 0,0382 \\
% \hline
% 147,287& 
% 114,583& 
% 149,87& 
% & 
% 0,0339& 
% 0,0233& 
% 0,0331 \\
% \hline
% 152,949& 
% 119,293& 
% 154,553& 
% & 
% 0,0384& 
% 0,0411& 
% 0,0312 \\
% \hline
% 156,681& 
% 121,621& 
% 159,414& 
% & 
% 0,0244& 
% 0,0195& 
% 0,0315 \\
% \hline
% 162,977& 
% 123,807& 
% 163,475& 
% & 
% 0,0402& 
% 0,0180& 
% 0,0255 \\
% \hline
% 165,311& 
% 126,013& 
% 166,655& 
% & 
% 0,0143& 
% 0,0178& 
% 0,0195 \\
% \hline
% 165,566& 
% 128,116& 
% 169,015& 
% & 
% 0,0015& 
% 0,0167& 
% 0,0142 \\
% \hline
% 165,071& 
% 130,579& 
% 169,377& 
% & 
% -0,0030& 
% 0,0192& 
% 0,0021 \\
% \hline
% 169,493& 
% 132,612& 
% 171,449& 
% & 
% 0,0268& 
% 0,0156& 
% 0,0122 \\
% \hline
% 172& 
% 133,298& 
% 175,665& 
% & 
% 0,0148& 
% 0,0052& 
% 0,0246 \\
% \hline
% 175,099& 
% 136,343& 
% 177,327& 
% & 
% 0,0180& 
% 0,0228& 
% 0,0095 \\
% \hline
% 178,819& 
% 138,931& 
% 184,029& 
% & 
% 0,0212& 
% 0,0190& 
% 0,0378 \\
% \hline
% 180,759& 
% 140,683& 
% 183,299& 
% & 
% 0,0108& 
% 0,0126& 
% -0,0040 \\
% \hline
% \end{tabular}
% \label{tab4}
% \end{center}
% \end{table}
% 
% \section{ANEXO D.2 DATOS DE VARIACIONES DE SERIES ECON\'{O}MICAS DEL ECUADOR}
% \label{subsec:mylabel11}
% \begin{table}[H]
% \begin{center}
% \begin{tabular}{|p{41pt}|l|l|l|l|l|l|l|l|l|l|}
% \hline
% & 
% IAE& 
% IPP& 
% & 
% & 
% IAE& 
% IPP& 
% & 
% & 
% IAE& 
% IPP \\
% \hline
% ene-04& 
% 3,04& 
% 5,82& 
% & 
% feb-08& 
% -6,34& 
% 1,9& 
% & 
% abr-12& 
% -2,25& 
% -4,02 \\
% \hline
% feb-04& 
% -1,84& 
% -0,17& 
% & 
% mar-08& 
% 0,75& 
% 4,75& 
% & 
% may-12& 
% 2,04& 
% -2,05 \\
% \hline
% mar-04& 
% -5,79& 
% -9,73& 
% & 
% abr-08& 
% 1,38& 
% -0,72& 
% & 
% jun-12& 
% -2,9& 
% 1,3 \\
% \hline
% abr-04& 
% -4,35& 
% -1,89& 
% & 
% may-08& 
% -0,63& 
% 0,98& 
% & 
% jul-12& 
% 0,2& 
% -3,65 \\
% \hline
% may-04& 
% 5,58& 
% 1,69& 
% & 
% jun-08& 
% 8,41& 
% 0,5& 
% & 
% ago-12& 
% -1,65& 
% 3,45 \\
% \hline
% jun-04& 
% -1,19& 
% 1,83& 
% & 
% jul-08& 
% 0,42& 
% -0,19& 
% & 
% sep-12& 
% -3,74& 
% 1,61 \\
% \hline
% jul-04& 
% -5,95& 
% -2,18& 
% & 
% ago-08& 
% -2,51& 
% 0,89& 
% & 
% oct-12& 
% 2,85& 
% 7,09 \\
% \hline
% ago-04& 
% 0,1& 
% -2,93& 
% & 
% sep-08& 
% 1,6& 
% 4,91& 
% & 
% nov-12& 
% -1,83& 
% -6,59 \\
% \hline
% sep-04& 
% -4,76& 
% 3,26& 
% & 
% oct-08& 
% 9,07& 
% 4,98& 
% & 
% dic-12& 
% -2,46& 
% 0,37 \\
% \hline
% oct-04& 
% 5,03& 
% 1,17& 
% & 
% nov-08& 
% -2,48& 
% -1,59& 
% & 
% ene-13& 
% 5,33& 
% 1,43 \\
% \hline
% nov-04& 
% 4,85& 
% 1,38& 
% & 
% dic-08& 
% 11,28& 
% 0,03& 
% & 
% feb-13& 
% -1,07& 
% 4,44 \\
% \hline
% dic-04& 
% 3,74& 
% 4,15& 
% & 
% ene-09& 
% -1,93& 
% 4,18& 
% & 
% mar-13& 
% 1,06& 
% 0,26 \\
% \hline
% ene-05& 
% 1,66& 
% 1,64& 
% & 
% feb-09& 
% -2,34& 
% 6,87& 
% & 
% abr-13& 
% 3,31& 
% -3,87 \\
% \hline
% feb-05& 
% 4,24& 
% 0,31& 
% & 
% mar-09& 
% -4,65& 
% 3,41& 
% & 
% may-13& 
% 0,46& 
% -8,24 \\
% \hline
% mar-05& 
% -4,09& 
% -0,02& 
% & 
% abr-09& 
% -11,45& 
% 7,4& 
% & 
% jun-13& 
% -0,93& 
% 2,19 \\
% \hline
% abr-05& 
% -1,67& 
% 6,56& 
% & 
% may-09& 
% 11,76& 
% 6,11& 
% & 
% jul-13& 
% 1,79& 
% 2,9 \\
% \hline
% may-05& 
% 3,18& 
% -3,9& 
% & 
% jun-09& 
% -4,63& 
% 1,66& 
% & 
% ago-13& 
% -4,38& 
% 2,44 \\
% \hline
% jun-05& 
% -6,6& 
% -0,44& 
% & 
% jul-09& 
% -7,12& 
% -11,29& 
% & 
% sep-13& 
% -0,3& 
% -1,88 \\
% \hline
% jul-05& 
% 1,11& 
% 6,48& 
% & 
% ago-09& 
% -9,3& 
% -7,48& 
% & 
% oct-13& 
% 4,03& 
% -3,37 \\
% \hline
% ago-05& 
% -3,23& 
% -4,81& 
% & 
% sep-09& 
% 2,77& 
% -13,54& 
% & 
% nov-13& 
% -4,43& 
% 0,6 \\
% \hline
% sep-05& 
% -1,09& 
% 8,05& 
% & 
% oct-09& 
% -5,66& 
% -14,86& 
% & 
% dic-13& 
% 0,13& 
% 4,14 \\
% \hline
% oct-05& 
% 1,1& 
% -8,24& 
% & 
% nov-09& 
% -1,07& 
% -11,11& 
% & 
% ene-14& 
% 2,37& 
% 1,17 \\
% \hline
% nov-05& 
% 4,97& 
% -3,09& 
% & 
% dic-09& 
% -5,78& 
% -3,45& 
% & 
% feb-14& 
% -3,12& 
% -1,34 \\
% \hline
% dic-05& 
% 4,04& 
% 1,34& 
% & 
% ene-10& 
% 0,99& 
% 1,84& 
% & 
% mar-14& 
% 5,01& 
% -1,01 \\
% \hline
% ene-06& 
% -0,24& 
% 4,21& 
% & 
% feb-10& 
% 3,07& 
% 15,17& 
% & 
% abr-14& 
% -0,13& 
% 1,16 \\
% \hline
% feb-06& 
% 0,96& 
% 11,54& 
% & 
% mar-10& 
% -1,67& 
% 0,22& 
% & 
% may-14& 
% 0,83& 
% 0,23 \\
% \hline
% mar-06& 
% -3,35& 
% -5,22& 
% & 
% abr-10& 
% 15,18& 
% 5,05& 
% & 
% jun-14& 
% 4,71& 
% 3,26 \\
% \hline
% abr-06& 
% -1,89& 
% -1,35& 
% & 
% may-10& 
% 0,8& 
% 12& 
% & 
% jul-14& 
% -4,02& 
% -1,15 \\
% \hline
% may-06& 
% 5,39& 
% 6,79& 
% & 
% jun-10& 
% 2,02& 
% -7,23& 
% & 
% ago-14& 
% 0,59& 
% 2,01 \\
% \hline
% jun-06& 
% -4,54& 
% 4,83& 
% & 
% jul-10& 
% -3,67& 
% 5,44& 
% & 
% sep-14& 
% 0,3& 
% -3,42 \\
% \hline
% jul-06& 
% 1,02& 
% 6,97& 
% & 
% ago-10& 
% 6,35& 
% 1,14& 
% & 
% oct-14& 
% 0,07& 
% -4,96 \\
% \hline
% ago-06& 
% -2,03& 
% -0,21& 
% & 
% sep-10& 
% -0,37& 
% 3,19& 
% & 
% nov-14& 
% -9,23& 
% 4,15 \\
% \hline
% sep-06& 
% -4,77& 
% -1,36& 
% & 
% oct-10& 
% -1,18& 
% 2,22& 
% & 
% dic-14& 
% 2,49& 
% -0,41 \\
% \hline
% oct-06& 
% 3,91& 
% -8,8& 
% & 
% nov-10& 
% 5,45& 
% -4,53& 
% & 
% ene-15& 
% 1,4& 
% 1,52 \\
% \hline
% nov-06& 
% -2,28& 
% 1,16& 
% & 
% dic-10& 
% -1,98& 
% 5,84& 
% & 
% feb-15& 
% -0,21& 
% -0,24 \\
% \hline
% dic-06& 
% 9,34& 
% 4,19& 
% & 
% ene-11& 
% 6,17& 
% -0,92& 
% & 
% mar-15& 
% 0,97& 
% 1,01 \\
% \hline
% ene-07& 
% -3,22& 
% -2,19& 
% & 
% feb-11& 
% -1,18& 
% 3,08& 
% & 
% abr-15& 
% -0,89& 
% -0,38 \\
% \hline
% feb-07& 
% 7,94& 
% 3,81& 
% & 
% mar-11& 
% -3,37& 
% 0,49& 
% & 
% may-15& 
% 6,52& 
% 0,14 \\
% \hline
% mar-07& 
% -2,09& 
% 6,49& 
% & 
% abr-11& 
% 3,41& 
% -7,66& 
% & 
% jun-15& 
% -3,33& 
% 0,9 \\
% \hline
% abr-07& 
% -2,01& 
% 2,98& 
% & 
% may-11& 
% 0,73& 
% 5,11& 
% & 
% & 
% & 
%  \\
% \hline
% may-07& 
% 0,96& 
% 0,06& 
% & 
% jun-11& 
% -3,8& 
% 1,55& 
% & 
% & 
% & 
%  \\
% \hline
% jun-07& 
% -2,56& 
% 5,99& 
% & 
% jul-11& 
% -2& 
% -0,46& 
% & 
% & 
% & 
%  \\
% \hline
% jul-07& 
% 2,21& 
% -4,25& 
% & 
% ago-11& 
% 1,04& 
% -0,79& 
% & 
% & 
% & 
%  \\
% \hline
% ago-07& 
% -4,63& 
% -7,03& 
% & 
% sep-11& 
% -1,07& 
% 4,8& 
% & 
% & 
% & 
%  \\
% \hline
% sep-07& 
% -6,77& 
% -4,32& 
% & 
% oct-11& 
% -3,52& 
% 2,95& 
% & 
% & 
% & 
%  \\
% \hline
% oct-07& 
% 1,95& 
% -1,24& 
% & 
% nov-11& 
% 6,91& 
% 2,25& 
% & 
% & 
% & 
%  \\
% \hline
% nov-07& 
% 0,1& 
% 3,78& 
% & 
% dic-11& 
% -2,52& 
% 1,88& 
% & 
% & 
% & 
%  \\
% \hline
% dic-07& 
% 7,28& 
% -6,16& 
% & 
% ene-12& 
% 6,61& 
% -1,91& 
% & 
% & 
% & 
%  \\
% \hline
% ene-08& 
% -1,65& 
% 7,25& 
% & 
% feb-12& 
% -2,25& 
% 9,99& 
% & 
% & 
% & 
%  \\
% \hline
% \end{tabular}
% \label{tab5}
% \end{center}
% \end{table}
% 
% \chapter{\'{I}NDICE ALFAB\'{E}TICO}
% \label{sec:mylabel1}
% \begin{center}
% 
% \printindex
% 
% \end{center}
% 
% \end{document}
