\chapter{Modelos Multivariantes de Series Temporales}
\label{sec:mylabel2}

Una serie temporal multivariante es un proceso estoc\'{a}stico ${(X_{t})}_{t\in Z}$, con $X_{t}$ un vector donde cada componente se define como una serie temporal univariante. En este documento se utilizar\'{a} la notaci\'{o}n de vectores como columnas. As\'{\i} se denota:
\[
X_{t}=(X_{1t}, \ldots, X_{kt})^{'} \text{ el vector de $k$ series univariantes en el instante $t$}
\]
Lo importante de tratar series multivariantes es que, a m\'{a}s de considerar simult\'{a}neamente observaciones de dos o m\'{a}s series univariantes, tambi\'{e}n se puede analizar las correlaciones existentes entre ellas; esto evidentemente enriquece el an\'{a}lisis, aunque los procesos operativos ser\'{a}n m\'{a}s complejos que en el caso univariante.

\section{Procesos Estacionarios}
\label{subsec:mylabel5}

Para poder estimar las caracter\'{i}sticas de los procesos se necesita suponer que son estables a lo largo del tiempo; esto implica, que son estacionarios. 

\subsection{Proceso estrictamente estacionario}
\label{subsubsec:mylabel1}

Un proceso estoc\'{a}stico multivariado ${(X_{t})}_{t\in Z}$, con $X_{t}=(X_{1t},\ldots ,X_{kt})'$, es estrictamente estacionario (o fuertemente estacionario) si las distribuciones conjuntas de cualquier 
conjunto finito de variables se mantienen por saltos. 

Es decir, si:
\[
F_{t_{1+l},\ldots, t_{k+l}}\left( x_{t_{1}+l},\ldots, x_{t_{k+l}} \right)=
F_{t_{1},\ldots ,t_{k}}\left( x_{t_{1}},\ldots, x_{t_{k}} \right)
\]
Para todo $k\in N$ y para todo $t_{1},\ldots , t_{k}$, $l\in Z$ 

Donde, $F_{t_{1},\ldots ,t_{k}}$ denota la distribuci\'{o}n conjunta de $X_{t_{1}},\ldots , X_{t_{k}}$.

\subsection{Proceso d\'{e}bilmente estacionario}
\label{subsubsec:mylabel2}

Un proceso estoc\'{a}stico multivariado ${(X_{t})}_{t\in Z}$, con $X_{t}=(X_{1t},\ldots, X_{kt})'$, se dice que es \textit{d\'{e}bilmente estacionario} si sus momentos de primer y segundo orden son invariantes en el tiempo (no dependen de t); es decir:

\begin{enumerate}
\item $E\left( X_{t} \right)=\mu \quad \forall t$\quad (el vector media es constante).
\item $Cov\left( X_{t}, X_{t-l} \right)=E\left[ \left( X_{t}-\mu \right)\left( X_{t-l}-\mu \right)' \right]=\Gamma_{l} \quad \forall t;$ es decir, la matriz de \textit{covarianzas cruzadas} entre $X_{t}$ y $X_{t-l}$ es independiente de t (solo depende del salto $l)$.
\end{enumerate}

La media $\mu$ es un vector $k-$dimensional compuesto por las esperanzas de las componentes de $X_{t}$. La matriz de covarianzas cruzadas es de orden $k* k$.

El i-\'{e}simo elemento de la diagonal de $\Gamma_{0}$ es la varianza de $X_{it}$; mientras que, el elemento $(i,j)$ de $\Gamma_{0}$ es la covarianza entre $X_{it}$ y $X_{jt}$. El elemento $(i,j)$ de $\Gamma_{l}$ es la covarianza entre $X_{it}$ y $X_{j,t-l}$. 

\begin{observacion}
Se puede demostrar que si un proceso ${(X_{t})}_{t\in Z}$ es d\'{e}bilmente estacionario entonces 
tambi\'{e}n lo ser\'{a} cada una de sus componentes.
\end{observacion}


\section{Matrices de Correlaci\'{o}n Cruzada (Cross-Correlation)}
\label{subsec:mylabel6}

En lo que sigue se considera que ${(X_{t})}_{t\in Z}$ es estacionaria.\newline

Sea D una matriz diagonal de orden $k*k$ compuesta por las desviaciones est\'{a}ndar de $X_{it}$ para $i=1,\ldots, k$, que se denota por: $D=diag\left\{ \sqrt{\Gamma_{11}(0)},\ldots ,\sqrt{\Gamma_{kk}(0)} \right\}$. La matriz de correlaciones cruzadas \index{Matriz!Correlaciones cruzadas} de $X_{t}$ se define como:

\[
\rho_{0}\equiv \left[ \rho_{ij}(0) \right]=D^{-1}\Gamma_{0}D^{-1}
\]

De manera particular, el elemento (i, j) de $\rho_{0}$ es:
\[
\rho_{ij}(0)=\frac{\Gamma_{ij}(0)}{\sqrt 
{\Gamma_{ii}(0) \Gamma_{jj}(0)}}=\frac{Cov(X_{it}, X_{jt})}{de(X_{it}) de(X_{jt})},
\]
donde, $de\left( . \right)$ es la desviaci\'{o}n est\'{a}ndar.\newline

$\rho_{ij}(0)$ es el coeficiente de correlaci\'{o}n lineal entre $X_{it}$ y $X_{jt}$. En el an\'{a}lisis de series de tiempo, dicho coeficiente se conoce como de concurrencia (en el mismo instante). Es f\'{a}cil ver que:

\begin{enumerate}
\item[i)] $\rho_{ij}\left( 0 \right)=\rho_{ji}\left( 0 \right)$
\item[ii)] $-1\leq \rho_{ij}\left( 0 \right)\leq 1$ 
\item[iii)] $\rho_{ii}\left( 0 \right)=1$
\end{enumerate}

As\'{i}, $\rho \left( 0 \right)$ es una matriz sim\'{e}trica con 1 en la diagonal.\newline

Hay que mencionar que las matrices $\Gamma_{l}$ contienen las \textbf{relaciones en retardo}\index{Relaciones en retardo} entre las componentes de las series. Por lo tanto, las matrices de correlaci\'{o}n cruzada se utilizan para medir la fuerza de la dependencia lineal entre las series de tiempo. 

La matriz de correlaci\'{o}n cruzada de $X_{t}$ con $X_{t-l}$ se define como:
\[
\rho_{l}\equiv \left[ \rho_{ij}(l) \right]=D^{-1}\Gamma_{l}D^{-1}
\]
donde, $D$ es la matriz diagonal de las desviaciones est\'{a}ndar de las series individuales. De la definici\'{o}n se tiene:
\[
\rho_{ij}\left( l \right)=\frac{\Gamma_{ij}(l)}{\sqrt{\Gamma_{ii}\left( 0 \right)\Gamma_{jj}(0)}}=\frac{Cov\left(X_{it}, X_{j,t-l} \right)}{de\left(X_{it} \right) de(X_{jt})}=\frac{Cov\left( X_{it},
X_{jt-l} \right)}{de\left(X_{it} \right)de(X_{jt-l})}
\]
que es el coeficiente de correlaci\'{o}n lineal entre $X_{it}$ y $X_{j,t-l}$. Cuando $l>0$, este coeficiente de 
correlaci\'{o}n mide la dependencia lineal de $X_{it}$ con respecto $X_{j,t-l}$ ($X_{t-l}$ ocurre con anterioridad al instante t). Consecuentemente, si $\rho_{ij}\left( l \right)\neq 0$ y $l>0$, se dice que la serie $X_{jt}$ \textbf{conduce} a la serie $X_{it}$ con retardo $l$.\newline

Similarmente, $\rho_{ji}(l)$ mide la dependencia lineal de $X_{jt}$ con respecto a $X_{i,t-l}$ y se puede decir que la serie $X_{it}$ \textbf{conduce} a la serie $X_{jt}$, con retardo $l$, si $\rho_{ij}(l)\neq 0$ y $l>0$.\newline

Se pueden mencionar las siguientes propiedades cuando $l>0$:

\begin{enumerate}
\item En general, $\rho_{ij}\left( l \right)\neq \rho_{ji}\left( l \right)$ para $i\neq j$, porque los dos coeficientes de correlaci\'{o}n miden diferentes relaciones lineales entre las series. Por lo tanto, $\Gamma_{l}$ y $\rho_{l}$ son, generalmente, no sim\'{e}tricas.
\item Utilizando $Cov\left(X, Y\right)=Cov\left(Y, X\right)$ y suponiendo que las series son estacionarias, se tiene:
\[
Cov(X_{it}, X_{j,t-l})=Cov\left(X_{j,t-l}, X_{it}\right)=Cov\left(X_{jt}, X_{i,t+l}\right)=Cov\left(X_{jt}, X_{i,t-\left( -l\right)} \right)
\]
\end{enumerate}

As\'{i} que $\Gamma_{ij}(l)=\Gamma_{ji}(-l)$, donde $\Gamma_{ji}(-l)$ es el elemento $(j, i)$ de $\Gamma_{-l}$; la igualdad se cumple para 1 $\leq$ i, j $\leq$ k. Es decir, $\Gamma_{l}=\Gamma_{-l}^{'}$.

\subsection{Dependencia Lineal}
\label{subsubsec:mylabel3}

Consid\'{e}rense\index{Dependencia lineal multivariante} las matrices de correlaci\'{o}n cruzada $\left\{\rho(l)\vert 
l=0, 1, 2,\ldots \right\}$ de una serie temporal vectorial estacionaria; \'{e}stas contienen la siguiente informaci\'{o}n:

\begin{enumerate}
\item Los elementos de la diagonal de la matriz de correlaci\'{o}n cruzada $\rho_{ii}\left( l \right)$ son las funciones de autocorrelaci\'{o}n de $X_{it}$.
\item El elemento fuera de la diagonal $\rho_{ij}(0)$ mide la relaci\'{o}n lineal de concurrencia entre $X_{it}$ y $X_{jt}$.
\item Para $l>0$, el elemento fuera de la diagonal $\rho_{ij}\left( l \right)$ mide la dependencia lineal de $X_{it}$ con respecto a $X_{j,t-l}$.
\end{enumerate}

Por lo tanto, si $\rho_{ij}\left( l \right)=0$ para todo $l>0$, $X_{it}$ no depende linealmente de ning\'{u}n valor del pasado $X_{j,t-l}$.\newline

\textbf{\textit{Resumen e interpretaci\'{o}n.}}\newline

En general, la relaci\'{o}n lineal entre dos series de tiempo $\left{X_{it}\right}$  y $\left{X_{jt}\right}$ puede 
resumirse en la siguiente forma:

\begin{enumerate}
\item $X_{it}$ y $X_{jt}$ no tienen relaci\'{o}n lineal si $\rho_{ij}\left( l \right)=\rho_{ji}\left( l \right)=0,\quad \forall l\geq 0$.
\item $X_{it}$ y $X_{jt}$ est\'{a}n al mismo tiempo correlacionadas si $\rho_{ij}\left( 0 \right)\neq 0$.
\item $X_{it}$ y $X_{jt}$ no tienen relaci\'{o}n de avance-retardo si $\rho_{ij}\left( l \right)=0$ y $\rho_{ji}\left( l \right)=0,\quad \forall l>0$. En este caso, se dice que las series son desacopladas.
\item Existe una \textit{relaci\'{o}n unidireccional} desde $X_{it}$ hacia $X_{jt}$ si $\rho_{ij}\left( l \right)=0,\quad \forall l>0$, pero $\rho_{ji}(v)\neq 0$ para alg\'{u}n $v>0$. En este caso, $X_{it}$ no depende de ning\'{u}n valor del pasado de $X_{jt}$, pero $X_{jt}$ depende de alg\'{u}n valor del pasado de $X_{it}$.
\item Existe una \textit{relaci\'{o}n de retroalimentaci\'{o}n} entre $X_{it}$ y $X_{jt}$ si $\rho_{ij}\left( l \right)\neq 0$ para alg\'{u}n $l>0$ y $\rho_{ji}(v)\neq 0$ para alg\'{u}n $v>0$.
\end{enumerate}

Las formulaciones anteriores son suficientes para analizar la dependencia lineal entre series temporales. Un enfoque m\'{a}s informativo para estudiar las relaciones entre las series temporales es construir un modelo multivariante para las series, porque un modelo correctamente especificado considera simult\'{a}neamente el n\'{u}mero de series y las correlaciones cruzadas de las mismas.\newline

En la pr\'{a}ctica se utilizan los estimadores de las matrices antes mencionadas; en particular para $\Gamma(l)$: 
\[
\hat{\Gamma}\left( l \right)=\frac{1}{T}\sum_{t=l+1}^T {\left( 
X_{t}-\bar{X} \right)\left(X_{t-l}-\bar{X} \right)'}, \quad l\geq 0 
\]
donde,

T: n\'{u}mero de observaciones
\[
\bar{X}=\frac{\left(\displaystyle\sum_{t=1}^T X_{t} \right)}{T}:\text{vector de medias muestrales}
\]
y para $\rho \left( l \right):$
\[
\hat{\rho }\left( l \right)=\hat{D}^{-1}\hat{\Gamma }_{X}\left(l \right)\hat{D}^{-1}, \quad l\geq 0 
\]
donde,

$\hat{D}:$ Es la matriz diagonal de orden $(k* k)$ que contiene las desviaciones est\'{a}ndar muestrales del vector $X_{t}$ en la diagonal.

\begin{ejemplo}
Se consideran tres series de datos econ\'{o}micos de un pa\'{i}s sudamericano: el producto interno bruto (PIB), denotada por $\left(X_{1t} \right)$; el consumo interno (CI), denotada por $\left(X_{2t} \right)$ y la demanda final interna (DFI), denotada por $\left(X_{3t} \right)$. Se dispone de 56 datos trimestrales, desde noviembre de 2010 hasta junio de 2015 (Ver Anexo D.1). Para efectos de comparaciones se trabajar\'{a} \'{u}nicamente con los primeros 50 datos y se guardar\'{a}n los 6 restantes para comparar con predicciones posteriores (enero 2015 - junio de 2015). Se desea estimar las matrices de correlaciones cruzadas de las series.
\end{ejemplo}

\textbf{Resoluci\'{o}n.}

Las matrices de correlaci\'{o}n cruzada se las construyen de manera manual, considerando cada escenario de posibles combinaciones entre las variables; as\'{i}, en este caso, se obtiene:

\begin{itemize}
      \item[a)] Estadísticos descriptivos de $x_{1t}, X_{2t}$ y $X_{3t}$.
\begin{table}[H]
\centering
\begin{tabular}{cccccccc}\hline
~ & Media & Mediana & M\'{a}ximo & M\'{i}nimo & Desv. Est.& Asimetr\'{i}a & Curtosis\\ \hline
$X_{1t}$ & 96,52 & 89,79 & 165,31 & 49,79 & 34,15 & 0,35 & 1,92 \\ 
$X_{2t}$ & 78,10 & 73,72 & 126,01 & 43,05 & 24,89 & 0,31 & 1,84 \\ 
$X_{3t}$ & 98,71 & 92,34 & 166,66 & 50,51 & 34,48 & 0,30 & 1,87 \\ \hline
\end{tabular}
\end{table}

      \item[b)] Matrices de correlación cruzada
\begin{table}[H]
\centering
\begin{tabular}{cccccccccc}\hline
~& \multicolumn{3}{c}{retardo 1} & \multicolumn{3}{c}{retardo 2} & \multicolumn{3}{c}{retardo 3} \\ \hline 
$X_{1t}$ & 0,94 & 0,94 & 0,94 & 0,87 & 0,88 & 0,87 & 0,81 & 0,82 & 0,81 \\
$X_{2t}$ & 0,93 & 0,94 & 0,94 & 0,87 & 0,88 & 0,87 & 0,80 & 0,81 & 0,81 \\
$X_{3t}$ & 0,93 & 0,94 & 0,94 & 0,87 & 0,88 & 0,87 & 0,80 & 0,81 & 0,81 \\ \hline
\end{tabular}
\end{table}

      \item[c)] Representación simplificada

\begin{table}[H]
\centering

\left|\begin{tabular}{ccc}
+ & + & + \\
+ & + & + \\
+ & + & + \\
\end{tabular} \right| \quad \left|
\begin{tabular}{ccc}
+ & + & + \\
+ & + & + \\
+ & + & + \\
\end{tabular} \right| \quad \left|
\begin{tabular}{ccc}
+ & + & + \\
+ & + & + \\
+ & + & + \\
\end{tabular}\right|
\caption{Resumen de estadísticas y matrices de correlación cruzada para $X_{1t}$, $X_{2t}$ y $X_{3t}$}
\label{tab13}
\end{table}
\end{itemize}

Para representar a las matrices de correlaci\'{o}n cruzada, se utiliza la forma gr\'{a}fica simplificada %(ver tabla 5.1c)
, que utiliza el hecho que $2/\sqrt{T}$ (0,28 en este caso) es el valor cr\'{i}tico de la correlaci\'{o}n muestral con nivel de significaci\'{o}n del $5\%$, bajo la suposici\'{o}n que $X_{t}$ es un ruido blanco:

\begin{itemize}
\item ``$+$'' representa a los coeficientes de correlaci\'{o}n que son mayores o iguales a $2/\sqrt T $.
\item ``-`` representa a los coeficientes de correlaci\'{o}n que son menores o iguales que $-2/\sqrt T $.
\item ``\textbf{.}'' Representa a los coeficientes que se encuentran entre a $-2/\sqrt T $ y $2/\sqrt T $.
\end{itemize}

Es f\'{a}cil ver que las correlaciones cruzadas son significativas en los primeros tres retardos. En algunos paquetes estad\'{i}sticos se puede encontrar el c\'{a}lculo de las matrices de correlaci\'{o}n cruzada. En Eviews, por ejemplo, se presenta la siguiente salida:

%\[
%X_{1t},X_{2t}(+i)\quad X_{1t},X_{2t}(-i)
%\]
\begin{figure}[H]
\centering
\includegraphics[width=0.7\textwidth]{Graficos/Cap4-5/STcap419.eps}
\caption{Correlaciones cruzadas entre $X_{1t}$ y $X_{2t}$}
\label{fig19}
\end{figure}

%\[
%X_{1t},X_{3t}(+i)\quad X_{1t},X_{3t}(-i)
%\]
\begin{figure}[H]
\centering
\includegraphics[width=0.7\textwidth]{Graficos/Cap4-5/STcap420.eps}
\caption{Correlaciones cruzadas entre $X_{1t}$ y $X_{3t}$}
\label{fig20}
\end{figure}

%\[
%X_{2t},X_{3t}(+i)\quad X_{2t},X_{3t}(-i)
%\]
\begin{figure}[H]
\centering
\includegraphics[width=0.7\textwidth]{Graficos/Cap4-5/STcap421.eps}
\caption{Correlaciones cruzadas entre $X_{2t}$ y $X_{3t}$}
\label{fig21}
\end{figure}


Como se puede observar, los valores calculados por el paquete son aquellos que est\'{a}n en la diagonal segundaria de las matrices calculadas manualmente. Para poder completar la matriz, se puede, ver a partir de la f\'{o}rmula de c\'{a}lculo que las $\hat{\rho }_{ii}\left( l \right)$ corresponden a las autocorrelaciones simples de orden $l$ de cada serie univariante dentro de $X_{t}$.


\section{Modelos de Vectores Autoregresivos (VAR)}
\label{subsec:modelos}

Este tipo de modelos no pertenecen a los modelos estoc\'{a}sticos desarrollados por Box y Jenkins; sin embargo, la representaci\'{o}n\index{Modelos VAR} VAR se puede considerar como la generalizaci\'{o}n de los modelos autoregresivos al caso multivariante.

\subsection{El caso bivariante}
\label{subsubsec:mylabel4}

Una representaci\'{o}n VAR bivariante\index{Modelos VAR!Bivariante} es aquella que consideran dos variables $X_{1t}$ y $X_{2t}$. Cada una de ellas se expresa en funci\'{o}n de sus propios valores del pasado y de los del presente y del pasado de la otra variable. Por ejemplo, se va a representar el modelo VAR bivariante de orden $p=3$ [VAR (3)]; se escribe:
\[
X_{1t}=v_{1}+\sum_{i=1}^3 {b_{1i}X_{1t-i}} +\sum_{i=1}^3 {c_{1i}X_{2t-i}} - d_{1}X_{2t}+u_{1t}
\]

\[
X_{2t}=v_{2}+\sum_{i=1}^3 {b_{2i}X_{1t-i}} +\sum_{i=1}^3 {c_{2i}X_{2t-i}} - d_{2}X_{1t}+u_{2t}
\]

Las variables $X_{1t}$ y $X_{2t}$ son estacionarias; las perturbaciones $u_{1t}$ y $u_{2t}$ son ruidos blancos de varianzas 
constantes y no correlacionados. Se puede ver inmediatamente la gran cantidad de par\'{a}metros a estimar (aqu\'{i} 16 coeficientes), con los problemas t\'{i}picos de p\'{e}rdida de grados de libertad. Hay que tomar en cuenta que $X_{1t}$ tiene un efecto inmediato en $X_{2t}$ y rec\'{i}procamente. Este sistema inicial se denomina \textbf{\textit{forma estructural}} de la representaci\'{o}n VAR\index{Modelos VAR!Forma estructural}. Su \textbf{\textit{forma matricial}}, \index{Modelos VAR!Forma matricial} se expresa como:

\[
BX_{t} = v+\sum_{i=1}^3 {\tilde{A}_{i}X_{t-i} + u_{t}} 
\]
con:

\[
B=\left[ {\begin{array}{*{20}c}
1 & d_{1}\\
d_{2} & 1\\
\end{array} } \right]\quad
X_{t}=\left[ 
{\begin{array}{*{20}c}
X_{1t}\\
X_{2t}\\
\end{array} } \right]\quad
v=\left[ {\begin{array}{*{20}c}
v_{1}\\
v_{2}\\
\end{array} } \right]\quad
\tilde{A}_{i}=\left[ 
{\begin{array}{*{20}c}
b_{1i} & c_{1i}\\
b_{2i} & c_{2i}\\
\end{array} } \right]\quad
u_{t}=\left[ {\begin{array}{*{20}c}
u_{1t}\\
u_{2t}\\
\end{array} } \right]
\]

Para obtener la \textbf{\textit{forma est\'{a}ndar}} de\index{Modelos VAR!Forma est\'{a}ndar} un modelo VAR, se multiplica la ecuaci\'{o}n anterior por $B^{-1}$ (que se supone existe); es decir, se expresa por:

\[
X_{1t}=v_{1}^{0}+\sum_{i=1}^3 {a_{1i}^{1}X_{1t-i}} +\sum_{i=1}^3 {a_{1i}^{2}X_{2t-i}} +\vartheta_{1t}
\]
\[
X_{2t}=v_{2}^{0}+\sum_{i=1}^3 {a_{2i}^{1}X_{1t-i}} +\sum_{i=1}^3 {a_{2i}^{2}X_{2t-i}} +\vartheta_{2t}
\]

En esta especificaci\'{o}n, los errores $\vartheta_{1t}$y $\vartheta_{2t}$ son funciones de las innovaciones $u_{1t}$ y $u_{2t}$; en efecto, de $\vartheta =B^{-1}u$, se obtiene:

\[
\vartheta_{1t}=\frac{\left( u_{1t}-d_{1}u_{2t} \right)}{\left( 1-d_{1}d_{2} \right)}\text{ y }\vartheta_{2t}=
\frac{\left( u_{2t}-d_{2}u_{1t} \right)}{\left( 1-d_{1}d_{2} \right)}
\]
Se puede ver que:
\[
E\left( \vartheta_{1t} \right)=0;\quad E\left( \vartheta_{2t} \right)=0;\quad E\left( \vartheta_{1t}\vartheta_{1t-i} 
\right)=0;\quad E\left( \vartheta_{2t}\vartheta_{2t-i} \right)=0
\]

Por lo tanto, los elementos de cada familia de errores tienen esperanza nula y son no correlacionados. Adem\'{a}s:

\[
E\left( \vartheta_{1t}^{2} \right)=\frac{\left( \sigma_{u_{1}}^{2}+d_{1}^{2}\sigma_{u_{2}}^{2} \right)}{\left( 1-d_{1}d_{2} 
\right)^{2}};\quad E\left( \vartheta_{2t}^{2} \right)=\frac{\left( \sigma_{u_{2}}^{2}+d_{2}^{2}\sigma_{u_{1}}^{2} \right)}{\left( 1-d_{1}d_{2} \right)^{2}}
\]

Donde $\sigma_{u_{1}}^{2}$ y $\sigma_{u_{2}}^{2}$ son las varianzas de $u_{1}$y $u_{2}$, respectivamente. As\'{i}, la varianza de los errores es constante (independiente del tiempo). Adem\'{a}s:

\[
E\left( \vartheta_{1t}\vartheta_{2t} \right)=-\frac{(d_{2}\sigma_{u_{1}}^{2}+d_{1}\sigma_{u_{2}}^{2})}{{(1-d_{1}d_{2})}^{2}}
\]

Si $d_{1}=d_{2}=0$, las variables $X_{1t}_{\mathrm{}}$ y $X_{2t}$ no tienen ninguna influencia sincr\'{o}nica entre s\'{i}, pues los errores $\vartheta_{1t}$y $\vartheta_{2t}$ ser\'{i}an no correlacionados. En caso contrario, los errores $\vartheta_{1t}$y $\vartheta_{2t}$ estar\'{i}an correlacionados y por tanto, una variaci\'{o}n de uno de estos errores en un instante dado tiene impacto en el otro.

\begin{proposicion}
El modelo VAR no permite distinguir entre variables end\'{o}genas (variables propias del fen\'{o}meno estudiado) y ex\'{o}genas (variables externas que ayudan a explicar las variables end\'{o}genas).
\end{proposicion}

\subsection{Representaci\'{o}n general de un VAR}
\label{subsubsec:mylabel5}

\textbf{Notaci\'{o}n. } Un modelo VAR\index{Modelos VAR!Representaci\'{o}n general} a k variables con p retardos se denota $VAR(p)$.

La generalizaci\'{o}n de la representaci\'{o}n VAR a $k$ variables con $p$ retardos se escribe en su forma est\'{a}ndar como:
\[
X_{t}=v_{0}+A_{1}X_{t-1}+A_{2}X_{t-2}+\ldots +A_{p}X_{t-p}+u_{t}
\]
donde, 
\[
X_{t}=\left[ {\begin{array}{*{20}c}
X_{1,t}\\
{\begin{array}{*{20}c}
X_{2,t}\\
\vdots \\
\end{array} }\\
X_{k,t}\\
\end{array} } \right];\quad 
v_{0}=\left[ {\begin{array}{*{20}c}
{\begin{array}{*{20}c}
v_{1}^{0}\\
v_{2}^{0}\\
\vdots \\
\end{array} }\\
v_{k}^{0}\\
\end{array} } \right];\quad 
A_{i}=\left[ 
{\begin{array}{*{20}c}
a_{1i}^{1} & \mathellipsis & a_{1i}^{k}\\
\vdots & \ddots & \vdots \\
a_{ki}^{1} & \mathellipsis & a_{ki}^{k}\\
\end{array} } \right];\quad
u_{t}=\left[ 
{\begin{array}{*{20}c}
{\begin{array}{*{20}c}
u_{1t}\\
u_{2t}\\
\vdots \\
\end{array} }\\
u_{kt}\\
\end{array} } \right]
\]

$u_{t}$ es el vector compuesto por los ruidos blancos de cada una de las $k$ ecuaciones del modelo.\newline

Se denota por: $\sum\nolimits_u = E(u_{t}u_{t}^{'})$, la matriz desconocida, de dimensi\'{o}n $k$, de varianzas-covarianzas de los errores.\newline

Esta representaci\'{o}n puede escribirse mediante el operador de retardo B, como: 
\[
\left( I-A_{1}B-A_{2}B^{2}-\mathellipsis -A_{p}B^{P} \right)X_{t}=v_{0}+u_{t}\text{,\quad o tambi\'{e}n:\quad}
A\left( B \right)X_{t}=v_{0}+u_{t}
\]
donde, el operador de retardo B se define de la siguiente manera:
\[
B^{i}X_{t}=X_{t-i},\quad i=1,2,\ldots 
\]
\[
B^{0}X_{t}=X_{t}
\]

\subsubsection{Estabilidad de un VAR}
Consid\'{e}rese\index{Modelos VAR!Estabilidad} un modelo $VAR(1)$:
\[
X_{t}=v_{0}+A_{1}X_{t-1}+u_{t}
\]

Se dice que un $VAR(1)$ es estable si todos los valores propios de $A_{1}$ son de valor absoluto menor que 1; lo que se puede expresar tambi\'{e}n por:
\[
\det \left( I_{k}-A_{1}z \right)\ne 0,\quad \text{para}\quad \left| z \right|\le 1
\]

Esto implica que todas las ra\'{i}ces del polinomio caracter\'{i}stico est\'{a}n fuera del c\'{i}rculo unidad.

\subsubsection{Representaci\'{o}n de un proceso VAR(p) en la forma de VAR(1)}

Un proceso $VAR(p)$ se puede\index{Modelos VAR!Representaci\'{o}n de VAR(1)} escribir como un proceso $VAR(1)$ si se plantea en la siguiente forma:
\[
X_{t}=A_{0}+AX_{t-1}+U_{t}
\]
donde,
\[
X_{t}=\left( {\begin{array}{*{20}c}
X_{t}\\
X_{t-1}\\
{\begin{array}{*{20}c}
\vdots \\
X_{t-p+1}\\
\end{array} }\\
\end{array} } \right)_{kp\ast 1}\quad
X_{t}=\left( {\begin{array}{*{20}c}
X_{1t}\\
{\begin{array}{*{20}c}
X_{2t}\\
\vdots \\
\end{array} }\\
X_{kt}\\
\end{array} } \right)_{k\ast 1}\quad
A_{0}=\left( {\begin{array}{*{20}c}
v_{0}\\
{\begin{array}{*{20}c}
0_{k\ast 1}\\
\vdots \\
\end{array} }\\
0_{k\ast 1}\\
\end{array} } \right)_{kp\ast 1}
U_{t}=\left( {\begin{array}{*{20}c}
\left( u_{t} \right)_{k\ast 1}\\
{\begin{array}{*{20}c}
0_{k\ast 1}\\
\vdots \\
\end{array} }\\
0_{k\ast 1}\\
\end{array} } \right)_{kp\ast 1}
\]

\[
A=\left( {\begin{array}{ccccc}
A_{1} & A_{2} & \ldots & A_{p-1} & A_{p} \\
I_{k} & 0 & \ldots & 0 & 0 \\
0 & I_{k} & \ldots & 0 & 0 \\
\vdots & \vdots & \ddots & \vdots & \vdots\\
0 & 0 & \ldots & I_{k} & 0 \\
\end{array}
\right)_{kp\ast kp}
\]

Lo importante de esta representaci\'{o}n es que para obtener las propiedades de los procesos VAR, es suficiente con probarlas para una $VAR\left( 1 \right)$

\subsubsection{Procesos VAR(p) estables}

Se dice que un $VAR(p)$ es estable si:
\[
\det \left( I_{kp}-Az \right)\ne 0,\mathrm{para}\left| z \right|\le 1
\]

Adem\'{a}s, se puede demostrar que:
\[
\det \left( I_{kp}-Az \right)=det\left( I-A_{1}z-A_{2}z^{2}-\mathellipsis -A_{p}z^{P} \right)
\]

\begin{observacion}
Se puede demostrar que si un proceso $VAR(1)$ es estable, entonces es estacionario.

En general, se puede demostrar que un proceso $VAR(p)$ es estacionario si el polinomio definido a partir de la expresi\'{o}n: $det\left( I-A_{1}z-A_{2}z^{2}-\mathellipsis -A_{p}z^{P} \right)$ tiene sus ra\'{i}ces fuera del c\'{i}rculo unidad del plano complejo; es decir:
\[
det\left( I-A_{1}z-A_{2}z^{2}-\mathellipsis -A_{p}z^{P} \right)\ne 0\forall ztalque\left| z \right|\le 1
\]
\end{observacion}

\begin{ejemplo}
Determine si el siguiente modelo es estacionario.
\[
\left[ {\begin{array}{*{20}c}
X_{1t}\\
X_{2t}\\
\end{array} } \right]=\left[ {\begin{array}{*{20}c}
2\\
5\\
\end{array} } \right]+\left[ {\begin{array}{*{20}c}
0,8 & 0,9\\
0,7 & 0,7\\
\end{array} } \right]\left[ {\begin{array}{*{20}c}
X_{1t-1}\\
X_{2t-1}\\
\end{array} } \right]+\left[ {\begin{array}{*{20}c}
u_{1t}\\
u_{2t}\\
\end{array} } \right]
\]

\textbf{Resoluci\'{o}n.}

Se tiene que:
\[
det\left( \left[ {\begin{array}{*{20}c}
1 & 0\\
0 & 1\\
\end{array} } \right]-\left[ {\begin{array}{*{20}c}
0,8 & 0,9\\
0,7 & 0,7\\
\end{array} } \right]z \right)=1-1,5z-0,07z^{2}\text{,\quad entonces\quad}
{\begin{array}{*{20}c}
z_{1}=-16,17\\
z_{2}=-15,97\\
\end{array} }
\]

Las dos ra\'{i}ces son superiores a 1 en valor absoluto; por lo tanto, el proceso es estable; lo que implica que es estacionario.
\end{ejemplo}


\section{Representaci\'{o}n VARMA de una Serie Multivariante}
\label{subsec:representaci}

\subsection{La representaci\'{o}n VMA}
\label{subsubsec:mylabel6}

Un modelo media m\'{o}vil vectorial de orden\index{Modelos VMA!Orden q} q ($VMA\left( q \right)$ por sus siglas en ingl\'{e}s), tiene la siguiente forma:
\[
X_{t}=m_{0}+u_{t}-M_{1}u_{t-1}-\mathellipsis -M_{q}u_{t-q}\quad \text{o}\quad X_{t}=m_{0}+M(B)u_{t}
\]

donde,\newline 
$m_{0}:$ Es un vector de dimensi\'{o}n $k$ constante$.$\newline
$M_{i}:$ Son matrices de dimensi\'{o}n $k\ast k$.\newline
$M\left( B \right)=I-M_{1}B-\mathellipsis -M_{q}B^{q}$ es el polinomio matriz MA en t\'{e}rminos del operador de retardo B.\newline
$\left\{ u_{t} \right\}_{t\in Z}:$ Es un ruido blanco multidimensional.\newline

De manera similar al caso univariante, los procesos $VMA\left( q \right)$ son d\'{e}bilmente estacionarios, siempre que la matriz de covarianzas $\left(\Sigma_{u} \right)$ de $u_{t}$ exista. Si se toma la esperanza de $X_{t}$, se tiene:
\[
\mu =E\left( X_{t} \right)=m_{0}
\]

As\'{i}, el vector constante $m_{0}$ es el vector media de $X_{t}$ para un modelo VMA.

Se define $\tilde{X}_{t}=X_{t}-m_{0}$ como el proceso corregido en media $VAR(q)$. Cuando se tiene un proceso $VMA(q)$ y considerando el hecho de que los $\left\{ u_{t} \right\}$ no est\'{a}n correlacionados, se obtiene:

\begin{enumerate}
      \item $Cov\left( \tilde{X}_{t},u_{t} \right)=\Sigma_{u}$
      \item $\Gamma_{0}=\Sigma_{u}+M_{1}\Sigma_{u}M_{1}^{'}+\mathellipsis +M_{q}\Sigma_{u}M_{q}^{'}$
      \item $\Gamma_{l}=0$ si $l>q$
      \item $\Gamma_{l}=\sum_{j=l}^q {\mathrm{M}_{j}\Sigma_{u}M_{j-l}^{'}}$\quad si $1\le l \leq q$,\quad donde $M_{0}=-I$
\end{enumerate}

Dado que $\Gamma_{l}=0$ para $l>q$, las matrices de correlaci\'{o}n cruzada de un proceso $VMA(q)$ satisfacen:
\[
\rho_{l}=0,\quad l>q
\]

\subsection{Representaci\'{o}n lineal de un VAR(p)}
\label{subsubsec:mylabel7}

Cuando se analizaron\index{Modelos VAR!Representaci\'{o}n lineal} las series temporales univariantes, se mostr\'{o} que bajo ciertas condiciones un proceso $AR(1)$ se puede representar como un proceso lineal. De la misma manera, para las series multivariantes se puede representar, en particular, un $VAR(1)$ como un proceso lineal (se dice que es la representaci\'{o}n 
lineal del proceso). Un modelo con esta forma permite medir el impacto en los valores presentes de una variaci\'{o}n de innovaciones (o choques).

Sea $X_{t}$ un $VAR(1)$ estable:
\[
X_{t}=v_{0}+A_{1}X_{t-1}+u_{t}
\]

Si se realizan sustituciones repetidas en el proceso hasta el i-\'{e}simo paso, se obtiene:
\[
\begin{array}{l}
 X_{t}=v_{0}+A_{1}\left( v_{0}+A_{1}X_{t-2}+u_{t-1} 
\right)+u_{t}\mathrm{=}\left( I+A_{1} \right)v_{0}+A_{1}^{2}X_{t-2}+\left( 
A_{1}u_{t-1}+u_{t} \right) \\ 
 X_{t}=v_{0}+A_{1}\left( \left( I+A_{1} 
\right)v_{0}+{A_{1}^{2}X}_{t-3}+A_{1}u_{t-2}+u_{t-1} \right)+u_{t} \\ 
 =\left( 
I+A_{1}+A_{1}^{2} 
\right)v_{0}+A_{1}^{3}X_{t-3}+(A_{1}^{2}u_{t-2}+A_{1}u_{t-1}+u_{t}) \\ 
 \vdots \\ 
 X_{t}=\left( I+A_{1}+\mathellipsis +A_{1}^{i} 
\right)v_{0}+A_{1}^{i+1}X_{t-i}+\displaystyle\sum_{j=0}^i {A_{1}^{j}u_{t-j}},\quad i=0,1,2,\mathellipsis \\ 
 \end{array}
\]

Por definici\'{o}n, $A^{0}=I$.\newline

Como el VAR es estable, se cumple que:
\[
\left( I+A_{1}+\mathellipsis +A_{1}^{i} \right)v_{0}\to \left( I-A_{1} 
\right)^{-1}v_{0}\quad \text{ si }\quad i\to \infty 
\]

Adem\'{a}s, $A_{1}^{i+1}\to 0$ r\'{a}pidamente; as\'{\i}, se lo puede ignorar. Por lo tanto, se obtiene la siguiente representaci\'{o}n:
\[
X_{t}=\left( I-A_{1} \right)^{-1}v_{0}+\displaystyle\sum_{i=0}^\infty {A_{1}^{i}u_{t-i}} 
\]

La generalizaci\'{o}n a un proceso $VAR(p)$ se la realiza aplicando la representaci\'{o}n de un VAR(p) como un VAR(1). As\'{i}, se obtiene:
\[
X_{t}=\mu +\sum_{i=0}^\infty {M_{i}u_{t-i}} 
\]
donde, 
\[
\mu ={(I-A_{1}-A_{2}-\mathellipsis -A_{p})}^{-1}v_{0}
\]
\[
M_{i}=\sum_{j=1}^{\mathrm{min}(p,i)} A_{j} M_{i-j}\quad i=1,2,\mathellipsis\quad \text{y}\quad M_{0}=I
\]

Las matrices $M_{i}$ aparecen como un ``\textbf{\textit{factor de impacto}}\index{Factor de impacto}'', a trav\'{e}s de las cuales se analiza el efecto de un choque a lo largo de todo el proceso.

\begin{observacion}
\quad\newline
\begin{enumerate}
      \item As\'{\i}, se obtiene que si un proceso $VAR(p)$ es estable, tiene una representaci\'{o}n lineal estacionaria.
      \item No se profundiza sobre la modelaci\'{o}n $VMA(q)$ porque no est\'{a} implementada en los programas comerciales usuales.
\end{enumerate}
\end{observacion}

\begin{ejemplo}
Consid\'{e}rese el proceso VMA (1):
\[
X_{t}=\mu +u_{t}-M_{1}u_{t-1}=\mu +u_{t}-Mu_{t-1}
\]
donde, por simplicidad, se ha quitado el sub\'{\i}ndice de $M_{1}$. Este modelo puede escribirse expl\'{\i}citamente como:
\[
\left[ {\begin{array}{*{20}c}
X_{1t}\\
X_{2t}\\
\end{array} } \right]=\left[ {\begin{array}{*{20}c}
\mu_{1}\\
\mu_{2}\\
\end{array} } \right]+\left[ {\begin{array}{*{20}c}
u_{1t}\\
u_{2t}\\
\end{array} } \right]-\left[ {\begin{array}{*{20}c}
m_{11} & m_{12}\\
m_{21} & m_{22}\\
\end{array} } \right]\left[ {\begin{array}{*{20}c}
u_{1t-1}\\
u_{2t-1}\\
\end{array} } \right]
\]
Se dice que la serie de retardos $\left( X_{t} \right)$ solo depende del presente y del pasado de $\left\{u_{t} \right\}$. Por lo tanto, el modelo es de memoria finita.\newline

El par\'{a}metro $m_{12}$ denota la dependencia lineal de $X_{1t}$ con $u_{2,t-1}$ en la presencia de $u_{1,t-1}$. Si $m_{12}=0$, $X_{1t}$ no depende de los retardos de $u_{2t}$ y, entonces tampoco, de los retardos de $X_{2t}$. De manera similar, si $m_{21}=0$, $X_{2t}$ no depende de los valores pasados de $X_{1t}$. Los elementos fuera de la diagonal de M muestran la dependencia entre las componentes de las series.\newline

Para este ejemplo, se pueden clasificar las relaciones entre $X_{1t}$ y $X_{2t}$ as\'{i}:
\begin{enumerate}
      \item Son series desacopladas si $m_{12}=m_{21}=0$.
      \item Hay una relaci\'{o}n din\'{a}mica unidireccional de $X_{1t}$ sobre $X_{2t}$ si $m_{12}=0$, pero $m_{21}\neq 0$ y viceversa.
      \item Hay una relaci\'{o}n de retroalimentaci\'{o}n entre $X_{1t}$ y $X_{2t}$ si $m_{12}\neq 0$ y $m_{21}\neq 0$.
\end{enumerate}

Finalmente, la correlaci\'{o}n actual entre los $m_{ij}$ (coeficientes estimados para el modelo VMA) es la misma que entre los $u_{it}$ . La descripci\'{o}n previa se puede generalizar para un modelo $VMA(q)$.
\end{ejemplo}


\subsection{La representaci\'{o}n VARMA}
\label{subsubsec:mylabel8}

La representaci\'{o}n VAR puede generalizarse (es una aplicaci\'{o}n multivariante del teorema de descomposici\'{o}n de Wold (1954)), por analog\'{i}a con los procesos $ARMA(pq)$.
\[
X_{t}=A_{0}+A_{1}X_{t-1}+A_{2}X_{t-2}+\ldots +A_{p}X_{t-p}+u_{t}+M_{1}u_{t-1}+M_{2}u_{t-2}+\ldots +M_{q}u_{t-q}
\]

Se trata de un proceso ARMA multivariante que se denota\index{Modelos VARMA}: VARMA.\newline

Las condiciones de estacionariedad son an\'{a}logas a las de un proceso ARMA univariante: 
\begin{itemize}
      \item Un proceso VAR es siempre invertible; es lineal (por ende estacionario) cuando es estable.
      \item Un proceso VMA es siempre estacionario. Es invertible si las ra\'{i}ces del polinomio caracter\'{i}stico asociado a $M(z)$ est\'{a}n fuera del c\'{i}rculo unitario complejo.
      \item Las condiciones de estacionariedad e invertibilidad de un VARMA est\'{a}n dadas, respectivamente, por la parte VAR y la parte VMA del VARMA.
\end{itemize}

La generalizaci\'{o}n de los modelos ARMA encuentra nuevos temas que no ocurren en el desarrollo de los modelos VAR y VMA. Uno de ellos es el \textit{problema de identificaci\'{o}n}. A diferencia de los modelos ARMA, los modelos VARMA pueden no estar  definidos de manera \'{u}nica.

\begin{ejemplo}
Considere un modelo bivariante $VMA(1)$:
\[
\left[ {\begin{array}{*{20}c}
X_{1t}\\
X_{2t}\\
\end{array} } \right]=\left[ {\begin{array}{*{20}c}
u_{1t}\\
u_{2t}\\
\end{array} } \right]-\left[ {\begin{array}{*{20}c}
0 & 2\\
0 & 0\\
\end{array} } \right]\left[ {\begin{array}{*{20}c}
u_{1,t-1}\\
u_{2,t-1}\\
\end{array} } \right]
\]
Es \textit{id\'{e}ntico} al modelo bivariante $VAR(1)$:
\[
\left[ {\begin{array}{*{20}c}
X_{1t}\\
X_{2t}\\
\end{array} } \right]-\left[ {\begin{array}{*{20}c}
0 & -2\\
0 & 0\\
\end{array} } \right]\left[ {\begin{array}{*{20}c}
X_{1,t-1}\\
X_{2,t-1}\\
\end{array} } \right]=\left[ {\begin{array}{*{20}c}
u_{1t}\\
u_{2t}\\
\end{array} } \right]
\]

La equivalencia de los modelos se puede examinar f\'{a}cilmente componente a componente. Es decir, para el modelo $VMA(1)$ se tiene:
\[
X_{1t}=u_{1t}-2u_{2,t-1}\quad \text{y}\quad X_{2t}=u_{2t}
\]

Por otro lado, para el modelo $VAR(1)$ se tiene:
\[
X_{1t}+2X_{2,t-1}=u_{1t}\quad \text{y}\quad X_{2t}=u_{2t}
\]

De los modelos se puede ver que:
\[
X_{2,t-1}=u_{2,t-1}
\]

Luego, los modelos para $X_{1t}$ son id\'{e}nticos. Este tipo de problema de identificaci\'{o}n es inofensivo porque cualquiera de los modelos puede ser utilizado en una aplicaci\'{o}n real. Sin embargo, existen casos en los que esta situaci\'{o}n si se convierte en un problema y hay que tener en cuenta muchas restricciones para poder estimar un modelo VARMA.
\end{ejemplo}


\section{Formulaci\'{o}n de un modelo VAR}
\label{subsec:mylabel7}

Los par\'{a}metros de un proceso VAR pueden estimarse solamente en las series temporales estacionarias. Se conoce que muchas series pueden volverse estacionarias a trav\'{e}s de un proceso de diferenciaci\'{o}n (en el caso de una tendencia determinista o una estacionalidad) o a trav\'{e}s de una transformaci\'{o}n de las variables (por ejemplo, una transformaci\'{o}n logar\'{i}tmica) en ciertos casos con heteroscedasticidad.

\subsection{Estimaci\'{o}n}
\label{subsubsec:mylabel9}

En el caso de un proceso VAR, las ecuaciones pueden\index{Modelos VAR!Estimaci\'{o}n de los coeficientes} estimarse por MCO independientemente una de la otra (o por un m\'{e}todo de m\'{a}xima verosimilitud).\newline

Sea el modelo $VAR(p)$ estimado:
\[
X_{t}=\hat{A}_{0}+\hat{A}_{1}X_{t-1}+\hat{A}_{2}X_{t-2}+\ldots +\hat{A}_{p}X_{t-p}+\hat{u}_{t}
\]

siendo, $\hat{u}_{t}$ el vector de dimensi\'{o}n $(k,1)$ de componentes $\hat{u}_{1t}, \hat{u}_{2t}, \ldots ,\hat{u}_{kt}$.\newline

Se denotar\'{a} por $\displaystyle\hat{\Sigma }_{u,p}$ la matriz de varianzas covarianzas estimada de los residuos del modelo. Para cualquier orden p, se define por:

\[
\displaystyle\hat{\Sigma }_{u,p}=\frac{1}{T-kp-1}\sum_{t=p+1}^T \hat{u}_{t} \left( \hat{u}_{t} \right)^{'},\quad p\ge 0
\]

\subsection{Determinaci\'{o}n del n\'{u}mero de retardos}
\label{subsubsec:mylabel10}

El mayor problema que debe enfrentarse a la hora de estimar los modelos VAR es el de la determinaci\'{o}n del n\'{u}mero de retardos a incluir en la estimaci\'{o}n; suele realizarse en forma cuantitativa, analizando los resultados de la estimaci\'{o}n y comparando los resultados obtenidos entre distintos modelos alternativos, ya que no es frecuente encontrar evidencias te\'{o}ricas al respecto.\newline

Los criterios com\'{u}nmente utilizados para la selecci\'{o}n entre modelos alternativos son el criterio informativo de Akaike (AIC), el criterio de informaci\'{o}n bayesiano (BIC), que tambi\'{e}n se conoce como el criterio de Schwarz (SC) o el criterio de Hanan-Quinn (HQ).\newline

Para el caso de la representaci\'{o}n VAR, estos criterios se pueden utilizar para determinar el orden $p$ del modelo. EL proceso de selecci\'{o}n del orden de la representaci\'{o}n consiste en estimar todos los modelos VAR para retardos de 0 a $p_{0}$ ($p_{0}$ es el m\'{a}ximo retardo admisible por la teor\'{\i}a econ\'{o}mica o por los datos disponibles y se fija de 
antemano). Los estad\'{\i}sticos AIC(p), SC(p) y HQ(p) para el caso multivariante tienen las siguientes expresiones:
\[
AIC\left( p \right)=\ln \Big[ \big| \sum_{u} \big| \Big]+
\frac{2k^{2}p}{T}
\]
\[
SC\left( p \right)=\ln \Big[ \big| \sum_{u} \big| \Big]+
\frac{k^{2}p\mathrm{ln}(T)}{T}
\]

\[
HQ\left( p \right)=\ln \Big[ \big| \sum_{u} \big| \Big]+\frac{2k^{2}p\mathrm{ln}(\ln \left( T) \right)}{T}
\]
donde,\newline

k$=$ n\'{u}mero de variables del sistema.\newline
T$=$ n\'{u}mero de observaciones.\newline
p$=$ n\'{u}mero de retardos.\newline
$\sum_{u}=$ matriz de varianzas covarianzas de residuos del modelo con retardo p (fijo).\newline

Otro criterio utilizado para determinar el retardo del modelo, es la raz\'{o}n de m\'{a}xima verosimilitud. Para utilizar este criterio, es necesario que el vector de las innovaciones tenga una distribuci\'{o}n normal; el logaritmo de la funci\'{o}n de verosimilitud tiene la siguiente expresi\'{o}n: 
\[
l=-\frac{Tp}{2}\left( 1+\ln \left( 2\pi \right) \right)-\frac{T}{2}\ln \left[ \left| \hat{\sum }_{u} \right| \right]
\]

\begin{observacion}
El retardo $p$ que minimice la mayor cantidad de los criterios de AIC, HQ, BIC; o, maximice el logaritmo de la funci\'{o}n de 
verosimilitud, se retiene. En la pr\'{a}ctica se aconseja que $p\le 5$, debido a que valores superiores implican incorporar una gran cantidad de par\'{a}metros.\newline

Tambi\'{e}n se puede utilizar el estad\'{i}stico $M\left( p \right)$ para probar la hip\'{o}tesis nula $H_{0}: \text{El modelo es un }VAR\left( p \right)$ contra la alternativa, $H_{1}: \text{El modelo es un }VAR(p-1)$. 

Este estad\'{\i}stico se define por:
\[
M\left( p \right)=-\left( T-k-p-\frac{3}{2} \right)\ln \left[ \frac{\left| \hat{\sum }_{u,p} \right|}{\left| \hat{\sum }_{u,p-1} \right|} \right]
\]
donde, 

$\hat{\sum }_{u,j}=$matriz de varianzas covarianzas del modelo con retardo j.\newline
$M\left( p \right)$ sigue asint\'{o}ticamente una distribuci\'{o}n Ji-cuadrado con $k^{2}$ grados de libertad.
\end{observacion}

\subsection{Diagn\'{o}stico y validaci\'{o}n del modelo}
\label{subsubsec:mylabel11}

Un buen punto de partida\index{Modelos VAR!Diagn\'{o}stico y validaci\'{o}n} para la verificaci\'{o}n de que el modelo estimado es el adecuado, es la significaci\'{o}n de los par\'{a}metros estimados, para no tener par\'{a}metros no deseados o par\'{a}metros que no aportan al modelo. Por otro lado, esto puede ser enga\~{n}oso, porque los par\'{a}metros estimados de un modelo pobre pueden ser tambi\'{e}n significativos. Por lo tanto, no se debe depender exclusivamente de la significaci\'{o}n de los 
par\'{a}metros para evaluar el modelo.\newline

Como en la mayor\'{i}a de situaciones de modelaci\'{o}n, la forma de evaluaci\'{o}n se realiza a trav\'{e}s del comportamiento de los residuos. Si el modelo es una representaci\'{o}n adecuada de un proceso generado por las series de tiempo, los residuos no deben tener ninguna tendencia significativa ni patr\'{o}n.\newline

Una forma de observar esto es considerar los elementos individuales de la matriz de autocorrelaci\'{o}n de los vectores de residuos. Otra forma es el uso del estad\'{i}stico \textbf{\textit{Portmanteau}}, que se analizar\'{a} posteriormente.

\subsubsection{Matrices de autocorrelaci\'{o}n multivariante}

Sea $\left\{ u_{t} \right\}$ un ruido blanco $k$-dimensional con matriz\index{Matriz!De covarianza residual} de covarianza $\Sigma_{u}$ y su correspondiente matriz de correlaci\'{o}n\index{Matriz!De correlaci\'{o}n residual} $R_{u}$ . La matriz de autocovarianza y la matriz de autocorrelaci\'{o}n muestral de $\left\{ u_{t} \right\}$ con respecto al retardo $i$ est\'{a}n dadas por:
\[
\hat{C}_{i}=\frac{1}{T}\sum_{t=i+1}^T {\hat{u}_{i}{\hat{u}'}_{t-i}} \quad i= 0, 1, \ldots ; i<T
\]

\[
\hat{R}_{i}=V_{u}^{-\frac{1}{2}}\hat{C}_{i}V_{u}^{-\frac{1}{2}}\quad i= 0, 1, \ldots ; i<T
\]

donde, T es el n\'{u}mero de observaciones de las series de tiempo y $V_{u}^{-\frac{1}{2}}$ es una matriz diagonal $(k\ast k)$ con el inverso de la ra\'{\i}z cuadrada de los elementos de la diagonal de $C_{0}$ en su diagonal.\newline

Sea $R_{l}^{\ast }=\left( R_{1},\mathellipsis ,R_{l} \right)'.$

\subsubsection{La prueba ``Portmanteau''}

La prueba de bondad de ajuste para los residuos de Box-Pierce (1970), la prueba \textit{Portmanteau}, fue extendida a modelos VAR multivariante\index{Prueba!Portmanteau multivariante} por Hosking (1980) y Li-McLeod (1981). Esta prueba determina si las autocorrelaciones residuales, sobre un retardo espec\'{i}fico, son estad\'{i}sticamente nulos.\newline

La hip\'{o}tesis que se prueba es:
\[
H_{0}:R_{l}^{\ast }=\left( R_{1},\mathellipsis ,R_{l} \right)'=0\qquad 
\text{contra}\qquad H_{a}:R_{l}^{\ast }=\left( R_{1},\mathellipsis ,R_{l} \right)'\neq 0
\]

Si no se rechaza la hip\'{o}tesis nula, se puede asumir que los residuos se comportan como un ruido blanco y, por lo tanto, es adecuado el modelo ajustado.\newline

La prueba multivariante \textit{Portmanteau} propuesta por Hosking (1980) considera el estad\'{i}stico:
\[
Q(l)=T\sum_{i=1}^l {tr\left( \hat{C}_{i}^{'}\hat{C}_{0}^{-1}\hat{C}_{i}\hat{C}_{0}^{-1} \right)} 
\]

Este estad\'{i}stico tiene aproximadamente una distribuci\'{o}n Ji-Cuadrada con $k^{2}(l-p)$ grados de libertad bajo la hip\'{o}tesis nula, donde $p $es el orden estimado del modelo VAR (p) y $l$ es el n\'{u}mero de retardos incluidos en la prueba para la significaci\'{o}n total. Ljung-Box (1978) propusieron una modificaci\'{o}n que conduce a propiedades mejores en el caso univariante; Hosking considera una modificaci\'{o}n similar para el caso multivariante. El estad\'{i}stico modificado de la prueba \textit{Portmanteau} est\'{a} dado por:
\[
Q^{'}(l)=T^{2}\sum_{i=1}^l {\left( T-i \right)^{-1}tr\left( \hat{C}_{i}^{'}\hat{C}_{0}^{-1}\hat{C}_{i}\hat{C}_{0}^{-1} \right)}
\]


\subsubsection{Prueba de Breusch - Godfrey o Prueba del Multiplicador de Lagrange (LM)}

Se utiliza para detectar autocorrelaci\'{o}n\index{Prueba!Multiplicador de Lagrange multivariante} de cualquier orden, especialmente en aquellos modelos con o sin variables dependientes retardadas. Permite determinar si existe correlaci\'{o}n en los residuos hasta un determinado orden.\newline

Se realiza la siguiente prueba de hip\'{o}tesis:
\[
H_{0}:\rho_{l}=0,\qquad \text{contra}\qquad H_{a}:\rho_{l}\neq 0,
\]
donde, $l$ es el orden del modelo VAR ajustado.

El estad\'{i}stico utilizado para la prueba es: 
\[
LM=TR^{2}
\]

donde, $T$ el n\'{u}mero de observaciones y $R^{2}$ corresponde a la bondad de ajuste de la regresi\'{o}n auxiliar entre las variables y los residuos.\newline

Este estad\'{\i}stico, bajo $H_{0}$, sigue asint\'{o}ticamente una distribuci\'{o}n Ji-cuadrado con $l$ grados de libertad, $\chi_{l}^{2}$.

\subsubsection{Prueba de Jarque-Bera }

Es una prueba asint\'{o}tica de normalidad para grandes muestras. La prueba de Jarque-Bera (JB) considera\index{Prueba!Jarque-Bera multivariante} la relaci\'{o}n entre los coeficientes de asimetr\'{\i}a y apuntamiento de los residuos de la ecuaci\'{o}n estimada y los correspondientes de una distribuci\'{o}n normal, de forma tal que si estas relaciones son suficientemente diferentes se rechazar\'{a} la hip\'{o}tesis nula de normalidad.

Se realiza la siguiente prueba de hip\'{o}tesis:
\[
\begin{tabular}{l}
 $H_{0}:$ los residuos siguen una distribuci\'{o}n normal multivariante \\ 
 $H_{1}:$ los residuos no siguen una distribuci\'{o}n normal multivariante\\ 
\end{tabular}
\]

Este estad\'{\i}stico se basa en las medidas de apuntamiento \textit{(curtosis)} y la asimetr\'{i}a a trav\'{e}s de la transformaci\'{o}n de Mahalanobis.\newline

La i-\'{e}sima componente del vector de asimetr\'{i}a estimado, se calcula de la siguiente manera:
\[
{as}_{i}=\frac{\displaystyle\frac{1}{T}\sum_{j=1}^T \hat{v}_{ij}^{3} }{\displaystyle\frac{1}{T}\sum_{j=1}^T \left( \hat{v}_{ij}^{2} \right)^{3/2}}=\frac{\displaystyle\sum_{j=1}^T \hat{v}_{ij}^{3} }{\displaystyle\sum_{j=1}^T \left( \hat{v}_{ij}^{2} \right)^{3/2} }
\]

La i-\'{e}sima componente del vector de apuntamiento estimado se calcula de la siguiente manera:
\[
k_{i}=\frac{\displaystyle\frac{1}{T}\sum_{j=1}^T \hat{v}_{ij}^{4} }{\displaystyle\frac{1}{T}\sum_{j=1}^T \left( \hat{v}_{ij}^{2} \right)^{2} }=\frac{\displaystyle\sum_{j=1}^T \hat{v}_{ij}^{4} }{\displaystyle\sum_{j=1}^T \left( \hat{v}_{ij}^{2} \right)^{2} }
\]

$\hat{v}_{ij}$ son los elementos de la matriz $\hat{V}$, que se define de la siguiente manera:
\[
\hat{V}=\hat{U}S_{\hat{U}}^{-1}
\]

donde, $\hat{U}$ es la matriz de los residuos obtenidos a trav\'{e}s de la estimaci\'{o}n de las variables utilizando el m\'{e}todo de m\'{i}nimos cuadrados; mientras que $S_{\hat{U}}$ es una matriz triangular superior tal que:
\[
\hat{U}^{'}\hat{U}=S_{\hat{U}}^{'}S_{\hat{U}}\quad \text{y}\quad {\left( \hat{U}^{'}\hat{U} 
\right)}^{-1}=S_{\hat{U}}^{-1}\left( S_{\hat{U}}^{-1} \right)^{'}
\]

En este caso, $\hat{V}$ es la matriz ortogonalizada de los residuos estimados; es decir, ${as}_{i}$ y $k_{i}$ corresponden a la asimetr\'{i}a y el apuntamiento individual estimados, respectivamente.

Entonces, se define a la asimetr\'{i}a y al apuntamiento estimados de la distribuci\'{o}n de la serie multivariante como:
\[
AS=\left( {as}_{1},\mathellipsis ,{as}_{T} \right)^{'}({as}_{1},\mathellipsis ,{as}_{T})
\]
\[
K=\left( k_{1}-3,\mathellipsis ,k_{T}-3 \right)^{'}(k_{1}-3,\mathellipsis ,k_{T}-3)
\]

El estad\'{i}stico utilizado para la prueba es: 
\[
JB=T\left[ \frac{AS}{6}+\frac{K}{24} \right]
\]

Este estad\'{i}stico se compara con una distribuci\'{o}n Ji-Cuadrada con 2T grados de libertad. 

\section{Predicci\'{o}n}
\label{subsec:mylabel8}

Con los coeficientes\index{Predicci\'{o}n!Modelos VAR} estimados del modelo\index{Modelos VAR!Predicci\'{o}n}, se puede calcular la predicci\'{o}n para un horizonte $h$, dada la informaci\'{o}n hasta el per\'{i}odo T; por ejemplo, para un VAR (1) se tiene:
\[
\hat{X}_{T}(1)=\hat{v}_{0}+\hat{A}_{1}X_{T}
\]
Al horizonte de 2 per\'{i}odos, la predicci\'{o}n es:
\[
\hat{X}_{T}\left( 2 \right)=\hat{v}_{0}+\hat{A}_{1}\hat{X}_{T}\left( 1 \right)=\hat{v}_{0}+\hat{A}_{1}\hat{v}_{0}+\hat{A}_{1}^{2}X_{T}
\]
Al horizonte de 3 per\'{i}odos, la predicci\'{o}n se escribe: 
\[
\hat{X}_{T}\left( 3 \right)=\hat{v}_{0}+\hat{A}_{1}\hat{X}_{T}\left( 2 \right)=\left( I+\hat{A}_{1}+\hat{A}_{1}^{2} 
\right)\hat{v}_{0}+\hat{A}_{1}^{3}X_{T}
\]
\[
\hat{X}_{T}\left( h \right)=\hat{v}_{0}+\hat{A}_{1}\hat{X}_{T}\left( h-1 \right)=\left( I+\hat{A}_{1}+\mathellipsis 
+\hat{A}_{1}^{h-1} \right)\hat{v}_{0}+\hat{A}_{1}^{h}X_{T},\quad h\geq 0
\]

Cuando $h\to \infty $, la previsi\'{o}n tiende a un valor constante (estado estacionario) puesto que $\hat{A}_{1}^{i}\to 0
$ si $i\to \infty $ y existe el l\'{i}mite de $\sum_{j=0}^\infty \hat{A}_{1}^{j} $, que es igual a $\left( I-\hat{A}_{1} \right)^{-1}$. Por tanto:
\[
\hat{X}_{T}\left( h \right)\to \left( I-\hat{A}_{1} \right)^{-1}\hat{v}_{0}\quad \text{cuando}\quad h\to \infty 
\]
El error de predicci\'{o}n al horizonte h viene dado por:
\[
e_{T}\left( h \right)=X_{T+h}-\hat{X}_{T}(h)
\]
En particular, para h$=$1 y h$=$2, se tiene:
\[
e_{T}\left( 1 \right)=u_{T+1}
\]
\[
e_{T}\left( 2 \right)=u_{T+2}+A_{1}u_{T+1}
\]
En general, para el horizonte h, se tiene:
\[
e_{T}\left( h \right)=\displaystyle\sum_{i=0}^{h-1} {A_{1}^{i}u_{T+h-i}} 
\]

La esperanza del error de predicci\'{o}n es nula. La matriz de varianza-covarianza del error de predicci\'{o}n es:
\[
\sum_{u}\left( h \right)=E\left[ \left( \sum_{i=0}^{h-1} {A_{1}^{i}u_{T+h-i}} \right)\left( \sum_{i=0}^{h-1} 
{A_{1}^{i}u_{T+h-i}} \right)^{'} \right]
\]

La varianza-covarianza estimada viene dada por:
\[
\hat{\sum }_{u}\left( h \right)=E\left[ \left( \sum_{i=0}^{h-1} {\hat{A}_{1}^{i}\hat{u}_{T+h-i}} \right)\left( \sum_{i=0}^{h-1} {\hat{A}_{1}^{i}\hat{u}_{T+h-i}} \right)^{'} \right]
\]
Luego,
\[
\hat{\sum }_{T}\left( h \right)=M_{0}\hat{\sum }_{u}{M'}_{0}+M_{1}\hat{\sum }_{u}{M'}_{1}+\mathellipsis {+M}_{h-1}\hat{\sum }_{u}{M'}_{h-1}
\]
donde $M_{i} $son las matrices de la representaci\'{o}n VMA.\newline

Por lo tanto, se tiene:
\[
M_{1}=\hat{A}_{1};\quad M_{2}=\hat{A}_{1}M_{1}+\hat{A}_{2}M_{0}=\hat{A}_{1}^{2}+\hat{A}_{2};
\]
\[
M_{3}=\hat{A}_{1}M_{2}+\hat{A}_{2} M_{1}+\hat{A}_{3} M_{0}=\hat{A}_{1}^{3}+\hat{A}_{1}\hat{A}_{2}+\hat{A}_{2}\hat{A}_{1}+\hat{A}_{3}
\]

La varianza del error de predicci\'{o}n para cada una de las predicciones de las $k$ variables $\left( \hat{\sigma }_{i}^{2}(h) \right)$ se lee sobre la primera diagonal de la matriz $\hat{\Sigma }_{u}(h)$. El intervalo de predicci\'{o}n al nivel $(1-\alpha )$ est\'{a} dado por: $\hat{X}_{iT}\left( h \right)\pm z_{1-\frac{\alpha }{2}}\hat{\sigma}_{i}(h)$ donde $z_{1-\frac{\alpha }{2}}$ es el cuantil de orden (1-$\alpha $/2) de la ley normal.

\begin{ejemplo}
Considerando las series del ejemplo 5.1, se busca modelar en su forma VAR. Sin embargo, las series en el ejemplo 5.1 est\'{a}n 
en niveles y como se pudo observar no son estacionarias; por esta raz\'{o}n, se trabajar\'{a} con las variaciones trimestrales de las series. As\'{i}, se tendr\'{a}: la variaci\'{o}n trimestral del PIB $\left( Y_{1t} \right)$, del CI $\left( Y_{2t} \right)$ y de la DFI$\left( Y_{3t} \right)$ de un pa\'{i}s sudamericano. N\'{o}tese que se ha denotado a las variaciones de 
las series con $Y$; para las variables originales se dejar\'{a} la notaci\'{o}n con $X$. Se trabajar\'{a} con 49 datos, dado que al calcular las variaciones, se pierde el primer dato. La serie Y iniciar\'{a} en la observaci\'{o}n 2 hasta la observaci\'{o}n 50. (Ver Anexo D.1).\newline

Determinar:

\begin{enumerate}
\item[a)] El orden del modelo VAR.
\item[b)] Los par\'{a}metros del modelo.
\item[c)] La predicci\'{o}n para las 6 siguientes observaciones y dar el intervalo de confianza al 95$\%$.
\end{enumerate}
\end{ejemplo}


\textbf{Resoluci\'{o}n.}\newline

Se inicia presentando el gr\'{a}fico de las series:
\begin{figure}[H]
\centering
\includegraphics[width=4.74in,height=3.38in]{Graficos/Cap4-5/STcap422.eps}
\caption{Gr\'{a}fico de las variaciones trimestrales de las series PIB, CI y DFI}
\label{fig22}
\end{figure}

Antes de realizar los procedimientos para estimar los modelos, se debe verificar si las series a ser analizadas son estacionarias; para ello se realiza la prueba de ra\'{i}ces unitarias para cada serie utilizando el programa EViews:

\begin{table}[H]
\centering
\begin{tabular}{p{110pt}p{70pt}ll}\hline\hline
& & t-Statistic & ~~Prob.* \\ \hline \hline
\multicolumn{2}{l}{Augmented Dickey-Fuller test statistic} & -6,689257& ~0,0000 \\ \hline
Test critical values: & 1$\%$ level & -4,161144 & \\
& 5$\%$ level & -3,506374 & \\
& 10$\%$ level & -3,183002 & \\ \hline \hline
\end{tabular}
\caption{Prueba DFA para $Y_{1t}$}
\label{tab14}
\end{table}

\begin{table}[H]
\centering
\begin{tabular}{p{110pt}p{70pt}ll}\hline\hline
& & t-Statistic & ~~Prob.* \\ \hline \hline
\multicolumn{2}{l}{Augmented Dickey-Fuller test statistic} & -6,231136 & ~0,0000 \\ \hline
Test critical values: & 1$\%$ level & -4,161144 & \\
& 5$\%$ level & -3,506374 & \\
& 10$\%$ level & -3,183002 & \\ \hline \hline
\end{tabular}
\caption{Prueba DFA para $Y_{2t}$}
\label{tab15}
\end{table}

\begin{table}[H]
\centering
\begin{tabular}{p{110pt}p{70pt}ll}\hline\hline
& & t-Statistic & ~~Prob.* \\ \hline \hline
\multicolumn{2}{l}{Augmented Dickey-Fuller test statistic} & -7,729086 & ~0,0000 \\ \hline
Test critical values: & 1$\%$ level & -4,161144 & \\
& 5$\%$ level & -3,506374 & \\
& 10$\%$ level & -3,183002 & \\ \hline \hline
\end{tabular}
\caption{Prueba DFA para $Y_{3t}$}
\label{tab16}
\end{table}

Como se puede ver en las tablas \ref{tab14}, \ref{tab15} y \ref{tab16}, las tres series son estacionarias.

\begin{enumerate}
      \item[a)] Se utilizar\'{a}n los criterios de Akaike, Schwarz y el logaritmo de m\'{a}xima verosimilutid para determinar el retardo $p$ entre 1 y 4. Se deben estimar cuatro modelos diferentes y retener aquel que satisfaga la mayor cantidad de criterios \'{o}ptimos.
\end{enumerate}

Inicialmente, se tiene un modelo de la forma, para p$=$1:
\[
\left[ {\begin{array}{*{20}c}
Y_{1t}\\
Y_{2t}\\
Y_{3t}\\
\end{array} } \right]=\left[ {\begin{array}{*{20}c}
a_{1}^{0}\\
a_{2}^{0}\\
a_{3}^{0}\\
\end{array} } \right]+\left[ {\begin{array}{*{20}c}
a_{11}^{1} & a_{11}^{2} & a_{11}^{3}\\
a_{21}^{1} & a_{21}^{2} & a_{21}^{3}\\
a_{31}^{1} & a_{31}^{2} & a_{31}^{3}\\
\end{array} } \right]\left[ {\begin{array}{*{20}c}
Y_{1t-1}\\
Y_{2t-1}\\
Y_{3t-1}\\
\end{array} } \right]+\left[ {\begin{array}{*{20}c}
u_{1t}\\
u_{2t}\\
u_{3t}\\
\end{array} } \right]
\]

Algunos paquetes econom\'{e}tricos utilizan la estimaci\'{o}n de MCO ecuaci\'{o}n por ecuaci\'{o}n ya que s\'{o}lo los valores rezagados de las variables end\'{o}genas aparecen en el lado derecho de la ecuaci\'{o}n, lo que hace que los estimadores sean eficientes.\newline

Se obtiene lo siguiente:
\[
\hat{Y}_{1t}=-0,1168Y_{1t-1}+0,7168Y_{2t-1}-0,0198Y_{3t-1}+0,0115
\]
\[
\hat{Y}_{2t}=-0,0575Y_{1t-1}+0,0674Y_{2t-1}+0,0900Y_{3t-1}+0,0197
\]
\[
\hat{Y}_{3t}=0,1972Y_{1t-1}+0,4192Y_{2t-1}-0,3462Y_{3t-1}+0,0185
\]

Con la ayuda del paquete EViews 7, se realiza la estimaci\'{o}n de los 4 modelos. As\'{i}, se obtuvieron los siguientes resultados:

\begin{table}[H]
\centering
\begin{tabular}{|l|p{30pt}|p{30pt}|p{30pt}|p{30pt}|}\hline
Criterio / Retardo &\quad 1 &\quad 2 &\quad 3 &\quad 4 \\ \hline
Log likelihood & 484,96 & 479,76 & 476,41 & 477,71 \\ \hline
Akaike & -18,92 & -18,72 & -18,60 & -18,67 \\ \hline
Schwarz & -18,46 & -17,91 & -17,43 & -17,13 \\ \hline
\end{tabular}
\caption{Criterios para escoger el retardo del VAR}
\label{tab17}
\end{table}

Como se puede observar en la tabla 5.5, es en el retardo 1 ($p=1)$ donde los criterios de Akaike y Schwarz se minimizan y el valor del \textit{log de verosimilitud} es el m\'{a}ximo. Por lo tanto se realiza la estimaci\'{o}n del $VAR(1)$.\newline

En el caso del programa EViews, para calcular el VAR se debe ingresar el n\'{u}mero de retardos como un rango; por ejemplo, para un $VAR(1)$ se debe especificar como ``1 1'', para un $VAR(4)$ se lo especifica como ``1 4'', ``2 4'', ``3 4'' o ``4 4'', dependiendo del rango de retardos que se requiera en el modelo.

\begin{figure}[H]
\centering
\includegraphics[width=4.26in,height=4.04in]{Graficos/Cap4-5/STcap423.eps}
\caption{Especificaci\'{o}n de un modelo VAR en EViews}
\label{fig23}
\end{figure}

\begin{enumerate}
      \item[b)] El modelo VAR estimado se escribe: 
\end{enumerate}
\[
Y_{1t}=-0,1168Y_{1t-1}+0,7168Y_{2t-1}-0,0198Y_{3t-1}+0,0115+\hat{u}_{1t}
\]
\[
\qquad\qquad(-0,55)\quad\qquad (2,05)\quad\qquad (-0,09)\qquad\qquad\qquad (-1,66)
\]
$R^{2}= 0,09;$ n $=$ 50; (.) $=$ estad\'{i}stico correspondiente a la distribuci\'{o}n t de Student.
\[
Y_{2t}=-0,0575Y_{1t-1}+0,0674Y_{2t-1}+0,0900Y_{3t-1}+0,0197+\hat{u}_{2t}
\]
\[
\qquad\qquad(-0,56)\quad\qquad (0,40)\quad\qquad (0,83)\qquad\qquad\qquad (5,85)
\]
$R^{2}= 0,03;$ n $=$ 50; (.) $=$ estad\'{i}stico correspondiente a la distribuci\'{o}n t de Student.
\[
Y_{3t}=0,1972Y_{1t-1}+0,4192Y_{2t-1}-0,3462Y_{3t-1}+0,0185+\hat{u}_{3t}
\]
\[
\qquad\qquad(0,92)\quad\qquad (1,18)\quad\qquad (-1,52)\qquad\qquad\qquad (2,61)
\]
$R^{2}= 0,06;$ n $=$ 50; (.) $=$ estad\'{i}stico correspondiente a la distribuci\'{o}n t de Student.\newline

Antes de realizar las predicciones, se debe verificar si el modelo cumple con el criterio de estabilidad (las ra\'{i}ces del polinomio caracter\'{i}stico est\'{a}n fuera del circulo unidad). Con ayuda del programa EViews se pueden calcular las \underline {inversas de las ra\'{i}ces} del polinomio caracter\'{i}stico autoregresivo, las que se espera que se encuentren dentro del c\'{i}rculo unidad. As\'{i} se obtiene:

\begin{table}[H]
\centering
\begin{tabular}{p{132pt}l}\hline\hline
~~Root & Modulus \\ \hline\hline
-0,246672 - 0,062904i & ~0,254566 \\
-0,246672 $+$ 0,062904i & ~0,254566 \\
~0,097725 & ~0,097725 \\ \hline\hline
\end{tabular}
\label{tab18}
\end{table}

\begin{figure}[H]
\centering
\includegraphics[width=2.60in,height=2.70in]{Graficos/Cap4-5/STcap424.eps}
\caption{Criterio de estabilidad para el VAR(1) estimado}
\label{fig24}
\end{figure}

Anal\'{i}tica y gr\'{a}ficamente, se concluye que las inversas de las ra\'{i}ces del polinomio caracter\'{i}stico se encuentran dentro del c\'{i}rculo unidad; por lo tanto, se concluye que el modelo es estable y, por tanto, es estacionario.

\begin{enumerate}
      \item[c)] Ahora, se necesita verificar que los residuos del modelo sean ruidos blancos; en general, se prueba la independencia. Para ello, se utilizar\'{a} el paquete EViews para obtener las pruebas sobre los residuos que se describieron anteriormente. As\'{i} se obtiene:
\end{enumerate}

\textbf{Prueba Portmanteau}

\begin{table}[H]
\centering
\begin{tabular}{cccccc}\hline\hline
Lags & Q-Stat & Prob. & Adj Q-Stat &  Prob. & df \\ \hline\hline
1 & ~0,628382 & NA* & ~0,641751 & NA* & NA* \\
2 & ~9,516110 & ~0,3911 & ~9,915902 & ~0,3573 & 9 \\
3 & ~16,74629 & ~0,5406 & ~17,62809 & ~0,4804 & 18 \\
4 & ~25,33301 & ~0,5558 & ~26,99543 & ~0,4640 & 27 \\
5 & ~35,85563 & ~0,4754 & ~38,74160 & ~0,3470 & 36 \\
6 & ~41,19646 & ~0,6338 & ~44,84542 & ~0,4784 & 45 \\
7 & ~51,95004 & ~0,5539 & ~57,43497 & ~0,3491 & 54 \\
8 & ~62,57741 & ~0,4913 & ~70,18781 & ~0,2494 & 63 \\
9 & ~72,52842 & ~0,4604 & ~82,43521 & ~0,1879 & 72 \\
10& ~80,22987 & ~0,5033 & ~92,16336 & ~0,1863 & 81 \\
11& ~88,06759 & ~0,5380 & ~102,3312 & ~0,1763 & 90 \\
12& ~89,89910 & ~0,7324 & ~104,7732 & ~0,3264 & 99 \\
13& ~94,98418 & ~0,8100 & ~111,7471 & ~0,3831 & 108 \\
14& ~99,40624 & ~0,8789 & ~117,9900 & ~0,4570 & 117 \\
15& ~103,7800 & ~0,9263 & ~124,3517 & ~0,5248 & 126 \\
16& ~115,1927 & ~0,8904 & ~141,4708 & ~0,3343 & 135 \\
17& ~118,4536 & ~0,9412 & ~146,5199 & ~0,4259 & 144 \\
18& ~127,4406 & ~0,9348 & ~160,8991 & ~0,3149 & 153 \\
19& ~130,5767 & ~0,9668 & ~166,0899 & ~0,3965 & 162 \\
20& ~136,6630 & ~0,9751 & ~176,5237 & ~0,3701 & 171 \\ \hline \hline
\end{tabular}
\caption{Prueba de autocorrelaci\'{o}n Pormanteau}
\label{tab19}
\end{table}

En la tabla 5.6 se observa que los p-valores (Prob.) para los retardos 2 al 10 son no significativos; de esto se concluye que los residuos no est\'{a}n autocorrelacionados.\newline

\textbf{Prueba LM}

\begin{table}[H]
\centering
\begin{tabular}{ccc}\hline\hline
Lags & LM-Stat & Prob \\ \hline\hline
1 & ~6,309247 & ~0,7086 \\
2 & ~8,996282 & ~0,4376 \\
3 & ~7,384406 & ~0,5972 \\
4 & ~8,563089 & ~0,4785 \\
5 & ~11,86241 & ~0,2212 \\
6 & ~5,615736 & ~0,7777 \\
7 & ~13,14360 & ~0,1562 \\
8 & ~12,15696 & ~0,2046 \\
9 & ~13,95395 & ~0,1240 \\
10& ~9,385154 & ~0,4025 \\
11& ~9,988141 & ~0,3514 \\
12& ~2,713061 & ~0,9746 \\
13& ~5,772031 & ~0,7625 \\
14& ~6,030074 & ~0,7369 \\
15& ~6,067111 & ~0,7332 \\
16& ~16,42881 & ~0,0584 \\
17& ~5,150654 & ~0,8210 \\
18& ~13,63067 & ~0,1361 \\
19& ~4,863835 & ~0,8460 \\
20& ~7,838041 & ~0,5505 \\ \hline\hline
\end{tabular}
\caption{Prueba LM}
\label{tab20}
\end{table}

Por los valores en la columna Prob. (ver tabla 5.7), se puede concluir que no existe correlaci\'{o}n serial. Esto confirma que no hay que reformular el modelo planteado.\newline

\textbf{Pruebas de Normalidad}

\begin{table}[H]
\centering
\begin{tabular}{cccc}\hline\hline
Component & Jarque-Bera & df & Prob. \\ \hline\hline
1 & ~1,142571 & 2 & ~0,5648 \\
2 & ~2,337811 & 2 & ~0,3107 \\
3 & ~0,393216 & 2 & ~0,8215 \\ \hline\hline
Joint & ~3,873599 & 6 & ~0,6938 \\ \hline\hline
\end{tabular}
\caption{Prueba de Normalidad de los residuos}
\label{tab21}
\end{table}

De la prueba de Jarque-Bera se concluye que la distribuci\'{o}n de los residuos es una distribuci\'{o}n normal multivariante.

\begin{observacion}
De los resultados obtenidos en los literales b y c se concluye que el modelo $VAR(1)$ es adecuado para los datos. 
\end{observacion}

\begin{enumerate}
      \item[d)] La predicci\'{o}n calculada por el modelo, de manera recurrente es:
\end{enumerate}
\[
\hat{Y}_{1t}=-0,1168Y_{1t-1}+0,7168Y_{2t-1}-0,0198Y_{3t-1}+0,0115
\]
\[
\hat{Y}_{1:51}=-0,1168\ast 0,0015+0,7168\ast 0,0167-0,0198\ast 0,0142+0,0115
\]
\[
\hat{Y}_{1:51}=0,023
\]
\[
\hat{Y}_{2t}=-0,0575Y_{1t-1}+0,0674Y_{2t-1}+0,0900Y_{3t-1}+0,0197
\]
\[
\hat{Y}_{2:51}=-0,0575\ast 0,0015+0,0674\ast 0,0167+0,0900\ast 0,0142+0,0197
\]
\[
\hat{Y}_{2:51}=0,022
\]
\[
\hat{Y}_{3t}=0,1972Y_{1t-1}+0,4192Y_{2t-1}-0,3462Y_{3t-1}+0,0185
\]
\[
\hat{Y}_{3:51}=0,1972\ast 0,0015+0,4192\ast 0,0167-0,3462\ast 0,0142+0,0185
\]
\[
\hat{Y}_{3:51}=0,021
\]
donde, $\hat{Y}_{i:j}$ significa, la previsi\'{o}n de la variable $Y_{i}$ para el per\'{i}odo j.

\begin{observacion}
A pesar que se trabaja con 49 datos, la \'{u}ltima observaci\'{o}n es la n\'{u}mero 50, por los motivos explicados en el enunciado del ejemplo; es por esto, que la primera observaci\'{o}n a predecir es la de orden 51 aunque en realidad corresponder\'{i}a al dato 50 de una serie temporal que inicie en el instante t$=$1.
\end{observacion}

De la misma manera se obtiene:
\[
\hat{Y}_{1:52}=0,024
\]
\[
\hat{Y}_{2:52}=0,022
\]
\[
\hat{Y}_{3:52}=0,025
\]
\[
\hat{Y}_{1:53}=0,023
\]
\[
\hat{Y}_{2:53}=0,022
\]
\[
\hat{Y}_{3:53}=0,023
\]
\[
\vdots 
\]
Para calcular la varianza del error de predicci\'{o}n, si fuera un VAR (1) se tiene:
\[
M_{1}=\hat{A}_{1};\quad M_{2}=\hat{A}_{1}M_{1}=\hat{A}_{1}^{2};\quad etc\mathellipsis.
\]
Para el caso del ejemplo, se obtiene:
\[
\hat{A}_{1}=\left[ {\begin{array}{*{20}c}
-0,1170 & 0,7168 & -0,0198\\
-0,0575 & 0,0674 & 0,0900\\
0,1972 & 0,04192 & -0,3462\\
\end{array} } \right]
\]
La matriz de varianza covarianza estimada de la predicci\'{o}n, para el horizonte $h=$\textit{1,} es: 
\[
\hat{\Sigma }_{T}\left( 1 \right)=\hat{\Sigma }_{u}=\left[ 
{\begin{array}{*{20}c}
0,0002 & 0,0001 & 0,0002\\
0,0001 & 0,0001 & 0,0001\\
0,0002 & 0,0001 & 0,0002\\
\end{array} } \right]
\]

As\'{i}, la varianza del error de predicci\'{o}n para $\hat{Y}_{1:51}$ es igual a 0,0002, la varianza del error de predicci\'{o}n para $\hat{Y}_{2:51}$ es igual a 0,0001 y la varianza del error de predicci\'{o}n para $\hat{Y}_{3:51}$ es igual a 0,0002.\newline

Los intervalos de confianza para $Y_{1:51}$, $Y_{2:51}$ y $Y_{3:51}$ vienen dados, respectivamente, por:
\[
0,024\pm 1,96\ast \sqrt {0,0002} =\left[ -0,007;0,053 \right]
\]
\[
0,022\pm 1,96\ast \sqrt {0,0001} =\left[ 0,007;0,037 \right]
\]
\[
0,025\pm 1,96\ast \sqrt {0,0002} =\left[ -0,009;0,051 \right]
\]

Para los horizontes $h=2$, $h=3$ se utilizan las siguientes f\'{o}rmulas:
\[
\hat{\sum }_{T}\left( 2 \right)=\hat{\sum }_{u}+\hat{A}_{1}\hat{\sum}_{u}\hat{A}_{1}^{'}=\left[ {\begin{array}{*{20}c}
0,0003 & 0,0001 & 0,0002\\
0,0001 & 0,0001 & 0,0001\\
0,0002 & 0,0001 & 0,0003\\
\end{array} } \right]
\]
\[
\hat{\sum }_{T}\left( 3 \right)=\hat{\sum }_{u}+\hat{A}_{1}\hat{\sum}_{u}\hat{A}_{1}^{'}+\hat{A}_{1}^{2}\hat{\sum }_{u}\hat{A}_{1}^{2'}=\left[ 
{\begin{array}{*{20}c}
0,0005 & 0,0001 & 0,0004\\
0,0001 & 0,0001 & 0,0001\\
0,0004 & 0,0001 & 0,0005\\
\end{array} } \right]
\]

Entonces los intervalos de confianza son:
\[
IC\left( Y_{1:52} \right)=0,024\pm 1,96\ast 0,016=\left[-0,007;0,056 \right]
\]
\[
IC\left( Y_{2:52} \right)=0,022\pm 1,96\ast 0,008=\left[0,007;0,037 \right]
\]
\[
IC\left( Y_{3:52} \right)=0,025\pm 1,96\ast 0,016=\left[-0,006;0,056 \right]
\]
\[
IC\left( Y_{1:52} \right)=0,024\pm 1,96\ast 0,023=\left[-0,020;0,068 \right]
\]
\[
IC\left( Y_{1:52} \right)=0,022\pm 1,96\ast 0,011=\left[0,001;0,043 \right]
\]
\[
IC\left( Y_{1:52} \right)=0,024\pm 1,96\ast 0,023=\left[-0,020;0,068 \right]
\]
\[
\vdots 
\]

\textbf{Comparaci\'{o}n con modelos univariantes}\newline

La teor\'{i}a VAR sugiere que las predicciones logradas son de mejor calidad que si se realiza la modelaci\'{o}n de las series de manera univariante. Para comprobar esto, se realiz\'{o} un modelo univariante para cada una de las series analizadas en este documento. As\'{i}, para las variaciones PIB se encontr\'{o} el modelo $Y_{1t}=0,94Y_{1t-5}+\hat{u}_{t}-0,87\hat{u}_{1t-5}$; para la variaci\'{o}n del CI se estim\'{o} el modelo $Y_{2t}=0,97Y_{2t-7}+\hat{u}_{t}-0,88\hat{u}_{2t-7}$ y para la variaci\'{o}n de la DFI se obtuvo el modelo $Y_{3t}=0,36Y_{3t-2}+0,53Y_{3t-6}+\hat{u}_{t}$. 

\begin{figure}[H]
\centering
\includegraphics[width=4.43in,height=3.40in]{Graficos/Cap4-5/STcap425.eps}
\caption{Comparaci\'{o}n de las predicciones VAR y UNIVARIANTE para $Y_{1t}$ (PIB)}
\label{fig25}
\end{figure}

\begin{figure}[H]
\centering
\includegraphics[width=4.43in,height=3.40in]{Graficos/Cap4-5/STcap426.eps}
\caption{Comparaci\'{o}n de las predicciones VAR y UNIVARIANTE para $Y_{2t}$ (CI) }
\label{fig26}
\end{figure}

\begin{figure}[H]
\centering
\includegraphics[width=4.43in,height=3.40in]{Graficos/Cap4-5/STcap427.eps}
\caption{Comparaci\'{o}n de las predicciones VAR y UNIVARIANTE para $Y_{3t}$ (DFI)}
\label{fig27}
\end{figure}

Se puede observar que el ajuste que tienen las predicciones del modelo VAR; pero, para comparar anal\'{i}ticamente se calcula del error cuadr\'{a}tico medio estimado, para determinar el mejor ajuste. As\'{i} se tiene:

\begin{table}[H]
\centering
\begin{tabular}{|l|c|c|c|}\hline
~& PIB & CI & DFI \\ \hline
VAR & 0,0005 & 0,0005 & 0,0006 \\ \hline
UNIVARIANTE & 0,0004 & 0,0004 & 0,0004 \\\hline
\end{tabular}
\caption{Error Medio Cuadr\'{a}tico estimado para los modelos VAR y univariante}
\label{tab22}
\end{table}

Como se puede observar, los errores son muy peque\~{n}os para ambos casos; la diferencia existente es estad\'{i}sticamente no significativa entre los dos tipos de predicciones. Esto se explica porque las correlaciones cruzadas entre las series son muy poco significativas; sin embargo, el objetivo de esta presentaci\'{o}n, es m\'{a}s bien, did\'{a}ctico; posteriormente se mostrar\'{a} un ejemplo m\'{a}s completo.

\section{La Causalidad}
\label{subsec:mylabel9}

En la teor\'{i}a, la demostraci\'{o}n de relaciones causales entre las variables de an\'{a}lisis proporciona los elementos de reflexi\'{o}n propicios para una mejor compresi\'{o}n de los fen\'{o}menos, sobre todo los econ\'{o}micos. De manera pr\'{a}ctica, ``\textbf{\textit{the causal knowlegedge}}'' (``el conocimiento de la causalidad\index{Causalidad}'') es 
necesario en la formulaci\'{o}n correcta de la pol\'{i}tica econ\'{o}mica. En efecto, saber la direcci\'{o}n de la causalidad es tambi\'{e}n importante en cuanto a poner un enlace entre las variables econ\'{o}micas.

\subsection{La causalidad seg\'{u}n Granger}
\label{subsubsec:mylabel12}

Granger (1969) propuso los conceptos de causalidad\index{Causalidad!Granger} y de exogeneidad: en el sentido de series de tiempo, se dir\'{i}a que la variable $X_{2t}$ es la causa de $X_{1t}$ , si la predicci\'{o}n de $X_{1t}$ mejora si la informaci\'{o}n relativa a $X_{2t}$ se incorpora al an\'{a}lisis (\textit{el t\'{e}rmino predicci\'{o}n parece preferible en el marco de la causalidad; en efecto, decir que }$Y_{t}$\textit{ causa }$X_{t}$\textit{ , solo significa que es preferible para predecir }$X_{t}$\textit{ conocer }$Y_{t}$\textit{ , que no conocerla)}. 

Sea el modelo $VAR(p)$ para el cual las variables $X_{1t}$ y $X_{2t}$ son estacionarias:

\begin{align*}
\left[ {\begin{array}{c}
X_{1t}\\
X_{2t}\\
\end{array} } \right] & =\left[ {\begin{array}{c}
a_{0}\\
b_{0}\\
\end{array} } \right]+\left[ {\begin{array}{cc}
a_{1}^{1} & b_{1}^{1}\\
a_{1}^{2} & b_{1}^{2}\\
\end{array} } \right]\left[ {\begin{array}{c}
X_{1t-1}\\
X_{2t-1}\\
\end{array} } \right]+\left[ {\begin{array}{cc}
a_{2}^{1} & b_{2}^{1}\\
a_{2}^{2} & b_{2}^{2}\\
\end{array} } \right]\left[ {\begin{array}{c}
X_{1t-2}\\
X_{2t-2}\\
\end{array} } \right]+\mathellipsis +\\
& \mathellipsis + \left[ {\begin{array}{cc}
a_{P}^{1} & b_{P}^{1}\\
a_{P}^{2} & b_{P}^{2}\\
\end{array} } \right]\left[ {\begin{array}{c}
X_{1t-P}\\
X_{2t-P}\\
\end{array} } \right]+\left[ {\begin{array}{c}
u_{1t}\\
u_{2t}\\
\end{array} } \right]
\end{align*}


El bloque de variables $(X_{2t-1},X_{2t-2},\mathellipsis ,X_{2t-p})$ se considera como ex\'{o}geno en comparaci\'{o}n del bloque de variables $(X_{1t-1}, X_{1t-2},\mathellipsis ,X_{1t-p})$ si el hecho de a\~{n}adir el bloque $X_{2t}$ no mejora significativamente la determinaci\'{o}n de las variables $X_{1t}$. Se trata de efectuar una prueba de restricciones sobre los coeficientes de las variables $X_{2t}$ de la representaci\'{o}n VAR (Se denotar\'{a} por RVAR al modelo VAR restringido). 
La determinaci\'{o}n del retardo $p $se efect\'{u}a por los criterios de Akaike, Shwarz o Hanan-Quinn. 

\begin{itemize}
\item $X_{2t}$ no causa $X_{1t}$ , si se acepta la siguiente hip\'{o}tesis:
\[
H_{0}:b_{1}^{1}=b_{2}^{1}=\mathellipsis =b_{p}^{1}=0
\]
\item $X_{1t}$ no causa $X_{2t}$ , si se acepta la siguiente hip\'{o}tesis:
\[
H_{0}:a_{1}^{2}=a_{2}^{2}=\mathellipsis =a_{p}^{2}=0
\]
\end{itemize}

Si se llegan a aceptar las dos hip\'{o}tesis: $X_{1t}$ causa a $X_{2t}$ y viceversa, se habla de efectos de retroalimentaci\'{o}n (``\textbf{\textit{feedback effect}}'').\newline 

Estas pruebas pueden llevarse a cabo con la ayuda de la prueba cl\'{a}sica de Fisher de la nulidad de los coeficientes (prueba de Wald), ecuaci\'{o}n por ecuaci\'{o}n o bien directamente comparando un modelo VAR sin restricciones (UVAR) y el modelo VAR restringido (RVAR).\newline

Se calcula el siguiente estad\'{i}stico:
\[
L^{\ast }=\left( T-c \right)(\ln \left| \Sigma_{RVAR} \right|-\ln \left| \Sigma_{UVAR} \right|)\sim \chi^{2}(2p)
\]
donde:

$\Sigma_{RVAR}\quad =$ matriz de varianzas-covarianzas de los residuos del modelo restringido,\newline
$\Sigma_{UVAR}\quad =$ matriz de varianzas-covarianzas de los residuos del modelo sin restricciones,\newline
T $=$ n\'{u}mero de observaciones,\newline
c $=$ n\'{u}mero de par\'{a}metros estimados de cada ecuaci\'{o}n del modelo sin restricciones.\newline
Si $L^{\ast }>\chi_{1-\alpha }^{2}(2p)$, entonces se rechaza la hip\'{o}tesis de la validez de la restricci\'{o}n con un nivel de significaci\'{o}n $\alpha $.\newline

Tambi\'{e}n se suele utilizar la prueba tradicional de Fisher:
\[
F^{\ast }=\frac{\frac{SCRR-SCRU}{c}}{\frac{SCRU}{n-k-1}}
\]

SRCU: es la suma de cuadrados de los residuos del modelo sin restricciones.\newline
SRRR: es la suma de cuadrados de los residuos del modelo restringido.\newline
C$=$ n\'{u}mero de restricciones (n\'{u}mero de coeficientes que pone a prueba la hip\'{o}tesis nula).
Si $F^{\ast }>F_{c;n-k-1}^{\alpha }$ se rechaza la hip\'{o}tesis nula.

\subsection{La causalidad seg\'{u}n Sims}
\label{subsubsec:mylabel13}
Sims (1980) presenta una especificaci\'{o}n\index{Causalidad!Sims} de prueba ligeramente diferente. Se considera que si los valores futuros de $X_{1t}$ permiten explicar los valores presentes de $X_{2t}$, entonces $X_{2t}$ es la causa de $X_{1t}$ . 

Esto se representa de la siguiente manera:
\[
X_{1t}=a_{1}^{0}+\sum_{i=1}^p {a_{1i}^{1}X_{1t-i}} +\sum_{i=1}^p {a_{1i}^{2}X_{2t-i}} +\sum_{i=1}^p {b_{i}^{2}X_{2t+i}} +u_{1t}
\]
\[
X_{2t}=a_{2}^{0}+\sum_{i=1}^p {a_{2i}^{1}X_{1t-i}} +\sum_{i=1}^p {a_{2i}^{2}X_{2t-i}} +\sum_{i=1}^p {b_{i}^{1}X_{1t+i}} +u_{2t}
\]

\begin{itemize}
      \item $X_{1t}$ no causa $X_{2t}$ , si se aceptan la siguiente hip\'{o}tesis:
\[
H_{0}:b_{1}^{2}=b_{2}^{2}=\mathellipsis =b_{p}^{2}=0
\]
      \item $X_{2t}$ no causa $X_{1t}$ , si se aceptan la siguiente hip\'{o}tesis:
\[
H_{0}:b_{1}^{1}=b_{2}^{1}=\mathellipsis =b_{p}^{1}=0
\]
\end{itemize}

Se sigue utilizando la prueba cl\'{a}sica de Fisher de nulidad de coeficientes.

\begin{observacion}
La prueba de Sims, se deja de como referencia dado que no se puede estimar directamente en EViews. Se puede crear un c\'{o}digo de programaci\'{o}n dentro del programa para poder realizar la estimaci\'{o}n, pero esto est\'{a} fuera del alcance de este documento. 
\end{observacion}

\begin{ejemplo}
A partir de la representaci\'{o}n VAR (1) estimada en el ejemplo 5.5, se va a proceder a realizar las pruebas de Granger.
\end{ejemplo}

\textbf{Resoluci\'{o}n.}\newline

Se procede con una prueba de Fisher, ecuaci\'{o}n por ecuaci\'{o}n.\newline

\textbf{Prueba de Granger}\newline

\begin{itemize}
      \item Ho\textbf{: }$Y_{2t}$ y $Y_{3t}$ no causan $Y_{1t}$

Se han estimado los siguientes modelos:
\[
Y_{1t}=-0,1168Y_{1t-1}+0,7168Y_{2t-1}-0,0198Y_{3t-1}+0,0115+\hat{u}_{1t}
\]
$R^{2}=0,09$;\quad $n=48$; $SCRU=0,010381$ (sin restricciones)
\[
Y_{1t}=-0,0197Y_{1t-1}+0,025+\hat{u}_{1t}
\]
$R^{2}=0,0004$;\quad $n=48$; \quad $SCRR=0,011474$ (restringido)

donde,

SRCU: es la suma de cuadrados de los residuos del modelo sin restricciones.\newline
SRRR: es la suma de cuadrados de los residuos del modelo restringido.
\[
F^{\ast }=\frac{\frac{SCRR-SCRU}{c}}{\frac{SCRU}{n-k-1}}=\frac{\frac{\mathrm{0,011474-0,010381}}{1}}{\frac{\mathrm{0,010381}}{48-3-1}}=4,6326
\]
c$=$ n\'{u}mero de restricciones (n\'{u}mero de coeficientes que pone a prueba la hip\'{o}tesis nula); en este caso c$=$1, dado que se elimina un coeficiente de cada ecuaci\'{o}n.\newline

$F^{\ast }>F_{1;48}^{0,05}\approx 4,05;$ por tanto, se rechaza la hip\'{o}tesis nula; $Y_{2t}$ y $Y_{3t}$ explica significativamente la variable $Y_{1t}$; existe causalidad seg\'{u}n Granger.

      \item Ho\textbf{: }$Y_{1t}$ y $Y_{3t}$ no causan $Y_{2t}$

Se han estimado los siguientes modelos: 
\[
Y_{2t}=-0,0575Y_{1t-1}+0,0674Y_{2t-1}+0,0900Y_{3t-1}+0,0197+\hat{u}_{2t}
\]
$R^{2}=0,0026$;\quad $n=48$;\quad $SCRU=0,002594$ (sin restricciones)
\[
Y_{2t}=0,1014Y_{2t-1}+0,0199+\hat{u}_{2t}
\]
$R^{2}=0,011$;\quad $n=48$;\quad $SCRR=0,002638$ (restringido)
\[
F^{\ast }=\frac{\frac{SCRR-SCRU}{c}}{\frac{SCRU}{n-k-1}}=\frac{\frac{0,002638-\mathrm{0,002594}}{1}}{\frac{\mathrm{0,002594}}{48-3-1}}=0,7463
\]
$F^{\ast }<F_{1;104}^{0,05}\approx 4,05;$ por tanto, se acepta la hip\'{o}tesis nula; $Y_{1t}$y $Y_{3t}$ no explican significativamente la variable $Y_{2t}$; no existe causalidad seg\'{u}n Granger.

      \item Ho\textbf{: }$Y_{1t}$ y $Y_{2t}$ no causan $Y_{3t}$

Se han estimado los siguientes modelos:
\[
Y_{3t}=0,1972Y_{1t-1}+0,4192Y_{2t-1}-0,3462Y_{3t-1}+
0,0185+\hat{u}_{3t}
\]
$R^{2}=0,0110$;\quad $n=48$;\quad $SCRU=0,011036$ (sin restricciones)
\[
Y_{3t}=-0,1394Y_{3t-1}+0,0285+\hat{u}_{3t}
\]
$R^{2}=0,011$;\quad $n=48$;\quad $SCRR=0,011487$ (restringido)
\[
F^{\ast }=\frac{\frac{SCRR-SCRU}{c}}{\frac{SCRU}{n-k-1}}=\frac{\frac{\mathrm{0,011487}-\mathrm{0,011036}}{1}}{\frac{\mathrm{0,011036}}{48-3-1}}=1,7981
\]
$F^{\ast }<F_{1;104}^{0,05}\approx 4,05;$ por tanto, se acepta la hip\'{o}tesis nula; $Y_{1t}$y $Y_{2t}$ no explican significativamente la variable $Y_{3t}$; no existe causalidad seg\'{u}n Granger.
\end{itemize}

El siguiente gr\'{a}fico muestra la salida del paquete EViews para la Prueba de Granger. Hay que considerar que el paquete considera la prueba de Wald y compara el valor obtenido con el estad\'{i}stico Chi-cuadrado.

\begin{table}[H]
\centering
\begin{tabular}{cccc}
\multicolumn{4}{l}{~Dependent variable: $Y_{1t}$} \\ \hline\hline
~Excluded~ & ~Chi-sq~ & ~df~ & ~Prob.~ \\ \hline\hline
$Y_{2t}$ & ~3.972118 & 1 & ~0.0463 \\ 
$Y_{3t}$ & ~0.007279 & 1 & ~0.9320 \\ \hline\hline
All & ~4.631706 & 2 & ~0.0987 \\ \hline\hline
\end{tabular}
\end{table}

\begin{table}[H]
\centering
\begin{tabular}{cccc}
\multicolumn{4}{l}{~Dependent variable: $Y_{2t}$} \\ \hline\hline
~Excluded~ & ~Chi-sq~ & ~df~ & ~Prob.~ \\ \hline\hline
$Y_{2t}$ & ~0.557515 & 1 & ~0.4553 \\ 
$Y_{3t}$ & ~0.661477 & 1 & ~0.4160 \\ \hline\hline
All & ~0.746494 & 2 & ~0.6885 \\ \hline\hline
\end{tabular}
\end{table}

\begin{table}[H]
\centering
\begin{tabular}{cccc}
\multicolumn{4}{l}{~Dependent variable: $Y_{3t}$} \\ \hline\hline
~Excluded~ & ~Chi-sq~ & ~df~ & ~Prob.~ \\ \hline\hline
$Y_{2t}$ & ~0.236725 & 1 & ~0.6266 \\ 
$Y_{3t}$ & ~1.415327 & 1 & ~0.2342 \\ \hline\hline
All & ~1.799419 & 2 & ~0.4067 \\ \hline\hline
\end{tabular}
\caption{Prueba de causalidad de Granger}
\end{table}

Como se puede ver, el p-valor (Prob.) es mayor que 0,05 en todos los casos, esto implica que no existe causalidad entre las variables. Sin embargo, al realizar el c\'{a}lculo inicial se concluy\'{o} que si existe causalidad de $Y_{3t}$ y $Y_{2t}$ hacia $Y_{1t}$. La diferencia se origina en los algoritmos de c\'{a}lculo que tienen los paquetes implementados.

\section{An\'{a}lisis de los ``choques''}
%\label{subsec:mylabel10}
El an\'{a}lisis de los choques consiste en medir el impacto de la variaci\'{o}n de una innovaci\'{o}n sobre las variables.\newline

Consid\'{e}rese el modelo estimado del ejemplo 5.5:
\[
Y_{1t}=-0,1168Y_{1t-1}+0,7168Y_{2t-1}-0,0198Y_{3t-1}+0,0115+\hat{u}_{1t}
\]
\[
Y_{2t}=-0,0575Y_{1t-1}+0,0674Y_{2t-1}+0,0900Y_{3t-1}+0,0197+\hat{u}_{2t}
\]
\[
Y_{3t}=0,1972Y_{1t-1}+0,4192Y_{2t-1}-0,3462Y_{3t-1}+0,0185+\hat{u}_{3t}
\]

Una variaci\'{o}n en un instante dado de $\hat{u}_{1t}$ tiene una consecuencia inmediata sobre $Y_{1t}$, y entonces sobre $Y_{1,t+1}$ y $Y_{2,t+1}$ ; por ejemplo, si se produce en \textit{t} un choque sobre $\hat{u}_{1t}$ igual a 1 y de orden 0 sobre $\hat{u}_{2t}$ (las otras variables permanecen iguales que en el tiempo ($t-1)$; para los valores subsiguientes, $u_{t+i}$ retorna a su valor $u_{t-1})$, se tiene los impactos siguientes:\newline

En el per\'{i}odo \textit{t}: 
\[
\left[ {\begin{array}{*{20}c}
\Delta Y_{1t}\\
\Delta Y_{2t}\\
\Delta Y_{3t}\\
\end{array} } \right]=\left[ {\begin{array}{*{20}c}
1\\
0\\
0\\
\end{array} } \right]
\]

En el per\'{i}odo \textit{t}$+$\textit{1}:
\[
\left[ {\begin{array}{*{20}c}
\Delta \hat{Y}_{1,t+1}\\
\Delta \hat{Y}_{2,t+1}\\
\Delta \hat{Y}_{3,t+1}\\
\end{array} } \right]=\left[ {\begin{array}{*{20}c}
-0,1170 & 0,7168 & -0,0198\\
-0,0575 & 0,0674 & 0,0900\\
0,1972 & 0,04192 & -0,3462\\
\end{array} } \right]\left[ {\begin{array}{*{20}c}
1\\
0\\
0\\
\end{array} } \right]=\left[ {\begin{array}{*{20}c}
-0,1170\\
-0,0575\\
0,1972\\
\end{array} } \right]
\]

En el per\'{i}odo \textit{t}$+$\textit{2}:
\[
\left[ {\begin{array}{*{20}c}
\Delta \hat{Y}_{1,t+2}\\
\Delta \hat{Y}_{2,t+2}\\
\Delta \hat{Y}_{3,t+2}\\
\end{array} } \right]=\left[ {\begin{array}{*{20}c}
-0,1170 & 0,7168 & -0,0198\\
-0,0575 & 0,0674 & 0,0900\\
0,1972 & 0,04192 & -0,3462\\
\end{array} } \right]\left[ {\begin{array}{*{20}c}
-0,1170\\
-0,0575\\
0,1972\\
\end{array} } \right]=\left[ {\begin{array}{*{20}c}
-0,0315\\
0,0206\\
-0,1154\\
\end{array} } \right]
\]
%\[
%\vdots 
%\]
Por otro lado, utilizando la representaci\'{o}n lineal del VAR, se logra realizar el an\'{a}lisis de las funciones de impulso-respuesta, ya que un choque en $Y_{1t}$ se reflejar\'{a} como la primera columna de $M_{i}=A_{1}^{i}$ (recuerde que $M_{0}=I)$; de forma similar, un impacto sobre $Y_{2t}$ se reflejar\'{i}a en la segunda columna de la matriz 
$M_{i}$. La generalizaci\'{o}n a un $VAR$ con $k$ variables, es inmediata.\newline

Considere el modelo estimado en el ejemplo 5.5:
\[
\hat{A}_{1}=\left[ {\begin{array}{*{20}c}
-0,1170 & 0,7168 & -0,0198\\
-0,0575 & 0,0674 & 0,0900\\
0,1972 & 0,04192 & -0,3462\\
\end{array} } \right]
\]
\[
\hat{A}_{1}^{2}=\left[ {\begin{array}{*{20}c}
-0,0315 & -0,0437 & 0,0737\\
0,0206 & 0,0011 & -0,0240\\
-0,1154 & 0,0245 & 0,1567\\
\end{array} } \right]
\]
%\[
%\vdots 
%\]
As\'{i}, los elementos de $\hat{A}_{1}^{i}$ representan los efectos de los choques unitarios de las variables del sistema luego de $i$ per\'{i}odos. Es por esta raz\'{o}n que, se les conoce como \textbf{\textit{multiplicadores din\'{a}micos}}\index{Multiplicadores!Din\'{a}micos}\textbf{\textit{ o respuestas al impulso.}}\newline

En las tablas siguientes se muestran los resultados de las funciones de impulso-respuesta que presenta el paquete EViews para las variables $Y_{1t}$, $Y_{2t}$ y $Y_{3t}$, que son los c\'{a}lculos hechos manualmente al inicio de esta secci\'{o}n. Esto tambi\'{e}n se muestra en los gr\'{a}ficos siguientes:

\begin{table}[H]
\centering
\begin{tabular}{cccc}\hline\hline
~Per\'{i}odo & $Y_{1t}$ & $Y_{2t}$ & $Y_{3t}$\\ \hline\hline
~1 & ~1,000000 & ~0,000000 & ~0,000000 \\
~2 & -0,193069 & -0,080383 & ~0,108038 \\
~3 & -0,016093 & ~0,019189 & -0,090439 \\
~4 & ~0,014628 & -0,005392 & ~0,036103 \\
~5 & -0,005860 & ~0,001663 & -0,012563 \\
~6 & ~0,002041 & -0,000533 & ~0,004198 \\
~7 & -0,000682 & ~0,000173 & -0,001384 \\
~8 & ~0,000225 & -5,66E-05 & ~0,000454 \\
~9 & -7,38E-05 & ~1,85E-05 & -0,000149 \\
~10& ~2,42E-05 & -6,06E-06 & ~4,88E-05 \\ \hline\hline
\end{tabular}
\caption{Respuesta de las variables $Y_{1t}$, $Y_{2t}$ y $Y_{3t}$ ante un choque unitario de $Y_{1t}$}
\label{tab24}
\end{table}

\begin{figure}[H]
\centering
\includegraphics[width=3.11in,height=6.36in]{Graficos/Cap4-5/STcap428.eps}
\caption{Respuesta de las variables $Y_{1t}$, $Y_{2t}$ y $Y_{3t}$ ante un choque unitario de $Y_{1t}$}
\label{fig28}
\end{figure}

\begin{table}[H]
\centering
\begin{tabular}{cccc}\hline\hline
~Per\'{i}odo & $Y_{1t}$ & $Y_{2t}$ & $Y_{3t}$\\ \hline\hline
~1 & ~0.000000 & ~1.000000 & ~0.000000 \\
~2 & ~0.689275 & ~0.075128 & ~0.424214 \\
~3 & -0.073293 & -0.011642 & -0.032978 \\
~4 & ~0.005504 & ~0.002053 & -0.002027 \\
~5 & ~0.000315 & -0.000470 & ~0.002131 \\
~6 & -0.000345 & ~0.000131 & -0.000865 \\
~7 & ~0.000140 & -4.02E-05 & ~0.000303 \\
~8 & -4.91E-05 & ~1.29E-05 & -0.000101 \\
~9 & ~1.65E-05 & -4.18E-06 & ~3.34E-05 \\
~10& -5.43E-06 & ~1.36E-06 & -1.10E-05 \\ \hline\hline
\end{tabular}
\caption{Respuesta de las variables $Y_{1t}$, $Y_{2t}$ y $Y_{3t}$ ante un choque unitario de $Y_{2t}$}
\label{tab25}
\end{table}

%\[
%RespuestadeY_{2t}paraY_{2t}
%RespuestadeY_{3t}paraY_{2t}
%RespuestadeY_{1t}paraY_{2t}
%\]

\begin{figure}[H]
\centering
\includegraphics[width=3.36in,height=6.54in]{Graficos/Cap4-5/STcap429.eps}
\caption{Respuesta de las variables $Y_{1t}$, $Y_{2t}$ y $Y_{3t}$ ante un choque unitario de $Y_{2t}$.}
\label{fig29}
\end{figure}

\begin{table}[H]
\centering
\begin{tabular}{cccc}\hline\hline
~Per\'{i}odo & $Y_{1t}$ & $Y_{2t}$ & $Y_{3t}$ \\ \hline\hline
~1 & ~0.000000 & ~0.000000 & ~1.000000 \\
~2 & ~0.018858 & ~0.089860 & -0.328412 \\
~3 & ~0.052104 & -0.024276 & ~0.148011 \\
~4 & -0.024001 & ~0.007288 & -0.053278 \\
~5 & ~0.008653 & -0.002311 & ~0.017996 \\
~6 & -0.002924 & ~0.000748 & -0.005955 \\
~7 & ~0.000968 & -0.000244 & ~0.001957 \\
~8 & -0.000318 & ~7.98E-05 & -0.000642 \\
~9 & ~0.000104 & -2.61E-05 & ~0.000210 \\
~10& -3.42E-05 & ~8.55E-06 & -6.88E-05 \\ \hline\hline
\end{tabular}
\caption{Respuesta de las variables $Y_{1t}$, $Y_{2t}$ y $Y_{3t}$ ante un choque unitario de $Y_{3t}$}
\label{tab26}
\end{table}

%\[
%RespuestadeY_{1t}paraY_{3t}
%RespuestadeY_{2t}paraY_{3t}
%RespuestadeY_{3t}paraY_{3t}
%\]

\begin{figure}[H]
\centering
\includegraphics[width=3.36in,height=6.65in]{Graficos/Cap4-5/STcap430.eps}
\label{fig30}
\end{figure}

La elecci\'{o}n de la direcci\'{o}n del impacto es muy importante y determina los valores obtenidos. Se puede observar que el efecto de la innovaci\'{o}n se desvanece con el tiempo; esto caracteriza a un proceso VAR estacionario.

\begin{itemize}
      \item[i.] Si las variables est\'{a}n medidas en escalas diferentes es com\'{u}n considerar las innovaciones iguales a su desviaci\'{o}n t\'{i}pica en lugar de los choques unitarios; es decir:
      \[
      \hat{u}_{t,0}=\sqrt {Var(\hat{u}_{t})} 
      \]
      Esto constituye \'{u}nicamente un reescalamiento de las funciones de impulso-respuesta.
      \item Si una variables no causa (en el sentido de Granger) al resto de variables en el sistema, entonces las respuestas al impulso sobre las otras variables ser\'{a}n cero.
\end{itemize}

\section{Descomposici\'{o}n de la varianza}
%\label{subsec:mylabel11}
\subsection{Representaci\'{o}n de errores ortogonales }
%\label{subsubsec:mylabel14}

Un problema a considerar en el an\'{a}lisis de las funciones de impulso-respuesta es el de la correlaci\'{o}n contempor\'{a}nea de errores y, por lo tanto, el impacto de un choque sobre una variable puede acompa\~{n}arse de un impacto en otra variable; ignorarla, puede distorsionar la verdadera relaci\'{o}n din\'{a}mica entre las variables. Es por esto que se trata de manera general de realizar el an\'{a}lisis a trav\'{e}s de la b\'{u}squeda de una representaci\'{o}n de errores ortogonales. \newline

Considerando la representaci\'{o}n lineal del VAR, se puede obtener lo siguiente:

\begin{enumerate}
      \item Dado que $\Sigma_{u}$ es sim\'{e}trica y definida positiva, entonces existe $P$ no singular tal que: 
      \[
      \Sigma_{u}=PP'
      \]
      donde, P es una matriz triangular obtenida a trav\'{e}s de la descomposici\'{o}n de Cholesky.
      
      \item Luego, se puede expresar lo siguiente:
      \[
      v_{t}=Pu_{t}
      \]
      donde, $v_{t}$ es un vector aleatorio con $E\left( v_{t} \right)=0\quad y \quad \mathrm{V}\left( v_{t} \right)=I$.

Entonces, $v_{t}$ son las \textit{innovaciones ortogonales}\index{Innovaci\'{o}n!Ortogonal} \quad de $u_{t}$. Luego, reemplazando: 
\[
X_{t}=\sum_{i=1}^\infty {\theta_{i}Pu_{t-i}} =\sum_{i=1}^\infty {M_{i}u_{t-i}},\quad M_{i}\equiv \theta_{i}P
\]

Entonces, $\theta_{0}=P$; por lo que, salvo el caso que $\Sigma_{u}$ sea diagonal, $\Sigma_{v}$ no ser\'{a} diagonal y sus elementos recoger\'{a}n las respuestas inmediatas del sistema de choques unitarios. Es por esto que se los conoce como \textbf{\textit{multiplicadores de impacto}}\index{Multiplicadores!De impacto}\textbf{\textit{. }}Adem\'{a}s, el hecho que $\theta_{0}=P$ sea una matriz triangular, implica que el orden de las variables en el vector es importante (\'{o}rdenes diferentes de descomposici\'{o}n pueden producir funciones de impluso respuesta diferentes). 

\subsection{Descomposici\'{o}n de la Varianza}
%\label{subsubsec:mylabel15}
La descomposici\'{o}n de la varianza del error de predicci\'{o}n tiene como objetivo calcular para cada una de las innovaciones su contribuci\'{o}n a la varianza del error. Por la t\'{e}cnica matem\'{a}tica de la descomposici\'{o}n de Cholesky\index{Descomposici\'{o}n de Cholesky} de la matriz $\Sigma_{u}$ que es sim\'{e}trica y definida positiva, se puede escribir la varianza del error de predicci\'{o}n para un horizonte $h$ en funci\'{o}n de la varianza del error atribuida a cada una de las variables; es suficiente dividir cada una de estas variaciones por la varianza total, para obtener su peso relativo en porcentaje.\newline

Se retoma el modelo VAR (1) de dos variables $X_{1t}$ y $X_{2t}$ La varianza del error de predicci\'{o}n para $X_{1,t+h}$ se puede escribir:
\[
\sigma_{x_{h}}^{2}=\sigma_{u_{1}}^{2}\left[ m_{11}^{2}\left( 0 \right)+\mathellipsis +m_{11}^{2}\left( h-1 \right) \right]+\sigma_{u_{2}}^{2}\left[ m_{22}^{2}\left( 0 \right)+\mathellipsis +m_{22}^{2}\left( h-1 \right) \right]
\]

Donde los $m_{ii}^{(j)}$ son los t\'{e}rminos de las matrices $M_{i} $(representaci\'{o}n lineal del proceso).

Al horizonte $h$, la descomposici\'{o}n de la varianza, en porcentaje, de las innovaciones propias de $X_{1t}$ sobre $X_{1t}$, est\'{a}n dadas por:
\[
\frac{\sigma_{u_{1}}^{2}\left[ m_{11}^{2}\left( 0 \right)+\mathellipsis +m_{11}^{2}\left( h-1 \right) \right]}{\sigma_{X_{1}}^{2}(h)}\ast 100
\]
Y la descomposici\'{o}n de la varianza, en porcentaje, de $X_{2t}$ sobre $X_{1t}$
\[
\frac{\sigma_{u_{2}}^{2}\left[ m_{22}^{2}\left( 0 \right)+\mathellipsis +m_{22}^{2}\left( h-1 \right) \right]}{\sigma_{X_{1}}^{2}(h)}\ast 100
\]

La interpretaci\'{o}n de los resultados es importante:
\begin{itemize}
      \item Si un choque sobre $u_{1t}$ no afecta la varianza del error de $X_{2t}$ independientemente del horizonte de predicci\'{o}n, entonces $X_{2t}$ puede considerarse como ex\'{o}gena porque $X_{2t}$ evoluciona independientemente de $u_{1t}$.
      \item En caso contrario, si un choque sobre $u_{1t}$ afecta fuertemente, en realidad totalmente, la varianza del error de $X_{2t}$, entonces $X_{2t}$ se considera como end\'{o}gena.
\end{itemize}

\subsection{Elecci\'{o}n del orden de descomposici\'{o}n}
%\label{subsubsec:mylabel16}
N\'{o}tese que el problema de la correlaci\'{o}n contempor\'{a}nea de los errores y, por lo tanto, el impacto de un choque sobre una variable, implica una elecci\'{o}n de descomposici\'{o}n que proporciona resultados asim\'{e}tricos en funci\'{o}n del orden de las variables. El problema es m\'{a}s complejo si el n\'{u}mero de las variables es importante.

%\textbf{Ejemplo 5.7:} 
\begin{ejemplo}
Consid\'{e}rese un modelo VAR con 4 variables $X_{1}, X_{2}, X_{3}$ y $X_{4}$.
\end{ejemplo}

Sup\'{o}ngase que se elige el siguiente orden para la descomposici\'{o}n de Cholesky: $X_{2} X_{3} X_{1} X_{4}$ y esto lleva a obtener la siguiente tabla hipot\'{e}tica:

\begin{table}[H]
\centering
\begin{tabular}{|c|c|c|c|c|}\hline
\multicolumn{5}{|c|}{Ordenaci\'{o}n de Cholesky: $X_{2} X_{3} X_{1} X_{4}$}\\ \hline
\multicolumn{5}{|c|}{Respuesta de $X_{2}$}\\ \hline
Per\'{i}odo & $X_{1}$ & $X_{2}$ & $X_{3}$ & $X_{4}$\\ \hline
1 & 0,000000 & 4, 583291 & 0,000000 & 0,000000\\ \hline
2 & 0,775767 & -0,251545 & 0,815017 & -0,905811\\ \hline
\multicolumn{5}{|c|}{Respuesta de $X_{3}$}\\ \hline
Per\'{i}odo & $X_{1}$ & $X_{2}$ & $X_{3}$ & $X_{4}$\\ \hline
1 & 0,000000 & 2,439203 & 4,54469 & 0,000000\\ \hline
2 & -0,441257 & -0,754324 & -3,564595 & -0,566602 \\ \hline
\multicolumn{5}{|c|}{Respuesta de $X_{1}$}\\ \hline
Per\'{i}odo & $X_{1}$ & $X_{2}$ & $X_{3}$ & $X_{4}$\\ \hline
1 & 4,603662 & 3,022459 & 1,802923 & 0,000000\\ \hline
2 & -0,795786 & -0,101091 & 0,388095 & -0,255476\\ \hline
\multicolumn{5}{|c|}{Respuesta de $X_{4}$}\\ \hline
Per\'{i}odo & $X_{1}$ & $X_{2}$ & $X_{3}$ & $X_{4}$\\ \hline
1 & 1,568803 & 0,486351 & 1,415711 & 3,191513\\ \hline
2 & -0,329818 & -4,104756 & -0,527459 & -3,186928 \\\hline
\end{tabular}
\caption{Descomposici\'{o}n hipot\'{e}tica de choques}
\label{tab27}
\end{table}

La interpretaci\'{o}n de la tabla 5.14 se har\'{i}a de la siguiente manera:
\begin{itemize}
      \item Un choque para el per\'{i}odo 1 sobre $X_{2}$ tiene un impacto solamente sobre $X_{2}$ y ausencia de correlaci\'{o}n contempor\'{a}nea con $X_{3}, X_{1}$ y $X_{4}$ .
      \item Un choque para el per\'{i}odo 1 sobre $X_{3}$ tiene un impacto sobre $X_{2}$ y $X_{3}$ y ausencia de correlaci\'{o}n contempor\'{a}nea con $X_{1}$ y $X_{4}$ .
      \item Un choque para el per\'{i}odo 1 sobre $X_{1}$ tiene un impacto sobre $X_{2}, X_{3}$ y $X_{1}$ y ausencia de correlaci\'{o}n contempor\'{a}nea con $X_{4}$ .
      \item Finalmente, un choque para el per\'{i}odo 1 sobre $X_{4}$ tiene un impacto sobre todas las variables.
\end{itemize}

Ahora, si se realiza el an\'{a}lisis considerando un orden diferente de las variables para la descomposici\'{o}n, se tendr\'{a}:

\begin{table}[H]
\centering
\begin{tabular}{|c|c|c|c|c|}\hline
\multicolumn{5}{|c|}{Ordenaci\'{o}n de Cholesky: $X_{3} X_{4} X_{2} X_{1}$} \\ \hline
\multicolumn{5}{|c|}{Respuesta de $X_{3}$} \\ \hline
Per\'{i}odo & $X_{1}$ & $X_{2}$ & $X_{3}$ & $X_{4}$ \\ \hline
1 & 0,000000 & 0,000000 & 2,805461 & 0,000000 \\ \hline
2 & -0,527459 & -3,186928 & 0,815017 & -0,905811 \\ \hline
\multicolumn{5}{|c|}{Respuesta de $X_{4}$} \\ \hline
Per\'{i}odo & $X_{1}$ & $X_{2}$ & $X_{3}$ & $X_{4}$ \\ \hline
1 & 0,000000 & 0,000000 & -2,231567 & 1,256043 \\ \hline
2 & -0,441257 & -0,754324 & 0,388095 & -0,255476 \\ \hline
\multicolumn{5}{|c|}{Respuesta de $X_{2}$} \\ \hline
Per\'{i}odo & $X_{1}$ & $X_{2}$ & $X_{3}$ & $X_{4}$ \\ \hline
1 & 0,000000 & -0,022459 & 1,256923 & -2,256123 \\ \hline
2 & 0,815017 & -0,905811 & 0,388095 & -0,255476 \\ \hline
\multicolumn{5}{|c|}{Respuesta de $X_{1}$} \\ \hline
Per\'{i}odo & $X_{1}$ & $X_{2}$ & $X_{3}$ & $X_{4}$ \\ \hline
1 & 1,457895 & -2,145627 & 1,711415 & 3,564281 \\ \hline
2 & -0,441257 & -0,754324 & -0,524784 & -3,968741 \\ \hline
\end{tabular}
\caption{Otra descomposici\'{o}n hipot\'{e}tica de choques}
\label{tab28}
\end{table}

La interpretaci\'{o}n de la tabla 5.15, se realiza de manera similar a la tabla 5.10.\newline

El orden de descomposici\'{o}n se deber\'{i}a efectuar desde la variable que se supone es m\'{a}s ex\'{o}gena hasta la variable menos ex\'{o}gena. En caso de duda, es necesario realizar diferentes combinaciones del orden de descomposici\'{o}n y analizar la robustez de los resultados.

\begin{ejemplo}
A partir de la representaci\'{o}n $VAR(1)$ estimada en el ejemplo 5.5, calcular e interpretar las funciones de impulso-respuesta ortogonales, y la descomposici\'{o}n de la varianza.
\end{ejemplo}

\textbf{Resoluci\'{o}n.}\newline

Dado que las variables del ejemplo son variaciones del PIB, el CI y el DFI, es l\'{o}gico pensar que un choque sobre la variable variaci\'{o}n del PIB influencie la variaci\'{o}n del CI y el DFI, m\'{a}s que si el choque fuera al rev\'{e}s. Esto ser\'{i}a: una innovaci\'{o}n sobre $Y_{1t}$ (variaci\'{o}n del PIB) influencia de manera instant\'{a}nea a $Y_{2t}$ (variaci\'{o}n del CI) y a $Y_{3t}$ (variaci\'{o}n del DFI); por otro lado, una innovaci\'{o}n sobre $Y_{2t}$ o $Y_{3t}$ no influencia de manera contempor\'{a}nea a $Y_{1t}$.

La matriz de varianza-covarianza estimada de los residuos es igual (ejemplo 5.5) a:
\[
\hat{\Sigma }_{u}=\left[ {\begin{array}{*{20}c}
0,0002 & 0,0001 & 0,0002\\
0,0001 & 0,0001 & 0,0001\\
0,0002 & 0,0001 & 0,0002\\
\end{array} } \right]
\]

Con el programa EViews, las salidas de las funciones de impulso-respuesta y la descomposici\'{o}n de varianza ser\'{i}an:

\begin{table}[H]
\centering
\begin{tabular}{ccccc}\hline\hline
\multicolumn{5}{c}{Variance Decomposition of $Y_{1}$:} \\ 
Per\'{i}odo & S.E. & $Y_{1}$ & $Y_{2}$ & $Y_{3}$ \\ \hline\hline
~1 & ~0,015360 & ~43,03994 & ~21,09375 & ~35,86631 \\
~2 & ~0,016092 & ~40,67934 & ~25,73053 & ~33,59013 \\
~3 & ~0,016102 & ~40,63720 & ~25,71162 & ~33,65118 \\
~4 & ~0,016104 & ~40,63653 & ~25,70823 & ~33,65524 \\
~5 & ~0,016104 & ~40,63654 & ~25,70806 & ~33,65540 \\
~6 & ~0,016104 & ~40,63655 & ~25,70805 & ~33,65540 \\
~7 & ~0,016104 & ~40,63655 & ~25,70805 & ~33,65540 \\
~8 & ~0,016104 & ~40,63655 & ~25,70805 & ~33,65540 \\ 
~9 & ~0,016104 & ~40,63655 & ~25,70805 & ~33,65540 \\
~10& ~0,016104 & ~40,63655 & ~25,70805 & ~33,65540 \\ \hline\hline
\end{tabular}
\end{table}

\begin{table}[H]
\centering
\begin{tabular}{ccccc}\hline\hline
\multicolumn{5}{c}{Variance Decomposition of $Y_{2}$:} \\ 
~Per\'{i}odo & S.E. & $Y_{1}$ & $Y_{2}$ & $Y_{3}$ \\ \hline\hline
~1 & ~0,007678 & ~0,000000 & ~100,0000 & ~0,000000 \\
~2 & ~0,007778 & ~1,084622 & ~98,63426 & ~0,281119 \\
~3 & ~0,007784 & ~1,144767 & ~98,54459 & ~0,310640 \\
~4 & ~0,007784 & ~1,149508 & ~98,53672 & ~0,313774 \\
~5 & ~0,007784 & ~1,149959 & ~98,53593 & ~0,314109 \\
~6 & ~0,007784 & ~1,150005 & ~98,53585 & ~0,314145 \\
~7 & ~0,007784 & ~1,150010 & ~98,53584 & ~0,314149 \\
~8 & ~0,007784 & ~1,150010 & ~98,53584 & ~0,314149 \\
~9 & ~0,007784 & ~1,150011 & ~98,53584 & ~0,314149 \\
~10& ~0,007784 & ~1,150011 & ~98,53584 & ~0,314149 \\ \hline\hline
\end{tabular}
\end{table}

\begin{table}[H]
\centering
\begin{tabular}{ccccc}\hline\hline
\multicolumn{5}{c}{Variance Decomposition of $Y_{3}$:} \\ 
~Per\'{i}odo & S.E. & $Y_{1}$ & $Y_{2}$ & $Y_{3}$ \\ \hline\hline
~1 & ~0,015837 & ~0,000000 & ~34,49091 & ~65,50909 \\
~2 & ~0,016226 & ~0,450218 & ~33,21246 & ~66,33733 \\
~3 & ~0,016293 & ~0,759358 & ~33,02585 & ~66,21479 \\
~4 & ~0,016303 & ~0,808238 & ~33,01068 & ~66,18109 \\
~5 & ~0,016304 & ~0,814151 & ~33,00932 & ~66,17653 \\
~6 & ~0,016305 & ~0,814812 & ~33,00919 & ~66,17600 \\
~7 & ~0,016305 & ~0,814883 & ~33,00917 & ~66,17594 \\
~8 & ~0,016305 & ~0,814891 & ~33,00917 & ~66,17594 \\ 
~9 & ~0,016305 & ~0,814892 & ~33,00917 & ~66,17594 \\
~10& ~0,016305 & ~0,814892 & ~33,00917 & ~66,17594 \\ \hline\hline
\multicolumn{5}{c}{Cholesky Ordering: $Y_{2} \quad Y_{2} \quad Y_{1}$} \\ \hline\hline
\end{tabular}
\caption{Descomposici\'{o}n de la Varianza (Orden de Cholesky $Y_{3}$ $Y_{2}$ $Y_{1}$)}
\label{tab29}
\end{table}

La descomposici\'{o}n de la varianza indica que la varianza del error de predicci\'{o}n de $Y_{1t}$ representa un 40,64{\%} con sus propias innovaciones, un 25,71{\%} con las de $Y_{2t}$ y un 33,66{\%} con las de $Y_{3t}$. La varianza del error de predicci\'{o}n de $Y_{2t}$ es de un 1,15{\%} con $Y_{1t}$, un 98,54{\%} con $Y_{2t}$ y un 0,31{\%} con $Y_{3t}$.  Finalmente, la varianza del error de predicci\'{o}n de $Y_{3t}$ es de 0,81{\%} con $Y_{1t}$, un 33,01{\%} con $Y_{2t}$ y un 66,18{\%} con sus propias innovaciones. Este efecto de asimetr\'{i}a se estudi\'{o} en la parte de la causalidad (ejemplo 5.5); lo que tambi\'{e}n se corrobora ahora.\newline

Por \'{u}ltimo, cabe se\~{n}alar que la tabla anterior muestra la desviaci\'{o}n est\'{a}ndar del error de previsi\'{o}n para $Y_{1t}$, $Y_{2t}$ y $Y_{3t}$, que se calcul\'{o} de manera tediosa en el ejemplo 5.5.\newline

\begin{observacion}
La modelizaci\'{o}n VAR se realiza \underline{siempre} sobre series estacionarias; sin embargo, el m\'{e}todo se puede aplicar a series que se las vuelve estacionarias a trav\'{e}s de diferenciaci\'{o}n, con la idea de que se debe recuperar la serie original una vez calculadas las predicciones (esto se puede hacer dado que la diferenciaci\'{o}n es una transformaci\'{o}n lineal). En algunos paquetes econom\'{e}tricos como EViews o Stata, entre otros, el programa da la opci\'{o}n autom\'{a}tica de recuperar la serie original de los datos. 
\end{observacion}

\section{Ejemplo Pr\'{a}ctico}
%\label{subsec:mylabel12}
Se consideran dos series de datos del Ecuador: las variaciones del \'{i}ndice de precios al productor (IPP), denotada por $\left( X_{1t} \right)$ y del \'{i}ndice de actividad econ\'{o}mica (IAE), denotada por $\left( X_{2t} \right)$. Se dispone de 132 datos mensuales desde enero de 2004 hasta junio de 2015 (Ver Anexo D.2). Para efectos de comparaciones se trabajar\'{a} \'{u}nicamente con los datos hasta diciembre de 2014 y se guardar\'{a}n los del a\~{n}o 2015. Los datos se tomaron de la p\'{a}gina oficial del INEC. Se desea estimar un modelo para realizar las predicciones de ambas series.\newline

\textbf{Resoluci\'{o}n.}

\begin{enumerate}
      \item[1.] {\bf Matrices de correlación cruzada} 
\end{enumerate}

\begin{itemize}
      \item[a)] Estad\'{i}sticos descriptivos de $X_{1t}$ y $X_{2t}$
\begin{table}[H]
\centering
\begin{tabular}{cccccccc}\hline
~& Media & Mediana & M\'{a}ximo & M\'{i}nimo & Desv. Est. & Asimetr\'{i}a & ~Curtosis \\ \hline
$X_{1t}$ & 0,64 & ~1,16 & 15,17 & -14,86 & 5,02 & -0,40 & 3,74 \\
$X_{2t}$ & 0,07 & -0,30 & 15,18 & -11,45 & 4,42 & 0,51 & 3,70 \\ \hline
\end{tabular}
\label{tab30}
\end{table}

      \item[b)] Matrices de correlaci\'{o}n cruzada

\begin{table}[H]
\centering
\begin{tabular}{ccccccccccc}\hline
& \multicolumn{2}{c}{retardo 1} & \multicolumn{2}{c}{retardo 2} & \multicolumn{2}{c}{retardo 3} & 
\multicolumn{2}{c}{retardo 4} & \multicolumn{2}{c}{retardo 5} \\ \hline
$X_{1t}$ & 0,17 & 0,12 & 0,08 & -0,02 & 0,01 & 0,08 & -0,12 & -0,03 & -0,03 & -0,09 \\
$X_{2t}$ & 0,21 & -0,20 & 0,05 & 0,08 & 0,14 & 0,02 & 0,13 & -0,09 & 0,01 & 0,08 \\ \hline
\end{tabular}
\label{tab31}
\end{table}

      \item[c)] Representación simplificada

\begin{table}[H]
\centering
\left|\begin{tabular}{cc}
. & . \\
+ & - \\
\end{tabular} \right| \quad \left|
\begin{tabular}{cc}
. & . \\
. & . \\
\end{tabular} \right| \quad \left|
\begin{tabular}{cc}
. & . \\
. & . \\
\end{tabular} \right| \quad \left|
\begin{tabular}{cc}
. & . \\
. & . \\
\end{tabular} \right| \quad \left|
\begin{tabular}{cc}
. & . \\
. & . \\
\end{tabular} \right|
\caption{Resumen de estadísticas y matrices de correlación cruzada para $X_{1t}$ y $X_{2t}$.}
\label{tab32}
\end{table}
\end{itemize}

Es f\'{a}cil ver que las correlaciones cruzadas son significativas al 5{\%} en el retardo 1 ($2/\sqrt T =$ 0,1747 en este caso). As\'{i}, $X_{2t}$ depende de los valores en el primer retardo de $X_{1t}$ en el primer retardo y del suyo propio. En EViews, se presenta la siguiente salida:

\begin{figure}[H]
\centering
\includegraphics[width=4.54in,height=4.47in]{Graficos/Cap4-5/STcap431.eps}
\caption{Correlaciones cruzadas entre $X_{1t}$ y $X_{2t}$}
\label{fig31}
\end{figure}

\begin{enumerate}
\item[2.] {\bf Estimaci\'{o}n y validaci\'{o}n del VAR}
\end{enumerate}
Se inicia presentando el gr\'{a}fico de las series:

\begin{figure}[H]
\centering
\includegraphics[width=5.23in,height=3.76in]{Graficos/Cap4-5/STcap432.eps}
\caption{Gr\'{a}fico de secuencia de las series}
\label{fig32}
\end{figure}

Antes de realizar los procedimientos para estimar los modelos, se debe verificar si las series a ser analizadas son estacionarias; para ello se realiza la prueba de ra\'{i}ces unitarias para cada serie utilizando el programa EViews:

\begin{table}[H]
\centering
\begin{tabular}{p{120pt}p{60pt}p{50pt}l} \hline \hline
& & t-Statistic & Prob.* \\ \hline \hline
\multicolumn{2}{p{180pt}}{Augmented Dickey-Fuller test statistic} & -9.001098 & 0.0000 \\ \hline
Test critical values: & 1{\%} level & -4.037668 & \\ 
 & 5{\%} level & -3.448348 & \\ 
 & 10{\%} level & -3.149326 & \\ \hline \hline
\multicolumn{4}{l}{*MacKinnon (1996) one-sided p-values.} \\
\end{tabular}
\caption{Prueba DFA para $X_{1t}$}
\label{tab33}
\end{table}

\begin{table}[H]
\centering
\begin{tabular}{p{120pt}p{60pt}p{50pt}l} \hline \hline
& & t-Statistic & Prob.* \\ \hline \hline
\multicolumn{2}{p{180pt}}{Augmented Dickey-Fuller test statistic} & -13.00154 & 0.0000 \\ \hline
Test critical values: & 1{\%} level & -4.037668 & \\ 
 & 5{\%} level & -3.448348 & \\ 
 & 10{\%} level & -3.149326 & \\ \hline \hline
\multicolumn{4}{l}{*MacKinnon (1996) one-sided p-values.} \\
\end{tabular}
\caption{Prueba DFA para $X_{2t}$}
\label{tab34}
\end{table}

Como se puede ver en las figuras 5.18 y 5.19, las dos series son estacionarias.
\begin{enumerate}
      \item[a)] Se utilizar\'{a}n los criterios de Akaike, Schwarz y el logaritmo de m\'{a}xima verosimilutid para determinar el retardo $p$ entre 1 y 4. Se deben estimar cuatro modelos diferentes y retener aquel que satisfaga la mayor cantidad de criterios \'{o}ptimos.
\end{enumerate}

Se obtiene lo siguiente:
\[
\hat{X}_{1t}=0,1479X_{1t-1}+0,2268X_{2t-1}+0,4928
\]
\[
\hat{X}_{2t}=0,1274X_{1t-1}-0,2187X_{2t-1}-0,0772
\]
Con la ayuda del paquete EViews 7, se realiza la estimaci\'{o}n de los 4 modelos. As\'{i}, se obtuvieron los siguientes resultados:

\begin{table}[H]
\centering
\begin{tabular}{|l|p{35pt}|p{35pt}|p{35pt}|p{35pt}|}\hline
Criterio / Retardo &\quad 1 &\quad 2 &\quad 3 &\quad 4 \\ \hline
Log likelihood & -761,42 & -756,05 & -746,56 & -736,01 \\ \hline
Akaike & ~11,81 & ~11,87 & ~11,88 & ~11,87 \\ \hline
Schwarz & ~11,93 & ~12,10 & ~12,20 & ~12,28 \\ \hline
\end{tabular}
\caption{Criterios para escoger el retardo del VAR}
\label{tab35}
\end{table}

Como se puede observar en la tabla 5.18, es en el retardo 1 ($p=1)$ donde los criterios de Akaike y Schwarz se minimizan aunque el valor del log de versoimilitud es m\'{i}nimo (en la pr\'{a}ctica, muy pocas veces se utiliza este criterio como decisivo para escoger el retardo del VAR). Por lo tanto se realiza la estimaci\'{o}n del $VAR(1)$.

\begin{enumerate}
      \item[i.] El modelo VAR estimado se escribe:
      \[
      X_{1t}=0,1479X_{1t-1}+0,2268X_{2t-1}+0,4928+\hat{u}_{1t}
      \]
      \[
      \quad (1,71)\qquad\quad (2,28)\qquad\quad (1,14)
      \]
      $R^{2} = 0,07$; \quad $n=130$;\quad (.) $=$ estad\'{i}stico correspondiente a la distribuci\'{o}n t de Student.
      \[
      X_{2t}=0,1274X_{1t-1}-0,2187X_{2t-1}-0,0772+\hat{u}_{2t}
      \]
      \[
      \quad (1,67)\qquad\quad (-2,49)\qquad\quad (-0,20)
      \]
      $R^{2} = 0,06$; \quad $n=130$;\quad (.) $=$ estad\'{i}stico correspondiente a la distribuci\'{o}n t de Student.\newline
      
      Antes de realizar las predicciones, se debe verificar si el modelo cumple con el criterio de estabilidad. Con ayuda del programa EViews se obtiene:

\begin{table}[H]
\centering
\begin{tabular}{p{120pt}l}\hline\hline
Ra\'{i}ces & M\'{o}dulo \\ \hline\hline
-0,285373 & ~0,285373 \\
~0,214590 & ~0,214590 \\ \hline\hline
\end{tabular}
\label{tab36}
\end{table}

\begin{figure}[H]
\centering
\includegraphics[width=2.53in,height=2.67in]{Graficos/Cap4-5/STcap433.eps}
\caption{Criterio de estabilidad para el VAR(1) estimado}
\label{fig33}
\end{figure}

Anal\'{i}tica y gr\'{a}ficamente, se concluye que las inversas de las ra\'{i}ces del polinomio caracter\'{i}stico se encuentran dentro del c\'{i}rculo unidad; por lo tanto, se concluye que el modelo es estable y, por tanto, es estacionario.

      \item[ii.] Ahora, se necesita verificar que los residuos del modelo sean ruidos blancos; en general, se prueba la independencia. Para ello, se utilizar\'{a} el paquete EViews para obtener las pruebas sobre los residuos que se describieron anteriormente. As\'{i} se obtiene:
\end{enumerate}

\textbf{Prueba Portmanteau}

\begin{table}[H]
\centering
\begin{tabular}{cccccc}\hline\hline
Lags & Q-Stat & Prob & Adj Q-Stat & Prob & df \\ \hline \hline
1 & ~0,050722 & NA* & ~0,051115 & NA* & NA* \\
2 & ~1,429068 & ~0,9641 & ~1,450997 & ~0,9627 & 6 \\
3 & ~6,825196 & ~0,7418 & ~6,974593 & ~0,7278 & 10 \\
4 & ~14,04034 & ~0,4467 & ~14,41879 & ~0,4190 & 14 \\
5 & ~15,45813 & ~0,6303 & ~15,89329 & ~0,6000 & 18 \\
6 & ~24,47227 & ~0,3230 & ~25,34360 & ~0,2809 & 22 \\
7 & ~26,60556 & ~0,4302 & ~27,59829 & ~0,3785 & 26 \\
8 & ~34,51039 & ~0,2609 & ~36,02148 & ~0,2074 & 30 \\
9 & ~37,10211 & ~0,3279 & ~38,80597 & ~0,2619 & 34 \\
10& ~51,10065 & ~0,0760 & ~53,97106 & ~0,0447 & 38 \\
11& ~53,74575 & ~0,1057 & ~56,86065 & ~0,0627 & 42 \\
12& ~68,78477 & ~0,0164 & ~73,42907 & ~0,0062 & 46 \\
13& ~72,41862 & ~0,0207 & ~77,46668 & ~0,0076 & 50 \\
14& ~76,83463 & ~0,0223 & ~82,41566 & ~0,0077 & 54 \\
15& ~77,52667 & ~0,0443 & ~83,19797 & ~0,0167 & 58 \\
16& ~80,19277 & ~0,0599 & ~86,23825 & ~0,0226 & 62 \\
17& ~82,43977 & ~0,0832 & ~88,82330 & ~0,0321 & 66 \\
18& ~86,47531 & ~0,0883 & ~93,50741 & ~0,0318 & 70 \\
19& ~90,20916 & ~0,0968 & ~97,88039 & ~0,0330 & 74 \\
20& ~93,29485 & ~0,1141 & ~101,5271 & ~0,0380 & 78 \\ \hline\hline
\end{tabular}
\caption{Prueba de autocorrelaci\'{o}n Pormanteau}
\label{tab37}
\end{table}

En la tabla 5.21 se observa que los p-valores (Prob.) para los retardos de 1 al 9 son no significativos; sin embargo, a partir del retardo 10 se vuelven significativos, esto sugiere que los residuos est\'{a}n autocorrelacionados.\newline

\textbf{Prueba LM}

\begin{table}[H]
\centering
\begin{tabular}{ccc}\hline\hline
Lags & LM-Stat & Prob \\ \hline\hline
1 & ~0,600602 & ~0,9630 \\
2 & ~1,492304 & ~0,8280 \\
3 & ~5,496276 & ~0,2401 \\
4 & ~7,679129 & ~0,1041 \\
5 & ~1,468612 & ~0,8322 \\
6 & ~9,278191 & ~0,0545 \\
7 & ~2,213578 & ~0,6965 \\
8 & ~8,222604 & ~0,0838 \\
9 & ~2,633746 & ~0,6209 \\
10& ~14,83753 & ~0,0051 \\
11& ~2,909367 & ~0,5731 \\
12& ~16,83962 & ~0,0021 \\
13& ~4,069454 & ~0,3967 \\
14& ~4,565627 & ~0,3348 \\
15& ~0,731412 & ~0,9474 \\
16& ~2,790990 & ~0,5934 \\
17& ~2,346395 & ~0,6723 \\
18& ~4,123520 & ~0,3895 \\
19& ~3,978227 & ~0,4090 \\
20& ~3,335910 & ~0,5033 \\\hline\hline
\end{tabular}
\caption{Prueba LM}
\label{tab38}
\end{table}

Por los valores en la columna Prob. (ver tabla 5.20), se puede concluir que existe autocorrelaci\'{o}n entre los residuos (retardos 10 y 12). Esto confirma que se hay que reformular el modelo planteado.\newline

\textbf{Reformulaci\'{o}n del modelo}

Se puede ver en las figuras 5.3 y 5.4 que a partir del retardo 10 existe autocorrelaci\'{o}n de los residuos, por lo que se prueba un nuevo modelo VAR (10) para corregir este inconveniente.\newline

Se estim\'{o} un VAR(10); sin embargo, al retardo 12 y 6 se ten\'{i}a autocorrelaci\'{o}n de residuos. Luego, se agregaron los retardos de orden 6 al VAR pero se encontr\'{o} autocorrelaci\'{o}n al retardo 12. Finalmente, se agreg\'{o} el retardo 12 y se consigui\'{o} que los residuos no est\'{e}n correlacionados, pero el problema era que no segu\'{i}an una distribuci\'{o}n normal multivariante. \newline

Luego, se procedi\'{o} a agregar un retardo de orden 14 y a quitar el retardo 12; con esto se consigui\'{o} que los residuos sean ruidos blancos, aunque presenta algo de correlaci\'{o}n en el retardo 12, se decidi\'{o} conservar este modelo ya que es el que cumple con m\'{a}s pruebas de independencia de residuos. Se debe mencionar que solamente se retienen los coeficientes significativos. As\'{i}, se obtuvo el siguiente modelo:

\begin{align*}
X_{1t} &= 0,1416X_{1t-1}-0,2403X_{1t-6}+0,0889X_{1t-10}-0,1047X_{1t-14}+0,2241X_{2t-1}\\
       & +0,0326X_{2t-6}-0,1393X_{2t-10}+0,1082X_{2t-14}+0,7716+\hat{u}_{1t}
\end{align*}

\begin{align*}
X_{2t} &= 0,0703X_{1t-1}-0,1787X_{1t-6}-0,1133X_{1t-10}-0,1893X_{1t-14}-0,2875X_{2t-1}\\
       & -0,0605X_{2t-6}-0,2330X_{2t-10}-0,0422X_{2t-14}+0,3525+\hat{u}_{2t}
\end{align*}

\begin{enumerate}
      \item[iii.] La predicci\'{o}n calculada por el modelo, de manera recurrente es:
      
\begin{align*}
\hat{X}_{1t} &= 0,1416X_{1t-1}-0,2403X_{1t-6}+0,0889X_{1t-10}-0,1047X_{1t-14}+0,2241X_{2t-1}\\
             & +0,0326X_{2t-6}-0,1393X_{2t-10}+0,1082X_{2t-14}+0,7716
\end{align*}
\begin{align*}
\hat{X}_{1,15:1} &= 0,1416\ast 4,15-0,2403\ast(3,26)+0,0889\ast(-1,34)-0,1047\ast(-3,37)\\
                 & +0,2241\ast(-9,23)+0,0326\ast(4,71)-0,1393\ast(-3,12)+0,1082\ast(4,03)\\
                 & +0,7716
\end{align*}
\[
\hat{X}_{1,15:1}=-0,235
\]
\begin{align*}
\hat{X}_{2t} &= 0,0703X_{1t-1}-0,1787X_{1t-6}-0,1133X_{1t-10}-0,1893X_{1t-14}-0,2875X_{2t-1}\\
             &  -0,0605X_{2t-6}-0,2330X_{2t-10}-0,0422X_{2t-14}+0,3525
\end{align*}
\begin{align*}
\hat{X}_{2,15:1} &= 0,0703\ast 4,15-0,1787\ast(-3,26)-0,1133\ast(-1,34)-0,1893\ast(-3,37)\\
                 & -0,2875\ast(-9,23)-0,0605\ast(4,71)-0,2330\ast(-3,12)-0,0422\ast(4,03)\\
                 & +0,3525
\end{align*}
\[
\hat{X}_{2,15:1}=3,777
\]
\end{enumerate}

donde, $\hat{X}_{i,15:j}$ significa, la previsi\'{o}n de la variable $X_{i}$ para el mes j del a\~{n}o 2015 (15:j).\newline

De la misma manera se obtiene:
\[
\hat{X}_{1,15:2}=0,400
\]
\[
\hat{X}_{2,15:2}=-1,281
\]
\[
\hat{X}_{1,15:3}=-0,221
\]
\[
\hat{X}_{2,15:3}=-0,536
\]

Para calcular la varianza del error de predicci\'{o}n, si fuera un VAR (1) se tiene:
\[
M_{1}=\hat{A}_{1};\quad M_{2}=\hat{A}_{1}M_{1}=\hat{A}_{1}^{2};\quad \mbox{etc}\mathellipsis.
\]

Dado que el modelo es un VAR (14) con coeficientes 1, 6, 10 y 14, se tendr\'{a}n las matrices $\hat{A}_{1}$, $\hat{A}_{6}$, $\hat{A}_{10}$, $\hat{A}_{14}$ y estar\'{a}n compuestas por los coeficientes de los retardos de las variables analizadas; as\'{i}, se obtiene:

\[
\hat{A}_{1}=\left[ {\begin{array}{*{20}c}
0,1416 & 0,2241\\
0,0703 & -0,2871\\
\end{array} } \right];\quad 
\hat{A}_{6}=\left[ {\begin{array}{*{20}c}
-0,2403 & 0,0326\\
-0,1787 & -0,0605\\
\end{array} } \right]
\]

\[
\hat{A}_{10}=\left[ {\begin{array}{*{20}c}
0,0889 & -0,1393\\
-0,1133 & -0,2330\\
\end{array} } \right];\quad 
\hat{A}_{12}=\left[ {\begin{array}{*{20}c}
-0,1047 & 0,1082\\
-0,1893 & -0,0422\\
\end{array} } \right] 
\]

La matriz de varianza covarianza estimada de la predicci\'{o}n, para el horizonte $h=1$, es: 
\[
\hat{\Sigma}_{T}\left( 1 \right)=\hat{\Sigma}_{u}=\left[ 
{\begin{array}{*{20}c}
23,821 & 0,422\\
0,422 & 16,602\\
\end{array} } \right]
\]

As\'{i}, la varianza del error de predicci\'{o}n para $\hat{X}_{1,15:1}$ es igual a 23,821 y la varianza del error de predicci\'{o}n para $\hat{X}_{2,15:1}$ es igual a 16,602.\newline

Los intervalos de confianza para $X_{1,15:1}$ y $X_{2,15:1}$ vienen dados, respectivamente, por:

\[
-0,235\pm 1,96\ast \sqrt {23,821} =\left[ -9,80;9,33 \right]
\]
\[
3,777\pm 1,96\ast \sqrt {16,602} =\left[ -4,21;11,76 \right]
\]

Para los horizontes $h=2$, $h=3$, se utilizan las siguientes f\'{o}rmulas:
\begin{align*}
\hat{\sum}_{T}(2) &= \hat{\sum}_{u}+\hat{A}_{1}\hat{\sum}_{u}\hat{A}_{1}^{'}+\hat{A}_{6}\hat{\sum }_{u}\hat{A}_{6}^{'}+\hat{A}_{10}\hat{\sum}_{u}\hat{A}_{10}^{'}+\hat{A}_{14}\hat{\sum }_{u}\hat{A}_{14}^{'}\\
                  &= \left[ 
{\begin{array}{*{20}c}
27,78 & 1,51\\
0,93 & 20,44\\
\end{array} } \right]
\end{align*}

\begin{align*}
\hat{\sum}_{T}(3) &= \hat{\sum }_{u}+\hat{A}_{1}\hat{\sum}_{u}\hat{A}_{1}^{'}+\mathellipsis +\hat{A}_{12}\hat{\sum}_{u}\hat{A}_{12}^{'}+\hat{A}_{1}^{2}\hat{\sum}_{u}\hat{A}_{1}^{2'}+\mathellipsis +\hat{A}_{14}^{2}\hat{\sum}_{u}\hat{A}_{14}^{2'}\\
                  &= \left[ 
{\begin{array}{*{20}c}
27,92 & 1,25\\
0,96 & 20,72\\
\end{array} } \right]
\end{align*}

Entonces los intervalos de confianza son:
\[
IC\left( X_{1,15:2} \right)=-1,394\pm 1,96\ast 5,43=\left[ -9,93 ; 10,73 \right]
\]
\[
IC\left( X_{2,15:2} \right)=0,268\pm 1,96\ast 4,59=\left[ -10,14 ; 7,58 \right]
\]
\[
IC\left( X_{1,15:3} \right)=-0,645\pm 1,96\ast 5,45=\left[ -10,58 ; 20,51 \right]
\]
\[
IC\left( X_{2,15:3} \right)=-0,832\pm 1,96\ast 4,62=\left[ -9,46 ; 17,99 \right]
\]

\textbf{Comparaci\'{o}n con modelos univariantes}\newline

La teor\'{i}a VAR sugiere que las predicciones logradas son de mejor calidad que si se realiza la modelaci\'{o}n de las series de manera univariante. Para comprobar esto, se realiz\'{o} un modelo univariante para cada una de las series analizadas en este ejemplo. As\'{i}, se encontr\'{o} que para el IPP el modelo univariante es $X_{1t}=0,18X_{1t-1}-0,24X_{1t-6}+\hat{u}_{t}$; mientras que, para el IAE es $X_{2t}=-0,25X_{2t-1}-0,27X_{2t-10}+\hat{u}_{t}-0,32\hat{u}_{t-8}$. Con estos modelos se realizaron las predicciones para el a\~{n}o 2014 y se obtiene lo siguiente: 

\begin{figure}[H]
\centering
\includegraphics[width=4.36in,height=3.66in]{Graficos/Cap4-5/STcap434.eps}
\caption{Comparaci\'{o}n de las predicciones VAR y UNIVARIANTE para $X_{1t}$ (IPP)}
\label{fig34}
\end{figure}

\begin{figure}[H]
\centering
\includegraphics[width=4.34in,height=3.68in]{Graficos/Cap4-5/STcap435.eps}
\caption{Comparaci\'{o}n de las predicciones VAR y UNIVARIANTE para $X_{2t}$ (IAE)}
\label{fig35}
\end{figure}

Se puede observar que el ajuste que tienen las predicciones del modelo VAR para el IPP es un tanto mejor que las del modelo univariante. Por otro lado, en el caso del IAE, las predicciones parecen bastante similares entre los 
dos modelos. Para poder determinar esto, se realiza el c\'{a}lculo del error cuadr\'{a}tico medio para determinar el mejor ajuste. As\'{i} se tiene:

%\begin{center}
%:
%\end{center}
%
\begin{table}[H]
\centering
\begin{tabular}{|c|c|c|} \hline
~ & ~IPP~ & ~IAE~ \\ \hline
VAR & 0,95 & 6,29 \\ \hline
UNIVARIANTE & 0,95 & 7,41 \\ \hline
\end{tabular}
\caption{Error Medio Cuadr\'{a}tico estimado para los modelos VAR y UNIVARIANTE}
\label{tab39}
\end{table}

Como se puede observar el modelo VAR es mejor en el caso del IAE; sin embargo, no lo es para el IPP. Se realiz\'{o} una prueba t de medias para determinar si existe diferencia estad\'{i}stica entre las medias de los errores cuadr\'{a}ticos generados por los modelos y se comprob\'{o} que en el caso del IAE el error medio cuadr\'{a}tico es diferente estad\'{i}sticamente entre los modelos (el modelo VAR ajusta mejor los datos); en el caso del IPP se determin\'{o} que no existe diferencia significativa entre los errores cuadr\'{a}ticos. Como conclusi\'{o}n, el modelo VAR predice de mejor manera que el modelo univariante.

\begin{enumerate}
      \item[3.] \textbf{La causalidad}
\end{enumerate}

El siguiente gr\'{a}fico muestra la salida del paquete EViews para la Prueba de Granger. 

\begin{table}[H]
\centering
\begin{tabular}{cccc}\hline
\multicolumn{3}{l}{Dependent variable: IPP} & \\ \hline
~Excluded~ & ~Chi-Square~ & ~df~ & ~Prob.~ \\ \hline\hline
IAE & 6.971411 & 4 & 0.1374 \\ \hline\hline
All & 6.971411 & 4 & 0.1374 \\ \hline\hline
\multicolumn{3}{l}{ } & \\ \hline
\multicolumn{3}{l}{Dependent variable: IAE} & \\ \hline
~Excluded~ & ~Chi-Square~ & ~df~ & ~Prob.~ \\ \hline\hline
IPP & 11.44772 & 4 & 0.0220 \\ \hline\hline
All & 11.44772 & 4 & 0.0220 \\ \hline\hline
\end{tabular}
\caption{Prueba de causalidad de Granger}
\label{tab40}
\end{table}

Como se puede ver, el p-valor (Prob.) es menor que 0,05 en el segundo caso y mayor que 0,05 en el primero; por lo que se concluye que $X_{1t}$ (IPP) explica significativamente la variable $X_{2t}$(IAE), pero $X_{2t}$ no explica significativamente la variable $X_{1t}$.

\begin{enumerate}
      \item[4.] \textbf{An\'{a}lisis de los ``choques''}
\end{enumerate}

En las tablas siguientes se muestran los resultados de las funciones de impulso-respuesta que presenta el paquete EViews para las variables $X_{1t}$ y $X_{2t}$; tambi\'{e}n se muestra los gr\'{a}ficos correspondientes:

\begin{table}[H]
\centering
\begin{tabular}{ccc}\hline
~Per\'{i}odo & IPP & IAE \\ \hline \hline
~1 & ~1,000000 & ~0,000000 \\
~2 & ~0,144733 & ~0,078893 \\
~3 & ~0,042822 & -0,010459 \\
~4 & ~0,003298 & ~0,006279 \\
~5 & ~0,002218 & -0,001481 \\
~6 & -8,96E-05 & ~0,000586 \\
~7 & -0,198874 & -0,156816 \\
~8 & -0,100462 & ~0,001894 \\
~9 & -0,022618 & -0,014730 \\
~10& -0,007966 & ~0,001527 \\ \hline \hline
\end{tabular}
\caption{Respuesta de las variables $X_{1t}$ y $X_{2t}$ ante un choque unitario de $X_{1t}$}
\label{tab41}
\end{table}

\begin{figure}[H]
\centering
\includegraphics[width=3.04in,height=4.52in]{Graficos/Cap4-5/STcap436.eps}
\caption{Respuesta de las variables $X_{1t}$ y $X_{2t}$ ante un choque unitario de $X_{1t}$}
\label{fig36}
\end{figure}

\begin{table}[H]
\centering
\begin{tabular}{ccc}\hline
~Per\'{i}odo & IPP & IAE \\ \hline \hline
~1 & ~0.000000 & ~1.000000 \\
~2 & ~0.224126 & -0.287467 \\
~3 & -0.032697 & ~0.098387 \\
~4 & ~0.017422 & -0.030581 \\
~5 & -0.004387 & ~0.010015 \\
~6 & ~0.001624 & -0.003187 \\
~7 & ~0.032155 & -0.059519 \\
~8 & -0.072018 & -0.003284 \\
~9 & ~0.000135 & -0.004230 \\
~10& -0.006113& -3.68E-05 \\ \hline\hline
\end{tabular}
\caption{Respuesta de las variables $X_{1t}$ y $X_{2t}$ ante un choque unitario de $X_{2t}$}
\label{tab42}
\end{table}

\begin{figure}[H]
\centering
\includegraphics[width=2.64in,height=3.99in]{Graficos/Cap4-5/STcap437.eps}
\caption{Respuesta de las variables $X_{1t}$ y $X_{2t}$ ante un choque unitario de $X_{2t}$}
\label{fig37}
\end{figure}

La elecci\'{o}n de la direcci\'{o}n del impacto es muy importante y determina los valores obtenidos. Se puede observar que el efecto de la innovaci\'{o}n se desvanece con el tiempo; esto caracteriza a un proceso VAR estacionario.

\begin{enumerate}
\item[5.] \textbf{Descomposici\'{o}n de la varianza}
\end{enumerate}

A partir de la representaci\'{o}n $VAR(14)$ estimada se calcula la descomposici\'{o}n de la varianza.\newline

Dado que las variables del ejemplo son variaciones del IPP y el IAE, es l\'{o}gico pensar que un choque sobre la variable variaci\'{o}n del IPP influencie la variaci\'{o}n del IAE m\'{a}s que si el choque fuera al rev\'{e}s. Esto ser\'{\i}a: una innovaci\'{o}n sobre $X_{1t}$ (variaci\'{o}n del IPP) influencia de manera instant\'{a}nea a $X_{2t}$ (variaci\'{o}n del IAE); por otro lado, una innovaci\'{o}n sobre $X_{2t}$ no influencia de manera contempor\'{a}nea a $X_{1t}$.\newline

La matriz de varianza-covarianza estimada de los residuos es igual (ejemplo 5.5) a:
\[
\hat{\Sigma}_{u}=\left[ {\begin{array}{*{20}c}
23,82 & 0,42\\
0,42 & 16,60\\
\end{array} } \right]
\]

Con el programa EViews, las salidas de las funciones de impulso-respuesta y la descomposici\'{o}n de varianza ser\'{i}an:

\begin{table}[H]
\centering
\begin{tabular}{ccccccccc}\hline\hline
\multicolumn{4}{c}{ Descomposici\'{o}n de la varianza de X1:} & 
 & 
\multicolumn{4}{c}{Descomposici\'{o}n de la varianza de X2:} \\ \hline\hline
 Periodo & S.E. & X1 & X2 &  &  Periodo & S.E. & X1 & X2 \\ \hline
 1 &  4,880657 &  100,0000 &  0,000000 &  &  1 &  4,074543 &  0,045030 &  99,95497 \\
 2 &  5,015877 &  96,68675 &  3,313246 &  &  2 &  4,251402 &  0,601299 &  99,39870 \\
 3 &  5,020588 &  96,62258 &  3,377415 &  &  3 &  4,270454 &  0,605403 &  99,39460 \\
 4 &  5,021112 &  96,60331 &  3,396688 &  &  4 &  4,272338 &  0,608029 &  99,39197 \\
 5 &  5,021150 &  96,60210 &  3,397905 &  &  5 &  4,272537 &  0,608160 &  99,39184 \\
 6 &  5,021154 &  96,60193 &  3,398072 &  &  6 &  4,272558 &  0,608181 &  99,39182 \\
 7 &  5,157177 &  96,71430 &  3,285697 &  &  7 &  4,368618 &  4,623096 &  95,37690 \\
 8 &  5,191890 &  96,43880 &  3,561199 &  &  8 &  4,368704 &  4,625945 &  95,37406 \\
 9 &  5,193082 &  96,44043 &  3,559566 &  &  9 &  4,369329 &  4,651664 &  95,34834 \\
 10 &  5,193258 &  96,43838 &  3,561625 &  &  10 &  4,369338 &  4,652042 &  95,34796 \\ \hline\hline
\multicolumn{9}{c}{Orden de Cholesky: X2 X1} \\ \hline\hline
\end{tabular}
\caption{Descomposici\'{o}n de la Varianza (Orden de Cholesky X1 X2)}
\label{tab43}
\end{table}
%
La descomposici\'{o}n de la varianza indica que la varianza del error de predicci\'{o}n de $X_{1t}$ representa un 96,63{\%} con sus propias innovaciones y un 3,56{\%} con las de $X_{2t}$. La varianza del error de predicci\'{o}n de $X_{2t}$ es de un 4,65{\%} con $X_{1t}$ y un 95,34{\%} con $X_{2t}$. Este efecto de asimetr\'{i}a se estudi\'{o} en la pate de la causalidad; lo que tambi\'{e}n se corrobora ahora.\newline

Por \'{u}ltimo, cabe se\~{n}alar que la tabla anterior muestra la desviaci\'{o}n est\'{a}ndar del error de previsi\'{o}n para $X_{1t}$ y $X_{2t}$.
