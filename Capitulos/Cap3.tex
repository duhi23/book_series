\chapter[Estimaci\'{o}n, Verificaci\'{o}n y Predicci\'{o}n en un Modelo ARIMA]{Estimaci\'{o}n, Verificaci\'{o}n y Predicci\'{o}n en un Modelo ARIMA por el  M\'{e}todo de Box--Jenkins}

\section{Metodolog\'{i}a Box--Jenkins}

El procedimiento para \index{Metodolog\'{i}a de Box y Jenkins} identificar, estimar, verificar y predecir a partir de un modelo ARIMA compatible con los datos, se hace de una manera secuencial, que se puede describir (en forma resumida) de la siguiente forma:

\begin{enumerate}
\item[a)] Se comienza por buscar, a partir de las observaciones, los valores plausibles para $(p,d,q)$. Esta etapa constituye la fase de\index{Metodolog\'{i}a de Box y Jenkins! Identificaci\'{o}n a priori} \emph{identificaci\'{o}n a priori} del modelo. Se requieren muchos c\'{a}lculos y, en general, no se obtiene una sola tripleta de valores $(p,d,q)$.\newline

Esto se explica porque un proceso estacionario puede aproximarse, tanto como se quiera, por un proceso $ARMA(p,q)$, donde el valor de $p$ puede ser cualquiera.

\item[b)] Para cada tripleta de valores de $(p,d,q)$ obtenida, se procede a la fase de \emph{estimaci\'{o}n de los coeficientes} \index{Metodolog\'{i}a de Box y Jenkins! Estimaci\'{o}n de los coeficientes}de los polinomios autoregresivo y media m\'{o}vil. Para fines pr\'{a}cticos, se los puede obtener con alguno de los paquetes estad\'{i}sticos (EViews, Statgraphics, R, etc.).

\item[c)] Al final de la etapa b), se dispone de varios modelos estimados. Estos pasan entonces por una fase de \emph{pruebas estad\'{i}sticas} (pruebas de significaci\'{o}n para los coeficientes, pruebas concernientes a las no correlaciones de los ruidos blancos, etc.), para verificar los resultados obtenidos con las hip\'{o}tesis hechas.\newline

Esta es la fase de \emph{verificaci\'{o}n} y \index{Metodolog\'{i}a de Box y Jenkins!Verificaci\'{o}n}puede resultar que todos los modelos estimados sean rechazados. Bajo condiciones de normalidad de los residuos (o independencia), las pruebas para los coeficientes son del tipo `$t$-student'' usuales.

\item[d)] Puede darse el caso de que varios modelos pasen la fase de verificaci\'{o}n. En este caso se escoge al modelo que tenga el mayor poder predictivo o la mayor cantidad de informaci\'{o}n (se explicar\'{a}n posteriormente estos conceptos).

\item[e)] Finalmente, una vez escogido el mejor modelo se procede a realizar la predicci\'{o}n para un per\'{i}odo posterior a la de los datos con los que se realiz\'{o} la modelaci\'{o}n.\newline

Las etapas c) y d) constituyen la fase \index{Metodolog\'{i}a de Box y Jenkins!Identificaci\'{o}n a posteriori} de \emph{identificaci\'{o}n a posteriori}; es decir, despu\'{e}s de la estimaci\'{o}n. Todas las etapas se explican en este cap\'{i}tulo tanto la parte te\'{o}rica como la pr\'{a}ctica.
\end{enumerate}


\section{Identificación \emph{A Priori} \index{Metodolog\'{i}a de Box y Jenkins!Identificaci\'{o}n a priori} }

\subsection{Elecci\'{o}n de $d$}

Si el proceso $(X_t)_{t\in \Z}$, generado por las observaciones, es un ARIMA con $d > 0$, no es estacionario; luego, su funci\'{o}n de autocorrelaci\'{o}n depende de $\ell$ y $t$, as\'{i}:
\[
\rho \left(\ell, t \right)=\frac{\cov(X_{t}, X_{t+\ell})}{\sqrt {\V(X_{t})} \cdot \sqrt {\V(X_{t+\ell})} }
\]

Se puede demostrar, adem\'{a}s, que la funci\'{o}n $\rho (\ell, t)$, tiende a 1 cuando $t$ tiende a $+\infty $. Si la funci\'{o}n $\widehat{\rho}(\ell)$ queda relativamente pr\'{o}xima a uno para un n\'{u}mero importante de valores de $\ell$, se puede entonces pensar que es necesario diferenciar la serie para volverla estacionaria.\newline

En la pr\'{a}ctica, sin embargo, el criterio de proximidad a 1 de los primeros valores de la funci\'{o}n $\widehat{\rho }(\ell)$ se remplaza por el de la proximidad entre si de los primeros valores de esta funci\'{o}n, aun cuando el valor $\widehat{\rho }(\ell)$ sea muy diferente de uno. A la serie diferenciada una vez se le aplica nuevamente este criterio para ver si es conveniente diferenciarla una segunda vez, y as\'{i} sucesivamente. Esto se muestra a continuaci\'{o}n:

\begin{ejemplo}
Para este ejemplo se ha utilizado el programa Statgraphics Centurion XV. Consid\'{e}rense los datos de variaci\'{o}n de un \'{i}ndice burs\'{a}til (ver Anexo C.2), cuyo gr\'{a}fico se muestra a continuaci\'{o}n:

\begin{figure}[H]
\centering
\includegraphics[width=0.7\textwidth]{Graficos/Cap3/STcap31.eps}
\caption{Serie de un \'{I}ndice Burs\'{a}til (SIB)}
\end{figure}

Al observar las autocorrelaciones estimadas del modelo original (Figura 3.2) se constata que se trata de un modelo donde es necesario realizar una diferenciaci\'{o}n para volver a la serie estacionaria (Figura 3.3); esto se podr\'{i}a deducir tambi\'{e}n al mirar la forma lineal creciente de la tendencia de la serie (Figura 3.1).

\begin{figure}[H]
\centering
\includegraphics[width=0.7\textwidth]{Graficos/Cap3/STcap32.eps}
\caption{Funci\'{o}n de autocorrelaci\'{o}n de la SIB}
\end{figure}

\begin{figure}[H]
\centering
\includegraphics[width=0.7\textwidth]{Graficos/Cap3/STcap33.eps}
\caption{Serie estacionaria obtenida a partir de la SIB}
\end{figure}

Para las series econ\'{o}micas los valores de $d$ que se obtienen, en general, son 0, 1 o 2; los valores superiores son muy raros. En el caso de la serie en consideraci\'{o}n, no se requiere una segunda diferenciaci\'{o}n no estacional.
\end{ejemplo}

\begin{observacion}
 El t\'{e}rmino utilizado para referirse a este tipo de diferenciaci\'{o}n es el de: \emph{diferenciaci\'{o}n no estacional}; la \emph{diferencia estacional} se explicar\'{a} m\'{a}s adelante.
\end{observacion}

La consecuencia de aplicar en un modelo un n\'{u}mero de diferencias mayor que el necesario es que el proceso final sea no invertible y con varianza mayor que si se hubiera aplicado un n\'{u}mero de diferencias correcto.

\subsection{Elecci\'{o}n de $p$ y $q$}

Para la elecci\'{o}n de $p$ y $q$ se analiza esencialmente la forma de la funci\'{o}n de autocorrelaci\'{o}n estimada $\widehat{\rho }(\ell)$ y de la funci\'{o}n de autocorrelaci\'{o}n parcial estimada $\widehat{r}(\ell)$ de la serie diferenciada d veces. Si la funci\'{o}n $\widehat{\rho }(\ell)$ se anula estad\'{i}sticamente para $\ell>q$, se puede pensar que se trata de un $MA(q)$; y si la funci\'{o}n $\widehat{r}(\ell)$ se anula estad\'{i}sticamente para valores de $\ell>p $se puede admitir que se trata de un $AR(p).$ As\'{i}, se obtienen valores de $p$ y $q$ (que pueden considerarse m\'{a}ximos) para el modelo; utilizando un paquete estad\'{i}stico (en este cap\'{i}tulo se ha elegido Statgraphics) se puede graficar la funci\'{o}n $\widehat{\rho }(\ell)$ y ver a partir de qu\'{e} orden $\ell$ los valores de la funci\'{o}n quedan dentro de la banda. Similarmente, se procede con la funci\'{o}n de autocorrelaci\'{o}n parcial.\newline

En cuanto a la obtenci\'{o}n de la banda de confianza para las dos funciones estimadas, para juzgar si los valores de $\widehat{\rho }(\ell)$ o $\widehat{r}(\ell)$ son significativamente diferentes de cero es bueno tener una idea de su desviaci\'{o}n est\'{a}ndar. Para un $MA(q)$, se tiene:
\[
\forall \ell>q,\qquad \V\left[ \widehat{\rho}\left(\ell \right) \right]\approx \frac{1}{T}\left[ 1+2\sum_{i=1}^{\ell-1} {\rho^{2}(i)} \right]
\]

Se puede probar que para identificar el orden q de un proceso media m\'{o}vil, se debe representar la sucesi\'{o}n de autocorrelaciones estimadas $\widehat{\rho }(\ell)$ y mirar a partir de qu\'{e} valor $q$ cada uno de los $\widehat{\rho }(\ell)$ se queda en el intervalo de extremos:

\[
\pm \left( \frac{1,96}{\sqrt T } \right)\left[ 1+2\left[ \rho^{2}\left( 1 \right)+\mathellipsis +\rho^{2}(\ell-1) \right]^{1/2} \right]
\]

Los valores te\'{o}ricos $\rho (j)$ deben reemplazarse por sus estimaciones. Se asume tambi\'{e}n un nivel de significaci\'{o}n del 5{\%} en la prueba de hip\'{o}tesis respectiva.\newline

Igualmente, se tiene que $\sqrt \frac{1}{T} $ es la desviaci\'{o}n est\'{a}ndar aproximada de $\widehat{r}(\ell)$ para $\ell>p$, si el proceso es un $AR(p)$. Para identificar el orden $p$ de un proceso autoregresivo, asumiendo un nivel de significaci\'{o}n del 5{\%}, se puede representar la sucesi\'{o}n de autocorrelaciones parciales estimadas $\widehat{r}(\ell)$ y mirar a partir de qu\'{e} valor $p$ cada una se queda en la banda dada por:

\[
\left[ -\frac{1,96}{\sqrt T };\frac{1,96}{\sqrt T } \right]
\]

\begin{ejemplo}
Volviendo al ejemplo 3.1, obs\'{e}rvense los gr\'{a}ficos correspondientes a las autocorrelaciones estimadas (ver Figura 3.4), donde se aprecia que a partir de $q=2$ la funci\'{o}n queda dentro de la banda de confianza, lo que en primera instancia hace suponer que se trata de un modelo $MA(2).$

\begin{figure}[H]
\centering
\includegraphics[width=0.7\textwidth]{Graficos/Cap3/STcap34.eps}
\caption{FAC estimada del modelo $ARIMA(0,1,0)$, para la SIB}
\end{figure}

Ahora bien, si adem\'{a}s se revisa la funci\'{o}n de autocorrelaci\'{o}n parcial estimada (Figura 3.5), donde se observa un decrecimiento exponencial de la serie hacia cero, parece confirmarse el hecho de que la serie original se debe modelar por un $MA(1)$ o un $MA(2)$.\newline

En este punto es necesario realizar un an\'{a}lisis m\'{a}s profundo. En principio, dado que $\widehat{r}(6)$ es estad\'{i}sticamente significativo, se podr\'{i}a pensar en que el Modelo m\'{a}s adecuado ser\'{i}a un $AR(6)$; es decir, globalmente se considerar\'{i}a un $ARMA(6,2)$, lo que realmente llevar\'{i}a a una sobreparametrizaci\'{o}n. En general, se aconseja que $p+d+q\le 5$.

\begin{figure}[H]
\centering
\includegraphics[width=0.7\textwidth]{Graficos/Cap3/STcap35.eps}
\caption{FACP estimada del modelo $ARIMA(0,1,0)$, para la SIB en primera 
diferencia}
\end{figure}
\end{ejemplo}



\section{Estimaci\'{o}n de un Modelo ARIMA}

Consid\'{e}rese el siguiente modelo\index{Metodolog\'{i}a de Box y Jenkins!Estimaci\'{o}n de los coeficientes }:
\[
 \Phi(B)\Delta^{d} X_{t}=  \theta_0  +  \Theta(B) u_{t}
\]
donde $\Phi(z)$ y $\Theta(z)$ son primos entre si y tienen ra\'{i}ces de m\'{o}dulo mayor a 1 y $(u_{t})$ ruido blanco de varianza $\sigma^{2}$ ; se tiene entonces que $W_{t} = \Delta^{d} X_{t}$ es asint\'{o}ticamente un $ARMA (p,q)$:
\[
 \Phi(B) W_{t} = \theta_{0} +  \Theta (B) u_{t}
\]

Sup\'{o}ngase que se conocen las observaciones $x_{1}, x_{2},\ldots ,x_{T}$ del proceso $(X_{t})$. Se pueden calcular las diferencias de orden $d$:
\begin{align*}
 w_{d+1}&=\Delta^{d}x_{d+1}\\
 w_{d+2}&=\Delta^{d}x_{d+2}\\
 &\ \ \vdots \\
 w_{T}&=\Delta^{d}x_{T}
\end{align*}

As\'{i}, se tienen $N=T- d$ observaciones, que se denotar\'{a}n por $z_{1} ,z_{2} ,\ldots,z_{N} $, para estimar los par\'{a}metros del modelo:
\[
\Phi (B) W_{t} = \theta_{0} + \Theta (B)u_{t} 
\]

Se puede suprimir $\theta_{0}$, remplazando $W_{t}$ por $\widehat W_{t} =W_{t} -\mu$, con $\mu =\theta_{0}/\left( {1-\varphi_{1} -\cdots -\varphi_{p} } \right)$. En lo que sigue se supondr\'{a} que $\theta_{0}= 0$.\newline

Se supondr\'{a} en esta secci\'{o}n que los ruidos blancos $u_{t} $ son normales (en realidad solo se requiere independencia). Para calcular la funci\'{o}n de verosimilitud $L\left( {\varphi_{1} ,\ldots, \varphi_{p} ,\theta_{1} ,\ldots, \theta_{q} ,\sigma^{2}} \right)=L\left( {\varphi ,\theta ,\sigma^{2}} \right)$ se requieren conocer $p$ valores iniciales $z_{0} ,\ldots,z_{1-p} $ y $q$ valores iniciales $u_{0}, \ldots, u_{1-q}$, que se expresar\'{a}n a trav\'{e}s de los vectores $z_{\ast }' =\left( {z_{0} ,\ldots, z_{1-p} } \right)$ y $u_{\ast }' =\left( {u_{0} ,\ldots,u_{1-q} } \right)$; existen dos formas alternativas de obtenerlos, las que se describen a continuaci\'{o}n.

\subsection{Procedimiento condicional}

En este caso se va a suponer que los elementos de $z_{\ast }$ y $u_{\ast } $ son iguales a sus esperanzas; es decir $\E u_{t} =0$ y $\E z_{t} =0$. Para el caso en que $\varphi \left( z \right)$ tenga ra\'{i}ces cercanas a la unidad o la serie sea estacional, una mejor soluci\'{o}n consiste en utilizar la f\'{o}rmula:
\[
u_{t} =z_{t} -\varphi_{1} z_{t-1} -\cdots -\varphi_{p} z_{t-p} 
+\theta_{1} u_{t-1} +\cdots +\theta_{q} u_{t-q} \qquad t=p+1,\ldots ,N
\]
tomando $u_{p} $, $u_{p-1}$,\dots nulos, pero considerando valores de $z_{t} $ realmente observados.\newline

La funci\'{o}n de verosimilitud en el caso condicional se expresa por:
\begin{align*}
 L
	&=L\left(\varphi ,\theta ,\sigma^{2}|z,z_{\ast } ,u_{\ast } \right)\\
	&=\left( 2\pi \sigma^{2} \right)^{-N/2}\exp \left(-\frac{1} {2 \sigma^{2}} \sum_{t=1}^N u_{t}^{2} \right) \\
	&=\left( {2\pi \sigma^{2}} \right)^{-N/2}\exp \left[ {-\frac{1}{2\sigma^{2}}\sum_{t=1}^N {\left( {z_{t} -\varphi_{1}z_{t-1} -\cdots -\varphi_{p} z_{t-p} +\theta_{1} u_{t-1} +\cdots +\theta_{q} u_{t-q} } \right)^{2}} } \right]
\end{align*}
tomando logaritmos se tiene que:
\[
\ell =\ln L=cte-\frac{N}{2}\ln\sigma^{2}-\frac{S_{\bullet } \left( {\varphi, \theta | {z,z_{\ast } ,u_{\ast } } } \right)}{2\sigma^{2}}
\]
se denotar\'{a}: 
\[
S_{\bullet } \left( {\varphi ,\theta } \right)=S\left( {\varphi, \theta | {z,z_{\ast } ,u_{\ast } }} \right)=\sum_{t=1}^N {u_{t}^{2} \left( {\varphi ,\theta ,\sigma ^{2}| {z,z_{\ast } ,u_{\ast } } } \right)} 
\]
llamada la suma cuadr\'{a}tica condicional. La expresi\'{o}n \emph{``cte''} representa de manera general a un valor constante.\newline

Derivando $\ell $ con respecto a $\sigma^{2}$ e igualando a cero se tiene:
\[
\frac{\partial \ell }{\partial \sigma^{2}}=-\frac{N}{2\sigma^{2}}+\frac{S_{\ast } }{2\sigma^{4}}=0\Longleftrightarrow \sigma^{2}=\frac{S_{\ast } \left( {\varphi ,\theta \left| {z,z_{\ast }, u_{\ast } } \right.} \right)}{N}=\frac{S_{\ast } }{N}
\]

Si se reemplaza esta expresi\'{o}n de $\sigma^{2}$ en $\ell$, se obtiene:

\[
\overline \ell =cte-\frac{N}{2}\ln\frac{S_{\ast } }{N}-\frac{N}{2}=cte+\frac{N}{2}\ln N-\frac{N}{2}\ln S_{\ast } -\frac{N}{2}=cte-\frac{N}{2}\ln S_{\ast } 
\]

As\'{i}, maximizar $\overline \ell $ equivale a minimizar $S_{\ast}$; esto es, a obtener una estimaci\'{o}n m\'{i}nimo cuadr\'{a}tica.

\begin{ejemplo}
Consid\'{e}rese una serie temporal que responde a un modelo $ARIMA (0,d,1)$:
\[
 z_{t} =u_{t} -\theta u_{t-1} =\left( {1-\theta B} \right)u_{t}
\]
con
\[
w_{t} =\Delta^{d} X_{t}, \qquad t=d+1,\ldots, T \qquad \text{ y }\qquad z_{t} =w_{t+d} ,\qquad t=1, \cdots , N
\]
En este caso:
\[
S_{\ast } \left( {\theta \left| {z,u_{0} } \right.} \right)=\sum_{t=1}^N {u_{t}^{2} =\sum_{t=1}^N {\left( {z_{t} +\theta u_{t-1} \left( \theta \right)} \right)^{2}} } 
\]
Si se considera entonces que $u_{0} =0,$ se obtiene:
\begin{align*}
 u_{1} \left( \theta \right)&=z_{1} +\theta u_{0} \left( \theta \right)=z_{1} +\theta u_{0} \\ 
 u_{2} \left( \theta \right)&=z_{2} +\theta u_{1} \left( \theta \right)=z_{2} +\theta \left( {z_{1} +\theta u_{0} } \right)=z_{2} +\theta z_{1} +\theta^{2}u_{0} \\ 
 u_{3} \left( \theta \right)&=z_{3} +\theta u_{2} \left( \theta \right)=z_{3} +\theta \left( {z_{2} +\theta z_{1} } \right)=z_{3} +\theta z_{2} +\theta^{2}z_{1} +\theta^{3}u_{0} \\ 
 &\ \ \vdots \\ 
 u_{t} \left( \theta \right)&=z_{t} +\theta z_{t-1} +\cdots +\theta ^{t-1}z_{1} +\theta^{t}u_{0} \\ 
 &=z_{t} +\theta z_{t-1} +\cdots +\theta^{t-1}z_{1} \qquad \text{(considerando $u_{0} =0$)}
\end{align*}


Si $N$ es suficientemente grande y puesto que $\left| \theta \right|<1,$ el valor que asuma $u_{0} $ no es muy importante; sin embargo, si $\left| \theta \right|$ est\'{a} cerca de 1 y $N$ es peque\~{n}o, la elecci\'{o}n de $u_{0} $ puede resultar significativa.\newline

En la tabla siguiente se presentan los c\'{a}lculos recursivos de $u_{t}$, para 11 valores dados de $X_{t} $ en un modelo $ARIMA\left( {0,1,1} \right),$ en el caso en que $\theta=0.6$. En este ejemplo:
\[
S_{\ast } \left( {\theta \left| {z,u_{0} } \right.} \right)=\sum_{t=1}^{10} {u_{t}^{2} \left( {\theta =0,6\left| {u_{0} =0} \right.} \right)=511,98} 
\]

\begin{table}[H]
\centering
\begin{tabular}{@{} cccc @{}}
\toprule
$t$& $X(t)$& $z(t)$& $u(t)$ \\
\midrule
0& 150& & 0,00 \\
1& 149& -1& -1,00 \\
2& 146& -3& -3,60 \\
3& 144& -2& -4,16 \\
4& 148& 4& 1,50 \\
5& 155& 7& 7,90 \\
6& 149& -6& -1,26 \\
7& 156& 7& 6,24 \\
8& 161& 5& 8,75 \\
9& 171& 10& 15,25 \\
10& 170& -1& 8,15 \\
\bottomrule
\end{tabular}
\caption{C\'{a}lculos para el caso condicionado}
\end{table}

Observe que con esta disposici\'{o}n de datos $z_{j} =w_{j}$ $\left( {j=1,\ldots , 10} \right).$\newline

Realizando el c\'{a}lculo para diversos valores de $\theta \in \left] {-1,1} \right],$se puede tener una idea de d\'{o}nde se alcanza el m\'{i}nimo para $S_{\ast } \left( {\theta \left| {z,u_{0} } \right.} \right).$ En general, la soluci\'{o}n se obtiene utilizando t\'{e}cnicas del an\'{a}lisis num\'{e}rico, situaci\'{o}n que no ser\'{a} abordada en este texto.
\end{ejemplo}


\subsection{Procedimiento no condicional}

Puede demostrarse que el logaritmo de la funci\'{o}n de verosimilitud no condicional se expresa por:
\[
\ell =\ell \left( {\varphi ,\theta ,\sigma^{2}\left| z \right.} \right)\approx -\frac{N}{2}\ln \sigma^{2}-\frac{S\left( {\varphi, \theta} \right)}{2\sigma^{2}}
\]
donde
\[
S\left( {\varphi ,\theta } \right)=\sum_{t=-\infty }^N {\left[ {u_{t} \left( {\varphi ,\theta ,z} \right)} \right]^{2}} 
\]
es la suma de cuadrados no condicional y, adem\'{a}s, los corchetes representan la esperanza condicional, definida por:
\[
\left[ {u_{t} \left( {\varphi ,\theta , z} \right)} \right]=\E\left({u_{t} \left| {\varphi ,\theta ,z} \right.} \right)
\]

Se pude tambi\'{e}n demostrar que los estimadores m\'{i}nimo cuadr\'{a}ticos, que se obtienen minimizando $S\left( {\varphi ,\theta } \right)$, son buenas aproximaciones para los estimadores de m\'{a}xima verosimilitud. Para ejecutar el proceso de minimizaci\'{o}n se requiere calcular las esperanzas condicionales de $z_{-j} \text{y}u_{-j} ,\quad j=0,1,2,\mathellipsis ,$ denotadas por $\left[ {z_{-j} } \right]\text{y}\left[ {\text{u}_{\text{-j}} } \right].$. Se utiliza para este efecto la representaci\'{o}n en avance del proceso:

\[
\Phi \left( F \right)z_{t} =\Theta \left( F \right)\varepsilon_{t} 
\]

donde $F$ es el operador de avance $\left( {FX_{t} =X_{t+1} } \right)$ y $\left( {\varepsilon_{t} } \right)$ es un ruido blanco con la misma varianza que $\left( {u_{t} } \right).$ En este caso $z_{t} $ se expresa solamente en t\'{e}rminos futuros de $z_{j}$ y $\in_{j} $. As\'{i} entonces, los $\left[ {z_{-j} } \right]$ son las predicciones hacia atr\'{a}s de $z_{-j} $ dadas las $z_{1} ,\ldots ,z_{N}$.\newline

Para ilustrar el proceso, consid\'{e}rese el siguiente ejemplo:

\begin{ejemplo}
Sea nuevamente el modelo
\[
z_{t} =\left( {1-\theta B} \right)u_{t} =u_{t} -\theta u_{t-1} 
\]
cuya representaci\'{o}n en avance se expresa por:
\[
z_{t} =\left( {1-\theta F} \right)\varepsilon_{t} =\varepsilon_{t} -\theta \varepsilon_{t+1}
\]
se tiene que:
\[
\left[ {u_{t} } \right]=\left[ {z_{t} } \right]+\theta \left[ {u_{t-1} } \right]\qquad \text{ y }\qquad \left[ {\varepsilon_{t} } \right]=\left[ {z_{t}} \right]+\theta \left[ {\varepsilon_{t+1} } \right]
\]
de donde se obtiene que:
\begin{itemize}
\item $\left[ {z_{t} } \right]=z_{t}$, si $t=1,\ldots,N$ y son predicciones hacia atr\'{a}s si $t\leq 0$.

\item $\left[ {\varepsilon_{0} } \right],\left[ {\varepsilon_{-1} } \right],\ldots $ son nulos, pues $\varepsilon_{0} ,\varepsilon_{-1} ,\ldots $ son ortogonales a $z$.

\item $\left[ {u_{-1} } \right],\left[ {u_{-2} } \right]$ son nulos pues son independientes de $z$ (en general son no nulos y se los obtiene por predicci\'{o}n hacia atr\'{a}s).
\end{itemize}

En la tabla siguiente se ilustran los c\'{a}lculos, considerando $\theta =0,6$

\begin{table}[H]
\centering
\begin{tabular}{@{}ccccc@{}}
\toprule
$t$& $X(t)$& $[z(t)]$& $[\varepsilon(t)]$& $[u(t)]$ \\
\midrule
-1&  &  & 0,00& 0,00 \\
0& 150& 0,95& 0,00& 0,95 \\
1& 149& -1,00& -1,59& -0,43 \\
2& 146& -3,00& -0,98& -3,26 \\
3& 144& -2,00& 3,36& -3,95 \\
4& 148& 4,00& 8,93& 1,63 \\
5& 155& 7,00& 8,22& 7,98 \\
6& 149& -6,00& 2,03& -1,21 \\
7& 156& 7,00& 13,38& 6,27 \\
8& 161& 5,00& 10,64& 8,76 \\
9& 171& 10,00& 9,40& 15,26 \\
10& 170& -1,00& -1,00& 8,15 \\
11& & & 0&  \\
\bottomrule
\end{tabular}
\caption{C\'{a}lculos para el caso no condicionado}
\end{table}

Se parte calculando $\left[ {\varepsilon_{10} } \right],$ bajo la suposici\'{o}n de que $\left[ {\varepsilon_{11} } \right]=0$:
\[
\left[ {\varepsilon_{10} } \right]=\left[ {z_{10} } \right]+0,6\left[ {\varepsilon_{11} } \right]=-1+0,6\cdot 0=-1
\]
de la misma manera se procede para $\left[ {\varepsilon_{9} } \right],\ldots,\left[ {\varepsilon_{1} } \right].$ Luego se tiene:
\[
0=\left[ {\varepsilon_{0} } \right]=\left[ {z_{0} } \right]+0,6\cdot \left[ 
{\varepsilon_{1} } \right]=\left[ {z_{0} } \right]+0,6\cdot \left( {-1,59} 
\right)
\]
de donde:
\[
\left[ {z_{0} } \right]=0,95
\]

Luego, los $\left[ {z_{-j} } \right]=0,$ para $j=1,2,\ldots$ Ahora se procede a calcular los $\left[ {u_{t} } \right]$ no nulos, de la siguiente manera:

\begin{align*}
\left[ {u_{0} } \right]&=\left[ {z_{0} } \right]+\theta \left[ {u_{-1} } 
\right]=\left[ {z_{0} } \right]+0=0,95 \\ 
 \left[ {u_{1} } \right]&=\left[ {z_{1} } \right]+\theta \left[ {u_{0} } 
\right]=-1+0,6\ast 0,95=-0,43 \\ 
 &\ \ \vdots 
\end{align*}

Adem\'{a}s:
\[
\left[ {X_{-1} } \right]=X_{0} -\left[ {z_{0} } \right]=150-0,95=149,05
\]
As\'{i},
\[
S\left( {0,6} \right)=\sum_{t=0}^{10} {\left[ {u_{t} } 
\right]^{2}=510,50} 
\]
Como puede observarse en las dos tablas precedentes, aunque se pueden constatar ciertas diferencias entre los valores de $u_{t}$ y $\left[ {u_{t} } \right]$ para valores peque\~{n}os de $t$, \'{e}stas van aminor\'{a}ndose conforme crece $t$, raz\'{o}n por la cual, a\'{u}n con un peque\~{n}o n\'{u}mero de observaciones, $S_{\ast } (0,6)$ es bastante similar a $S\left( {0,6} \right)$.\newline

Como se mencion\'{o} anteriormente, las minimizaciones se deben efectuar utilizando t\'{e}cnicas num\'{e}ricas, para lo cual se requiere obtener valores iniciales para los par\'{a}metros.\newline

El procedimiento general para obtener la funci\'{o}n de verosimilitud exacta de un proceso ARMA, as\'{i} como para las estimaciones de las esperanzas condicionales para $z_{0} ,z_{-1} ,\ldots $ se presentan en el Anexo C.1. En todo caso, hay que recordar que los estimadores de m\'{a}xima verosimilitud son normalmente asint\'{o}ticos.
\end{ejemplo}


\begin{ejemplo}
En el ejemplo planteado (de la SIB) se ha considerado una serie que luego de realizar una diferenciaci\'{o}n no estacional muestra su comportamiento estacionario; es decir, se trata de un modelo $ARIMA(0,1,0)$.\newline

Con ayuda del utilitario Statgraphics y considerando el par $p=2$ y  $q=2$ (es decir, se ha establecido la posibilidad de que se trate de un modelo $ARIMA(2,1,2))$, se han obtenido los siguientes resultados:

\begin{table}[H]
\centering
\begin{tabular}{@{}ccccc@{}}
\toprule
Par\'{a}metro& Estimado& Error Estd.& Valor $t$& Valor $P$ \\
\midrule
$AR(1)$& 1,44980& 0,20709& 7,00091& 0,00000 \\
$AR(2)$& -0,58809& 0,12222& -4,81164& 0,00000 \\
$MA(1)$& -0,10109& 0,23630& -0,42781& 0,66948 \\
$MA(2)$& 0,60732& 0,20394& 2,97796& 0,00344 \\
Media& 0,75431& 0,30518& 2,47172& 0,01469 \\
Constante& 0,10431& & &  \\
\bottomrule
\end{tabular}
\caption{Resumen estad\'{i}stico para el modelo $ARIMA(2,1,2)$ de la SIB}
\end{table}

\end{ejemplo}


Esta estimaci\'{o}n la realiza el paquete internamente a trav\'{e}s de los m\'{e}todos descritos anteriormente. En la secci\'{o}n siguiente se realizar\'{a} la verificaci\'{o}n, validaci\'{o}n y elecci\'{o}n del mejor modelo.

\section{Identificaci\'{o}n \emph{A Posteriori}}

Una vez realizadas las estimaciones\index{Metodolog\'{i}a de Box y Jenkins!Identificaci\'{o}n a posteriori } de los modelos, se procede a efectuar las fases de verificaci\'{o}n y de elecci\'{o}n del mejor modelo.

\subsection{Fase de verificaci\'{o}n}

La finalidad de esta fase \index{Metodolog\'{i}a de Box y Jenkins!Verificaci\'{o}n}consiste en analizar la adecuaci\'{o}n entre el modelo y los datos. Las pruebas estad\'{i}sticas utilizadas son de dos tipos: las concernientes a los par\'{a}metros $a_{j}$ y $b_{j}$ de los procesos autoregresivo y media m\'{o}vil, respectivamente, del modelo propuesto $ARMA(p,q)$ de la serie diferenciada $d$ veces y las pruebas concernientes a las hip\'{o}tesis del ruido blanco $u_{t}$

\subsubsection{Pruebas concernientes a los par\'{a}metros:}

Interesa saber si se puede disminuir el n\'{u}mero de par\'{a}metros; es decir, si se dispone de un modelo $ARMA(p,q)$, si es posible formular un modelo $ARMA(p-1,  q)$ o un modelo $ARMA  (p,q-1)$; esto lleva a considerar la siguiente prueba de hip\'{o}tesis:
\[
\begin{cases}
 H_0:& \phi_{p} =0 \\ 
 H_{1} :& \phi_{p} \ne 0 \\ 
\end{cases}
\]

Si se acepta $H_{0}$, se asume que el modelo es $ARMA(p-1,  q)$; en caso contrario, se retiene la formulaci\'{o}n $ARMA(p,q)$. Utilizando la normalidad asint\'{o}tica de los estimadores, su regi\'{o}n de rechazo est\'{a} determinada por:

\[
W=\left\{ {\frac{\left| {\widehat{\varphi}_{p} } \right|}{\sqrt {\widehat{\V}\widehat{\varphi}_{p} } }>u_{1-\frac{\alpha }{2}} } \right\}
\]

Si ahora se considera $p_{1} = p+1$, la prueba a realizar es: 

\[
\begin{cases}
 H_0:& \phi_{p+1} =0 \\ 
 H_{1} :& \phi_{p+1} \ne 0 \\ 
\end{cases}
\]

Si se acepta $H_{0}$ se retiene la forma $ARMA (p, q)$; en caso contrario, se acepta el modelo $ARMA (p_{1}, q)$. En este caso la regi\'{o}n de rechazo est\'{a} determinada por:

\[
W=\left\{ {\frac{\left| {\widehat{\varphi}_{p+1} } \right|}{\sqrt 
{\widehat{\V}\widehat{\varphi}_{p+1} } }>u_{1-\frac{\alpha 
}{2}} } \right\}
\]
De manera similar se procede con el modelo $ARMA (p , q_{1} )$ donde $q_{1} = q + 1$ o $q_{1} = q - 1$.

\begin{observacion}
No se deben disminuir o aumentar simult\'{a}neamente los dos par\'{a}metros. 
\end{observacion}

Con el paquete estad\'{i}stico Statgraphics, se obtiene una tabla (ver tabla 3.3) con los valores de los par\'{a}metros, el valor del estad\'{i}stico $T$ y el $p$-valor: la probabilidad obtenida con los datos observados de aceptar la hip\'{o}tesis $H_{1}$ de que el par\'{a}metro sea diferente de cero, cuando en verdad sea cero (el valor p, o nivel de significaci\'{o}n emp\'{i}rico del contraste, es el m\'{i}nimo nivel de significaci\'{o}n tal que se rechaza la hip\'{o}tesis nula).

\begin{ejemplo}
Retomando el ejemplo de la SIB, con prop\'{o}sitos ilustrativos, para mostrar los efectos de la sobreparametrizaci\'{o}n, se elegi\'{o} el par $p=2$ y $q=2$; es decir, se ha establecido la posibilidad de que se trate de un modelo $ARIMA(2,1,2)$. Se obtienen los siguientes resultados:\newline

Para el modelo $ARIMA(2,1,2)$ se observa que los coeficientes pertenecientes a la parte MA son estad\'{i}sticamente nulos (se tomar\'{a} siempre como nivel de significaci\'{o}n el 5{\%}. Ver Tabla 3.4), lo que indica la necesidad de reformularlo. Este hecho suele suceder como efecto de la sobreparametrizaci\'{o}n (en este caso $q$ es muy grande). 

\begin{table}[H]
\centering
\begin{tabular}{@{}ccccc@{}}\hline
Par\'{a}metro & Estimado & Error Estd. & Valor $t$ & Valor $P$ \\ \hline
$AR(1)$ & 1,44980& 0,20709& 7,00091& 0,00000 \\
$AR(2)$ & -0,58809& 0,12222& -4,81164& 0,00000 \\
$MA(1)$ & -0,10109& 0,23630& -0,42781& 0,66948 \\
$MA(2)$ & 0,60732& 0,20394& 2,97796& 0,00344 \\
Media & 0,75431& 0,30518& 2,47172& 0,01469 \\
Constante& 0,10431& & &  \\ \hline
\end{tabular}
\caption{Resumen estad\'{i}stico para el modelo $ARIMA(2,1,2)$ de la SIB}
\end{table}

Considerando el modelo $ARIMA(2,1,1)$, se ve que el coeficiente que corresponde a $AR(2)$ es estad\'{i}sticamente nulo (ver tabla 3.5), seg\'{u}n lo cual se debe volver a cambiar de modelo.

\begin{table}[H]
\centering
\begin{tabular}{@{}ccccc@{}}\hline
Par\'{a}metro & Estimado & Error Estd. & $t$ & Valor-$P$ \\ \hline
$AR(1)$ & 0,745854& 0,100892& 7,39258& 0,000000 \\
$AR(2)$ & -0,116622& 0,097009& -1,20218& 0,231382 \\
$MA(1)$ & -0,825507& 0,0626035& -13,1863& 0,000000 \\
Media   & 0,765563& 0,416395& 1,83855& 0,068163 \\
Constante & 0,283847& & &  \\ \hline
\end{tabular}
\caption{Resumen estad\'{i}stico para el modelo ARIMA(2,1,1) de la SIB}
\end{table}

Considerando el modelo $ARIMA(1,1,1)$ se obtiene que el coeficiente que corresponde a la constante en el modelo es estad\'{i}sticamente igual a cero; por lo que, es necesario modificarlo nuevamente (ver Tabla 3.6). 

\begin{table}[H]
\centering
\begin{tabular}{@{}ccccc@{}}\hline
Par\'{a}metro & Estimado & Error Estd. & $t$ & Valor-$P$ \\ \hline
$AR(1)$ & 0,64614& 0,0645419& 10,0112& 0,000000 \\
$MA(1)$ & -0,855448& 0,0491277& -17,4127& 0,000000 \\
Media   & 0,847445& 0,45797& 1,85044& 0,066405 \\
Constante & 0,299877& & &  \\ \hline
\end{tabular}
\caption{Resumen estad\'{i}stico para el modelo $ARIMA(1,1,1)$ de la SIB}
\end{table}

Finalmente, se adopta el mismo modelo $ARIMA(1,1,1)$ pero sin la constante (ver Tabla 3.7). Luego, esta ser\'{a} la mejor estimaci\'{o}n del modelo.

\begin{table}[H]
\centering
\begin{tabular}{@{}ccccc@{}}\hline
Par\'{a}metro& Estimado& Error Estd.& $t$&Valor-$P$ \\ \hline
$AR(1)$& 0,66579& 0,0603313& 11,0356& 0,000000 \\
$MA(1)$& -0,841889& 0,0515922& -16,3181& 0,000000 \\ \hline
\end{tabular}
\caption{Resumen estad\'{i}stico para el modelo $ARIMA(1,1,1)$ de la SIB sin constante}
\end{table}

\end{ejemplo}


\subsubsection{Pruebas concernientes al ruido blanco}

Se pretende probar si la serie de residuos estimados:
\[
\widehat{u}_{t}=\frac{\widehat{\Phi }(B)}{\widehat{\Theta }(B)}X_{t}
\]
es coherente con la hip\'{o}tesis de que los $\widehat{u}_{t}$ forman un ruido blanco; es decir, verificar si los residuos son no correlacionados y gaussianos (se puede probar que siempre son centrados). Estas pruebas pueden ser sustituidas por una prueba de independencia. 

\begin{itemize}
 \item \textbf{Utilizaci\'{o}n de las FAC y FACP residuales estimadas}
 
 Las pruebas para los residuos consisten en verificar que son no correlacionados. Para ello, lo m\'{a}s f\'{a}cil es verificar que no tienen una estructura ARMA; es decir, que sus funciones de autocorrelaci\'{o}n y autocorrelaci\'{o}n parcial residuales estimadas son estad\'{i}sticamente nulas (se encuentran dentro de las bandas de confianza). 

  \item \textbf{Prueba de Box-Pierce (``Overcoat'')}
  
  La prueba se basa en que el estad\'{i}stico:
 \[
 Q=T\sum_{l=1}^k {\widehat{\rho }_{\ell}^{2}\left( \widehat{u}_{t} \right)} 
 \]
 sigue asint\'{o}ticamente una ley $\chi^{2}$ con $(k-p-q)$ grados de libertad, bajo la hip\'{o}tesis de independencia de los $u_{t}$ $(\widehat{\rho }_{\ell}\left( \widehat{u}_{t} \right)$ representa la correlaci\'{o}n estimada de orden $\ell$, de los residuos estimados). Luego, se rechaza la hip\'{o}tesis de independencia de ruido blanco al nivel $\alpha $, si $Q>\chi_{1-\alpha }^{2}[k-p-q]$ (cuantil de orden $1-\alpha $ de la ley $\chi^{2}$ con ($k-p-q$) grados de libertad).
\end{itemize}

El n\'{u}mero $k$ debe ser suficientemente grande; habitualmente se lo toma entre 15 y 20.\newline

Cuando la hip\'{o}tesis se declara inadecuada, se pueden utilizar las funciones de autocorrelaci\'{o}n y autocorrelaci\'{o}n parcial estimadas de los residuos, para ver en qu\'{e} forma se modifica el modelo (ver la secci\'{o}n 3.8).

\begin{figure}[H]
\centering
\includegraphics[width=0.7\textwidth]{Graficos/Cap3/STcap38.eps}
\caption{FAC residual estimada del Modelo $ARIMA(2,1,2)$ de la SIB}
\end{figure}


\begin{figure}[H]
\centering
\includegraphics[width=0.7\textwidth]{Graficos/Cap3/STcap39.eps}
\caption{FACP residual estimada del Modelo $ARIMA(2,1,2)$ de la SIB}
\end{figure}


\begin{ejemplo}
En la serie planteada para el ejemplo y modelada por un $ARIMA(2,1,2)$, los gr\'{a}ficos de las funciones de autocorrelaci\'{o}n residual estimada (Figura 3.6) y de autocorrelaci\'{o}n parcial residual estimada (Figura 3.7), muestran los valores dentro de las bandas de confianza; lo mismo se verifica en la alternativa presentada con el modelo $ARIMA(2,1,1)$ (Figuras 3.8 y 3.9) y, finalmente, para el modelo escogido 
$ARIMA(1,1,1)$ sin constante, que se presentan en las Figuras 3.10 y 3.11.

\begin{figure}[H]
\centering
\includegraphics[width=0.7\textwidth]{Graficos/Cap3/STcap310.eps}
\caption{FAC residual estimada del Modelo $ARIMA(2,1,1)$ de la SIB}
\end{figure}


\begin{figure}[H]
\centering
\includegraphics[width=0.7\textwidth]{Graficos/Cap3/STcap311.eps}
\caption{FACP residual estimada del Modelo $ARIMA(2,1,1)$ de la SIB}
\end{figure}


\begin{figure}[H]
\centering
\includegraphics[width=0.7\textwidth]{Graficos/Cap3/STcap312.eps}
\caption{FAC residual estimada del Modelo $ARIMA(1,1,1)$ de la SIB}
\end{figure}


\begin{figure}[H]
\centering
\includegraphics[width=0.7\textwidth]{Graficos/Cap3/STcap313.eps}
\caption{FACP residual estimada del Modelo $ARIMA(1,1,1)$ de la SIB}
\end{figure}


Considerando el modelo seleccionado para la SIB ($ARIMA(1,1,1)$), se presenta la salida en el paquete EViews de las autocorrelaciones simples y parciales estimadas del modelo, con el respectivo estad\'{i}stico $Q$ y su valor $p$ asociado (ver la columna Prob. de la figura 3.12). Dado que en todos los casos el valor $p$ es mayor que 0,05, se consideran que los residuos no son significativos y por lo tanto se acepta la independencia de los mismos.

\begin{figure}[H]
\centering
\includegraphics[width=0.7\textwidth]{Graficos/Cap3/STcap314.eps}
\caption{FAC y FACP residuales estimadas del Modelo $ARIMA(1,1,1)$ de la SIB en EViews}
\end{figure}
\end{ejemplo}


\subsection{Utilizaci\'{o}n de los residuos estimados para modificar el Modelo}

Cuando la funci\'{o}n de autocorrelaci\'{o}n de los residuos estimados de un modelo ajustado indica que este es inadecuado, es necesario considerar de qu\'{e} manera podr\'{i}a ser modificado. 

Sup\'{o}ngase que los residuos estimados $\widehat{u}_{t}$ del modelo:
\[
{\Phi }_{0}\left( B \right){\Delta }^{d_{0}}X_{t}={\Theta }_{0}(B)u_{t}
\]

parecen no verificar las hip\'{o}tesis de independencia; utilizando las funciones de autocorrelaci\'{o}n y autocorrelaci\'{o}n parcial estimadas de los $\widehat{u}_{t}$ y los m\'{e}todos explicados anteriormente, se puede ajustar para la serie $u_{t}$ un modelo de la forma:

\[
{\Phi }'\left( B \right){\Delta }^{d'}u_{t}={\Theta }'(B)\varepsilon_{t}
\]

Eliminando $u_{t}$, de las ecuaciones anteriores, se puede a asumir el siguiente modelo:

\[
{\Phi }_{0}\left( B \right){\Phi }'\left( B \right){\Delta }^{d_{0}}{\Delta}^{d'}X_{t}={\Theta }_{0}(B){\Theta }'(B)\varepsilon_{t}
\]

que puede ser ajustado nuevamente y verificado. Lo importante en el modelo anterior son los grados de los nuevos polinomios asociados a los operadores respectivos (se suman en cada producto):

\[
{\Phi }\left( {B} \right){=}{\Phi }_{0}\left( B \right){\Phi }'\left( B \right) \qquad \text{ y }\qquad {\Theta }\left( {B} \right){=}{\Theta }_{0}(B){\Theta }'(B)
\]

Una aplicaci\'{o}n de estos procedimientos se presenta en el cap\'{i}tulo 4.

\subsection{Pruebas de estabilidad}

En algunos casos, estimar un modelo de series temporales puede verse afectado por la presencia de cambios en la estructura de la serie (o series) en estudio, lo que puede llevar a conclusiones inexactas o incorrectas al momento de realizar predicciones o en la interpretaci\'{o}n del comportamiento de la serie.\newline

Una de las etapas de validaci\'{o}n del modelo estimado consiste en estudiar si los par\'{a}metros estimados por el modelo son estables en toda la muestra. Si no se cumple este supuesto, la capacidad de predictibilidad del modelo ser\'{a} limitada.\newline

Una de los m\'{e}todos comunes para comprobar la estabilidad del modelo es dividir las $T$ observaciones en dos grupos:

\begin{itemize}
\item $T_{1}$: observaciones utilizadas en la estimaci\'{o}n del modelo
\item $T_{2}=T-T_{1}$: observaciones que se utilizar\'{a}n para las pruebas y la evaluaci\'{o}n del modelo.
\end{itemize}

Lo usual es que las primeras $T_{1}$ observaciones se utilicen para modelar y las \'{u}ltimas observaciones para las pruebas.\newline

Una de las pruebas tradicionales para evaluar esta estabilidad es la prueba de Chow, de punto de quiebre.

\subsubsection*{Prueba de punto de quiebre de Chow}

La idea de esta prueba consiste en estimar un modelo de cada submuestra por separado y ver si existen diferencias significativas en los modelos estimados, lo que se manifiesta en un cambio estructural en la relaci\'{o}n.\newline

La prueba de Chow compara la suma de residuos al cuadrado obtenidos a partir de un modelo estimado con toda la muestra con la suma de residuos al cuadrado que se obtiene cuando se estiman los modelos para cada subconjunto de las observaciones.\newline

Para la prueba se utiliza el estad\'{i}stico $F$, que se calcula de la siguiente manera:

\[
F=\frac{\left( \tilde{u}'\tilde{u}-(u_{1}'u_{1}+u_{2}'u_{2}) \right)/k}{(u_{1}'u_{1}+u_{2}'u_{2})/(T-2k)}
\]

donde,
\begin{itemize}
 \item $\tilde{u}'\tilde{u}$: Es la suma de cuadrados de los residuos considerando todas las observaciones.
 \item $u_{i}'u_{i}$: Es la suma de cuadrados de los residuos de la submuestra $i$  ($i=1,  2$)
 \item $T$: N\'{u}mero de observaciones.
 \item $k$: N\'{u}mero de par\'{a}metros en la ecuaci\'{o}n estimada.
\end{itemize}

Si no hubiera diferencias sustanciales en la estructura entre los dos modelos, el numerador ser\'{i}a muy peque\~{n}o; por tanto, se lo puede utilizar para contrastar la siguiente prueba de hip\'{o}tesis:

\[
\left\{ {\begin{array}{l}
 H_{0}\text{: No  existen  cambios  estructurales} \\ 
 H_{1}\text{: Existen  cambios  estructurales}  
 \end{array}} \right.
\]

El estad\'{i}stico $F$ tiene una distribuci\'{o}n de Fisher-Snedcor asint\'{o}ticamente, bajo $H$; es exacta si los errores son normales i.i.d.\newline

Otra forma de evaluar este fen\'{o}meno es mediante el estad\'{i}stico raz\'{o}n del logaritmo de verosimilitud o el estad\'{i}stico de Wald. Las interpretaciones de los resultados de estos tres estad\'{i}sticos son similares.

\begin{observacion}
El tratamiento para dos submuestras se puede generalizar a m\'{a}s de un punto de quiebre. 
\end{observacion}

\begin{ejemplo}
Considerando la SIB, se realiza la prueba de punto de quiebre para la observaci\'{o}n 71 (aproximadamente, la mitad de los datos). As\'{i}, se obtienen los siguientes resultados:

\begin{figure}[H]
\centering
\includegraphics[width=0.7\textwidth]{Graficos/Cap3/STcap315.eps}
\caption{Prueba de Chow para el punto de quiebre.}
\end{figure}

Dado que el $p$-valor correspondiente al estad\'{i}stico $F$ es mayor que 0,05, se acepta la hip\'{o}tesis nula de no existencia de cambios estructurales de la serie.
\end{ejemplo}

\begin{observacion}
En caso de que exista un cambio estructural de la serie, hay que tener mucho cuidado al realizar predicciones, ya que un cambio estructural al inicio del per\'{i}odo de pron\'{o}stico afecta en gran medida el poder predictivo del modelo. 
\end{observacion}


\subsection{Elecci\'{o}n del Modelo}

Puede suceder que varios modelos pasen la fase de verificaci\'{o}n. En este caso existen criterios que no poseen una base te\'{o}rica s\'{o}lida, pero son \'{u}tiles a la hora de realizar una elecci\'{o}n.

\subsubsection{Criterio del mayor poder predictivo}

En un modelo ARMA, el error de predicci\'{o}n al horizonte 1 ($e_{t}(1)$), es tal que su varianza $\V[e_{t}(1)]=\sigma^{2}$. Se puede entonces, proponer escoger el Modelo que conduzca a un error de predicci\'{o}n m\'{a}s peque\~{n}o; es decir, aquel que tenga la varianza estimada del ruido blanco m\'{a}s peque\~{n}a. Diversos criterios son aceptables:

\begin{enumerate}
\item[a)] La estimaci\'{o}n de la varianza ($\widehat{\sigma }^{2}$).
\item[b)] El coeficiente de determinaci\'{o}n:

\[
R^{2}=1-\frac{(\widehat{\sigma} )^{2}}{V}
\]

donde $V$ es la varianza muestral de la serie diferenciada $d$ veces. Este segundo criterio es simplemente una versi\'{o}n normada del anterior.

\item[c)] El coeficiente de determinaci\'{o}n modificado:
\[
\bar{R}^{2}=1-\frac{(\widehat{\sigma})^{2}/(N-p-q)}{V/(N-1)}
\]
que toma en cuenta los grados de los polinomios autoregresivo y media m\'{o}vil.

\item[d)] El estad\'{i}stico de Fisher:
\[
F=1-\frac{(V-(\widehat{\sigma} )^{2})/(p+q)}{\widehat{\sigma} ^{2}/(N-p-q)}
\]
introducido por analog\'{i}a con el caso del modelo lineal. 
\end{enumerate}

El criterio a) es para minimizar mientras que los otros son para maximizar.


\subsubsection{Criterio de informaci\'{o}n}

Otra aproximaci\'{o}n, introducida por Akaike (1969), consiste en suponer que los modelos $ARMA  (p,q)$ dan aproximaciones de la realidad y que la verdadera distribuci\'{o}n desconocida de las observaciones ${\Delta }^{d}X_{t}$ (observaciones de $\left( X_{t} \right)$ diferenciadas $d$ veces) no satisface forzosamente tal modelo. Se puede entonces fundamentar la elecci\'{o}n del modelo sobre una medida de la desviaci\'{o}n entre la verdadera ley desconocida y el modelo propuesto.\newline

La medida habitual es la cantidad de informaci\'{o}n de Kullback. Sean $f_{0}(X)$ la densidad desconocida de las observaciones y $\left\{ f\left( x \right),f\in F_{p,q} \right\}$ la familia de densidades correspondientes al modelo $ARMA  (p,  q)$; la desviaci\'{o}n entre la verdadera ley y el Modelo se mide por:

\[
I\left( f_{0},F_{p,q} \right)=\int {\log\frac{f_{0}\left( x \right)}{f\left( 
x \right)}f_{0}\left( x \right)dx} 
\]

Esta cantidad es siempre positiva y solo se anula si la verdadera densidad $f_{0}$ pertenece a $F_{p,q}$. $I\left( f_{0},F_{p,q} \right)$, el valor de la cantidad de informaci\'{o}n es evidentemente desconocido; pero si se dispone de un buen estimador $I\left( f_{0},F_{p,q} \right)$ de esta cantidad, ser\'{a} posible utilizarlo como criterio.\newline

Se escoge entonces el modelo $F_{p,q}$ que conduce al valor m\'{a}s peque\~{n}o de la estimaci\'{o}n $I\left( f_{0},F_{p,q} \right)$. Los estimadores de la cantidad de informaci\'{o}n que han sido propuestos son:

\begin{enumerate}
 \item $AIC\left( p,q \right)=\log{(\widehat{\sigma })}^{2}+\frac{2(p+q)}{T}$ (Akaike, 1969)
 \item $BIC\left( p,q \right)=\log{(\widehat{\sigma })}^{2}+(p+q)\frac{\log  T}{T}$ (Schwarz, 1978)
 \item $\varphi \left( p,q \right)=\log{(\widehat{\sigma })}^{2}+\left( p+q \right)c\frac{\log\left( \log  T \right)}{T}$ con  $c>2$ (Hannan-Quinn, 1979)
\end{enumerate}

El primero de estos estimadores es el m\'{a}s utilizado; sin embargo, solo los dos restantes son convergentes y conducen a una selecci\'{o}n asint\'{o}tica correcta del modelo.\newline

El objetivo es minimizar el error de predicci\'{o}n final, para lo cual los par\'{a}metros $p$ y $q$ pueden ser cambiados hasta obtener un valor m\'{i}nimo del estimador. Para modelos autoregresivos, simulaciones de Monte Carlo sugieren que AIC tiene una tendencia a sobrestimar el par\'{a}metro $p$; para corregir esta tendencia, se introduce el criterio modificado BIC (criterio de informaci\'{o}n bayesiano), donde $\widehat{\sigma }^{2}$ es el estimador de m\'{a}xima verosimilitud para la varianza del ruido blanco.\newline

En forma general, la introducci\'{o}n de estos criterios ayudan a la identificaci\'{o}n del modelo. Sin embargo, la b\'{u}squeda de un modelo que los minimice puede ser muy larga si no se tiene alguna idea de la clase de modelo a explorar.

\begin{observacion}
En ciertas ocasiones, no se obtienen los mismos resultados con todos los criterios; en este caso, se elige el modelo que cumpla con el mayor n\'{u}mero de criterios (los paquetes estad\'{i}sticos tambi\'{e}n suelen incorporar otros criterios como, por ejemplo, la raz\'{o}n de verosimilitud) 
\end{observacion}


\section{Predicci\'{o}n de los Modelos ARIMA}

\subsection{C\'{a}lculo de las predicciones \'{o}ptimas en un modelo ARIMA\index{Predicci\'{o}n! Modelo ARIMA}}

\begin{definicion}
La predicci\'{o}n lineal \'{o}ptima $\widehat{X}_{T} (h)$, al horizonte $h$ ($h>0$, entero), en el instante $T$, se define como la regresi\'{o}n af\'{i}n de $X_{{T+h}}$ sobre el espacio vectorial generado por $\{X_t, t\in \Z\}$.
\end{definicion}

Sup\'{o}ngase que $(X_{t}, t \in \Z)$ es un proceso $ARIMA(p,d,q)$. La representaci\'{o}n media m\'{o}vil de tal proceso se escribe:
\[
X_{t} =m_{t} +\sum_{i=0}^{t+s-p'-1} {\psi_{i} u_{t-i} } \qquad  t\in \Z
\]
Por tanto, para el instante $T+h$, se tiene:
\[
X_{T+h} =m_{T+h} +\sum_{i=0}^{T+h+s-p'-1} {\psi_{i} u_{T+h-i} } 
\]
donde $(u_{t})$ es un ruido blanco.\newline

La regresi\'{o}n af\'{i}n de $X_{{T+h}}$ sobre el e.v. generado por las $X_{t}$, $t\le T$, es la misma que la regresi\'{o}n af\'{i}n de $X_{{T+h}}$ sobre el e.v. generado por los $\{ u_{t} , t \le  T\}$; adem\'{a}s, \'{e}sta viene dada por:
\begin{equation}
\label{eq:3.01}
{\widehat{X}_{T} (h)=m_{T+h} 
+\sum_{i=h}^{T+h+s-p'-1} {\psi_{i} u_{T+h-i} } =m_{T+h} 
+\sum_{j=0}^{T+s-p'-1} {\psi_{j+h} u_{T-j} } }
\end{equation}

Esta f\'{o}rmula da la regresi\'{o}n afin de $X_{{T+h}}$ sobre los $X_{t}$ , $t\le T$, puesto que los $u_{t}$ son no correlacionados. Si, adem\'{a}s, los $u_{t}$ son normales o independientes entre ellos, $\widehat{X}_{T} (h)$ es igualmente la esperanza condicional de $X_{{T+h}}$ conocidas $X_{t}$, $t \le T$.\newline

El error de predicci\'{o}n se deduce directamente de (\ref{eq:3.01}):
\[
e_{T} (h)=X_{T+h} -\widehat{X} _{T} 
(h)=\sum_{i=0}^{h-1} {\psi_{i} u_{T+h-i} } 
\]

En particular: $e_{T} (1)=u_{T+1}$.\newline

Adem\'{a}s, el error en media cuadr\'{a}tica de $\widehat{X}_{T} (h)$, que en este caso coincide con la varianza de $\widehat{X}_{T} (h)$, es: 
\[
EMC_{T} (h)=\E(X_{T+h} -\widehat{X} _{T} (h))^{2}=\sigma ^{2}\sum_{i=0}^{h-1} {\psi_{i}^{2} } 
\]

En particular: $EMC_{T} (1)=\sigma^{2}$

\begin{observacion}
Por lo general, cuando $X_{t} $ se puede expresar en la forma denominada $MA(\infty )$:
\[
X_{t} =\sum_{i=0}^\infty {\psi_{i} u_{t-i} } \qquad t\in \Z
\]

se obtienen los mismos resultados para el error de predicci\'{o}n y el EMC de horizonte h. Los coeficientes $\psi_{i} $ se pueden obtener de la expresi\'{o}n $\psi (B)\phi (B)=\Theta (B),$ donde:

\[
\psi (B)=\sum_{i=0}^{\infty } \psi_{i} B^{i}
\]

Obs\'{e}rvese que $MA(\infty )$ es solamente una notaci\'{o}n, pues la serie $\psi (B)$ no siempre es convergente; no confundir con la definici\'{o}n de un proceso lineal ni con la de una media m\'{o}vil infinita.
\end{observacion}

La f\'{o}rmula \eqref{eq:3.01} no es utilizable directamente, pues los $u_{t}$ no son observables; es necesario entonces buscar otras expresiones de $\widehat{X} _{T} (h)$.\newline

Consid\'{e}rese tambi\'{e}n la representaci\'{o}n AR con $s$ grande (se pueden despreciar los t\'{e}rminos con los $\pi^{\ast}_{j, T+h}$):
\[
X_{T+h} =-\sum_{i=1}^{T+h+s-p'-1} {\pi_{i} X_{T+h-i} +u_{T+h} } 
\]
entonces:
\begin{equation}
\label{eq:3.02}
 \widehat{X} _{T} (h)=-\sum_{i=1}^{T+h+s-p'-1} {\pi_{i} \widehat{X}_{T+h-i} } 
\end{equation}
donde: 
\[
\widehat{X}_{T+h-i} =\begin{cases}
                          \widehat{X} _{T} (h-i) & \text{si } i<h \\
                          X_{T+h-i} & \text{si } i\ge h
                         \end{cases}
\]

Si se considera la representaci\'{o}n ARIMA:
\[
X_{T+h} =\sum_{i=1}^{P'} {\varphi_{i} X_{T+h-i} +u_{T+h} -} 
\sum_{i=1}^q {\theta_{i} u_{T+h-i} } 
\]
se obtiene: 
\begin{equation}
\label{eq:3.03}
\widehat{X} _{T} (h)=\sum_{i=1}^{p'} {\varphi_{i} \widehat{X} _{T+h-i} -\sum_{i=1}^q {\theta_{i} \widehat{u} _{T+h-i} } } 
\end{equation}
donde: 
\[
\widehat{u}_{T+h-i} =\begin{cases}
                      0 &  \text{si } i<h \\ 
                      u_{T+h-i}&  \text{si } i\ge h
                     \end{cases}
\]
Finalmente, consid\'{e}rese ahora la predicci\'{o}n de $X_{{T+h}}$ en el instante $T+1$:
\[
\widehat{X}_{T+1} (h-1)=m_{T+h}+\sum_{i=h-1}^{T+h+s-p'-1} {\psi_{i} u_{T+h-i} } 
\]
por tanto:
\begin{equation}
\label{eq:3.04}
\widehat{X} _{T+1} (h-1)-\widehat{X}_{T} (h)=\psi_{h-1} u_{T+1} =\psi_{h-1} (X_{T+1} -\widehat{X}  _{T} (1))
\end{equation}

Con esta f\'{o}rmula se puede proceder a actualizar las predicciones, cuando se ha observado $X_{T+1}$.\newline

En resumen:
\begin{enumerate}
\item Con \eqref{eq:3.02} se calcula $\widehat{X} _{T} (h)$, $h=1,2,3,\ldots,H$
\item Si se conoce $X_{{T+1}}$ se calcula con \eqref{eq:3.04}, $\widehat{X}_{T+1} (h)$, $h=1, 2,\ldots, H-1$ (actualizaci\'{o}n).
\item $\widehat{X} _{T} (H)$ se calcula con \eqref{eq:3.03} , bajo la condici\'{o}n de que $H > q$ (caso frecuente).
\end{enumerate}

\begin{observacion}
Para el c\'{a}lculo pr\'{a}ctico de las predicciones se suponen conocidos los par\'{a}metros $\psi_{i} $ y $\theta _{j} $ (luego de que se han sido estimados); lo mismo se asume con $u_{1} ,\ldots,u_{T} .$ En cuanto a $u_{0} ,u_{-1} ,\ldots,$ se los toma igual a sus esperanzas (es decir, como cero) o, alternativamente, se los estima.
\end{observacion}


\begin{ejemplo}
A continuaci\'{o}n se presentan algunos casos en los que se explicitan los predictores y los EMC:

\begin{enumerate}
\item Sea el modelo $ARIMA (0, 0, 2)$:
\[
X_{t} =u_{t} -\theta_{1} u_{t-1} -\theta_{2} u_{t-2} \quad t\in \Z
\]
En este caso:
\[
 \psi_{1} =-\theta_{1}; \qquad \psi_{2} =-\theta_{2}\qquad \text{y}\qquad \psi_{j} =0 \text{ para }j>2
\]

Se calculan a continuaci\'{o}n para varios per\'{i}odos de predicci\'{o}n el valor real u observado, el predictor y el correspondiente error medio cuadr\'{a}tico.

\begin{center}\small
\begin{tabular}{@{}cccc@{}}
\toprule
\textbf{Per\'{i}odo}& \textbf{Valor Real}& \textbf{Predictor}& \textbf{EMC} \\
\midrule
$T+1$& $X_{T+1} =u_{T+1} -\theta_{1} u_{T} -\theta_{2} u_{T-1} $& $\widehat{X}_{T} (1)=-\theta_{1} u_{T} -\theta_{2} u_{T-1} $& $\sigma_{u}^{2} $ \\[5pt]
$T+2$& $X_{T+2} =u_{T+2} -\theta_{1} u_{T+1} -\theta_{2} u_{T} $& $\widehat{X}_{T} (2)=-\theta_{2} u{ }_{T}$& $\left( {1+\theta_{1}^{2} } \right)\sigma_{u}^{2} $ \\[5pt]
$T+3$& $X_{T+3} =u_{T+3} -\theta_{1} u_{T+2} -\theta_{2} u_{T+1} $& $\widehat{X}_{T} (3)=0$& $\left( {1+\theta_{1}^{2} +\theta_{2}^{2} } \right)\sigma_{u}^{2} $ \\
\bottomrule
\end{tabular}
\end{center}

Puede verse que para $h>2$, resultar\'{a} que $\widehat{X}_{T} (h)=0$ y que $EMC(h)=\left( {1+\theta_{1}^{2} +\theta_{2}^{2} } \right)\sigma_{u}^{2} $. En un modelo MA(q), se verifica que $\widehat{X}_{T} (h)=0$ para $h > q$ y que el error medio cuadr\'{a}tico se estabiliza en el valor que toma para $h=q+1.$

\item Sea un modelo $ARIMA (0, 1, 0)$; es decir, un modelo ``marcha aleatoria''
\[
X_{t} =X_{t-1} +u_{t} \qquad t\in \Z
\]
Al pasar a la forma $MA( \infty )$ se tiene que $\psi_{i} =1$ para $i=1,2,3\ldots$.\newline

Para los per\'{i}odos $T+1$, $T+2$ y $T+3$ se obtienen los siguientes resultados: 

\begin{center}
\small
\begin{tabular}{@{}cccc@{}}
\toprule
\textbf{Per\'{i}odo}& \textbf{Valor Real}& \textbf{Predictor}& \textbf{ECM} \\
\midrule
$T+1$& $X_{T+1} =X_{T} +u_{T+1} $& $\widehat{{X}}_{T} \left( 1 \right)=X_{T} $&
$\sigma_{u}^{2}$ \\[5pt]
$T+2$& $X_{T+2} =X_{T+1} +u_{T+2} $& $\widehat{{X}}_{T} \left( 2 \right)=\widehat{{X}}_{T} \left( 1 \right)=X_{T} $& $2\sigma_{u}^{2}$ \\[5pt]
$T+3$& $X_{T+3} =X_{T+2} +u_{T+3} $& $\widehat{{X}}_{T} \left( 3 \right)=\cdots =X_{T} $& $3\sigma_{u}^{2}$ \\
\bottomrule
\end{tabular}
\end{center}


En un modelo ``marcha aleatoria'' para todos los per\'{i}odos futuros se predice con el valor correspondiente al \'{u}ltimo per\'{i}odo muestral. El EMC se incrementa en $\sigma_{u}^{2}$ para cada per\'{i}odo adicional.

\item Sea un modelo $ARIMA( {1,1,0})$
\[
\left( {1-\varphi_{1} B} \right)\left( {1-B} \right)X_{t} =u_{t} 
\]

Pasando a un ARMA mediante la utilizaci\'{o}n de par\'{a}metros autorregresivos generalizados se tiene que

\[
X_{t} -\phi_{1} X_{t-1} -\phi_{2} X_{t-2} =u_{t} 
\]

donde $\phi_{1} =\varphi_{1} +1y\phi_{2} =-\varphi_{1}$.\newline

A partir de la relaci\'{o}n:
\[
\psi \left( B \right)\phi \left( B \right)=1
\]

Se tiene que $\psi_{0} =1,$ y $\psi_{1} =\varphi_{1} .$ Tomando $\psi_{0} y\psi_{1} $ como valores iniciales, los siguientes valores de $\psi_{i} $ se determinan recursivamente mediante la f\'{o}rmula:

\[
\psi_{k} =\phi_{1} \psi_{k-1} +\phi_{2} \psi_{k-2} ,\qquad k\ge 2
\]

Para los per\'{i}odos $T+1,T+2yT+3$ se obtienen los siguientes resultados:

\begin{center}\small
\begin{tabular}{@{}cccc@{}}
\toprule
\textbf{Per\'{i}odo}& \textbf{Valor Real}& \textbf{Predictor}& \textbf{ECM} \\
\midrule
$T+1$& $X_{T+1} =\varphi_{1} X_{T} +\varphi_{2} X_{T-1} +u_{T+1} $& $\widehat{{X}}_{T} \left( 1 \right)=\varphi_{1} X_{T} +\varphi_{2} X_{T-1} $& $\sigma_{u}^{2}$ \\[5pt]
$T+2$& $X_{T+2} =\varphi_{1} X_{T+1} +\varphi_{2} X_{T} +u_{T+2} $& $\widehat{{X}}_{T} \left( 2 \right)=\varphi_{1} \widehat{{X}}_{T} \left( 1 \right)+\varphi_{2} X_{T} $& 
$\left( {1+\psi_{1}^{2} } \right)\sigma_{u}^{2} $ \\[5pt]
$T+3$& $X_{T+3} =\varphi_{1} X_{T+2} +\varphi_{2} X_{T+1} +u_{T+3} $& $\widehat{{X}}_{T} \left( 3 \right)=\varphi_{1} \widehat{{X}}_{T} \left( 2 \right)+\varphi_{2} \widehat{{X}}_{T} \left( 1 \right)$& $\left( {1+\psi_{1}^{2} +\psi_{2}^{2} } \right)\sigma_{u}^{2} $ \\
\bottomrule
\end{tabular}
\end{center}


\item Sea el modelo $ARIMA( {1,0,1})$
\[
X_{t} =\varphi_{1} X_{t-1} +u_{t} -\theta_{1} u_{t-1} 
\]

El coeficiente $\psi_{1} =\varphi_{1} -\theta_{1} .$ Los siguientes valores se calculan de forma recursiva a partir de

\[
\psi_{k} =\varphi_{1} \psi_{k-1} ,\quad \quad k\ge 1
\]

Para este modelo, al predecir en los per\'{i}odos $T+1$,$T+2$ y $T+3$ se obtienen los siguientes resultados:

\begin{center}\small
\begin{tabular}{@{}cccc@{}}
\toprule
\textbf{Per\'{i}odo}& \textbf{Valor Real}& \textbf{Predictor}& \textbf{EMC} \\
\midrule
$T+1$& $X_{T+1} =\varphi_{1} X_{T} +u_{T+1} -\theta_{1} u_{T} $& $\widehat{X}_{T} (1)=\varphi_{1} X_{T} -\theta_{1} u_{T} $& $\sigma_{u}^{2} $ \\[5pt]
$T+2$& $X_{T+2} =\varphi_{1} X_{T+1} +u_{T+2} -\theta_{1} u_{T+1} $& $\widehat{X}_{T} (2)=\varphi_{1} \widehat{X}_{T} (1)$& $\left( {1+\psi_{1}^{2} } \right)\sigma_{u}^{2} $ \\[5pt]
$T+3$& $X_{T+3} =\varphi_{1} X_{T+2} +u_{T+3} -\theta_{1} u_{T+2} $& $\widehat{X}_{T} (3)=\varphi_{1} \widehat{X}_{T} (2)$& $\left( {1+\psi_{1}^{2} +\psi_{2}^{2} } \right)\sigma_{u}^{2} $ \\
\bottomrule
\end{tabular}

\end{center}
\end{enumerate}
\end{ejemplo}


\subsection{Intervalos de predicci\'{o}n}

Si\index{Predicci\'{o}n!Intervalos} se asume que $u_{t} \sim N\left( {0,\sigma^{2}} \right)$, entonces:
\[
 e_{T} (h)=X_{T+h} -\widehat{X} _{T} (h)=\sum_{i=0}^{h-1} {\psi_{i} u_{T+h-i} }  \sim N\left({0,\sigma^{2}\sum_{i=0}^{h-1} {\psi_{i}^{2} } } \right)
\]

As\'{i}, se puede obtener un intervalo de confianza, llamado intervalo de predicci\'{o}n, para $X_{T+h}$:

\[
I=\widehat{X} _{T} (h)\pm u_{1-\frac{\alpha }{2}} \sigma 
\sqrt {\sum_{i=0}^{h-1} {\psi_{i}^{2} } } 
\]

donde $u_{1-\frac{\alpha}{2}}$ es el  cuantil de orden $(1-\frac{\alpha}{2})$  de la distribuci\'{o}n $N(0,1)$.\newline

Para calcular este intervalo, se requiere reemplazar $\sigma$ por su estimaci\'{o}n.

\subsection{Funciones de predicci\'{o}n}

Se\index{Predicci\'{o}n!Funciones de} conoce que:
\[
\widehat{X}_{_{t} } (h)=\sum_{i=1}^{p'} {\varphi_{i} 
\widehat{X}_{t} (h-i)} \Longleftrightarrow \widehat{X}_{t} (h)-\sum_{i=1}^{p'} {\varphi_{i} \widehat{X}_{t}  (h-i)=0}  \qquad h>q
\]

Si $\left(h\le i,\ \widehat{X}_{t} (h-i)=X_{t+h+i} \right);$ es decir, se tiene una ecuaci\'{o}n en recurrencia.\newline

Se llamar\'{a} funci\'{o}n de predicci\'{o}n a $\widehat{X}_{t} (h)$, (con $t$ fijo). Entonces la soluci\'{o}n general de $\widehat{X} _{t} (h)$ es:

\[
\widehat{X} _{t} (h)=\sum_{i=1}^{p'} {b_{i} (t)f_{i} (h)} 
\]

donde los $f_{i}(h)$ son las soluciones elementales de la ecuaci\'{o}n de recurrencia que tiene por polinomio caracter\'{i}stico a:

\[
z^{p'}-\sum_{i=1}^{p'} {\varphi_{i} z^{p'-i}} =z^{p'}\left[ 
{1-\sum_{i=1}^{p'} {\varphi_{i} } \left( {\frac{1}{z}} \right)^{i}} 
\right]=z^{p'}\varphi \left( {\frac{1}{z}} \right)
\]

es decir, las inversas de las ra\'{i}ces de $\varphi \left( z \right)=0$. Los $b_{i}(t)$ se determinan de los $p'$ valores iniciales (ver Anexo B.3).

\begin{observacion}

\begin{enumerate}
\item[i.] Si no hay diferenciaci\'{o}n: $\varphi (z)=\Phi (z);$ adem\'{a}s, si las ra\'{i}ces son distintas, se tiene:
\[
 \widehat{X} _{t} (h)=\sum_{i=1}^p {b_{i} (t)\lambda_{i}^{h} }
\]
(los $\lambda_i$ inversas de las ra\'{i}ces de $\Phi(z)=0$)
\[
\therefore \widehat{X} _{t} (h) \xrightarrow[m.c.]{h-\infty} 0 
\]
El resultado es cierto tambi\'{e}n en el caso general, si las ra\'{i}ces de $\Phi (z)$ son de $\left| \right|>1$.

\item[ii.] Si $\phi(z)=(1-z)^{d} \Phi (z)$, $\widehat{X}_{T}(h)$ admite como l\'{i}mite un polinomio asint\'{o}tico de grado $(d-1)$.

\item[iii.] Si adem\'{a}s en el modelo existe una constante $\theta_{0}$ , hay que a\~{n}adir una soluci\'{o}n particular; por ejemplo:
\[
\frac{\theta_{0} h^{d}}{d!\Phi (1)}
\]
\end{enumerate}
\end{observacion}

\begin{ejemplo}
Consid\'{e}rese el modelo;
\[
(1-B)^{2} (1-0,5B) X_{t}=2+(1- 0,8B) u_{t} 
\]
con los siguientes valores iniciales; $\widehat{X} _{t} (1)= 1$, $X_{t}= 2$, $X_{t-1} =-1$.\newline

La forma general de $\widehat{X} _{T} (h)$ es:
\[
\widehat{X} _{T} (h)= a + bh + c(1/2)^{h} +\frac{2h^{2}}{2!(1/2)}
\]

Para determinar los coeficientes $a$, $b$ y $c$, se deben utilizar los valores iniciales de $\widehat{X} _{T} (h)$ con $h=1, 0, -1$:

\begin{align*}
\widehat{X} _{T} (1)&= a+b+c(1/2)+2 =1\\
\widehat{X} _{T} (0)&= a+c = 2\\
\widehat{X} _{T} (-1)= a- b+ 2c + 2= -1 
\end{align*}

Se debe resolver entonces, el sistema de ecuaciones:
\[
\systeme{ 2a+2b+c=-2, a-b+2c=-3 , a+c=2 }
\]
que tiene como soluci\'{o}n: $a=18$, $b=-11$, $c=-16$; entonces:
\[
 \widehat{X}_{T} (h)=18 - 11 h -16(1/2)^{h} + 2h^{2}
\]
\end{ejemplo}


\subsection{Contraste de la estabilidad estructural en la predicci\'{o}n }

Cuando se ha utilizado un modelo\index{Predicci\'{o}n!Estabilidad estructural} ARIMA, ya validado, para predecir los valores de la variable en per\'{i}odos extramuestrales, se puede plantear el siguiente interrogante: ?`sigue siendo v\'{a}lido el modelo para los per\'{i}odos de predicci\'{o}n?; en otras palabras, ?`se ha producido un cambio estructural en los per\'{i}odos de predicci\'{o}n?\newline

Con objeto de obtener una respuesta a esta cuesti\'{o}n se puede utilizar el siguiente estad\'{i}stico.

\begin{equation}\label{eq:3.05}
 E_{m}^{\ast }=\frac{\displaystyle\sum_{i=0}^m {e_{T+h}^{2}(i)} }{\widehat{\sigma}^{2}} 
\end{equation}

donde $e_{T+h}( i)$ es el error de predicci\'{o}n de $X_{T+h+i} $ utilizando la informaci\'{o}n disponible hasta el momento $T+h$.\newline

Se puede demostrar que el estad\'{i}stico tiene asint\'{o}ticamente una distribuci\'{o}n $\chi^{2}$ con $m$ grados de libertad, bajo la hip\'{o}tesis nula de que no se ha dado una ruptura estructural al pasar a los per\'{i}odos de predicci\'{o}n. Para comprender el significado de este contraste debe tenerse en cuenta que si el modelo sigue siendo correcto se verificar\'{a} que:

\[
e_{T+`h} ( i )\approx u_{T+h+i} 
\]

Por el contrario, en la medida que esto no ocurra, tender\'{a} a aumentar el EMC de cada predicci\'{o}n, y por tanto el estad\'{i}stico $E_{m}^{\ast},$ lo que llevar\'{i}a a rechazar la hip\'{o}tesis nula.

\begin{observacion}
Esta prueba tambi\'{e}n se la encuentra con el nombre de \emph{Prueba de predicciones de Chow}\index{Prueba de Chow!Predicci\'{o}n}. Se puede utilizar el siguiente estad\'{i}stico:

\[
F=\frac{\left( \tilde{u}'\tilde{u}-u_{1}'u_{1} 
\right)/T_{2}}{u_{1}'u_{1}/(T_{1}-k)}
\]

donde
\begin{itemize}
      \item $\tilde{u}'\tilde{u}$: Es la suma de cuadrados de los residuos considerando todas las observaciones.
      \item $u_{1}'u_{1}$: Es la suma de cuadrados de los residuos de la submuestra $T_{1}$
      \item $T_{1}$: Es un subper\'{i}odo de $T$.
      \item $T$: N\'{u}mero de observaciones.
      \item $k$: N\'{u}mero de coeficientes estimados.
      \item $T_{2}$: N\'{u}mero de puntos de predicci\'{o}n.
\end{itemize}

$F$ sigue una distribuci\'{o}n exacta si los errores son i.i.d.\newline

Por otro lado, tambi\'{e}n se puede considerar la estimaci\'{o}n del logaritmo de la raz\'{o}n de verosimilitud que est\'{a} basado en la comparaci\'{o}n del m\'{a}ximo de las funciones de verosimilitud restringida y sin restringir. Ambos logaritmos de verosimilitud son obtenidos a partir de la regresi\'{o}n utilizando toda la muestra. La prueba tiene una distribuci\'{o}n asint\'{o}tica $\chi^{2}$ con $T_{2}$ grados de libertad, bajo $H_{0}$, para la siguiente prueba de hip\'{o}tesis:

\[
\begin{cases}
 H_{0}:&\text{No  existen  cambios  estructurales}\\
 H_{1}:&\text{Existen  cambios  estructurales}
\end{cases}
\]

Cabe recalcar que esta formulaci\'{o}n es similar a la presentada anteriormente.
\end{observacion}


\begin{ejemplo}
Considerando la SIB se realiza la prueba anterior, teniendo en cuenta que se predicen desde la observaci\'{o}n 135 hasta la 141. El resultado es el siguiente:

\begin{figure}[H]
\centering\small
\begin{tabular}{lccc}
\multicolumn{4}{l}{Chow Forecast Test}   \\
\multicolumn{4}{l}{Equation: UNTITLED} \\
\multicolumn{4}{l}{Specification: $D(SER1,1)$ $AR(1)$ $MA(1)$}   \\
\multicolumn{4}{l}{Test predictions for observations from 135 to 141} \\
\toprule
 & Value& df& Probability  \\
 \midrule
$F$-statistic& ~0.363066& (7, 130)& ~0.9221 \\
Likelihood ratio& ~2.691185& ~7& ~0.9120 \\
\bottomrule
\end{tabular}
\caption{Prueba de predicciones de Chow}
\end{figure}

Como se puede observar, el valor $p$ asociado al estad\'{i}stico $F$ es mayor que 0,05; por lo tanto, se acepta la hip\'{o}tesis nula de que no existen cambios estructurales en las predicciones.
\end{ejemplo}


\section{Transformaci\'{o}n de Datos}

El operador $\Delta^{d}= (1-B)^{d}$ aplicado a $X_{t} $ tiene por objeto volver a la serie estacionaria. Sin embargo, para ciertos tipos de series no se podr\'{a} obtener la estacionariedad de esta manera. Por ejemplo: si la serie $X_{t} $ tiene una esperanza matem\'{a}tica que es funci\'{o}n exponencial de $t$ (tendencia exponencial), ning\'{u}n operador $\Delta^{d}$ podr\'{a} anularla, ya que $\Delta^{d}\left( e^{at} \right)=e^{a}(e^{t}-e^{t-d})$; una idea natural en este caso consiste en transformar\index{Transformaci\'{o}n de datos!Logar\'{i}tmica} la serie $X_{t} $ (supuesta positiva) por la funci\'{o}n $\ln$  antes de aplicar un operador $\Delta^{d}$.\newline

De igual manera, si la serie es del tipo $X_{t}=(at +b)Z_{t}$, donde $(Z_{t})$ es un proceso estacionario con $\E Z_{t} = 1$, $\V Z_{t}= \sigma^{2}$ y $a> 0$, se tiene:

\begin{itemize}
\item $\E X_{t}  = at+b$.
\item $\V(X_{t}) =\sigma^{2}(at +b)^{2}$.
\item $\E\Delta X_{t} = a$.
\item $\V(\Delta X_{t}) = \sigma^{2}\left\{ {\left[ {at+b} \right]^{2}+\left[ {a(t-1)+b} \right]^{2}-2\left[ {at+b} \right]\left[ {a(t-1)+b} \right]\rho (1)} \right\}$\\ $\phantom{\V(\Delta X_{t}) }=\sigma^{2}\left\{ {a^{2}+2(1-\rho (1))\left[ {at+b} \right]\left[ {a(t-1)+b} \right]} \right\}$
\end{itemize}

As\'{i}, $\Delta X_{t}$ tiene una esperanza constante y una varianza creciente con $t; $si se toma previamente logaritmos, se obtiene: 

\[
\Delta \ln X_{t} =\ln \frac{at+b}{a(t-1)+b}+\Delta \ln Z_{t} 
\approx \Delta \ln Z_{t} 
\]

Si $(Z_{t})$ es estacionario (en sentido estricto), $\Delta \ln X_{t}$ lo ser\'{a} asint\'{o}ticamente.\newline

Se constata que la transformaci\'{o}n $\ln$ puede ser \'{u}til y las indicaciones generalmente admitidas de su utilidad son: un crecimiento exponencial de la tendencia o una crecimiento de la variabilidad de $\Delta X_{t}$ junto con una estabilidad en media.\newline

Si se supone que la serie $Y_{t} =\ln X_{t} $ es un proceso ARIMA, se le puede aplicar las t\'{e}cnicas vistas y obtener as\'{i} $\widehat{Y}_{t} (h)=\E\left[ Y_{t+h} | Y_{t}  ,Y_{t-1} ,\ldots  \right]$. El problema es entonces deducir de $\widehat{Y}_{t} (h)$ una predicci\'{o}n $X_{t}^{\ast } (h)$ de $X_{t+h} $. La primera idea consiste en tomar:
\[
X_{t}^{\ast } (h)=\exp \left[ {\widehat{Y} _{t} (h)} \right]
\]
Es claro, sin embargo, que $X_T^{\ast}(h)$ es diferente de la predicci\'{o}n \'{o}ptima $\widehat{X}_t(h)=\E[X_{t+h}|X_t,X_{t+1},\ldots]$, pues por la desigualdad de Jensen (para convexidad) se tiene:
\begin{align*}
\widehat{X} _{t} (h)&=\E\left[ X_{t+h} | X_{t} ,X_{t-1} ,\ldots \right]\\
	&=\E\left[ \exp (\ln X_{t+h}) | X_{t} ,X_{t-1} ,\ldots \right]\\
	&>\exp \E\left[ \ln X_{t+h} | X_{t} ,X_{t-1} ,\ldots  \right]\\
	&=\exp \E\left[ Y_{t+h} | Y_{t} ,Y_{t-1} ,\ldots \right]\\
	&=\exp \left[ \widehat{Y}_{t} (h) \right]=X_{t}^{\ast } (h) 
\end{align*}

Sup\'{o}ngase ahora que $Y_{t}$ es normal (o $X_{t}$ $\log$-normal). Se tiene entonces:
\[
\widehat{X} _{t} (h)=\E\left[ \exp Y_{t+h} | Y_{t} ,Y_{t+1} ,\ldots \right]=\exp \left[\widehat{Y}_{t} (h) \right]\exp \left[ \frac{1}{2}\V(Y_{t+h} )| Y_{t},Y_{t-1} ,\ldots \right]
\]
ya que la esperanza de la ley $\log$-normal asociada a la ley normal $N(m, 
\sigma^{2})$ es:
\[
\exp \left[ {m+\frac{\sigma^{2}}{2}} \right]
\]
As\'{i}, se tiene:
\[
\widehat{X} _{t} (h)=X_{t}^{\ast } (h)\exp \left[ {\frac{1}{2}V(Y_{t+h} )| Y_{t} ,Y_{t-1} ,\ldots} \right]
\]

Para calcular el factor de correcci\'{o}n, recordando que los $u_{t}$ son el ruido blanco asociado a $Y_{t}$, obs\'{e}rvese que:

\begin{align*}
\V(Y_{t+h} | Y_{t} ,Y_{t-1} ,\ldots)
	&=\V(Y_{t+h}| u_{t},u_{t-1} ,\ldots) \\ 
	&=\V\left(\sum_{i=0}^{h-1} \psi_{i} u_{t-1} \right) \\ 
	&=\sigma^{2}\sum_{i=0}^{h-1} {\psi_{i}^{2} }
\end{align*}

Finalmente, se toma como predictor de $X_{{t+h}}$:

\[
\widehat{X} _{t} (h)=\exp \left[ {\widehat{Y}_{t} (h)+\frac{\sigma^{2}}{2}\sum_{i=0}^{h-1} {\psi _{i}^{2} } } \right]
\]

En la pr\'{a}ctica se reemplazan los par\'{a}metros por sus estimadores.\newline

La transformaci\'{o}n logar\'{i}tmica es sin duda la m\'{a}s utilizada para las series econ\'{o}micas. Sin embargo, esta transformaci\'{o}n es un elemento de una clase m\'{a}s general (llamada clase de transformaciones de Box-Cox\index{Transformaci\'{o}n de datos!Box-Cox}) definida por:

\[
T_{\lambda } (X_{t} )=\begin{cases}
                       \displaystyle\frac{X_{t}^{\lambda } -1}{\lambda }&\lambda \ne 0\\[1mm]
                       \ln X_{t} & \lambda =0
                      \end{cases}
\]

$X_{t}$ se supone positiva. Obs\'{e}rvese que $\ln X_{t}=\lim_{\lambda \to 0} T_{\lambda } (X_{t} )$.


\section{Ra\'{i}ces Unitarias}

En el desarrollo de la metodolog\'{i}a de Box-Jenkins para el tratamiento de las series temporales (ver cap\'{i}tulo 2), se exige que antes de la fase de identificaci\'{o}n, si las series no son estacionarias se hagan las transformaciones oportunas para lograr la estacionariedad. Una de las transformaciones m\'{a}s comunes es la ``diferenciaci\'{o}n\index{Diferenciaci\'{o}n}'', que consiste en considerar la transformaci\'{o}n:

\[
Y_{t}=\left( 1-B \right)X_{t}=X_{t}-X_{t-1}
\]

Puesto que el polinomio autoregresivo asociado ${\Phi }\left( z \right)=1-z$ tiene como ra\'{i}z al valor 1, se dice que el proceso ($Y_{t})$ tiene una ra\'{i}z unitaria\index{Ra\'{i}z unitaria}; tambi\'{e}n se dice que la serie ($Y_{t})$ es integrada de orden 1 y se denota por $I(1)$. Obviamente, de manera similar, se pueden definir ra\'{i}ces unitarias de orden superior.\newline

El gr\'{a}fico de la serie temporal o el de sus correlogramas se utilizan frecuentemente para determinar la necesidad de ``diferenciar'' la serie; son instrumentos emp\'{i}ricos para detectar la presencia de ra\'{i}ces unitarias, pero pueden resultar algo imprecisos.\newline

A partir de 1979 se han desarrollado pruebas de hip\'{o}tesis apropiadas para saber si la serie tiene o no ra\'{i}ces unitarias; \'{e}stas se realizan utilizando la metodolog\'{i}a propuesta por Dickey y Fuller (de ah\'{i} se desprende el nombre de la prueba). En este documento se utilizar\'{a} el paquete EViews, que proporciona las pruebas Dickey-Fuller Aumentada (ADF) y la de Phillips-Perron.

\subsection{La prueba de Dickey-Fuller}

La forma m\'{a}s f\'{a}cil de presentar esta prueba\index{Ra\'{i}z unitaria!Prueba de Dickey-Fuller} es considerar un modelo $AR(1)$:

\[
X_{t}=\rho X_{t-1}+u_{t},\qquad        
t  \epsilon   Z
\]

con $\left| \rho \right|<1$ y $(u_{t})$ un ruido blanco. La condici\'{o}n $\left| \rho \right|<1$ es necesaria y suficiente para que el proceso sea estacionario y lineal. En este caso $X_{t}$ es la innovaci\'{o}n en el instante t; es decir, $u_{t}$ influye sobre los valores de $X_{t}$ correspondientes al mismo per\'{i}odo, o a per\'{i}odos posteriores, pero nunca ejerce influencia sobre los valores de $X_{t}$ correspondientes a per\'{i}odos anteriores.\newline

Consid\'{e}rese ahora el modelo $AR(1)$ precedente. Si el coeficiente de $X_{t-1}$ es en realidad igual a 1, surge el problema de ra\'{i}z unitaria; es decir, una situaci\'{o}n de no estacionariedad. Por consiguiente si se realiza la regresi\'{o}n:

\[
X_{t}=\rho X_{t-1}+u_{t}
\]

y si se encuentra que $\rho =1$, entonces se dice que la variable $X_{t}$ tiene una ra\'{i}z unitaria (en este caso se dice que la serie es una \emph{caminata aleatoria}\index{Caminata aleatoria}). En este caso se realiza la prueba de hip\'{o}tesis:

\[
\begin{cases}
H_{0}:&\rho =1 \\ 
H_{1}:&\rho <1
\end{cases}
\]

Con frecuencia, a la ecuaci\'{o}n anterior se la expresa de forma alternativa, utilizando el operador de primera diferencia $\Delta =1-B$ y sustrayendo de cada lado $X_{t-1}$:

\[
{\Delta }X_{t}=\left( \rho -1 \right)X_{t-1}+u_{t}
\]
\[
{\Delta }X_{t}=\left( \rho -1 \right)X_{t-1}+u_{t}=\delta 
X_{t-1}+u_{t}
\]

donde $\delta =(\rho -1)$ y ${\Delta }X_{t}=X_{t}-X_{t-1}$. Ahora las hip\'{o}tesis nula y alternativa son: 

\[
\begin{cases}
 H_{0}:&\delta =0 \\ 
 H_{1}:&\delta <0 
\end{cases}
\]

Para averiguar si una serie temporal es o no estacionaria se efect\'{u}a una regresi\'{o}n, utilizando cualquiera de los modelos anteriores y se determina si $\rho =1$ estad\'{i}sticamente (o  $\delta =0$). Esta es la forma en que aparece la prueba en el paquete EViews.\newline

Bajo la hip\'{o}tesis nula que $\rho =1$, el estad\'{i}stico $t$ no sigue una distribuci\'{o}n $t$-student convencional. Dickey y Fuller mostraron que sigue una distribuci\'{o}n especial y calcularon sus valores cr\'{i}ticos con base en simulaciones de Monte Carlo. De esta manera, al estad\'{i}stico $t$ calculado convencionalmente se le denota como el estad\'{i}stico $\tau $, y a su prueba como la de Dickey y Fuller.\newline

Note que $H_{0}$ se acepta si el valor observado de $\tau $ es mayor al valor cr\'{i}tico correspondiente, de acuerdo al nivel de significaci\'{o}n que se adopte para la prueba. Si la hip\'{o}tesis nula se rechaza; es decir, si la serie de tiempo es estacionaria, se puede utilizar la prueba $t$-student usual.\newline

Las tablas iniciales no eran del todo adecuadas, por lo cual fueron ampliadas por MacKinnon mediante simulaciones; \'{e}stas son las que utiliza el paquete Eviews.\newline

Existen tres versiones del modelo de regresi\'{o}n:
\begin{enumerate}
 \item[a)] Modelo sin componentes deterministas, que ya ha sido tratado:
\[
X_{t}=\rho X_{t-1}+u_{t}   \qquad   t=2,3,\ldots ,T
\]
\item[b)] Modelo con un t\'{e}rmino constante:
\[
X_{t}=\alpha +\rho X_{t-1}+u_{t} \qquad t=2,3,\ldots,T
\]

en el cual la hip\'{o}tesis nula conjunta es: $H_{0}:\alpha =0,  \rho =1.$ Si se rechaza la hip\'{o}tesis nula $H_{0}:\alpha =0,  \rho =1$ y se concluye que $\alpha \ne 0$ y $\rho <1$, se tendr\'{a} que el proceso es estacionario con media no nula. Si $\alpha =0$, se obtiene el caso precedente.

\item[c)] Modelo con un t\'{e}rmino constante y una tendencia lineal
\[
X_{t}=\alpha +\beta t+\rho X_{t-1}+u_{t} \qquad t=2,3,\ldots,T
\]

en el cual, si al rechazar la hip\'{o}tesis nula $H_{0}:\alpha =\beta =0$,  $\rho =1$, se concluye que $\beta \ne 0$ y $\rho <1$, se dice que el proceso es estacionario alrededor de una tendencia lineal. Note que en este caso el proceso no es estacionario en el sentido de la definici\'{o}n; sin embargo, luego de una diferenciaci\'{o}n se torna en estacionario. As\'{i}, el modelo solamente ahorra un paso en el an\'{a}lisis.
\end{enumerate}

Si se ha aceptado $H_{0}$ y se desea saber si se requiere una segunda diferenciaci\'{o}n, se aplica el mismo procedimiento a la serie $\Delta X_{t}$. Los modelos de regresi\'{o}n precedentes tambi\'{e}n se pueden emplear en este caso.

\subsection{Extensiones de la prueba de Dickey-Fuller}

La prueba de ra\'{i}z unitaria descrita hasta el momento es v\'{a}lida solamente si la serie obedece a un proceso $AR(1)$. Si la serie tiene una correlaci\'{o}n de un orden mayor, se viola el supuesto de ruido blanco sobre los residuos, por lo cual se requiere aplicar la prueba de Dickey -- Fuller\index{Ra\'{i}z unitaria!Prueba de Dickey-Fuller Aumentada} aumentada (ADF), en la que se realiza una correcci\'{o}n param\'{e}trica para correlaciones de orden superior asumiendo que la serie $X_{t}$ sigue un proceso $AR(p)$; esto es, a\~{n}ade retardos de la primera diferencia de $X_{t-1}$ al lado derecho de la regresi\'{o}n. El modelo de manera general se expresa por:

\[
\Delta X_{t}=\alpha +\beta t+\delta X_{t-1}+\sum_{i=0}^{p-1} {\gamma 
_{i}\Delta X_{t-i}} +u_{t}
\]

De esta manera, la hip\'{o}tesis nula y alternativa a contrastarse son: $H_{0}:\delta =0$ y $H_{1}:\delta <0$.\newline

Said y Dickey (1985) mostraron que esta prueba es asint\'{o}ticamente v\'{a}lida incluso para series que presenten una componente media m\'{o}vil; solamente, debe escogerse un valor de $k=p-1$ adecuado (en general se utiliza como $k$ a la parte entera de la ra\'{i}z cuarta del tama\~{n}o muestral); de manera que, para series grandes con componente media m\'{o}vil pueden utilizarse los mismos valores cr\'{i}ticos.\newline

El contraste de Phillips-Perron (PP), es una generalizaci\'{o}n del contraste de DF, que permite flexibilidad en las hip\'{o}tesis relativas a la distribuci\'{o}n de errores.

\begin{observacion}
Si se a\~{n}ade el t\'{e}rmino independiente y/o la tendencia determinista en los modelos de regresi\'{o}n de DF o de DFA sin que lo exija el proceso, se reducen los grados de libertad, se aumenta el tama\~{n}o de la regi\'{o}n cr\'{i}tica y se reduce adem\'{a}s la potencia de la prueba, lo que puede conducir a una interpretaci\'{o}n err\'{o}nea del resultado de los contrates. Por tanto, es muy importante determinar si se incluyen o no el t\'{e}rmino independiente y/o tendencia determinista en la regresi\'{o}n seleccionada.\newline

En la pr\'{a}ctica, es mejor empezar tratando el modelo (c) y de acuerdo a los resultados de los coeficientes $\alpha$ y $\beta$ adoptar el modelo adecuado. Tambi\'{e}n, se debe tener cuidado de elegir los coeficientes $\gamma_{i}$ necesarios (estad\'{i}sticamente significativos); en caso contrario, se puede incurrir en errores en la decisi\'{o}n sobre la prueba. El paquete Eviews, incluye una opci\'{o}n autom\'{a}tica para esta selecci\'{o}n.
\end{observacion}

\begin{ejemplo}
Se considera nuevamente la SIB (cap\'{i}tulo 2), que se consider\'{o} que se deb\'{i}a diferenciar, a partir del gr\'{a}fico de la serie y de su FAC estimada. La informaci\'{o}n estad\'{i}stica sobre la prueba de ra\'{i}ces unitarias m\'{a}s adecuada aparece en la Tabla 3.8 (modelo (b)), en la que se observa que con los niveles de significaci\'{o}n usuales (10{\%}, 5{\%} y 1{\%}) se debe aceptar la presencia de una ra\'{i}z unitaria; sin embargo, el procedimiento aplicado a la serie diferenciada muestra que \'{e}sta ya no admite una ra\'{i}z unitaria (la SIB es, entonces, $I(1)$). Ver la Tabla 3.9.

\begin{table}[H]
\centering\small
\caption{Informaci\'{o}n estad\'{i}stica para la prueba ADF de la SIB}
\begin{tabular}{lccc}
\toprule
& & $t$-Statistic& Prob.* \\
\midrule
\multicolumn{2}{l}{Augmented Dickey-Fuller test statistic} & -1.112915& ~0.7096 \\
\midrule
Test critical values:& 1{\%} level&  -3.479656&  \\
& 5{\%} level& -2.883073&  \\
& 10{\%} level& -2.578331&  \\
\bottomrule
\multicolumn{4}{l}{*MacKinnon (1996) one-sided $p$-values.}
\end{tabular}

\vspace{8mm}
\begin{tabular}{ccccc}
\multicolumn{4}{l}{Augmented Dickey-Fuller Test Equation}  \\
\multicolumn{4}{l}{Dependent Variable: D(SIB)} \\
\multicolumn{4}{l}{Method: Least Squares} \\
\multicolumn{4}{l}{Date: 05/20/13 Time: 13:14} \\
\multicolumn{4}{l}{Sample (adjusted): 8 141}  \\
\multicolumn{4}{l}{Included observations: 134 after adjustments}  \\
\toprule
Variable& Coefficient& Std. Error& $t$-Statistic& Prob. \\
\midrule
SIB(-1)& -0.003310& 0.002974& -1.112915& 0.2679 \\
D(SIB(-1))& 1.466814& 0.084727& 17.31223& 0.0000 \\
D(SIB(-2))& -1.214989& 0.146390& -8.299676& 0.0000 \\
D(SIB(-3))& 0.930050& 0.165351& 5.624695& 0.0000 \\
D(SIB(-4))& -0.816091& 0.165477& -4.931760& 0.0000 \\
D(SIB(-5))& 0.514208& 0.145415& 3.536147& 0.0006 \\
D(SIB(-6))& -0.243578& 0.082814& -2.941259& 0.0039 \\
C& 0.375276& 0.165658& 2.265358& 0.0252 \\
\bottomrule
\end{tabular}
\end{table}


\begin{table}[H]
\centering\small
\caption{Informaci\'{o}n estad\'{i}stica para la prueba ADF de la SIB diferenciada}
\begin{tabular}{lccc}
\toprule
& & $t$-Statistic& Prob.* \\
\midrule
\multicolumn{2}{l}{Augmented Dickey-Fuller test statistic} &  
-5.820465& ~0.0000 \\
Test critical values:& 1{\%} level&  -3.479656&  \\
& 5{\%} level& -2.883073&  \\
& 10{\%} level& -2.578331&  \\
\bottomrule
\multicolumn{4}{l}{*MacKinnon (1996) one-sided $p$-values.}
\end{tabular}

\vspace{8mm}
\begin{tabular}{ccccc}
\multicolumn{4}{l}{Augmented Dickey-Fuller Test Equation}  \\
\multicolumn{4}{l}{Dependent Variable: D(SIB,2)} \\
\multicolumn{4}{l}{Method: Least Squares} \\
\multicolumn{4}{l}{Date: 05/20/13 Time: 13:15} \\
\multicolumn{4}{l}{Sample (adjusted): 8 141}  \\
\multicolumn{4}{l}{Included observations: 134 after adjustments}  \\
\toprule
Variable& Coefficient& Std. Error& $t$-Statistic& Prob. \\
\midrule
D(SIB(-1))& -0.361056& 0.062032& -5.820465& 0.0000 \\
D(SIB(-1),2)& 0.834828& 0.083637& 9.981516& 0.0000 \\
D(SIB(-2),2)& -0.387908& 0.095122& -4.078010& 0.0001 \\
D(SIB(-3),2)& 0.544866& 0.099687& 5.465761& 0.0000 \\
D(SIB(-4),2)& -0.274558& 0.090012& -3.050232& 0.0028 \\
D(SIB(-5),2)& 0.240218& 0.082837& 2.899894& 0.0044 \\
C& 0.224892& 0.095922& 2.344538& 0.0206 \\
\bottomrule
\end{tabular}
\end{table}

\end{ejemplo}



\section{Modelos Estacionales}

Se conoce que, por ejemplo, ciertas series mensuales tienen un perfil estacional marcado; es decir, los datos relativos a un mismo mes de diferentes a\~{n}os tienen tendencia a situarse de manera similar respecto a la media anual. Esto hace pensar en incluir en el modelo ARIMA retardos m\'{u}ltiplos de 12. En teor\'{i}a, nada impide el incluir par\'{a}metros p, q lo suficientemente grandes para que esos retardos sean tomados en cuenta; sin embargo, su c\'{a}lculo se convertir\'{i}a en una tarea pr\'{a}cticamente imposible. Para evitar este aumento dr\'{a}stico de par\'{a}metros, Box y Jenkins propusieron un tipo particular de modelos ARIMA estacionales, que son modelos multiplicativos del tipo:

\[
\Delta^{d}\Phi_{p} (B)\Delta_{S}^{D} \Phi_{P} (B^{S})X_{t} =\Theta_{q} 
(B)\Theta_{Q} (B^{S})u_{t} 
\]

donde $S$ es el periodo de la estacionalidad\index{Per\'{i}odo de la estacionalidad} ($S=12$ para series mensuales, $S=4$ para series trimestrales,\dots). $\Delta =1-B$, $\Delta_{S} =1-B^{S}$; $\Phi_{P}$, $\Phi_{p}$, $\Theta_{q}$, $\Theta_{Q}$ son polinomios de grado $p$, $P$, $q$ y $Q$, respectivamente, con ra\'{i}ces de m\'{o}dulo superior al 1; ($u_{t}$ ) es un ruido blanco.\newline

Un proceso $X_{t}$ que satisface la ecuaci\'{o}n precedente se llama un proceso\index{Procesos SARIMA!Orden $[(p,d,q) (P,D,Q)]$} $SARIMA_S[(p, d, q) (P, D, Q)]$.\newline

La justificaci\'{o}n de este modelo consiste en aplicar a las $S$ series obtenidas a partir de $X_{t}$, para los meses id\'{e}nticos, la misma transformaci\'{o}n:
\[
\frac{\Delta_{S}^{D} \Phi_{P} (B^{S})}{\Theta_{Q} (B^{S})}
\]
y suponer que la serie obtenida:
\[
\alpha_{t} =\frac{\Delta_{S}^{D} \Phi_{P} (B^{S})}{\Theta_{Q} 
(B^{S})}X_{t} 
\]
ya no es estacional y se la puede modelar por un $ARIMA (p, d, q)$.
\[
\Delta^{d}\Phi_{p} (B)\alpha_{t} =\Theta_{q} (B)u_{t} 
\]

Las series estacionales pueden detectarse examinando las funciones de autocorrelaci\'{o}n y autocorrelaci\'{o}n parcial estimadas pues \'{e}stas toman valores altos, en valor absoluto, en los \'{i}ndices m\'{u}ltiplos de $S$.\newline

La identificaci\'{o}n de los par\'{a}metros $P$, $D$, $Q$ de la parte estacional se hace de manera an\'{a}loga a lo descrito anteriormente, tomando en cuenta para las $\widehat{\rho}(h)$ y $\widehat{r}(h)$, los valores de $h$ m\'{u}ltiplos de $S$.\newline

Las fases de estimaci\'{o}n y verificaci\'{o}n se realizan de la misma manera que para los modelos no estacionales.

\begin{ejemplo}[Un ejemplo con estacionalidad] Los datos que se utilizar\'{a}n (ver Anexo C.3) corresponden a las ventas mensuales (SVM) en d\'{o}lares, de una compa\~{n}\'{i}a comercial en el per\'{i}odo comprendido desde enero de 2006 hasta diciembre de 2015 (Figura No. 3.15). Se modelar\'{a} la serie con los datos hasta el a\~{n}o 2014 y se guardar\'{a}n los del a\~{n}o 2015 para poder compararlos con las predicciones. Siguiendo la metodolog\'{i}a explicada se procede a la identificaci\'{o}n a priori del modelo.

\begin{figure}[H]
\centering
\includegraphics[width=0.7\textwidth]{Graficos/Cap3/STcap318.eps}
\caption{Grafico de SVM}
\end{figure}

\end{ejemplo}


\subsubsection*{Identificaci\'{o}n a Priori}

\subsubsection*{Elecci\'{o}n de $d$.}

La modelaci\'{o}n de esta serie se realizar\'{a} utilizando el paquete Eviews.\newline

Se observa que los primeros valores de la funci\'{o}n de autocorrelaci\'{o}n (Figura 4.16) no son significativamente grandes, por lo cual no es necesario realizar una diferenciaci\'{o}n no estacional de la serie ($d=0$).

\subsubsection*{Elecci\'{o}n de $D$.}

Los valores alrededor de 12 y 24 son grandes (Figura 4.16), por lo que se requiere una diferenciaci\'{o}n estacional de orden 12; es decir, $D=1$.

\begin{figure}[H]
\centering
\includegraphics[width=0.7\textwidth]{Graficos/Cap3/STcap319.eps}
\caption{FAC y FACP estimadas de SVM}
\end{figure}

Luego de observar la serie diferenciada no estacionalmente de orden 12, de acuerdo a lo detallado, se observa que no hace falta una segunda diferenciaci\'{o}n estacional.

\begin{observacion}
Si existiera la posibilidad de diferenciar estacionalmente y no estacionalmente, siempre se debe realizar primero la diferencia estacional, pues hay que reconocer que el operador de diferenciaci\'{o}n no estacional ${\Delta }=1-B$ es un factor del operador de diferencia estacional\index{Diferenciaci\'{o}n!Estacional} ${\Delta }^{S}=1-B^{S};$ es decir, al diferenciar estacionalmente a la serie, ya se la diferencia no estacionalmente.
\end{observacion}


\begin{observacion}
Un error frecuente en que se suele incurrir es pensar que la prueba ADF o similares, que determinan la existencia de ra\'{i}ces unitarias (no estacionales), sirven para detectar estacionalidad; esto no es correcto, pues la estacionalidad es un fen\'{o}meno m\'{a}s complejo.
\end{observacion}

\begin{figure}[H]
\centering
\includegraphics[width=0.7\textwidth]{Graficos/Cap3/STcap320.eps}
\caption{FAC Y FACP estimadas de la SVM con una diferencia estacional}
\end{figure}


\subsubsection*{Especificaci\'{o}n del Modelo}

En la Figura 3.17 de la serie diferenciada estacionalmente de orden 12, se observa que la autocorrelaci\'{o}n estimada de orden 13 se sale de la banda de confianza (95{\%}), con lo que se escoge un coeficiente $MA(13)$; de la misma manera, se observa que las autocorrelaciones parciales de orden 12 y 13 son significativamente distintas de cero, por lo que se eligen coeficientes $AR(12)$ y $AR(13)$.\newline

Se puede entonces considerar como modelo inicial, que se llamar\'{a} Modelo 1, al siguiente:

\[
SVMDE\left( t \right)=c_{1}+c_{2}\cdot SVMDE\left( t-12 \right)+c_{3}\cdot 
SVMDE\left( t-13 \right)+c_{4}\cdot u(t-13)
\]

donde SVMDE representa a la serie de ventas mensuales con una diferencia estacional. El modelo ingresa a un proceso iterativo de estimaci\'{o}n, verificaci\'{o}n y modificaci\'{o}n (de ser el caso).

\begin{table}[H]
\centering\small
\caption{Informaci\'{o}n estad\'{i}stica del Modelo 1}
\begin{tabular}{@{}crrrr@{}}
\toprule
Variable& Coefficient& Std. Error& t-Statistic& Prob. \\
\midrule
C& 152.7693& 28.75136& 5.313462& 0.0000 \\
$AR(12)$& -0.284706& 0.118481& -2.402962& 0.0186 \\
$AR(13)$& 0.171932& 0.235285& 0.730737& 0.4671 \\
$MA(13)$& 0.260089& 0.253103& 1.027601& 0.3073 \\
\bottomrule
\end{tabular}

\begin{tabular}{@{}lrlr@{}}
\toprule
R-squared& 0.172040& Mean dependent var & 151.3253 \\
Adjusted R-squared& 0.140599& S.D. dependent var & 254.2020 \\
S.E. of regression& 235.6552& Akaike info criterion& 13.80961 \\
Sum squared resid& 4387135.& Schwarz criterion & 13.92618 \\
Log likelihood& -569.0988& Hannan-Quinn criter. & 13.85644 \\
F-statistic& 5.471747& Durbin-Watson stat & 1.337219 \\
Prob(F-statistic)& 0.001811&\\
\bottomrule
\end{tabular}
\end{table}


\subsubsection*{Verificaci\'{o}n}

\subsubsection*{Pruebas para los Par\'{a}metros.}

Con el software Eviews, se realiz\'{o} la estimaci\'{o}n del modelo inicial, como se muestra en la Tabla 3.10; analizando la \'{u}ltima columna de los $p$-valores se concluye que no todos los coeficientes tienen un $p$-valor menor a 0,05 (95{\%} de confianza), por lo cual no son estad\'{i}sticamente significativos (diferentes de cero). Es el caso de los coeficientes asociados a $AR(13)$ y $MA(13)$.\newline

Por lo tanto, se debe modificar el modelo. As\'{i}, se deben eliminar los coeficientes que no son significativos; no es recomendable eliminar m\'{a}s de un coeficiente a la vez, sino de uno en uno. En la tabla 3.11, se pueden ver los resultados obtenidos luego de quitar el coeficiente $AR(13)$ (el que tiene el $p$-valor m\'{a}s grande); se denominar\'{a} Modelo 2:

\begin{table}[H]
\centering\small
\caption{Informaci\'{o}n estad\'{i}stica del Modelo 2}
\begin{tabular}{@{}crrrr@{}}
\toprule
Variable& Coefficient& Std. Error& t-Statistic& Prob. \\
\midrule
$C$& 154.4945& 26.07034& 5.926062& 0.0000 \\
$AR(12)$& -0.348977& 0.119777& -2.913542& 0.0046 \\
$MA(13)$& 0.408495& 0.117771& 3.468550& 0.0008 \\
\bottomrule
\end{tabular}

\begin{tabular}{@{}lrlr@{}}
\toprule
R-squared& 0.193596& Mean dependent var & 157.1548 \\
Adjusted R-squared& 0.173685& S.D. dependent var & 258.2531 \\
S.E. of regression& 234.7569& Akaike info criterion & 13.79004 \\
Sum squared resid& 4463975.& Schwarz criterion & 13.87685 \\
Log likelihood& -576.1816& Hannan-Quinn criter. & 13.82494 \\
F-statistic& 9.722955& Durbin-Watson stat & 1.285551 \\
Prob(F-statistic)& 0.000164&  \\
\bottomrule
\end{tabular}
\end{table}

Como se puede ver en la Tabla 3.11, los coeficientes tienen un $p$-valor menor que 0,05, por lo cual son significativos. A continuaci\'{o}n, se presentan las autocorrelaciones y autocorrelaciones parciales estimadas de los residuos del Modelo 2.

\begin{figure}[H]
\centering
\includegraphics[width=0.7\textwidth]{Graficos/Cap3/STcap321.eps}
\caption{FAC y FACP residuales estimadas del Modelo 1 para SVM.}
\end{figure}

A partir de la Figura 3.18, se deduce que se deber\'{i}a modificar el Modelo 2, pues aunque pasa la prueba de los coeficientes, los residuos est\'{a}n correlacionados (se constata con la observaci\'{o}n de los gr\'{a}ficos y del estad\'{i}stico $Q$, con sus correspondientes $p$-valores, todos menores que 0,05).


\subsubsection*{Modificaci\'{o}n del Modelo}

La Figura 3.18 muestra que las autocorrelaciones y autocorrelaciones parciales estimadas son significativas al orden 1, lo que sugiere modificar el modelo aumentando un par\'{a}metro no estacional auto-regresivo y/o media m\'{o}vil. Los resultados fueron los siguientes:\newline

Se aument\'{o} un coeficiente para $AR(1)$ que result\'{o} significativo (se denominar\'{a} Modelo 3); adem\'{a}s los residuos se pueden considerar ruido blanco (todos los $p$-valores son mayores que 0,05).

\begin{table}[H]
\centering\small
\caption{Informaci\'{o}n estad\'{i}stica del Modelo 3 para la SVM}
\begin{tabular}{@{}crrrr@{}}
\toprule
Variable& Coefficient& Std. Error& t-Statistic& Prob. \\
\midrule
$C$& 156.1661& 34.82541& 4.484257& 0.0000 \\
$AR(1)$& 0.325909& 0.107494& 3.031876& 0.0033 \\
$AR(12)$& -0.335945& 0.114892& -2.924008& 0.0045 \\
$MA(13)$& 0.480540& 0.111556& 4.307611& 0.0000 \\
\bottomrule
\end{tabular}

\begin{tabular}{@{}lrlr@{}}
\toprule
R-squared& 0.278803& Mean dependent var & 157.1548 \\
Adjusted R-squared& 0.251758& S.D. dependent var & 258.2531 \\
S.E. of regression& 223.3914& Akaike info criterion & 13.70218 \\
Sum squared resid& 3992299.& Schwarz criterion & 13.81793 \\
Log likelihood& -571.4914& Hannan-Quinn criter. & 13.74871 \\
F-statistic& 10.30889& Durbin-Watson stat & 1.912194 \\
Prob(F-statistic)& 0.000008& &  \\
\bottomrule
\end{tabular}
\label{tab17}
\end{table}

\begin{figure}[H]
\centering
\includegraphics[width=0.7\textwidth]{Graficos/Cap3/STcap322.eps}
\caption{FAC y FACP estimadas del Modelo 3 para la SVM}
\end{figure}


\subsubsection*{Prueba de predicciones de Chow}

A continuaci\'{o}n, se realiza la prueba de estabilidad de las predicciones (12 en este caso). As\'{i}, se obtienen los siguientes resultados:

\begin{figure}[H]
\centering
\includegraphics[width=0.7\textwidth]{Graficos/Cap3/STcap323.eps}
\caption{Prueba de Chow para las predicciones.}
\end{figure}

Dado que el $p$-valor correspondiente al estad\'{i}stico F es menor que 0,05, se rechaza la hip\'{o}tesis nula de no existencia de cambios estructurales en las predicciones. Sin embargo, este fen\'{o}meno puede deberse a la estacionalidad de la serie y a un efecto de ``volatilidad'' que se estudiar\'{a} en el siguiente cap\'{i}tulo.

\subsubsection*{Predicciones}

Como se puede ver, el Modelo 3 es el mejor modelo que describe los datos de ventas mensuales; se lo utilizar\'{a} para comparar los valores de predicci\'{o}n del modelo retenido con los verdaderos valores para el a\~{n}o 2015. Para confirmar que el modelo retenido realiza buenas predicciones a corto plazo, se grafican \'{e}stas con los 12 datos correspondientes al a\~{n}o 2015 (Figura 3.17); tambi\'{e}n aparecen los intervalos de confianza correspondientes (nivel de confianza 95{\%}).

\begin{figure}[H]
\centering
\includegraphics[width=0.7\textwidth]{Graficos/Cap3/STcap324.eps}
\caption{Comparaci\'{o}n de predicciones con valores reales (2015)}
\end{figure}


\section{Ejercicios Propuestos}

\begin{enumerate}
 \item Sea $X=(X_{n} )_{n\in \Z} $ proceso estacionario, centrado que satisface la ecuaci\'{o}n ARMA:
\[
9X_{n} -X_{n-2} =W_{n} +3W_{n-1} 
\]
donde $(W_{n})$ es un r.b. de varianza $\sigma^{2}$.\newline

Calcular expl\'{i}citamente el predictor lineal $\widehat{X}_{n+1} $ de $X_{n+1} $ conociendo el pasado $\{ {X_{p} ,p\le n} \}$, luego $\widehat{X}_{n+2} $ y mostrar como proseguir por recurrencia el c\'{a}lculo de $\widehat{X}_{n+3}$, $\widehat{X}_{n+4} $, etc.

\item Sea el siguiente modelo ajustado con una muestra de 64 observaciones
\[
(1-0,8B)Y_{t} =(1+0,5){\widehat{u}_{t} } 
\]
La FAC y la FACP estimadas de los residuos vienen dadas por:

\begin{center}
\begin{tabular}{@{}cccccc@{}}
\toprule
&\multicolumn{5}{c}{\textbf{Retardo}} \\
\cmidrule{2-6}
\textbf{Coeficiente} & 1& 2& 3& 4& 5 \\
\midrule
$\widehat{r}_{k} $ & 0,38& 0,16& 0,05& 0,02& 0,01 \\
$\widehat{\rho}_{k} $ & 0,38& 0,03& 0,07& 0,02& 0,03 \\
\bottomrule
\end{tabular}
\end{center}

?`Considerar\'{i}a aceptable el modelo ajustado? En caso contrario, ?`C\'{o}mo podr\'{i}a reformular el modelo? Justifique cada paso de su procedimiento.

\item Se ha estimado con una muestra de 110 observaciones del proceso $(X_{t} )$ con el siguiente modelo:
\[
X_{t} =u_{t} +0,5u_{t-1} +0,4u_{t-2} +18;\qquad\sigma_{u}^{2} =4
\]
La informaci\'{o}n acerca de $u_{t}$ y $ X_{t}$ se ha perdido, excepto para los siguientes valores de $X_{t}$:

\[
X_{{106}}=20\qquad X_{{107}}=21\qquad X_{108}=19 X_{109}=19\qquad X_{{110}}=17
\]

Bajo este supuesto:
\begin{enumerate}
\item Efectuar la predicci\'{o}n de $X_{t+h} $ para los per\'{i}odos 111, 112 y 113 con origen en $t = 110$.
\item Calcular el EMC para las anteriores predicciones.
\item Suponiendo que se dispone de una observaci\'{o}n adicional: $X_{111}=17$, actualizar la predicci\'{o}n de los per\'{i}odos 112 y 113.
\end{enumerate}

\item Sea el modelo estimado:
\[
(1+0,5B)(1-B)X_{t} =(1-0,5B)u_{t} +0,8
\]
con $\sigma^{2}=1.$ Se pide:
\begin{enumerate}
\item Dado $X_{79}= 40$, $X_{80}=41$, $u_{80} =0,2$. Calcular: $\widehat{X}_{80} (1)$, $\widehat{X}_{80} (2)$ y $\widehat{X}_{80} (3)$.
\item Calcular el error medio cuadr\'{a}tico de las predicciones.
\item Establecer una banda de confianza del 90{\%} para las predicciones.
\end{enumerate}

\item Sea el proceso $(X_{t})_{t\in \Z} $, definido por:
\[
(1-4B)X_{t} =3+(1-0,5B)u_{t} 
\]
donde $(u_{t})$ r.b. de varianza 1.

\begin{enumerate}
\item Encuentre la covarianza entre $X_{t}$ y $u_{t-i}$, $i\in \Z$
\item ?`Cu\'{a}l es la funci\'{o}n de autocorrelaci\'{o}n del proceso?
\item Expresar $X_{t}$ en funci\'{o}n de los $u_{t-i}$, $i\in \Z$. 
\item Exprese el proceso en su forma can\'{o}nica (y centrado). ?`Cu\'{a}l es la expresi\'{o}n como proceso lineal?
\item ?`Cu\'{a}l es la varianza del error de predicci\'{o}n con horizonte 1?
\end{enumerate}	

\item Sea $\displaystyle f(\lambda )=\frac{\frac{13}{12}+\cos \lambda }{\frac{41}{40}+\cos \lambda}\cdot\frac{1}{2\pi },\qquad \lambda \in [-\pi ,\pi]$.

\begin{enumerate}
\item ?`Cu\'{a}l es la ecuaci\'{o}n ARMA can\'{o}nica satisfecha por $X$? ?`Cu\'{a}l es la varianza de la innovaci\'{o}n?
\item Dar una representaci\'{o}n expl\'{i}cita de la innovaci\'{o}n $u_{t} $ en funci\'{o}n de las $X_{j} ,j\le t$. 
\item Calcular $\widehat{X}_{T+1} $, conociendo $X_{t} \left( {t\le T} \right)$. Determinar el error de predicci\'{o}n.
\end{enumerate}

?`Existe autocorrelaci\'{o}n entre estas observaciones?

\item Suponga que se ha estimado un modelo $ARIMA (0, 2, 1)$ con par\'{a}metro 0,30 para representar la serie trimestral de la poblaci\'{o}n activa de un pa\'{i}s, cuyos valores durante el a\~{n}o 2014 fueron:

\begin{center}
\begin{tabular}{@{}ccccc@{}}
\toprule
\textbf{Trimestre} & 
\textbf{(2014, I)}& 
\textbf{(2014, II)}& 
\textbf{(2014, III)}& 
\textbf{(2014, IV)} \\
\midrule
Poblaci\'{o}n activa & 14.000& 14.150& 14.200& 14.350 \\
(Miles de personas)& \\
\bottomrule
\end{tabular}
\end{center}

Elabore predicciones para la poblaci\'{o}n activa durante los cuatro trimestres de 2015, sabiendo que la predicci\'{o}n que se hab\'{i}a hecho en (2014, III) para (2014, IV) era de 14.500.

\item Se considera el modelo $(1-B) (1-0,5B) X_{t}=(1-0,8B) u_{t}$, donde $(u_{t})$ es un r.b. normal, con varianza 0,25. Se supone que $X_{T} =12$ y $\widehat{X}_{T} (1)=10$. Dar un intervalo de predicci\'{o}n, de nivel de confianza 95{\%}, para $X_{T+1} $.

\begin{center}
\begin{tabular}{@{}ccccccccccc@{}}
\toprule
$t$& 1& 2& 3& 4& 5& 6& 7& 8& 9& 10 \\
\midrule
$X_{t}$& 7& 9& 10& 6& 5& 3& 1& 2& 3& 4 \\
\bottomrule
\end{tabular}
\end{center}

\begin{enumerate}
 \item Calcular $\widehat{\rho}_{1}$, $\widehat{\rho}_{2}$ y $\widehat{\rho}_{3}$
\item Encontrar un intervalo de confianza ($\alpha =5\%$) para $\widehat{\rho}_{1}$, $\widehat{\rho}_{2}$ y $\widehat{\rho}_{3}$
\item Calcular $\widehat{r}_{1}$, $\widehat{r}_{2}$ y $\widehat{r}_{3}$.
\item Encontrar un intervalo de confianza ($\alpha =5\%$) para $\widehat{r}_{1}$, $\widehat{r}_{2}$ y $\widehat{r}_{3}$
\end{enumerate}

\item Sea el modelo $ARIMA ( 1, 1, 1)$: $(1-\varphi B) \Delta  X_{t}=\theta_{0} +(1-\theta B) u_{t}$.
\begin{enumerate}
 \item Calcular $\widehat{X}_{t} (h)$, $h= 1, 2$
\item Dar una f\'{o}rmula para $\widehat{X}_{t} (h)$, $h \ge 2$
\end{enumerate}

\item Sea el proceso $(X_{t})$ definido por $X_{t} =u_{t} -0,6u_{t-1} $.
\begin{enumerate}
 \item Calcular $S=\sum {u_{t}^{2} }$ para las observaciones:
 \begin{center}
\begin{tabular}{@{}cccccc@{}}
\toprule
$t$& 1& 2& 3& 4& 5 \\
\midrule
$X_{t}$& 3& 2& 0& 2& 1 \\
\bottomrule
\end{tabular}
\end{center}
condicionado al valor de $u_{0}$.

\item Calcular el valor de $u_{0}$ que minimiza $S$.
\end{enumerate}

\end{enumerate}