\chapter{Modelos de Heteroscedasticidad Condicional}

\section{Modelos Arch--Garch Sim\'{e}tricos}

En esta secci\'{o}n, se presentar\'{a} un breve resumen de la teor\'{i}a desarrollada alrededor de los modelos ARCH\index{ARCH!Definici\'{o}n} -- GARCH, los cuales determinan un patr\'{o}n de comportamiento estad\'{i}stico para la varianza condicional, denominados modelos Auto-Regresivos con Heteroscedasticidad Condicional.\newline

El an\'{a}lisis de series temporales econ\'{o}micas, tradicionalmente se ha centrado en el estudio de modelos para la media condicional en los que se asume que la varianza condicional es constante; en este caso se dice que existe homoscedasticidad. Si este no fuera el caso, se estar\'{i}a enfrentado un problema de no estacionariedad de la serie.\newline

En 1982, Robert Engle revolucion\'{o} los modelos de volatilidad introduciendo el estudio de las estructuras cuadr\'{a}ticas, ampliando as\'{i} la visi\'{o}n de la metodolog\'{i}a Box--Jenkins, en la cual los modelos lineales de tipo ARIMA admiten que las innovaciones son un ruido blanco, con media cero y varianza constante.\newline

Los modelos de volatilidad condicional son importantes por el papel que juega el riesgo y el concepto de incertidumbre en el desarrollo de las teor\'{i}as modernas de modelos financieros, que relacionan de forma directa el riesgo con la volatilidad. Estos modelos permiten relacionar el valor de la varianza condicional (no constante) en funci\'{o}n del conjunto de informaci\'{o}n disponible en periodos anteriores; est\'{a}n espec\'{i}ficamente dise\~{n}ados para modelar y pronosticar varianzas condicionales.

\begin{definicion}
 Un Modelo $ARCH(r)$ se define por:
\begin{align*}
 Z_{t} &= \sqrt h_{t} u_{t} \\ 
 h_{t} &= \alpha_{0}+\alpha_{1}Z_{t-1}^{2}+\mathellipsis +\alpha_{r}Z_{t-r}^{2}
\end{align*}

donde los $(u_{t})$ son independientes e id\'{e}nticamente distribuidos con media cero y varianza uno (i.i.d. $(0,1)$), $\alpha_{0}>0$ y $\alpha_{i}\geq 0$ para $i>0$.
\end{definicion}

En la pr\'{a}ctica se supone que $u_{t}\sim N(0,1)$ o que siguen una distribuci\'{o}n t-student. Los coeficientes $\alpha_{i}$ deben satisfacer ciertas condiciones (en general no negatividad: $\alpha_{0}>0$ y $\alpha_{i}\geq 0$ para $i>0$ y $\sum_{i=1}^r \alpha_{i} <1$), dependiendo del tipo de restricciones que se coloquen sobre el proceso $Z_{t}$.\newline

Las \index{ARCH!Restricciones}restricciones de signo de los coeficientes de la ecuaci\'{o}n de varianza garantizan que la varianza condicional sea positiva en todos los per\'{i}odos. Lo que se necesita es que, una vez que se haya estimado el modelo, genere una serie de varianzas positiva, lo que puede suceder a\'{u}n si algunos de los coeficientes $\alpha_{i}$ fueran negativos. Esto se puede considerar como un contraste de validez del modelo.\newline

Por la propia definici\'{o}n, a valores grandes de $Z_{t}$ les siguen otros valores grandes de la serie.\newline

$h_{t}$ es la varianza condicional de $Z_{t}$ dado $Z_{s},\quad s<t$.

\begin{observacion}
Se puede demostrar que un proceso $ARCH(r)$ implica que se puede representar $X_{t}^{2}$ como un proceso $AR(r)$, con residuos que no son gaussianos. Adem\'{a}s, el coeficiente de apuntamiento (curtosis) es mayor que 3, por lo cual las colas de la distribuci\'{o}n ser\'{a}n m\'{a}s pesadas que en la distribuci\'{o}n normal.
\end{observacion}

\begin{definicion}
Un modelo $GARCH(r,s)$, est\'{a}\index{GARCH!Definici\'{o}n} definido por:
\begin{align*}
 Z_{t}&=\sqrt h_{t} u_{t} \\ 
 h_{t}&=\alpha_{0}+\sum_{i=1}^r {\alpha_{i}Z_{t-i}^{2}} +\sum_{j=1}^s {\beta_{j}h_{t}} 
\end{align*}
donde los $u_{t}$ son i.i.d. $(0,1)$ con $\alpha_{0}>0$, $\alpha_{i}\geq 0$, $\beta_{j}\geq 0$ para $i>0$ y
\[
\sum_{i=1}^r \alpha_{i} +\sum_{i=1}^s \beta_{i} <1
\]
Como en el caso de los modelos ARCH, usualmente se supone que los $u_{t}$ son normales o siguen una distribuci\'{o}n $t$-student.
\end{definicion}

Las restricciones\index{ARCH!Restricciones} de no negatividad impuestas sobre los coeficientes en la definici\'{o}n son para asegurarse que la varianza no llegue a tomar un valor negativo. Adem\'{a}s, en algunos casos se pueden encontrar condiciones menos restrictivas para estos coeficientes, que aseguren que la varianza sea positiva. 

\begin{observacion}
 Se puede demostrar que un proceso $GARCH(r, s)$ implica que se puede representar $Z_{t}^{2}$ como un proceso $ARMA(r, s)$, con residuos que no son gaussianos (m\'{a}s precisamente con residuos que representan una diferencia martingala). Tambi\'{e}n, en este caso, el coeficiente de apuntamiento (curtosis) es mayor que 3, por lo cual las colas de la distribuci\'{o}n ser\'{a}n m\'{a}s pesadas que en la distribuci\'{o}n normal.
\end{observacion}

\subsection*{Modelos IGARCH}

Este modelo fue descrito originalmente por Engle y Bollerslev (1986). Si el modelo polinomial AR del modelo GARCH tiene una ra\'{i}z unitaria, se tiene un modelo IGARCH. Los modelos IGARCH\index{IGARCH!Definici\'{o}n} son modelos GARCH con una ra\'{i}z unitaria. Estos modelos tienen la caracter\'{i}stica que los impactos de los choques al cuadrado sobre $Z_{t}^{2}$ son persistentes; esto se puede expresare por:
\[
\eta_{t-i}=Z_{t-i}^{2}-h_{t-i},\qquad\text{para }i>0
\]

\begin{definicion}
 Un modelo $IGARCH(r,s)$, se define por:
\begin{align*}
Z_{t}&=\sqrt h_{t} u_{t} \\ 
h_{t}&=\alpha_{0}+\sum_{i=1}^r {\alpha_{i}Z_{t-i}^{2}} +\sum_{j=1}^s {{(1-\beta }_{j})h_{t-j}}
\end{align*}
tal que,
\[
\sum_{j=1}^s \beta_{j} +\sum_{i=1}^r \alpha_{i} =1
\]

Como en el caso de los modelos ARCH, usualmente se supone que los $u_{t}$ son normales o siguen una distribuci\'{o}n $t$-student.
\end{definicion}

Los modelos IGARCH son un caso espec\'{i}fico dentro de la familia de los ``modelos con varianza persistente'' en los que la informaci\'{o}n actual (en el instante ``t'') es importante para realizar predicciones \'{o}ptimas con cualquier horizonte temporal.

\section{Modelos GARCH asim\'etricos}

Una caracter\'{i}stica de los modelos GARCH, dado que la varianza depende fundamentalmente de valores cuadr\'{a}ticos pasados, es que la volatilidad que generan frente a cambios positivos o negativos inesperados de la variable (pi\'{e}nsese en retornos de inversiones), dan una respuesta sim\'{e}trica a estos. Sin embargo, se ha probado emp\'{i}ricamente que la reacci\'{o}n que tiene la volatilidad o varianza condicional de muchas variables financieras a este tipo de cambios es asim\'{e}trica; es decir, existe diferencia en la respuesta de la volatilidad de la variable, dependiendo de si el cambio es positivo o negativo.\newline

Con el fin de modelar esta respuesta asim\'{e}trica se han desarrollado una variedad de modelos asim\'{e}tricos; los m\'{a}s representativos son los modelos EGARCH, TARCH, PARCH. 

\subsection*{Modelos EGARCH}

Las restricciones de no negatividad para asegurar la positividad de la varianza en los modelos GARCH en muchas ocasiones son dif\'{i}ciles de lograr. Nelson (1991) propuso los modelos EGARCH (\textit{Exponential GARCH)}, como soluci\'{o}n a este problema; \'{e}stos, adem\'{a}s, incorporan efectos asim\'{e}tricos.

\begin{definicion}
Un modelo $EGARCH(r,s)$, se define\index{EGARCH!Definici\'{o}n} por:
\begin{align*}
 Z_{t}&=\sqrt h_{t} u_{t} \\ 
 \ln \left( h_{t} \right)&=\alpha_{0}+\sum_{j=1}^s {\beta_{j}\ln \left( h_{t-j} \right)} +\sum_{i=1}^r \left( \alpha_{i}\left| u_{t} \right|+\gamma_{i}u_{t} \right)
\end{align*}
\end{definicion}

N\'{o}tese que el lado izquierdo de la ecuaci\'{o}n es el logaritmo de la varianza condicional; esto implica que su efecto es exponencial y garantiza que las predicciones de \'{e}sta ser\'{a}n no negativas. Cuando $\gamma_{i}\neq 0$, el efecto asim\'{e}trico deber incorporarse al Modelo GARCH.\newline

Obs\'{e}rvese tambi\'{e}n que se puede escribir $\frac{Z_{t}}{\sqrt h_{t} }$ en lugar de $u_{t}$, en las f\'{o}rmulas anteriores.\newline

Como en el caso de los modelos ARCH, usualmente se supone que los $u_{t}$ son normales o siguen una distribuci\'{o}n $t$-student.

\subsection*{Modelos TARCH}

Los modelos\index{TARCH!Definici\'{o}n} TARCH (\textit{Threshold} \textit{ARCH)} fueron introducidos independientemente por Zakoian (1990) y Glosten, Jaganathan y Runklen (1993), por lo que tambi\'{e}n se conocen como GJR--GARCH. Estos modelos incluyen una variable adicional $d_{t}$, que determina el car\'{a}cter asim\'{e}trico del modelo.

\begin{definicion}
Un modelo $TARCH(r,s)$, se define por:
\begin{align*}
  Z_{t}&=\sqrt h_{t} u_{t} \\ 
 h_{t}&=\alpha_{0}+\sum_{i=1}^r {\alpha_{i}Z_{t-i}^{2}} +\gamma Z_{t-1}^{2}d_{t-1}+\sum_{j=1}^s {\beta_{j}h_{t-j}}
\end{align*}
donde los $u_{t}$ son i.i.d. $(0,1)$ con $\alpha_{0}>0$, $\alpha_{i}\geq 0$, $\beta_{j}\geq 0$ para $i,j>0$ y $\gamma \neq 0$
\[
d_{t}=\begin{cases}
       1,& \text{si }Z_{t}<0 \\ 
	   0,& \text{si }Z_{t}\ge 0
      \end{cases}
\]
Si $\gamma =0$ se pierde el efecto asim\'{e}trico del modelo.
\end{definicion}

En este modelo, las malas noticias $(Z_{t}<0)$ y las buenas noticias $(Z_{t}\geq 0)$ (pi\'{e}nsese otra vez en retornos), tienen efectos diferentes sobre la varianza condicional.

\subsection*{Modelos PARCH}

Los modelos\index{PARCH!Definici\'{o}n} PARCH (\textit{Power }ARCH) desarrollados independientemente por Taylor (1986) y Schwert (1989), introducen la desviaci\'{o}n est\'{a}ndar a los modelos ARCH; donde se modela la desviaci\'{o}n est\'{a}ndar en lugar de la varianza. Este modelo fue generalizado por Ding y otros (1993).\newline

En el modelo PARCH, el par\'{a}metro de potencia $\delta $ de la desviaci\'{o}n est\'{a}ndar puede ser estimado antes que impuesto y los par\'{a}metros opcionales $\gamma $ se agregan para capturar la asimetr\'{i}a dentro de los datos. 

\begin{definicion}
Un modelo $PARCH(r,s)$, se define por:
\begin{align*}
 Z_{t}&=\sqrt h_{t} u_{t} \\ 
 h_{t}^{\delta }&=\alpha_{0}+\sum_{i=1}^r {\alpha_{i}(\left| u_{t-i}\right|-} \gamma_{i}{u_{t-i})}^{\delta }+\sum_{j=1}^s {\beta_{j}h_{t-j}^{\delta }}
\end{align*}
donde $\delta >0$, es el par\'{a}metro del t\'{e}rmino de la potencia. $\gamma_{i}$ se dicen los par\'{a}metros de apalancamiento. 
\end{definicion}

En series de valores sim\'{e}tricos $\gamma_{i}=0$ para todo $i$. N\'{o}tese que si $\delta =1$ y $\gamma_{i}=0$ para todo $i$, el modelo PARCH es simplemente una especificaci\'{o}n GARH est\'{a}ndar. Si los $\gamma_{i}=0$ se pierde el efecto asim\'{e}trico del modelo.\newline

Los modelos GARCH asim\'{e}tricos, se estiman por el m\'{e}todo de m\'{a}xima verosimilitud condicional, por lo cual se requiere de ciertos supuestos acerca del comportamiento de los errores. Por lo general, se suponen i.i.d con distribuci\'{o}n normal o incluso con una distribuci\'{o}n $t$-student.

\section{Metodolog\'ia de la Modelici\'on}

El objetivo es encontrar un modelo que represente adecuadamente a los datos hist\'{o}ricos de una determinada variable, combinando especificaciones tanto para la media como para la varianza condicional. Los tipos de modelos que se considerar\'{a}n ser\'{a}n los ARIMA -- GARCH, de tal manera que la media condicional de la serie sea descrita por un modelo del tipo ARIMA y su varianza condicional por uno de la familia de modelos ARCH -- GARCH o de sus extensiones asim\'{e}tricas PARCH, TARCH y EGARCH. La modelaci\'{o}n se realizar\'{a} utilizando el paquete EViews.\newline

El primer paso es, por tanto, modelar la serie de datos por un modelo del tipo ARIMA o incluso SARIMA, con lo que se obtiene un modelo para la media condicional de la serie.\newline

Luego de haberse eliminado toda correlaci\'{o}n lineal en la serie, se debe indagar si existe heteroscedasticidad condicional residual, para lo cual deben analizarse los residuos estandarizados estimados al cuadrado; el correleograma correspondiente, permite llevar a cabo un an\'{a}lisis gr\'{a}fico de identificaci\'{o}n, para ver si alg\'{u}n valor es estad\'{i}sticamente diferente de cero, y por tanto, existe autocorrelaci\'{o}n en su forma residual cuadr\'{a}tica.\newline

Si se verifica la existencia de heteroscedasticadad condicional en los residuos, se rechaza el supuesto de la varianza constante; se intentar\'{a} entonces obtener una especificaci\'{o}n para la varianza condicional, a trav\'{e}s de la modelaci\'{o}n de los residuos estimados obtenidos por el modelo ARIMA, mediante un modelo del tipo ARCH -- GARCH o sus extensiones asim\'{e}tricas.\newline

Inicialmente se mantiene la estructura para la media condicional, obtenida por el modelo ARIMA, pero esta puede modificarse con la nueva especificaci\'{o}n. Los residuos estimados deben analizarse, tanto en su forma simple como en la cuadr\'{a}tica, para eliminar toda evidencia de autocorrelaci\'{o}n lineal (deben aceptarse como un ruido blanco).\newline

La estimaci\'{o}n y verificaci\'{o}n permiten encontrar uno o varios modelos que cumplan las condiciones que se impusieron en la modelaci\'{o}n ARIMA; es decir, todos los coeficientes deben ser significativos; las ra\'{i}ces de los polinomios caracter\'{i}sticos, tanto de la parte auto-regresiva como de la media m\'{o}vil, deben estar fuera del c\'{i}rculo unidad, para as\'{i} asegurar la estacionariedad e invertibilidad del proceso. Adem\'{a}s, los coeficientes de la ecuaci\'{o}n de la varianza condicional deben satisfacer las restricciones de no negatividad para la varianza (modelos ARCH -- GARCH).\newline

Para la verificaci\'{o}n de la presencia de una estructura ARIMA en los residuos (simples o cuadr\'{a}ticos) pueden utilizarse la FAC y la FACP; adem\'{a}s, tambi\'{e}n se debe realizar la prueba global (estad\'{i}stico $Q$) de Box -- Pierce -- Ljung.\newline

Una vez que un modelo ha sido estimado y ha superado las diversas verificaciones, se convierte en un instrumento \'{u}til para las predicciones de valores futuros. Como en la modelaci\'{o}n ARIMA, si varios modelos son plausibles, se elige entre estos al mejor, mediante los criterios ya citados previamente.

\section{Ejemplos con Heteroscedasticidad Condicional}

Aunque los datos de las ventas que se vienen utilizando no corresponden al \'{a}mbito financiero, sirven muy bien para ilustrar la modelaci\'{o}n para la varianza condicional. En esta ocasi\'{o}n se adoptar\'{a} el Modelo 3 con el cual se model\'{o} la media condicional (SARIMA). La Figura 4.2 no permite aceptar la hip\'{o}tesis de que la serie tenga una varianza constante.\newline

Una posibilidad para amortiguar los efectos de varianza no constante es utilizar la transformaci\'{o}n logaritmo o, en general, la transformaci\'{o}n de Box y Cox; sin embargo, en esta ocasi\'{o}n se tratar\'{a} de modelar directamente la varianza a trav\'{e}s de los Modelos ARCH-GARCH o sus extensiones asim\'{e}tricas.\newline

En la Tabla 4.1 y en las figuras 4.1 y 4.2 se presentan la informaci\'{o}n estad\'{i}stica y residual para el Modelo 3 de la SVM:

\begin{table}[H]
\centering
\begin{tabular}{ccccc}\hline\hline
Variable & Coefficient & Std. Error & $t$-Statistic & Prob. \\ \hline\hline
$C$& 156.1661& 34.82541& 4.484257& 0.0000 \\
$AR(1)$& 0.325909& 0.107494& 3.031876& 0.0033 \\
$AR(12)$& -0.335945& 0.114892& -2.924008& 0.0045 \\
$MA(13)$& 0.480540& 0.111556& 4.307611& 0.0000 \\ \hline\hline
\end{tabular}
\caption{Informaci\'{o}n sobre los coeficientes del Modelo 3 para la SVM}
\end{table}

\begin{figure}[H]
\centering
\includegraphics[width=0.7\textwidth]{Graficos/Cap4-5/STcap41.eps}
\caption{FAC y FACP estimadas residuales del Modelo 3 para la SVM}
\end{figure}

\begin{figure}[H]
\centering
\includegraphics[width=0.7\textwidth]{Graficos/Cap4-5/STcap42.eps}
\caption{FAC y FACP estimadas de los residuos cuadr\'{a}ticos del Modelo 
3 para la SVM}
\end{figure}

Se observan fuertes correlaciones entre los residuos cuadr\'{a}ticos estandarizados estimados, por lo cual se hace necesaria la modelaci\'{o}n de la varianza condicional del Modelo 3.\newline

En general, es dif\'{i}cil establecer el orden para los modelos ARCH-GARCH. Lo usual es probar los modelos con par\'{a}metros (1,0), (1,1), (1,2) o (2,2). Para este caso se empez\'{o} probando con el modelo $ARCH(1)$; los resultados aparecen en la Tabla 4.2 y el las Figuras 4.3 y 4.4. 

\begin{table}[H]
\centering
\begin{tabular}{ccccc}\hline\hline
Variable & Coefficient & Std. Error & $z$-Statistic & Prob. \\ \hline\hline
$C$& 115.3958& 36.57140& 3.155355& 0.0016 \\
$AR(1)$& 0.483776& 0.103019& 4.695993& 0.0000 \\
$AR(12)$& -0.449722& 0.078830& -5.704935& 0.0000 \\
$MA(13)$& 0.798959& 0.041313& 19.33915& 0.0000 \\ \hline\hline
\multicolumn{5}{c}{Variance Equation}\\ \hline\hline
$C$& 16129.50& 3530.889& 4.568113& 0.0000 \\
$RESID(-1)^2$& 0.610436& 0.249248& 2.449115& 0.0143 \\ \hline\hline
\end{tabular}
\end{table}

\begin{table}[H]
\centering
\begin{tabular}{cccc}\hline\hline
R-squared& 0.335639& Mean dependent var & 157.1548 \\
Adjusted R-squared& 0.310726& S.D. dependent var& 258.2531 \\
S.E. of regression& 214.4082& Akaike info criterion & 13.23093 \\
Sum squared resid& 3677671.& Schwarz criterion & 13.40456 \\
Log likelihood& -549.6992& Hannan-Quinn criter. & 13.30073 \\
Durbin-Watson stat& 2.201103& \\ \hline\hline
\end{tabular}

\begin{tabular}{ccccc} \hline\hline
Inverted AR Roots& $.96-.24i$& $.96+.24i$& $.71+.65i$& $.71-.65i$ \\
& $.28-.89i$& $.28+.89i$& $-.21+.90i$& $-.21-.90i$ \\
& $-.63-.66i$& $-.63+.66i$& $-.87+.24i$& $-.87-.24i$ \\
Inverted MA Roots& .95-.24i& .95$+$.24i& .74-.65i& .74$+$.65i \\
& $.35+.92i$& $.35-.92i$& $-.12-.98i$& $-.12+.98i$ \\
& $-.56-.81i$& $-.56+.81i$& $-.87+.46i$& $-.87-.46i$ \\
& $.98$ & &  \\ \hline\hline
\end{tabular}
\caption{Informaci\'{o}n estad\'{i}stica para el Modelo 3-ARCH(1) para la SVM}
\end{table}


\begin{figure}[H]
\centering
\includegraphics[width=0.7\textwidth]{Graficos/Cap4-5/STcap43.eps}
\caption{FAC y FACP estimadas residuales del Modelo 3-ARCH(1) para la SVM}
\end{figure}

\begin{figure}[H]
\centering
\includegraphics[width=0.7\textwidth]{Graficos/Cap4-5/STcap44.eps}
\caption{FAC y FACP estimadas de los residuos cuadr\'{a}ticos del Modelo 3-ARCH(1) para la SVM}
\end{figure}

Las Figuras 4.3 y 4.4 evidencian que existen problemas ya no solo en los residuos cuadr\'{a}ticos, sino tambi\'{e}n en los residuos simples. En la figura 4.4, la FACP en el orden 13 es significativo (y cercano a la estacionalidad 12); por lo cual, se decidi\'{o} incluir un t\'{e}rmino AR(13) en el Modelo 3; esto tampoco solucion\'{o} totalmente la falta de independencia de los residuos cuadr\'{a}ticos. Luego, de algunas pruebas se encontr\'{o} como modelo final aquel que contiene t\'{e}rminos c, SAR(12), MA(12) y AR(13) para la media (se lo llamar\'{a} Modelo 4) y ARCH(1) para la varianza. Los resultados se muestran en la Tabla 4.3 y en las Figuras 4.5 y 4.6.
