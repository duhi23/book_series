\chapter{Modelos de Heteroscedasticidad Condicional}

\section{Modelos Arch--Garch Sim\'{e}tricos}

En esta secci\'{o}n, se presentar\'{a} un breve resumen de la teor\'{i}a desarrollada alrededor de los modelos ARCH\index{ARCH!Definici\'{o}n} -- GARCH, los cuales determinan un patr\'{o}n de comportamiento estad\'{i}stico para la varianza condicional, denominados modelos Auto-Regresivos con Heteroscedasticidad Condicional.\newline

El an\'{a}lisis de series temporales econ\'{o}micas, tradicionalmente se ha centrado en el estudio de modelos para la media condicional en los que se asume que la varianza condicional es constante; en este caso se dice que existe homoscedasticidad. Si este no fuera el caso, se estar\'{i}a enfrentado un problema de no estacionariedad de la serie.\newline

En 1982, Robert Engle revolucion\'{o} los modelos de volatilidad introduciendo el estudio de las estructuras cuadr\'{a}ticas, ampliando as\'{i} la visi\'{o}n de la metodolog\'{i}a Box--Jenkins, en la cual los modelos lineales de tipo ARIMA admiten que las innovaciones son un ruido blanco, con media cero y varianza constante.\newline

Los modelos de volatilidad condicional son importantes por el papel que juega el riesgo y el concepto de incertidumbre en el desarrollo de las teor\'{i}as modernas de modelos financieros, que relacionan de forma directa el riesgo con la volatilidad. Estos modelos permiten relacionar el valor de la varianza condicional (no constante) en funci\'{o}n del conjunto de informaci\'{o}n disponible en periodos anteriores; est\'{a}n espec\'{i}ficamente dise\~{n}ados para modelar y pronosticar varianzas condicionales.

\begin{definicion}
 Un Modelo $ARCH(r)$ se define por:
\begin{align*}
 Z_{t} &= \sqrt h_{t} u_{t} \\ 
 h_{t} &= \alpha_{0}+\alpha_{1}Z_{t-1}^{2}+\mathellipsis +\alpha_{r}Z_{t-r}^{2}
\end{align*}

donde los $(u_{t})$ son independientes e id\'{e}nticamente distribuidos con media cero y varianza uno (i.i.d. $(0,1)$), $\alpha_{0}>0$ y $\alpha_{i}\geq 0$ para $i>0$.
\end{definicion}

En la pr\'{a}ctica se supone que $u_{t}\sim N(0,1)$ o que siguen una distribuci\'{o}n t-student. Los coeficientes $\alpha_{i}$ deben satisfacer ciertas condiciones (en general no negatividad: $\alpha_{0}>0$ y $\alpha_{i}\geq 0$ para $i>0$ y $\sum_{i=1}^r \alpha_{i} <1$), dependiendo del tipo de restricciones que se coloquen sobre el proceso $Z_{t}$.\newline

Las \index{ARCH!Restricciones}restricciones de signo de los coeficientes de la ecuaci\'{o}n de varianza garantizan que la varianza condicional sea positiva en todos los per\'{i}odos. Lo que se necesita es que, una vez que se haya estimado el modelo, genere una serie de varianzas positiva, lo que puede suceder a\'{u}n si algunos de los coeficientes $\alpha_{i}$ fueran negativos. Esto se puede considerar como un contraste de validez del modelo.\newline

Por la propia definici\'{o}n, a valores grandes de $Z_{t}$ les siguen otros valores grandes de la serie.\newline

$h_{t}$ es la varianza condicional de $Z_{t}$ dado $Z_{s},\quad s<t$.

\begin{observacion}
Se puede demostrar que un proceso $ARCH(r)$ implica que se puede representar $X_{t}^{2}$ como un proceso $AR(r)$, con residuos que no son gaussianos. Adem\'{a}s, el coeficiente de apuntamiento (curtosis) es mayor que 3, por lo cual las colas de la distribuci\'{o}n ser\'{a}n m\'{a}s pesadas que en la distribuci\'{o}n normal.
\end{observacion}

\begin{definicion}
Un modelo $GARCH(r,s)$, est\'{a}\index{GARCH!Definici\'{o}n} definido por:
\begin{align*}
 Z_{t}&=\sqrt h_{t} u_{t} \\ 
 h_{t}&=\alpha_{0}+\sum_{i=1}^r {\alpha_{i}Z_{t-i}^{2}} +\sum_{j=1}^s {\beta_{j}h_{t}} 
\end{align*}
donde los $u_{t}$ son i.i.d. $(0,1)$ con $\alpha_{0}>0$, $\alpha_{i}\geq 0$, $\beta_{j}\geq 0$ para $i>0$ y
\[
\sum_{i=1}^r \alpha_{i} +\sum_{i=1}^s \beta_{i} <1
\]
Como en el caso de los modelos ARCH, usualmente se supone que los $u_{t}$ son normales o siguen una distribuci\'{o}n $t$-student.
\end{definicion}

Las restricciones\index{ARCH!Restricciones} de no negatividad impuestas sobre los coeficientes en la definici\'{o}n son para asegurarse que la varianza no llegue a tomar un valor negativo. Adem\'{a}s, en algunos casos se pueden encontrar condiciones menos restrictivas para estos coeficientes, que aseguren que la varianza sea positiva. 

\begin{observacion}
 Se puede demostrar que un proceso $GARCH(r, s)$ implica que se puede representar $Z_{t}^{2}$ como un proceso $ARMA(r, s)$, con residuos que no son gaussianos (m\'{a}s precisamente con residuos que representan una diferencia martingala). Tambi\'{e}n, en este caso, el coeficiente de apuntamiento (curtosis) es mayor que 3, por lo cual las colas de la distribuci\'{o}n ser\'{a}n m\'{a}s pesadas que en la distribuci\'{o}n normal.
\end{observacion}

\subsection*{Modelos IGARCH}

Este modelo fue descrito originalmente por Engle y Bollerslev (1986). Si el modelo polinomial AR del modelo GARCH tiene una ra\'{i}z unitaria, se tiene un modelo IGARCH. Los modelos IGARCH\index{IGARCH!Definici\'{o}n} son modelos GARCH con una ra\'{i}z unitaria. Estos modelos tienen la caracter\'{i}stica que los impactos de los choques al cuadrado sobre $Z_{t}^{2}$ son persistentes; esto se puede expresare por:
\[
\eta_{t-i}=Z_{t-i}^{2}-h_{t-i},\qquad\text{para }i>0
\]

\begin{definicion}
 Un modelo $IGARCH(r,s)$, se define por:
\begin{align*}
Z_{t}&=\sqrt h_{t} u_{t} \\ 
h_{t}&=\alpha_{0}+\sum_{i=1}^r {\alpha_{i}Z_{t-i}^{2}} +\sum_{j=1}^s {{(1-\beta }_{j})h_{t-j}}
\end{align*}
tal que,
\[
\sum_{j=1}^s \beta_{j} +\sum_{i=1}^r \alpha_{i} =1
\]

Como en el caso de los modelos ARCH, usualmente se supone que los $u_{t}$ son normales o siguen una distribuci\'{o}n $t$-student.
\end{definicion}

Los modelos IGARCH son un caso espec\'{i}fico dentro de la familia de los ``modelos con varianza persistente'' en los que la informaci\'{o}n actual (en el instante ``t'') es importante para realizar predicciones \'{o}ptimas con cualquier horizonte temporal.

\section{Modelos GARCH asim\'etricos}

Una caracter\'{i}stica de los modelos GARCH, dado que la varianza depende fundamentalmente de valores cuadr\'{a}ticos pasados, es que la volatilidad que generan frente a cambios positivos o negativos inesperados de la variable (pi\'{e}nsese en retornos de inversiones), dan una respuesta sim\'{e}trica a estos. Sin embargo, se ha probado emp\'{i}ricamente que la reacci\'{o}n que tiene la volatilidad o varianza condicional de muchas variables financieras a este tipo de cambios es asim\'{e}trica; es decir, existe diferencia en la respuesta de la volatilidad de la variable, dependiendo de si el cambio es positivo o negativo.\newline

Con el fin de modelar esta respuesta asim\'{e}trica se han desarrollado una variedad de modelos asim\'{e}tricos; los m\'{a}s representativos son los modelos EGARCH, TARCH, PARCH. 

\subsection*{Modelos EGARCH}

Las restricciones de no negatividad para asegurar la positividad de la varianza en los modelos GARCH en muchas ocasiones son dif\'{i}ciles de lograr. Nelson (1991) propuso los modelos EGARCH (\textit{Exponential GARCH)}, como soluci\'{o}n a este problema; \'{e}stos, adem\'{a}s, incorporan efectos asim\'{e}tricos.

\begin{definicion}
Un modelo $EGARCH(r,s)$, se define\index{EGARCH!Definici\'{o}n} por:
\begin{align*}
 Z_{t}&=\sqrt h_{t} u_{t} \\ 
 \ln \left( h_{t} \right)&=\alpha_{0}+\sum_{j=1}^s {\beta_{j}\ln \left( h_{t-j} \right)} +\sum_{i=1}^r \left( \alpha_{i}\left| u_{t} \right|+\gamma_{i}u_{t} \right)
\end{align*}
\end{definicion}

N\'{o}tese que el lado izquierdo de la ecuaci\'{o}n es el logaritmo de la varianza condicional; esto implica que su efecto es exponencial y garantiza que las predicciones de \'{e}sta ser\'{a}n no negativas. Cuando $\gamma_{i}\neq 0$, el efecto asim\'{e}trico deber incorporarse al Modelo GARCH.\newline

Obs\'{e}rvese tambi\'{e}n que se puede escribir $\frac{Z_{t}}{\sqrt h_{t} }$ en lugar de $u_{t}$, en las f\'{o}rmulas anteriores.\newline

Como en el caso de los modelos ARCH, usualmente se supone que los $u_{t}$ son normales o siguen una distribuci\'{o}n $t$-student.

\subsection*{Modelos TARCH}

Los modelos\index{TARCH!Definici\'{o}n} TARCH (\textit{Threshold} \textit{ARCH)} fueron introducidos independientemente por Zakoian (1990) y Glosten, Jaganathan y Runklen (1993), por lo que tambi\'{e}n se conocen como GJR--GARCH. Estos modelos incluyen una variable adicional $d_{t}$, que determina el car\'{a}cter asim\'{e}trico del modelo.

\begin{definicion}
Un modelo $TARCH(r,s)$, se define por:
\begin{align*}
  Z_{t}&=\sqrt h_{t} u_{t} \\ 
 h_{t}&=\alpha_{0}+\sum_{i=1}^r {\alpha_{i}Z_{t-i}^{2}} +\gamma Z_{t-1}^{2}d_{t-1}+\sum_{j=1}^s {\beta_{j}h_{t-j}}
\end{align*}
donde los $u_{t}$ son i.i.d. $(0,1)$ con $\alpha_{0}>0$, $\alpha_{i}\geq 0$, $\beta_{j}\geq 0$ para $i,j>0$ y $\gamma \neq 0$
\[
d_{t}=\begin{cases}
       1,& \text{si }Z_{t}<0 \\ 
	   0,& \text{si }Z_{t}\ge 0
      \end{cases}
\]
Si $\gamma =0$ se pierde el efecto asim\'{e}trico del modelo.
\end{definicion}

En este modelo, las malas noticias $(Z_{t}<0)$ y las buenas noticias $(Z_{t}\geq 0)$ (pi\'{e}nsese otra vez en retornos), tienen efectos diferentes sobre la varianza condicional.

\subsection*{Modelos PARCH}

Los modelos\index{PARCH!Definici\'{o}n} PARCH (\textit{Power }ARCH) desarrollados independientemente por Taylor (1986) y Schwert (1989), introducen la desviaci\'{o}n est\'{a}ndar a los modelos ARCH; donde se modela la desviaci\'{o}n est\'{a}ndar en lugar de la varianza. Este modelo fue generalizado por Ding y otros (1993).\newline

En el modelo PARCH, el par\'{a}metro de potencia $\delta $ de la desviaci\'{o}n est\'{a}ndar puede ser estimado antes que impuesto y los par\'{a}metros opcionales $\gamma $ se agregan para capturar la asimetr\'{i}a dentro de los datos. 

\begin{definicion}
Un modelo $PARCH(r,s)$, se define por:
\begin{align*}
 Z_{t}&=\sqrt h_{t} u_{t} \\ 
 h_{t}^{\delta }&=\alpha_{0}+\sum_{i=1}^r {\alpha_{i}(\left| u_{t-i}\right|-} \gamma_{i}{u_{t-i})}^{\delta }+\sum_{j=1}^s {\beta_{j}h_{t-j}^{\delta }}
\end{align*}
donde $\delta >0$, es el par\'{a}metro del t\'{e}rmino de la potencia. $\gamma_{i}$ se dicen los par\'{a}metros de apalancamiento. 
\end{definicion}

En series de valores sim\'{e}tricos $\gamma_{i}=0$ para todo $i$. N\'{o}tese que si $\delta =1$ y $\gamma_{i}=0$ para todo $i$, el modelo PARCH es simplemente una especificaci\'{o}n GARH est\'{a}ndar. Si los $\gamma_{i}=0$ se pierde el efecto asim\'{e}trico del modelo.\newline

Los modelos GARCH asim\'{e}tricos, se estiman por el m\'{e}todo de m\'{a}xima verosimilitud condicional, por lo cual se requiere de ciertos supuestos acerca del comportamiento de los errores. Por lo general, se suponen i.i.d con distribuci\'{o}n normal o incluso con una distribuci\'{o}n $t$-student.

\section{Metodolog\'ia de la Modelici\'on}

El objetivo es encontrar un modelo que represente adecuadamente a los datos hist\'{o}ricos de una determinada variable, combinando especificaciones tanto para la media como para la varianza condicional. Los tipos de modelos que se considerar\'{a}n ser\'{a}n los ARIMA -- GARCH, de tal manera que la media condicional de la serie sea descrita por un modelo del tipo ARIMA y su varianza condicional por uno de la familia de modelos ARCH -- GARCH o de sus extensiones asim\'{e}tricas PARCH, TARCH y EGARCH. La modelaci\'{o}n se realizar\'{a} utilizando el paquete EViews.\newline

El primer paso es, por tanto, modelar la serie de datos por un modelo del tipo ARIMA o incluso SARIMA, con lo que se obtiene un modelo para la media condicional de la serie.\newline

Luego de haberse eliminado toda correlaci\'{o}n lineal en la serie, se debe indagar si existe heteroscedasticidad condicional residual, para lo cual deben analizarse los residuos estandarizados estimados al cuadrado; el correleograma correspondiente, permite llevar a cabo un an\'{a}lisis gr\'{a}fico de identificaci\'{o}n, para ver si alg\'{u}n valor es estad\'{i}sticamente diferente de cero, y por tanto, existe autocorrelaci\'{o}n en su forma residual cuadr\'{a}tica.\newline

Si se verifica la existencia de heteroscedasticadad condicional en los residuos, se rechaza el supuesto de la varianza constante; se intentar\'{a} entonces obtener una especificaci\'{o}n para la varianza condicional, a trav\'{e}s de la modelaci\'{o}n de los residuos estimados obtenidos por el modelo ARIMA, mediante un modelo del tipo ARCH -- GARCH o sus extensiones asim\'{e}tricas.\newline

Inicialmente se mantiene la estructura para la media condicional, obtenida por el modelo ARIMA, pero esta puede modificarse con la nueva especificaci\'{o}n. Los residuos estimados deben analizarse, tanto en su forma simple como en la cuadr\'{a}tica, para eliminar toda evidencia de autocorrelaci\'{o}n lineal (deben aceptarse como un ruido blanco).\newline

La estimaci\'{o}n y verificaci\'{o}n permiten encontrar uno o varios modelos que cumplan las condiciones que se impusieron en la modelaci\'{o}n ARIMA; es decir, todos los coeficientes deben ser significativos; las ra\'{i}ces de los polinomios caracter\'{i}sticos, tanto de la parte auto-regresiva como de la media m\'{o}vil, deben estar fuera del c\'{i}rculo unidad, para as\'{i} asegurar la estacionariedad e invertibilidad del proceso. Adem\'{a}s, los coeficientes de la ecuaci\'{o}n de la varianza condicional deben satisfacer las restricciones de no negatividad para la varianza (modelos ARCH -- GARCH).\newline

Para la verificaci\'{o}n de la presencia de una estructura ARIMA en los residuos (simples o cuadr\'{a}ticos) pueden utilizarse la FAC y la FACP; adem\'{a}s, tambi\'{e}n se debe realizar la prueba global (estad\'{i}stico $Q$) de Box -- Pierce -- Ljung.\newline

Una vez que un modelo ha sido estimado y ha superado las diversas verificaciones, se convierte en un instrumento \'{u}til para las predicciones de valores futuros. Como en la modelaci\'{o}n ARIMA, si varios modelos son plausibles, se elige entre estos al mejor, mediante los criterios ya citados previamente.

\section{Ejemplos con Heteroscedasticidad Condicional}

Aunque los datos de las ventas que se vienen utilizando no corresponden al \'{a}mbito financiero, sirven muy bien para ilustrar la modelaci\'{o}n para la varianza condicional. En esta ocasi\'{o}n se adoptar\'{a} el Modelo 3 con el cual se model\'{o} la media condicional (SARIMA). La Figura 4.2 no permite aceptar la hip\'{o}tesis de que la serie tenga una varianza constante.\newline

Una posibilidad para amortiguar los efectos de varianza no constante es utilizar la transformaci\'{o}n logaritmo o, en general, la transformaci\'{o}n de Box y Cox; sin embargo, en esta ocasi\'{o}n se tratar\'{a} de modelar directamente la varianza a trav\'{e}s de los Modelos ARCH-GARCH o sus extensiones asim\'{e}tricas.\newline

En la Tabla 4.1 y en las figuras 4.1 y 4.2 se presentan la informaci\'{o}n estad\'{i}stica y residual para el Modelo 3 de la SVM:

\begin{table}[H]
\centering
\begin{tabular}{ccccc}\hline\hline
Variable & Coefficient & Std. Error & $t$-Statistic & Prob. \\ \hline\hline
$C$& 156.1661& 34.82541& 4.484257& 0.0000 \\
$AR(1)$& 0.325909& 0.107494& 3.031876& 0.0033 \\
$AR(12)$& -0.335945& 0.114892& -2.924008& 0.0045 \\
$MA(13)$& 0.480540& 0.111556& 4.307611& 0.0000 \\ \hline\hline
\end{tabular}
\caption{Informaci\'{o}n sobre los coeficientes del Modelo 3 para la SVM}
\end{table}

\begin{figure}[H]
\centering
\includegraphics[width=0.7\textwidth]{Graficos/Cap4-5/STcap41.eps}
\caption{FAC y FACP estimadas residuales del Modelo 3 para la SVM}
\end{figure}

\begin{figure}[H]
\centering
\includegraphics[width=0.7\textwidth]{Graficos/Cap4-5/STcap42.eps}
\caption{FAC y FACP estimadas de los residuos cuadr\'{a}ticos del Modelo 
3 para la SVM}
\end{figure}

Se observan fuertes correlaciones entre los residuos cuadr\'{a}ticos estandarizados estimados, por lo cual se hace necesaria la modelaci\'{o}n de la varianza condicional del Modelo 3.\newline

En general, es dif\'{i}cil establecer el orden para los modelos ARCH-GARCH. Lo usual es probar los modelos con par\'{a}metros (1,0), (1,1), (1,2) o (2,2). Para este caso se empez\'{o} probando con el modelo $ARCH(1)$; los resultados aparecen en la Tabla 4.2 y el las Figuras 4.3 y 4.4. 

\begin{table}[H]
\centering
\begin{tabular}{ccccc}\hline\hline
Variable & Coefficient & Std. Error & $z$-Statistic & Prob. \\ \hline\hline
$C$& 115.3958& 36.57140& 3.155355& 0.0016 \\
$AR(1)$& 0.483776& 0.103019& 4.695993& 0.0000 \\
$AR(12)$& -0.449722& 0.078830& -5.704935& 0.0000 \\
$MA(13)$& 0.798959& 0.041313& 19.33915& 0.0000 \\ \hline\hline
\multicolumn{5}{c}{Variance Equation}\\ \hline\hline
$C$& 16129.50& 3530.889& 4.568113& 0.0000 \\
$RESID(-1)^2$& 0.610436& 0.249248& 2.449115& 0.0143 \\ \hline\hline
\end{tabular}
\end{table}

\begin{table}[H]
\centering
\begin{tabular}{cccc}\hline\hline
R-squared& 0.335639& Mean dependent var & 157.1548 \\
Adjusted R-squared& 0.310726& S.D. dependent var& 258.2531 \\
S.E. of regression& 214.4082& Akaike info criterion & 13.23093 \\
Sum squared resid& 3677671.& Schwarz criterion & 13.40456 \\
Log likelihood& -549.6992& Hannan-Quinn criter. & 13.30073 \\
Durbin-Watson stat& 2.201103& \\ \hline\hline
\end{tabular}

\begin{tabular}{ccccc} \hline\hline
Inverted AR Roots& $.96-.24i$& $.96+.24i$& $.71+.65i$& $.71-.65i$ \\
& $.28-.89i$& $.28+.89i$& $-.21+.90i$& $-.21-.90i$ \\
& $-.63-.66i$& $-.63+.66i$& $-.87+.24i$& $-.87-.24i$ \\
Inverted MA Roots& .95-.24i& .95$+$.24i& .74-.65i& .74$+$.65i \\
& $.35+.92i$& $.35-.92i$& $-.12-.98i$& $-.12+.98i$ \\
& $-.56-.81i$& $-.56+.81i$& $-.87+.46i$& $-.87-.46i$ \\
& $.98$ & &  \\ \hline\hline
\end{tabular}
\caption{Informaci\'{o}n estad\'{i}stica para el Modelo 3-ARCH(1) para la SVM}
\end{table}


\begin{figure}[H]
\centering
\includegraphics[width=0.7\textwidth]{Graficos/Cap4-5/STcap43.eps}
\caption{FAC y FACP estimadas residuales del Modelo 3-ARCH(1) para la SVM}
\end{figure}

\begin{figure}[H]
\centering
\includegraphics[width=0.7\textwidth]{Graficos/Cap4-5/STcap44.eps}
\caption{FAC y FACP estimadas de los residuos cuadr\'{a}ticos del Modelo 3-ARCH(1) para la SVM}
\end{figure}

Las Figuras 4.3 y 4.4 evidencian que existen problemas ya no solo en los residuos cuadr\'{a}ticos, sino tambi\'{e}n en los residuos simples. En la figura 4.4, la FACP en el orden 13 es significativo (y cercano a la estacionalidad 12); por lo cual, se decidi\'{o} incluir un t\'{e}rmino AR(13) en el Modelo 3; esto tampoco solucion\'{o} totalmente la falta de independencia de los residuos cuadr\'{a}ticos. Luego, de algunas pruebas se encontr\'{o} como modelo final aquel que contiene t\'{e}rminos c, SAR(12), MA(12) y AR(13) para la media (se lo llamar\'{a} Modelo 4) y ARCH(1) para la varianza. Los resultados se muestran en la Tabla 4.3 y en las Figuras 4.5 y 4.6.

\begin{observacion}
El Modelo adoptado para la serie es el siguiente: 
\end{observacion}

\[
SVMDE\left( t \right)=9,27+0,32\ast SVMDE\left( t-13 \right)+0,41\ast 
SVMDE\left( t-12P \right)+u\left( t \right)+0,71\ast u(t-12)
\]
\[
Z_{t}=\sqrt h_{t} \varepsilon_{t}
\]
\[
h_{t}^{2}=323,59+7,21\ast Z_{t-1}^{2}
\]

Donde los $\varepsilon_{t}$ son i.i.d(0,1).
 
\begin{table}[H]
\centering
\begin{tabular}{ccccc}\hline\hline
Variable & Coefficient & Std. Error & z-Statistic & Prob. \\
C & 9.265438 & 3.978654 & 2.328787 & 0.0199 \\
AR(13) & 0.319152 & 0.025925 & 12.31043 & 0.0000 \\
SAR(12) & 0.407339 & 0.014784 & 27.55205 & 0.0000 \\
MA(12) & -0.712141 & 0.006475 & -109.9861 & 0.0000 \\ \hline\hline
\multicolumn{5}{c}{Variance Equation} \\ \hline\hline
C & 323.5866 & 330.1205 & 0.980208 & 0.3270 \\
RESID(-1)$\^$ 2& 7.215214 & 1.233581 & 5.848999 & 0.0000 \\ \hline\hline
R-squared & 0.005121 & \multicolumn{2}{l}{Mean dependent var} & 153.2254 \\
Adjusted R-squared & -0.039426 & \multicolumn{2}{l}{S.D. dependent var} & 260.5928 \\
S.E. of regression & 265.6801 & \multicolumn{2}{l}{Akaike info criterion} & 12.91269 \\
Sum squared resid & 4729258& \multicolumn{2}{l}{Schwarz criterion} & 13.10390 \\
Log likelihood & -452.4003 & \multicolumn{2}{l}{Hannan-Quinn criter.} & 12.98872 \\
Durbin-Watson stat & 1.033534 &  &  &  \\ \hline\hline
\end{tabular}
\caption{Informaci\'{o}n estad\'{i}stica para el Modelo 4-ARCH(1) para la SVM}
% \label{tab3}
\end{table}


\begin{figure}[H]
\centering
\includegraphics[width=4.51in,height=4.30in]{Graficos/Cap4-5/STcap45.eps}
\caption{FAC y FACP estimadas residuales del Modelo 4-ARCH(1) para la SVM}
%\label{fig5}
\end{figure}


\begin{figure}[H]
\centering
\includegraphics[width=4.54in,height=4.31in]{Graficos/Cap4-5/STcap46.eps}
\caption{FAC y FACP estimadas de los residuos cuadr\'{a}ticos} del Modelo 4-ARCH(1) para la SVM
%\label{fig6}
\end{figure}

\begin{observacion}
En la Tabla 4.3, se puede observar que el coeficiente correspondiente a la constante del ARCH no es significativo; sin embargo, se lo debe conservar (siempre) para asegurar la positividad de la varianza.
\end{observacion}


A continuaci\'{o}n se presenta el ajuste de la varianza de la SVM con este Modelo (Figura 4.7), la cual era dif\'{i}cil de percibir en el gr\'{a}fico original debido a la presencia de una fuerte estacionalidad. 

\begin{figure}[H]
\centering
\includegraphics[width=5.32in,height=3.68in]{Graficos/Cap4-5/STcap47.eps}
\caption{Evoluci\'{o}n de la varianza de la SVM seg\'{u}n el Modelo 4-ARCH(1)}
%\label{fig7}
\end{figure}

\begin{observacion}
En el anexo C se presenta una serie de datos de temperaturas que se deben modelar por un modelo m\'{a}s general que el presentado aqu\'{i}; esto es, un modelo ARIMA(1,1,1)-GARCH(2,1).
\end{observacion}

\textbf{Modelos con una extensi\'{o}n ARCH asim\'{e}trica}\newline

Considerando el ejemplo anterior, se plantea buscar un modelo que considere alguna de las extensiones asim\'{e}tricas de los modelos ARCH. As\'{i}, se probaron varios modelos utilizando las representaciones EGARCH y TGARCH; sin embargo, no se encontraron buenos resultados. 

\begin{ejemplo}
(Aplicaci\'{o}n a una serie financiera): Para el siguiente ejemplo se consideran los precios diarios de cierre del \'{i}ndice Standar and Poor's 500 (S{\&}P 500), que es uno de los \'{i}ndices burs\'{a}tiles m\'{a}s importante de los Estados Unidos. El per\'{i}odo de an\'{a}lisis est\'{a} comprendido entre el 2 de marzo de 2009 y el 31 de diciembre de 2014, obteni\'{e}ndose un total de 1.471 datos (http://finance.yahoo.com/q/hp?s$=${\%}5EGSPC$+$Historical$+$Prices).
\end{ejemplo}

A continuaci\'{o}n se presenta el gr\'{a}fico de la serie a ser analizada:

\begin{figure}[H]
\centering
\includegraphics[width=5.83in,height=3.96in]{Graficos/Cap4-5/STcap48.eps}
\caption{Serie de los precios de cierre del \'{\i}ndice S{\&}P 500}
%\label{fig8}
\end{figure}

Evidentemente la serie no es estacionaria, por lo que se deber\'{i}a realizar alg\'{u}n procedimiento para volverla estacionaria. En el caso de los \'{i}ndices burs\'{a}tiles es muy com\'{u}n el uso de los logaritmos de los retornos de los precios de las series para poder realizar el an\'{a}lisis de las mismas. En este caso se utiliza la siguiente f\'{o}rmula:

\[
r_{t}=\mathrm{ln?}\left( \frac{P_{t}}{P_{t-1}} \right)
\]

Los $P_{t}$ son los precios de cierre diarios; los valores $r_{t}$ posteriormente se transforman en porcentajes. Ahora, se tiene la siguiente forma de la serie de log retornos:

\begin{figure}[H]
\centering
\includegraphics[width=5.73in,height=3.96in]{Graficos/Cap4-5/STcap49.eps}
\caption{Serie de las rentabilidades de los precios de cierre del \'{i}ndice S{\&}P 500}
%\label{fig9}
\end{figure}

A partir de este punto se trabajar\'{a} con la serie de log retornos en lugar de los precios del \'{i}ndice. Se realiza una prueba de Dickey -- Fuller para verificar la estacionariedad de la serie:

\begin{table}[H]
\centering
\begin{tabular}{p{110pt}p{70pt}ll}\hline\hline
& & t-Statistic & ~~Prob.* \\ \hline \hline
\multicolumn{2}{l}{Augmented Dickey-Fuller test statistic} & -41,62920 & ~0,0000 \\ \hline
Test critical values: & 1$\%$ level & -3,964304 & \\
& 5$\%$ level & -3,412872 & \\
& 10$\%$ level & -3,128424 & \\ \hline \hline
\end{tabular}
\caption{Prueba de Dickey - Fuller para $r_t$}
%\label{tab16}
\end{table}

Como se puede ver en la tabla anterior, la serie es estacionaria (Prob. menor que 0,05). Una vez verificada la estacionariedad de la serie, se procede a plantear un modelo ARIMA para eliminar la correlaci\'{o}n serial de los datos. Para ello se presentan los correlogramas de la serie en niveles:

\begin{figure}[H]
\centering
\includegraphics[width=4.51in,height=5.23in]{Graficos/Cap4-5/STcap410.eps}
\caption{FAC y FACP estimadas $r_{t}$}
%\label{fig10}
\end{figure}


En este tipo de series se vuelve un tanto complicado poder fijar de antemano los valores de p y q. Para este ejemplo, se escogen los siguientes coeficientes: AR(1) y MA(1) (se denominar\'{a} Modelo 1). As\'{i}, se obtiene: 

\begin{table}[H]
\centering
\begin{tabular}{ccccc}\hline\hline
Variable & Coefficient & Std. Error & t-Statistic & Prob.\\
C & 0,000734 & 0,000274 & 2,680441 & 0,0074 \\
AR(1) & -0,809534 & 0,066741 & -12,12955 & 0,0000 \\
MA(1) & 0,748635 & 0,076324 & 9,808685 & 0,0000 \\ \hline\hline
R-squared & 0,025091 & \multicolumn{2}{l}{Mean dependent var} & 0,000738 \\
Adjusted R-squared & 0,023761 & \multicolumn{2}{l}{S.D. dependent var} & 0,010989 \\
S.E. of regression & 0,010858 & \multicolumn{2}{l}{Akaike info criterion} & -6,205849 \\
Sum squared resid & 0,172825 & \multicolumn{2}{l}{Schwarz criterion} & -6,195041 \\
Log likelihood & 4561,196 & \multicolumn{2}{l}{Hannan-Quinn criter.} & -6,201819 \\
F-statistic & 18,86514 & \multicolumn{2}{l}{Durbin-Watson stat} & 2,001260 \\
Prob(F-statistic) & 0,000000 & & & \\ \hline\hline
\end{tabular}
\caption{Resumen de estad\'{i}sticas del Modelo 1 para $r_{t}$}
% \label{tab5}
\end{table}

\begin{figure}[H]
\centering
\includegraphics[width=4.51in,height=4.31in]{Graficos/Cap4-5/STcap411.eps}
\caption{FAC y FACP estimadas del Modelo 1 para $r_{t}$}
%\label{fig11}
\end{figure}

Como se puede observar en la tabla 4.5, los coeficientes son significativos. Sin embargo, al revisar los residuos (figura 4.11) se puede constatar que no son un ruido blanco; por lo tanto, hay que reformular el modelo. Observando las FAC y FACP, parecer\'{i}a adecuado agregar un retardo de orden 14 ya que en este punto los residuos se vuelven significativos.\newline

Se agreg\'{o} un AR(14), pero result\'{o} no significativo. Luego, se ingres\'{o} un retardo MA(14) que result\'{o} ser no significativo tambi\'{e}n. Se agregaron los coeficientes AR(2) y MA(2) que resultaron ser significativos, pero los residuos no eran un ruido blanco.\newline

Finalmente, luego de realizar varios intentos con diferentes coeficientes, se obtuvo un modelo en el que todos ellos fueron significativos (se denominar\'{a} Modelo 2):

\begin{table}[H]
\centering
\begin{tabular}{ccccc}
Variable & Coefficient & Std. Error & t-Statistic & Prob.\\ \hline\hline
C & 0,000757 & 0,000231 & 3,280979 & 0,0011 \\
AR(1) & -0,070311 & 0,025963 & -2,708108 & 0,0068 \\
AR(3) & -0,073718 & 0,025820 & -2,855059 & 0,0044 \\
AR(5) & -0,082387 & 0,025782 & -3,195503 & 0,0014 \\ \hline\hline
R-squared & 0,018832 & \multicolumn{2}{l}{Mean dependent var} & 0,000760 \\
Adjusted R-squared & 0,016818 & \multicolumn{2}{l}{S.D. dependent var} & 0,010923 \\
S.E. of regression & 0,010831 & \multicolumn{2}{l}{Akaike info criterion} & -6,210057 \\
Sum squared resid & 0,171395 & \multicolumn{2}{|p{120pt}|}{Schwarz criterion} & -6,195614 \\
Log likelihood & 4552,866 & \multicolumn{2}{l}{Hannan-Quinn criter.} & -6,204670 \\
F-statistic & 9,347342 & \multicolumn{2}{l}{Durbin-Watson stat} & 1,968465 \\
Prob(F-statistic) & 0,000004 & & &  \\ \hline\hline
\end{tabular}
\caption{Resumen de estad\'{i}sticas del Modelo 2 para $r_{t}$}
% \label{tab6}
\end{table}

Se puede observar que todos los coeficientes son significativos. Se procede ahora a realizar el an\'{a}lisis de los residuos para validar el modelo. Se presenta el correlograma que contiene las autocorrelaciones simples y parciales del modelo:

\begin{figure}[H]
\centering
\includegraphics[width=4.57in,height=6.20in]{Graficos/Cap4-5/STcap412.eps}
\caption{FAC y FACP estimadas del Modelo 2 para $r_{t}$}
%\label{fig12}
\end{figure}

Se puede observar un fen\'{o}meno que suele ocurrir cuando se modela este tipo de series: los residuos se comportan ``bien'' hasta un cierto punto y luego se vuelven significativos. En este caso, a partir del retardo 25 los residuos dejan de ser un ruido blanco; esto sugiere que se debe agregar un retardo de orden 25 al modelo.\newline

Cabe recalcar que es inusual el uso de un coeficiente tan alto como el de orden 25; sin embargo, en este tipo de series suele suceder que se necesitan valores altos de los coeficientes y no es problema usar uno tan alto dada la gran cantidad de datos que se tiene, As\'{i} se obtiene lo siguiente (Modelo 3):

% \begin{table}[H]
% \begin{center}
% \begin{tabular}{|p{100pt}|l|l|p{60pt}|l|}
% \hline
% & 
% & 
% & 
% & 
%  \\
% \hline
% & 
% & 
% & 
% & 
%  \\
% \hline
% Variable& 
% Coefficient& 
% Std, Error& 
% t-Statistic& 
% Prob,~~ \\
% \hline
% & 
% & 
% & 
% & 
%  \\
% \hline
% & 
% & 
% & 
% & 
%  \\
% \hline
% C& 
% 0,000746& 
% 0,000207& 
% 3,599620& 
% 0,0003 \\
% \hline
% AR(3)& 
% -0,077899& 
% 0,025863& 
% -3,011928& 
% 0,0026 \\
% \hline
% AR(5)& 
% -0,080599& 
% 0,025807& 
% -3,123110& 
% 0,0018 \\
% \hline
% AR(1)& 
% -0,070427& 
% 0,025972& 
% -2,711692& 
% 0,0068 \\
% \hline
% MA(25)& 
% -0,097947& 
% 0,026205& 
% -3,737729& 
% 0,0002 \\
% \hline
% & 
% & 
% & 
% & 
%  \\
% \hline
% & 
% & 
% & 
% & 
%  \\
% \hline
% R-squared& 
% 0,027780& 
% \multicolumn{2}{|p{120pt}|}{~~~~Mean dependent var} & 
% 0,000760 \\
% \hline
% Adjusted R-squared& 
% 0,025116& 
% \multicolumn{2}{|p{120pt}|}{~~~~S,D, dependent var} & 
% 0,010923 \\
% \hline
% S,E, of regression& 
% 0,010785& 
% \multicolumn{2}{|p{120pt}|}{~~~~Akaike info criterion} & 
% -6,217853 \\
% \hline
% Sum squared resid& 
% 0,169832& 
% \multicolumn{2}{|p{120pt}|}{~~~~Schwarz criterion} & 
% -6,199799 \\
% \hline
% Log likelihood& 
% 4559,577& 
% \multicolumn{2}{|p{120pt}|}{~~~~Hannan-Quinn criter,} & 
% -6,211119 \\
% \hline
% F-statistic& 
% 10,42938& 
% \multicolumn{2}{|p{120pt}|}{~~~~Durbin-Watson stat} & 
% 1,966958 \\
% \hline
% Prob(F-statistic)& 
% 0,000000& 
% & 
% & 
%  \\
% \hline
% & 
% & 
% & 
% & 
%  \\
% \hline
% & 
% & 
% & 
% & 
%  \\
% \hline
% \end{tabular}
% \label{tab7}
% \end{center}
% \end{table}
% 
% \begin{center}
% Tabla 4.7: Resumen de estad\'{\i}sticas del Modelo 3 para $r_{t}$
% \end{center}
% 
% \begin{figure}[H]
% \includegraphics[width=4.54in,height=4.30in]{STcap413.eps}
% \label{fig13}
% \end{figure}
% 
% \begin{center}
% Figura 4.13: FAC y FACP estimadas del Modelo 3 para $r_{t}$
% \end{center}
% 
% Como se puede observar, el Modelo 3 result\'{o} ser adecuado para los datos, 
% todos los coeficientes son significativos y los residuos pueden considerarse 
% como un ruido blanco, Con este modelo se elimina la correlaci\'{o}n serial 
% de los datos; ahora se analiza la posibilidad de que exista volatilidad en 
% los datos, para ello se presentan los correlogramas de los residuos al 
% cuadrado:
% 
% \begin{figure}[H]
% \includegraphics[width=4.52in,height=4.31in]{STcap414.eps}
% \label{fig14}
% \end{figure}
% 
% \begin{center}
% Figura 4.14: FAC y FACP de los residuos al cuadrado estimadas del Modelo 3 
% para $r_{t}$
% \end{center}
% 
% Los residuos cuadr\'{a}ticos del modelo indican que existe volatilidad, por 
% lo que, es necesario plantear un modelo ARCH -- GARCH o sus extensiones 
% sim\'{e}tricas o asim\'{e}tricas para poder modelar el efecto de 
% volatilidad, Resulta complicado plantear el orden de los modelos ARCH -- 
% GARCH; es recomendable probar los modelos con retardos 0, 1 o 2 y las 
% posibles combinaciones entre ellos para los modelos sim\'{e}tricos (ARCH, 
% GARCH y TARCH),
% 
% Se ensayaron todos los posibles modelos sim\'{e}tricos, pero ninguno dio 
% buenos resultados; por lo que se inici\'{o} la b\'{u}squeda de un modelo 
% asim\'{e}trico, Se plantea un EGARCH (1,0) con 1 grado de asimetr\'{\i}a a 
% partir del Modelo 3, Se obtiene lo siguiente:
% 
% \begin{table}[H]
% \begin{center}
% \begin{tabular}{|p{100pt}|l|p{60pt}|l|l|}
% \hline
% & 
% & 
% & 
% & 
%  \\
% \hline
% & 
% & 
% & 
% & 
%  \\
% \hline
% Variable& 
% Coefficient& 
% Std, Error& 
% z-Statistic& 
% Prob,~~ \\
% \hline
% & 
% & 
% & 
% & 
%  \\
% \hline
% & 
% & 
% & 
% & 
%  \\
% \hline
% C& 
% 0,000757& 
% 0,000210& 
% 3,610536& 
% 0,0003 \\
% \hline
% AR(3)& 
% -0,079949& 
% 0,016916& 
% -4,726127& 
% 0,0000 \\
% \hline
% AR(5)& 
% -0,076020& 
% 0,016552& 
% -4,592778& 
% 0,0000 \\
% \hline
% AR(1)& 
% -0,086650& 
% 0,023149& 
% -3,743161& 
% 0,0002 \\
% \hline
% MA(25)& 
% -0,099414& 
% 0,022763& 
% -4,367456& 
% 0,0000 \\
% \hline
% & 
% & 
% & 
% & 
%  \\
% \hline
% & 
% & 
% & 
% & 
%  \\
% \hline
% & 
% \multicolumn{2}{|p{115pt}|}{Variance Equation} & 
% & 
%  \\
% \hline
% & 
% & 
% & 
% & 
%  \\
% \hline
% & 
% & 
% & 
% & 
%  \\
% \hline
% C(6)& 
% -9,221394& 
% 0,032689& 
% -282,0918& 
% 0,0000 \\
% \hline
% C(7)& 
% 0,182044& 
% 0,039466& 
% 4,612698& 
% 0,0000 \\
% \hline
% C(8)& 
% -0,163505& 
% 0,029388& 
% -5,563628& 
% 0,0000 \\
% \hline
% & 
% & 
% & 
% & 
%  \\
% \hline
% & 
% & 
% & 
% & 
%  \\
% \hline
% \end{tabular}
% \label{tab8}
% \end{center}
% \end{table}
% 
% \begin{center}
% Tabla 4.8: Resumen de estad\'{\i}sticas del Modelo 3-EGARCH(1,0) para 
% $r_{t}$
% \end{center}
% 
% \begin{figure}[H]
% \includegraphics[width=4.51in,height=4.30in]{STcap415.eps}
% \label{fig15}
% \end{figure}
% 
% \begin{center}
% Figura 4.15: FAC y FACP de los residuos al cuadrado estimadas del Modelo 
% 3-EGARCH(1,0) para $r_{t}$
% \end{center}
% 
% Como se puede ver en la tabla 4.8, todos los coeficientes son 
% significativos, por lo que se puede considerar como un candidato a modelo 
% adecuado para los datos, Sin embargo, al revisar el correlograma de los 
% residuos al cuadrado (figura 4.15) se puede concluir que estos no son un 
% ruido blanco, por lo que el modelo no es adecuado para los datos y es 
% necesario reformularlo,
% 
% As\'{\i}, se prueba con los modelos EGARCH(0,1) con un grado de 
% asimetr\'{\i}a; pero no result\'{o} bueno, Tambi\'{e}n se intent\'{o} con un 
% modelo EGARCH(1,1) con un grado de asimetr\'{\i}a que tampoco fue adecuado, 
% Luego, se consider\'{o} el modelo EGARCH (2,1) con un grado de 
% asimetr\'{\i}a (se lo denominar\'{a} Modelo 4); as\'{\i}, se obtiene
% 
% \begin{table}[H]
% \begin{center}
% \begin{tabular}{|p{100pt}|l|p{60pt}|l|l|}
% \hline
% & 
% & 
% & 
% & 
%  \\
% \hline
% & 
% & 
% & 
% & 
%  \\
% \hline
% Variable& 
% Coefficient& 
% Std, Error& 
% z-Statistic& 
% Prob,~~ \\
% \hline
% & 
% & 
% & 
% & 
%  \\
% \hline
% & 
% & 
% & 
% & 
%  \\
% \hline
% C& 
% 0,000374& 
% 0,000179& 
% 2,091286& 
% 0,0365 \\
% \hline
% AR(1)& 
% -0,031360& 
% 0,022832& 
% -1,373532& 
% 0,1696 \\
% \hline
% AR(3)& 
% -0,012167& 
% 0,025680& 
% -0,473782& 
% 0,6357 \\
% \hline
% AR(5)& 
% -0,036932& 
% 0,024399& 
% -1,513683& 
% 0,1301 \\
% \hline
% MA(25)& 
% -0,059493& 
% 0,019810& 
% -3,003174& 
% 0,0027 \\
% \hline
% & 
% & 
% & 
% & 
%  \\
% \hline
% & 
% & 
% & 
% & 
%  \\
% \hline
% & 
% \multicolumn{2}{|p{115pt}|}{Variance Equation} & 
% & 
%  \\
% \hline
% & 
% & 
% & 
% & 
%  \\
% \hline
% & 
% & 
% & 
% & 
%  \\
% \hline
% C(6)& 
% -0,650709& 
% 0,070982& 
% -9,167198& 
% 0,0000 \\
% \hline
% C(7)& 
% -0,119657& 
% 0,058242& 
% -2,054483& 
% 0,0399 \\
% \hline
% C(8)& 
% 0,332807& 
% 0,060940& 
% 5,461182& 
% 0,0000 \\
% \hline
% C(9)& 
% -0,193797& 
% 0,018401& 
% -10,53209& 
% 0,0000 \\
% \hline
% C(10)& 
% 0,947998& 
% 0,006261& 
% 151,4142& 
% 0,0000 \\
% \hline
% & 
% & 
% & 
% & 
%  \\
% \hline
% & 
% & 
% & 
% & 
%  \\
% \hline
% \end{tabular}
% \label{tab9}
% \end{center}
% \end{table}
% 
% \begin{center}
% Tabla 4.9: Resumen de estad\'{\i}sticas del Modelo 4 para $r_{t}$
% \end{center}
% 
% \begin{figure}[H]
% \includegraphics[width=4.52in,height=4.33in]{STcap416.eps}
% \label{fig16}
% \end{figure}
% 
% \begin{center}
% Figura 4.16: FAC y FACP de los residuos al cuadrado estimadas del Modelo 4 
% para $r_{t}$
% \end{center}
% 
% En la tabla 4.9 se muestra el resumen de estad\'{\i}sticas para el Modelo 4: 
% Se puede observar que algunos coeficientes son no significativos; sin 
% embargo, con esta especificaci\'{o}n se obtiene que los residuos al cuadrado 
% se pueden considerar como un ruido blanco, Por lo tanto, es necesario 
% cambiar la parte ARMA del modelo, 
% 
% Se inicia eliminando el coeficiente correspondiente al AR(3), pero esto hizo 
% que los coeficientes AR(1) y AR(5) sean no significativos, Luego, se elimina 
% el coeficiente AR(1), pero el coeficiente AR(5) es no significativo y 
% tambi\'{e}n se lo saca del modelo. Lo anterior hace que la constante sea no 
% significativa y tambi\'{e}n se la retira del modelo, Finalmente, se obtiene 
% un modelo ARMA-EGARCH (0,25) -- (2,1) con un grado de asimetr\'{\i}a (Modelo 
% 5), Se obtiene lo siguiente:
% 
% \begin{table}[H]
% \begin{center}
% \begin{tabular}{|p{100pt}|l|p{60pt}|p{60pt}|l|}
% \hline
% & 
% & 
% & 
% & 
%  \\
% \hline
% & 
% & 
% & 
% & 
%  \\
% \hline
% Variable& 
% Coefficient& 
% Std, Error& 
% z-Statistic& 
% Prob,~~ \\
% \hline
% & 
% & 
% & 
% & 
%  \\
% \hline
% & 
% & 
% & 
% & 
%  \\
% \hline
% MA(25)& 
% -0,060381& 
% 0,019660& 
% -3,071285& 
% 0,0021 \\
% \hline
% & 
% & 
% & 
% & 
%  \\
% \hline
% & 
% & 
% & 
% & 
%  \\
% \hline
% & 
% \multicolumn{2}{|p{115pt}|}{Variance Equation} & 
% & 
%  \\
% \hline
% & 
% & 
% & 
% & 
%  \\
% \hline
% & 
% & 
% & 
% & 
%  \\
% \hline
% C(2)& 
% -0,662879& 
% 0,071336& 
% -9,292324& 
% 0,0000 \\
% \hline
% C(3)& 
% -0,110414& 
% 0,055841& 
% -1,977307& 
% 0,0480 \\
% \hline
% C(4)& 
% 0,334526& 
% 0,059085& 
% 5,661796& 
% 0,0000 \\
% \hline
% C(5)& 
% -0,213870& 
% 0,018972& 
% -11,27318& 
% 0,0000 \\
% \hline
% C(6)& 
% 0,946505& 
% 0,006513& 
% 145,3187& 
% 0,0000 \\
% \hline
% & 
% & 
% & 
% & 
%  \\
% \hline
% & 
% & 
% & 
% & 
%  \\
% \hline
% R-squared& 
% 0,003666& 
% \multicolumn{2}{|p{120pt}|}{~~~~Mean dependent var} & 
% 0,000733 \\
% \hline
% Adjusted R-squared& 
% 0,003666& 
% \multicolumn{2}{|p{120pt}|}{~~~~S,D, dependent var} & 
% 0,010987 \\
% \hline
% S,E, of regression& 
% 0,010967& 
% \multicolumn{2}{|p{120pt}|}{~~~~Akaike info criterion} & 
% -6,578077 \\
% \hline
% Sum squared resid& 
% 0,176675& 
% \multicolumn{2}{|p{120pt}|}{~~~~Schwarz criterion} & 
% -6,556473 \\
% \hline
% Log likelihood& 
% 4840,887& 
% \multicolumn{2}{|p{120pt}|}{~~~~Hannan-Quinn criter,} & 
% -6,570021 \\
% \hline
% Durbin-Watson stat& 
% 2,155425& 
% & 
% & 
%  \\
% \hline
% & 
% & 
% & 
% & 
%  \\
% \hline
% & 
% & 
% & 
% & 
%  \\
% \hline
% \end{tabular}
% \label{tab10}
% \end{center}
% \end{table}
% 
% \begin{center}
% Tabla 4.10: Resumen de estad\'{\i}sticas del Modelo 5 para $r_{t}$
% \end{center}
% 
% Se puede observar que todos los coeficientes son significativos y, por lo 
% tanto, es un posible modelo adecuado para los datos, Para verificar la 
% validez de los datos, se presentan los correologramas de los residuos 
% estimados (figura 4.17) y de los residuos al cuadrado estimados del modelo 
% (figura 4.18):
% 
% \begin{figure}[H]
% \includegraphics[width=4.52in,height=4.32in]{STcap417.eps}
% \label{fig17}
% \end{figure}
% 
% \begin{center}
% Figura 4.17: FAC y FACP de los residuos al cuadrado estimadas del Modelo 5 
% para $r_{t}$
% \end{center}
% 
% \begin{figure}[H]
% \includegraphics[width=4.52in,height=4.31in]{STcap418.eps}
% \label{fig18}
% \end{figure}
% 
% \begin{center}
% Figura 4.18: FAC y FACP de los residuos al cuadrado estimadas del Modelo 5 
% para $r_{t}$
% \end{center}
% 
% A partir de las figuras anteriores se puede concluir que el modelo planteado 
% es adecuado para los datos. El modelo final obtenido es:
% \[
% r_{t}=\varepsilon_{t}-0,06\varepsilon_{t-25}+\sqrt h_{t} +u_{t}
% \]
% \[
% \ln \left( h_{t} \right)=-0,66-0,11\left| u_{t-1} \right|+0,33\left| u_{t-2} 
% \right|-0,21u_{t}+0,94\ln \left( h_{t-1} \right)
% \]